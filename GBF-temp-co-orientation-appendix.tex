\documentclass[12pt]{article}
\paperheight=11in
%\paperwidth=8.5in
\renewcommand{\baselinestretch}{1.04}
\usepackage{amsmath,amsthm,verbatim,amssymb,amsfonts,amscd, graphicx, mathrsfs, hyperref,mathtools,multicol, enumitem,bbm,diagrams}
\usepackage[usenames,dvipsnames]{color}
\usepackage{amstext}
%\usepackage{cite}
%\usepackage[notref]{showkeys}
\usepackage{color}


\newcommand\purple[1]{\marginpar{replacement}\textcolor{Purple}{#1}}     
\newcommand\blue[1]{\marginpar{new}\textcolor{blue}{#1}}                       %
\newcommand\red[1]{\marginpar{??}\textcolor{red}{#1}}                         %
\newcommand\green[1]{\marginpar{delete ok?}\textcolor{green}{#1}}
%\newcommand\ownremark[1]{\marginpar{remark for own use}\textcolor[rgb]{0.5,0.5,0}{#1}}                   %


\topmargin0.0cm
\headheight0.0cm
\headsep0.0cm
\oddsidemargin0.0cm
\textheight23.0cm
\textwidth16.5cm
\footskip1.0cm
\theoremstyle{plain}
\newtheorem{theorem}{Theorem}[section]
\newtheorem{corollary}[theorem]{Corollary}
\newtheorem{lemma}[theorem]{Lemma}
\newtheorem{proposition}[theorem]{Proposition}
\newtheorem*{theorem*}{Theorem}
\newtheorem*{lemma*}{Lemma}
\newtheorem*{proposition*}{Proposition}


\theoremstyle{definition}
\newtheorem{definition}[theorem]{Definition}

\theoremstyle{remark}
\newtheorem{remark}[theorem]{Remark}
\newtheorem{example}[theorem]{Example}
\newtheorem{exercise}[theorem]{Exercise}
\newcommand{\codim}{\text{codim}}
 

\newcommand{\primeset}[1]{#1}
\newcommand{\id}{\textup{id}}
\newcommand{\onto}{\twoheadrightarrow}
\newcommand{\hra}{\hookrightarrow} 
\newcommand{\Td}[1]{\Tilde{#1}}
\newcommand{\td}[1]{\tilde{#1}}
\newcommand{\into}{\hookrightarrow}
\newcommand{\PP}{\mathbb{P}}
\newcommand{\bS}{\mathbb{S}}
\newcommand{\X}{\mathbb{X}}
\newcommand{\Z}{\mathbb{Z}}
\newcommand{\Q}{\mathbb{Q}}
\newcommand{\R}{\mathbb{R}}
\newcommand{\G}{\mathbb{G}}
\newcommand{\N}{\mathbb{N}}
\newcommand{\F}{\mathbb{F}}
\newcommand{\C}{\mathbb{C}}
\newcommand{\D}{\mathbb{D}}

\renewcommand{\L}{\mathbb{L}}
\newcommand{\bd}{\partial}
\newcommand{\pf}{\pitchfork}
\newcommand{\ra}{\rightarrow}
\newcommand{\la}{\leftarrow}
\newcommand{\Ra}{\Rightarrow}
\renewcommand{\H}{\mathbb H}
\newcommand{\rla}{\RightLeftarrow}
\newcommand{\mc}[1]{\mathcal{#1}}
\newcommand{\ms}[1]{\mathscr{#1}}
\newcommand{\bb}[1]{\mathbb{#1}}
\newcommand{\dlim}{\varinjlim}
\newcommand{\vg}{\varGamma}
\newcommand{\blm}[2]{\langle  #1 , #2 \rangle}
\newcommand{\bl}[2]{\left( #1 , #2 \right)}
\newcommand{\vs}{\varSigma}
\newcommand{\holink}{\text{holink}}
\newcommand{\map}{\operatorname{map}}
\newcommand{\hl}{\operatorname{holink}}
\newcommand{\wt}{\widetilde}
\renewcommand{\hom}{\text{Hom}}
\newcommand{\Hom}{\text{Hom}}
\newcommand{\SHom}{\text{\emph{Hom}}}
\newcommand{\Ext}{\text{Ext}}
\newcommand{\mf}{\mathfrak}
\newcommand{\ih}{IH^{\bar p}}
\newcommand{\di}{\text{dim}}
\newcommand{\im}{\text{im}}
\newcommand{\cok}{\text{cok}}
\newcommand{\coim}{\text{coim}}
\newcommand{\bp}{\boxplus}
%\renewcommand{\P}{\mathbb P}
\newcommand{\q}{\mathfrak q}
\newcommand{\supp}{\text{supp}}
\newcommand{\singsupp}{\text{singsupp}}
\newcommand{\Dom}{\text{Dom}}
\newcommand{\LPDO}{\text{LPDO}}
\newcommand{\PsiDO}{\Psi\text{DO}}

\newcommand{\ka}{\kappa}

\newcommand{\fl}{\text{FL}}
\newcommand{\wfl}{\text{WFL}}
\newcommand{\Ker}{\mbox{Kernel }}
%\newcommand{\p}{\mf{p}}
\newcommand{\p}{\mathbbm{p}}
%\newcommand{\p}{\mathpzc{p}}
\newcommand{\Vol}{\text{Vol}}
\newcommand{\uW}{\underline{W}}
\newcommand{\udW}{\underline{\partial W}}
\newcommand{\uV}{\underline{V}}



\newcommand{\sect}[1]{\vskip1cm \noindent\paragraph{#1}}

\newcommand{\ttau}{\text{\texthtt}}

\newcommand{\xr}{\xrightarrow}
\newcommand{\xl}{\xleftarrow}

\DeclareRobustCommand{\zvec}[1]{%
  \mathrlap{\vec{\mkern-2mu\phantom{#1}}}#1%
}

\DeclareMathAlphabet{\mathpzc}{OT1}{pzc}{m}{it}
\newcommand{\cman}{\mathrm{cMan}}

\newcommand{\Or}{{\rm Det}}

\begin{document}

In \cite{Lipy14}, Lipyanskiy uses a different notion of co-orientations, which he initially refers to as orientations of maps, than the one we have used to define geometric cochains. We here discuss Lipyanskiy's co-orientation and show that it is equivalent to our definition, up to possible sign conventions.

To define, co-orientations, Lipyanskiy utilizes the determinant line bundles of Donaldson and Kronheimer in \cite[Section 5.2.1]{DoKr90}. There, in a slightly more general setting, the determinant line bundle of a map of vector bundles $F:E\to E'$ over $M$ is defined, in our notation from Section REF, as $$\Or(\ker F)\otimes \Or(\cok(F))^*,$$ 
where the $*$ over $\Or(\cok(F))$ denotes the dual bundle. In the setting of a smooth map $f:M\to N$, we can think of the derivative $Df$ as a map $Df:TM\to f^*(TN)$ (CHECK), and then a co-orientation of $f$ is an orientation of the determinant line bundle associated to $Df$. If $\ker(Df)$ and $\cok(Df)$ are well-defined vector bundles. Then we can relate this notion of co-orientation to ours as follows.

Returning to the general setting of the bundle map $F:E\to E'$ and choosing splittings, we can write $E= \ker(F)\oplus \im(F)$ and $E'=\im(F)\oplus \cok(F)$, and then $\Or(E)\cong \Or(\ker(F))\otimes \Or(\im(F))$ and $\Or(E')\cong \Or(\im(F))\otimes \Or(\cok(F))$. 
So $\Or(E)$ and $\Or(E')$ are isomorphic (and hence there exists a non-zero section to $\Hom(E,E')$) if and only if  $\Or(\ker(F))\cong \Or(\cok(F))$. In this case we observe that $\Hom(\Or(E),\Or(E'))$, $$\Hom(\Or(\ker(F)), \Or(\cok(F)))\cong \Or(\ker(F))^*\otimes \Or(\cok(F)),$$ 
and $$(\Or(\ker(F))^*\otimes \Or(\cok(DF)))^*\cong \Or(\ker(F))\otimes \Or(\cok(F))^*$$
are all trivial lines bundles, and hence isomorphic. 

So, up to the actual co-orientation conventions, the two notions of co-orientability coincide. We leave it to the reader to define the isomorphisms in sufficient detail to carry a particular co-orientation as defined in \cref{S: co-orientations} to one as defined here.

There are two problems with the preceding analysis: the assumption that the kernel and cokernel are well-defined vector bundles and the possible reliance of the isomorphisms on a choice of splittings. The latter issue is dispensed with pretty easily by \cref{??} below. The first requires a bit more work, as in general $\ker(Df)$ and $\cok(Df)$ do not even have the same dimensions from fiber to fiber. The solution is to generalize the determinant line bundle as in \cite{DoKr90} to a definition that is always well defined and that isomorphic to $\Or(\ker F)\otimes \Or(\cok(F))^*$ when it is also well defined. 

For this, let $\underline{\C}^N$ be the trivial $\R^N$ bundle over $M$, and suppose we have a map $\psi:\underline{\R}^N\to E'$ such that $F\oplus \psi:E\oplus \underline{\R}^N$ is surjective\footnote{Donaldson and Kronheimer work with complex vector bundles, so \cite{DoKr90} features $\underline{\C}^N$ rather than $\underline{\R}^N$.}. This will always be true in our setting, as we observed in Section \ref{S: manifolds with corners} that work of Joyce and Melrose implies that smooth manifolds with corners can always be embedded in finite dimensional Euclidean space. Hence tangent bundles are subbundles of trivial bundles and the images of projections of trivial bundles (or, up isomorphism, quotients of the trivial bundle by their orthogonal bundles). The gain is that $F\oplus \psi$ now has trivial cokernel and a kernel that is a vector bundle as now the fibers of the kernel have a fixed dimension. Now let $$\ms L=\Or(\ker(F\oplus \psi))\otimes \Or(\underline{\R}^N)^*\cong \Or(\ker(F\oplus \psi)).$$ 

In the case where the $\ker(F)$ and $\cok(F)$ were already vector bundles, $\ms L$ is isomorphic to the determinant line bundle of $F$ using the following two lemmas: 

\begin{lemma}
If $F:E\to E'$ and $\psi:\underline{\R}^N\to E'$  are bundle maps with $F\oplus \psi: E\oplus \underline{\R}^N$ surjective, then the following sequence of vector bundles is exact:
\begin{diagram}
0&\rTo&\ker(F)&\rTo&\ker(F\oplus \psi)&\rTo& \underline{\R}^N&\rTo& \cok(F)&\rTo&0.
\end{diagram} 
\end{lemma}

\begin{lemma}\label{L: det sequence}
Given an exact sequence of vector bundles
\begin{diagram}
0&\rTo&V_1&\rTo& \cdots &\rTo& V_m&]rTo&0,
\end{diagram}
there is a canonical isomorphism 
$$\underset{i\text{ odd}}{\otimes}\Or(V_i)\cong \underset{i\text{ even}}{\otimes}\Or(V_i).$$
\end{lemma}

Assuming these lemmas for now, they combine to give us an isomorphism
$$\Or(\ker(F))\otimes \Or(\underline{\R}^N)\cong \Or(\ker(F\oplus \psi))\otimes \Or(\cok(F)).$$
Multiplying both sides by $\Or(\underline{\R}^N)^*\otimes \Or(\cok(F))^*$ and using that in general for a line bundle $L$ we have $L\otimes L^*\cong \underline{\R}$, we obtain
$$\Or(\ker(F))\otimes \Or(\cok(F))^*\cong \Or(\ker(F\oplus \psi))\otimes \Or(\underline{\R}^N)^*.$$
So, as promised, the two definitions agree (up to isomorphism?) when $\ker(F)$ and $\cok(F)$ are defined. 

We also observe that in the context of a map $f:M\to N$ of manifolds, if $\psi:\underline{R}^N\to f^*(TN)$ is surjective, then applying \cref{L: det sequence} to exact sequence
\begin{diagram}
0&\rTo&\ker(Df\otimes \psi)&\rTo&TM\oplus\underline{\R}^N&\rTo^{Df\oplus \psi}&f^*(TN)&\rTo&0,
\end{diagram}
we obtain an isomorphism
$$\Or(\ker(Df\otimes \psi))\otimes\Or(f^*(TN))\cong \Or(TM\oplus\underline{\R}^N)\cong \Or(TM)\otimes \Or(\underline{\R}^N)).$$
So once again employing the duals, we obtain an isomorphism between the Donaldson-Kronheimer line bundle $\Or(\ker(Df\otimes \psi))\otimes \Or(\underline{\R}^N))^*$ used by Lipyanskiy and $\Or(TM)\otimes\Or(f^*(TN))^*\cong \Hom(\Or(f^*(TN)),\Or(TM))$, while our notion of co-orientation utilizes the line bundle $\Hom(\Or(TM), \Or(f^*(TN)))$, which is dual to  $\Hom(\Or(f^*(TN)),\Or(TM))$. So, once again we observe that our line bundle $\Hom(\Or(TM), \Or(f^*(TN)))$ is trivial, and so possesses a non-zero section, if and only if the Donaldson-Kronheimer determinant line bundle for the map $Df$ is trivial. This also shows that the Donaldson-Kronheimer construction is independent of the choice of $\C^N$ and $\psi$. 






\bibliographystyle{amsplain}
\bibliography{../../bib}




\end{document}
