\documentclass{amsart}
\usepackage{amssymb, amsmath, amsfonts, verbatim, mathtools, stmaryrd, tabularx}

%AMM: Paul Taylor's package with the fix for NEVER HAPPEN error
\usepackage{diagrams}
\makeatletter
\def\foo#1\endgraf\unskip#2\foo{\def\row@to@buffer{#1\endgraf\unskip\unskip#2}}
\expandafter\foo\row@to@buffer\foo
\makeatother

\renewcommand{\arraystretch}{1.5}
\setlength{\tabcolsep}{15pt}

\usepackage{bm}
\usepackage{mathbbol}
\usepackage{ulem} % Used in the To-Do list
\usepackage[all]{xy}
\usepackage{tikz-cd}
\usepackage{tikz}
\usepackage{caption}
\usepackage{subcaption}

\usepackage[bookmarks=true, linktocpage=true,
bookmarksnumbered=true, breaklinks=true,
pdfstartview=FitH, hyperfigures=false,
plainpages=false, naturalnames=true,
colorlinks=true, pagebackref=true,
pdfpagelabels]{hyperref}
\hypersetup{
	linkcolor  = blue,
	citecolor  = blue,
	urlcolor   = blue,
	colorlinks = true,
}
\usepackage[capitalise,noabbrev,nosort]{cleveref}

\addtolength{\textwidth}{1.4in}
\oddsidemargin=0in
\evensidemargin=0in

\newtheorem{theorem}{Theorem}[section]
\newtheorem{lemma}[theorem]{Lemma}
\newtheorem{corollary}[theorem]{Corollary}
\newtheorem{proposition}[theorem]{Proposition}

\theoremstyle{definition}
\newtheorem{definition}[theorem]{Definition}
\newtheorem{remark}[theorem]{Remark}
\newtheorem{example}[theorem]{Example}
\newtheorem{convention}[theorem]{Convention}
\newtheorem*{notation}{Notation}

\newcommand{\codim}{\text{codim}}
\newcommand{\mc}[1]{\mathcal{#1}}
\newcommand{\R}{\mathbb{R}}
\newcommand{\Z}{\mathbb{Z}}
\newcommand{\N}{\mathbb{N}}
\newcommand{\id}{\mathrm{id}}
\newcommand{\e}{\mathbf{e}}
\newcommand{\f}{\mathbf{f}}
\newcommand{\bb}{\mathbf{b}}
\newcommand{\interval}{\mathbb{I}}
\newcommand{\simplex}{\mathbb{\Delta}}
\newcommand{\I}{\square}
\newcommand{\chains}{C_*}
\newcommand{\chainsn}{C_n}
\newcommand{\cochains}{C^*}
\newcommand{\chain}[1]{C_{#1}}
\newcommand{\cochain}[1]{C^{#1}}
\newcommand{\cd}{{\rm cd}}
\DeclareMathOperator*{\colim}{colim}
\newcommand{\ori}{\mathfrak{or}}
\newcommand{\cor}{\mathfrak{cor}}
\newcommand{\sh}{\mathfrak{sh}}
\newcommand{\cs}{{\rm cst}}
\newcommand{\os}{{\rm gst}}
\newcommand{\Or}{{\rm Det}}
\renewcommand{\Im}{{\rm Im} \;}
\newcommand{\im}{\text{im}}
\newcommand{\into}{\hookrightarrow}
\newcommand{\init}{{\rm Init}}
\newcommand{\term}{{\rm Term}}
\newcommand{\sms}{\smallsmile}
\newcommand{\pf}{\pitchfork}
\newcommand{\bd}{\partial}
\newcommand{\td}{\tilde}
\newcommand{\Hom}{\textup{Hom}}
\newcommand\red[1]{\textcolor{red}{#1}}
\newcommand\blue[1]{\textcolor{blue}{#1}}
\newcommand\purple[1]{\textcolor{purple}{#1}}
\newcommand{\uW}{\underline{W}}
\newcommand{\udW}{\underline{\partial W}}
\newcommand{\uV}{\underline{V}}
\newcommand{\uM}{\underline{M}}
\newcommand{\uN}{\underline{N}}
\newcommand{\xr}{\xrightarrow}
\newcommand{\xl}{\xleftarrow}
\newcommand{\uX}{\underline{X}}
%\newcommand{\ker}{\rm ker}
\hyphenation{to-pol-o-gy}
\newcommand{\cman}{\mathrm{cMan}}
\newcommand{\Cre}{\mathrm{Cre}}
\newcommand{\aug}{\mathbf{a}}
\newcommand{\vertices}{{\rm Vert}}       % vertices
\newcommand{\jinterval}{\mathbb{J}}
\newcommand{\onto}{\twoheadrightarrow}
\newcommand{\diag}{\mathbf{d}}
\newcommand{\Ext}{\text{Ext}}
\newcommand{\mf}{\mathfrak}
\newcommand{\creflastconjunction}{, and\nobreakspace} %makes cleveref use serial commas

%GBF: Indents the table of contents.
\setcounter{tocdepth}{4}% to get subsubsections in toc
\let\oldtocsection=\tocsection
\let\oldtocsubsection=\tocsubsection
\let\oldtocsubsubsection=\tocsubsubsection
\renewcommand{\tocsection}[2]{\hspace{0em}\oldtocsection{#1}{#2}}
\renewcommand{\tocsubsection}[2]{\hspace{1em}\oldtocsubsection{#1}{#2}}
\renewcommand{\tocsubsubsection}[2]{\hspace{2em}\oldtocsubsubsection{#1}{#2}}

%%%%%%%%%%%%%%%%%%%%%%%%%%%%%%%%%%%%%%%%%%%%%%%%%%%%%%%%%%%%%%%

\title{Geometric cohomology}

% Greg
\author[G. Friedman]{Greg Friedman}
\address{Department of Mathematics, Texas Christian University}
\email{\href{mailto:g.friedman@tcu.edu}{g.friedman@tcu.edu}}
\thanks{This work was partially supported by a grant from the Simons Foundation (\#839707 to Greg Friedman)}

% Anibal
\author[A. Medina-Mardones]{Anibal M. Medina-Mardones}
\address{LAGA, Universit\'e Sorbonne Paris Nord}
\email{\href{mailto:medina-mardones@math.univ-paris13.fr}{medina-mardones@math.univ-paris13.fr}}

% Dev
\author[D. Sinha]{Dev Sinha}
\address{Mathematics Department, University of Oregon}
\email{\href{mailto:dps@uoregon.edu}{dps@uoregon.edu}}

\begin{document}
	\begin{abstract}
		We set up the theory of geometric cohomology.
	\end{abstract}
	\maketitle
	\tableofcontents

	% !TEX root = ../foundations.tex

\section{Introduction}\label{intro}

We fully develop a geometric approach to ordinary homology and cohomology on smooth manifolds, with classes represented by smooth maps from manifolds to our target manifold of interest.
Such a development is in line with thinking about homology dating back to Poincar\'e and Lefschetz, but is also a time honored approach to cohomology through intersection theory and Thom classes.
Thom's seminal work on bordism theory showed that not all homology classes can be represented by pushing forward their fundamental classes and thus necessitates a broader notion of representing manifold.
The one we utilize here is that of manifolds with corners, the smallest category containing manifolds with boundary, as needed to define homologies, and which is closed under transverse pullback, which we use to define multiplicative structures.
The study of these multiplicative structures -- a partially defined ring structure on cochains and a module over it on chains -- is also in line with the classical perspective, and their rigorous development is a central goal of this project.
While these give rise to the usual cup and cap products on cohomology and homology, they are markedly different at the chain and cochain level when compared to simplicial, cubical, or singular approaches, providing greater geometric access and enjoying strict commutativity when defined.

Approaches to ordinary homology and cohomology were an active area of development eighty years ago, and indeed we highlight that our work is parallel to de Rham theory, allowing one to make calculations and invoke geometry for manifolds, rather than rely on transcendental approaches to cochains as formal linear duals.
Compared with de Rham theory, our present work has the key advantage of being defined over the integers.
Slightly more recent developments with similar goals include Goresky's work on geometric homology and cohomology of Whitney stratified objects \cite{goresky1981stratified} and the book by Buoncristiano, Rourke, and Sanderson \cite{buoncristiano1976homology}.
For concrete applications of ideas in line with our viewpoint, we mention Cochran's work \cite{cochran1990milnor} relating Seifert surface intersections with Massey products in the context of Milnor invariants for links, as well as the work of our third-named author and his collaborator on group cohomology through configuration spaces, where key calculations are made through intersections \cite{giusti2012symmetric, giusti2021alternating}.
But this project is more contemporary than one might assume.
Symplectic geometers have been revisiting these ideas as a parallel to work on Floer theory, with both Lipyanskiy \cite{Lipy14} and Joyce \cite{Joyc15} offering versions.
In a similar vein, Kreck's ``differential algebraic topology'' \cite{Krec10}, provides homology and cohomology on smooth manifolds using maps from \textit{stratifolds}, a certain kind of singular space.
Even the foundations of a theory of transversality for manifolds with corners suitable for our work has only been worked out in the last decade by Joyce \cite{Joy12}.

We take Lipyanskiy's approach and extend it to fully treat multiplicative structures, along the way filling in details needed.
While it is expected that ordinary homology can be captured in this way -- after all simplices and cubes are manifolds with corners -- the technicalities in this setting are surprising.
In particular, the boundary of a boundary of a manifold with corners is not empty or even zero as a chain.
Following Lipyanskiy, we simply quotient by the image of the boundary squared, but then our chains and cochains are themselves equivalence classes.
Once one is working with such equivalence classes and needs, for example, transversality to define products, truly substantial difficulties arise.
Indeed, at one point we doubted that a well-defined multiplication exists.

While the idea of homology as represented by fundamental classes of submanifolds or, more generally, manifold mappings is quite familiar, it was only in some corners of 20th century geometric topology -- including intersection theory -- that \textit{co}homology classes were also represented by appropriate maps from manifolds.
More specifically, we consider proper and co-oriented smooth maps from manifolds with corners, with associated degree of the class given by the codimension of the map.
Such cochains are geometric objects in their own right, which partially evaluate on chains through intersection.
A great benefit to such thinking is that the classical operations of algebraic topology, such as cup and cap products, can be described \textit{at the level of chains and cochains} by simple geometric operations based on intersection, without recourse to chain approximations to the diagonals, Alexander--Whitney maps, or other such combinatorics.
This again is reminiscent of the original thinking about such products in terms of intersections, and parallels modern work such as intersection theory in the PL category as summarized in \cite{McC06}.
The trade-off for such a pleasant description is that these intersections are not always defined; they require transversality.
This limitation is also classically anticipated by the famous commutative cochain problem.
Loosely speaking, no integral cochain construction computing ordinary cohomology can be made canonically into a (graded) commutative ring.
Since the process of forming intersections is commutative, the ring structure it induces in our theory cannot be fully defined.
We find the trade-off worthwhile, and in work building on these foundations \cite{FMS-flows} we have already ``married'' multiplicatively the theory we develop here, which is commutative and partially defined, and the one defined by cubical cochains with the Serre product, which is not commutative but everywhere defined.

Lipyanskiy's unpublished manuscript \cite{Lipy14}, on which we build, gives a fairly thorough account of geometric homology, but a much more lightly sketched account of geometric cohomology, which leaves several major theorems  unproven.
Some other expected results are not stated at all, including an isomorphism between geometric and ordinary cohomology, either as graded abelian groups or as rings.
So one of our main goals is to give a thorough account, with detailed proofs, of geometric homology and cohomology, with our primary focus on geometric cohomology, both because Lipyanskiy's account of this requires more filling in and also because cohomology with its algebra structure is of more interest to us.
In addition to research applications, we find these ideas helpful in teaching graduate students as, for example, cohomology of projective spaces follows from linear algebra, and pushforward or umkher maps are defined just by taking images.
At the research level, we found Lipyanskiy's work, thanks to Mike Miller, while working on \cite{FMS-flows} and looking for a rigorous foundation to geometrically model the cup product.
Ultimately our main aim is to obtain a full -- but partially defined -- $E_\infty$-algebra structure on geometric cochains, with the $E_\infty$-structure ``resolving'' partial definedness (though itself being partially defined!) rather than non-commutativity.
We also see plenty of room for development to make geometric cochains more broadly applicable, for example a conjectural theory for CW complexes with smooth attaching maps.
Our careful treatment here is offered to facilitate such work, and we invite others who would like to use geometric reasoning in algebraic topology to contact us if they are interested in such development.

\section*{Acknowledgments}

The authors thank Mike Miller, for pointing us to \cite{Lipy14}, and Dominic Joyce, for answering questions about his work.

\section*{Conventions}

Throughout we will denote the dimension of a manifold represented by an upper case character by the corresponding lower case character, for example, $\dim(M)=m$, $\dim(V)=v$, etc. % Introduction
	% !TEX root = ../foundations.tex

\section{Manifolds with corners}\label{S: manifolds with corners}

In this section, we provide an overview of manifolds with corners, which are the main geometric objects in the definitions of geometric chains and cochains.
Our reference for this material is Joyce \cite{Joy12}.
By \textbf{smooth} we always mean differentiable of all orders.
Throughout the paper, all manifolds and maps will be in the smooth category unless noted otherwise.

\begin{definition}
	If $A \subset \R^n$ and $B \subset \R^m$ are any subsets we say that $f \colon A \to B$ is \textbf{smooth} if it extends to a smooth map from an open neighborhood of $A$ to $\R^m$.
	We say that $f$ is a \textbf{diffeomorphism} if $f$ is a homeomorphism and both $f$ and $f^{-1}$ are smooth\footnote{Joyce requires $n = m$ in his definition for $f$ to be a diffeomorphism.
	But suppose $f$ is a diffeomorphism as defined here and, without loss of generality, $n>m$.
	Let $\R^m \subset \R^n$ in the usual way.
	Then as $f$ extends to a smooth map from a neighborhood of $A$ to $\R^m$, we can certainly consider $f$ as extending to a smooth map from a neighborhood of $A$ to $\R^n$.
	Similarly, as $f^{-1}$ extends to a smooth map from a neighborhood of $B$ in $\R^m$ to $\R^n$, we can extend $f^{-1}$ to a neighborhood $B$ in $\R^n$ by precomposing with the projection $\R^n \to \R^m$.
	So diffeomorphisms in our sense can be made into diffeomorphisms in Joyce's sense.}.
\end{definition}

With this definition, the notions of smooth charts and atlases for manifolds in the standard setting can be extended to define (smooth) manifolds with corners; see \cite[Section 2]{Joy12}: An $n$-dimensional chart $(U,\phi)$ of the space $W$ has domain $U$ an open subset of $\R^n_k = [0,\infty)^k \times \R^{n-k} \subset \R^n$ and $\phi$ is a homeomorphism from $U$ to $\phi(U) \subset W$.
Two $n$-dimensional charts $(U,\phi)$ and $(V,\psi)$ are compatible if $\psi^{-1}\phi \colon \phi^{-1}(\phi(U) \cap \psi(V)) \to \psi^{-1}(\phi(U) \cap \psi(V))$ is a diffeomorphism of subsets of $\R^n$.
An $n$-dimensional atlas for $W$ is a family of pairwise compatible $n$-dimensional charts that cover $W$, and an $n$-dimensional manifold with corners is a paracompact Hausdorff space with a maximal $n$-dimensional atlas.

While essentially equivalent, we choose to work entirely with subspaces of $\R^\infty$ in order to have a set of such objects, so rather than directly employ Joyce's definition, we define manifolds with corners for the purposes of this text as follows.

\begin{definition}\label{D: MWC}
	An $n$-dimensional \textbf{manifold with corners} $W$ is a subspace of some $\R^N \subset \R^\infty$ such that each point of $W$ possesses a neighborhood diffeomorphic to an open subset of $\R^n_k$ for some $k$.
\end{definition}

We note that these local diffeomorphisms provide charts $\phi \colon U \subset \R^n_k \to W$, and these charts are compatible, with $\psi^{-1}\phi \colon \phi^{-1}(\phi(U) \cap \psi(V)) \to \psi^{-1}(\phi(U) \cap \psi(V))$ being a diffeomorphism, even using Joyce's more restrictive notion of diffeomorphism.
These charts collectively give an atlas, and every atlas extends to a unique maximal one.
So our manifolds with corners are also manifolds with corners by Joyce's definition.

On the other hand, Joyce notes on page 231 of \cite{Joy12} that his manifolds with corners are what Melrose calls t-manifolds, and Melrose shows in \cite[Proposition 1.15.1]{Melrose} that t-manifolds can be embedded into manifolds without boundary.
These can then be embedded into Euclidean space by the Whitney Embedding Theorem.
So any one of Joyce's manifolds with corners is diffeomorphic to a manifold with corners in our sense.

By modeling on $\R^n_k$, our category includes manifolds ($k = 0$ in all charts) and manifolds with boundary ($k\leq 1$ in all charts) as well as cubes and simplices, but not the octahedron, for example, as the cone on $[0,1] \times [0,1]$ is not modeled by any $\R^n_k$.

\begin{comment}
	The smooth real-valued functions on a manifold with corners $W$ are those $f$ such that for each chart $\phi: U \subset \R^n_k \to W$ the composition $f \circ \phi:U \to \R$ is smooth.
\end{comment}

\begin{definition}
	A map $f \colon W \to M$ between manifolds with corners is {(weakly) \bf smooth} if whenever $(U,\phi)$, $(V,\psi)$ are charts for $W$ and $M$ respectively then
	$$\psi^{-1}f \phi \colon (f\phi)^{-1}\psi(V) \to V$$
	is smooth.

	The \textbf{tangent bundle} of a manifold with corners is the space of derivations of the ring of smooth real-valued functions.
	Analogously to smooth manifolds with boundary, if $(U,\phi)$ is a chart of $W$ then $d\phi$ takes the tangent space to $\R^n$ at $x \in U$ isomorphically to the tangent spaces of $W$ at $\phi(x)$.
	In particular, if $W$ is $n$-dimensional and $x \in W$, then the tangent space at $x$, denoted $T_xW$, is isomorphic to $\R^n$.
\end{definition}

In \cite{Joy12}, Joyce reserves the word \textit{smooth} for weakly smooth maps that also satisfy an additional condition concerning how they interact with the boundaries of their codomains.
When the codomain is a manifold without boundary, which will be our primary situation of interest, the weaker and stronger notions coincide.
In the few other cases in which we need to consider maps whose codomains have boundary, in particular boundary immersions and projections of pullbacks, the maps will also satisfy Joyce's stronger condition, though we will never need to utilize this explicitly.
Thus we will feel justified in simply using the word ``smooth'' throughout, referring the reader to \cite[Definition 3.1]{Joy12} for the full definition.

\begin{notation}
	Our default notation for manifolds with corners will be capital letters with the corresponding lower case letter denoting the dimension.
	In other words, $\dim(V) = v$, $\dim(W) = w$, $\dim(M) = m$, etc.
	We generally reserve $M$ for a manifold without boundary.
\end{notation}

\subsection{Boundaries}\label{S: boundaries}

We next need to describe boundaries of manifolds with corners.
Again see \cite[Section 2]{Joy12} for further details.

\begin{definition}
	A point $x$ in an $n$-dimensional manifold with corners $W$ has \textbf{depth} $k$ if there is a chart from an open subset of $\R^n_k$ which sends the origin to $x$.
	The set $S^k(W) \subseteq W$ of elements having depth~$k$ is called the \textbf{stratum of depth $k$}.
	By \cite[Proposition 2.4.]{Joy12}, $S^k(W)$ is an $n-k$ manifold without boundary.
\end{definition}

\begin{example}
	If $W$ is a smooth manifold with boundary in the classical sense, then $S^0(W)$ is its interior, $S^1(W)$ is its boundary, and $S^k(W) = \emptyset$ for $k>1$.
	If $S^k(W) = \emptyset$ for all $k>0$, then $W$ is a manifold without boundary.
\end{example}

When $W$ is a general manifold with corners, the boundary is more naturally a space equipped with a map to $W$, rather than a subspace of $W$.
The reason can be seen, for example, in the teardrop space of \cref{F: teardrop}, where the boundary should be considered to be homeomorphic to the closed interval mapping both endpoints to the vertex of the tear drop.
To explain in more detail, we have the following definitions.

\begin{figure}[h]
	\includegraphics[scale = 1.25, angle = -90]{figures/teardrop.pdf}
	\caption{The 2-dimensional ``teardrop'' is a manifold with corners whose boundary inclusion is not injective.}
	\label{F: teardrop}
\end{figure}

\begin{definition}
	A \textbf{local boundary component} $\beta$ of $W$ at $x \in W$ is a consistent choice of connected component $\mathbf{b}_V$ of $S^1(X) \cap V$ for any neighborhood $V$ of $x$, with consistent meaning that $\mathbf{b}_{V \cap V'} \subset \mathbf{b}_{V} \cap \mathbf{b}_{V'}$.
\end{definition}

Since this notion is local, the number of such components is determined by the depth.
Considering the origin in $\R^n_k$, for any $k \geq 0$, points having depth $k$ have exactly $k$ local boundary components.
As another example, letting $\interval = [0,1]$ and so $\interval^3$ the 3-cube, the set $S^1(\interval^3)$ consists of the interiors of two-dimensional faces, $S^3(\interval^3)$ is the set of eight corners, and any sufficiently small neighborhood of a corner intersects exactly three of the two-dimensional faces.

\begin{definition}\label{D: MWC boundary}
	Let $W$ be a manifold with corners.
	Define its \textbf{boundary} $\partial W$ to be the space of pairs $(x, \bb)$ with $x \in W$ and $\bb$ a local boundary component\footnote{Note that if $x \in S^0(W)$ then $x$ does not have any local boundary components and so does not appear in such a pair.} of $W$ at $x$.
	Define $i_{\partial W} \colon \partial W \to W$ by sending $(x,\bb)$ to $x$.
\end{definition}

In the teardrop example, $i_{\partial W} \colon \bd W \to W$ takes $\interval$ to the topological manifold boundary with both endpoints going to the unique point of $S^2(W)$.
For $\interval^3$, the boundary consists of six closed two-dimensional squares each mapping homeomorphically to a face of the cube.
In general, $|i_{\bd W}^{-1}(x)|$ is equal to the depth of $x$.

As in Douady \cite{Doua61}, $\partial W$ is itself a manifold with corners, and the boundary map $i_{\partial W}$ is a smooth immersion \cite[Theorem 3.4]{Joy12}. Note that $S^1(W)$ will always be diffeomorphic to the interior of $\bd W$, i.e.\ $S^0(\bd W) = S^1(W)$.
Inductively, we let $\partial^k W$ denote $\partial (\partial^{k-1} W)$ with $\partial^0 W = W$, and we let $i_{\partial^k W}$ denote the composite of the boundary maps sending $\partial^k W$ to $W$. The map $i_{\partial^k W}$ takes $S^a(\bd^k W)$ onto $S^{a+k}(W)$, but not, in general, injectively.

\begin{remark}\label{R: bd diff}
	If $f:V \to W$ is a diffeomorphism of manifolds with corners then it must, in particular, take components of $S^i(V)$ diffeomorphically onto corresponding components of $S^i(W)$, and, consequently, given a local boundary component $\mathbf{b}$ at a point $x \in V$, the map $f$ picks out a corresponding local boundary component, say $f_*(\mathbf{b})$ in $W$. We obtain a diffeomorphism $f_\bd:\bd V \to \bd W$ by $(x,\mathbf{b})\mapsto (f(x),f_*(\mathbf{b}))$ with $fi_{\bd V} = i_{\bd W} f_\bd$.
\end{remark}

While for manifolds with boundary $\partial^2W$ is empty, for general manifolds with corners it will not be.
However, given a co-oriented map $W \to M$, $\bd^2W$ does come equipped with a natural co-orientation reversing $\Z_2$ action.
This will be explained below and is a key component in showing that $\partial$ is suitable for defining boundary maps of geometric chain and cochain complexes.

N.B. When no confusion is likely to occur, we will sometimes abuse notation and use $\bd W$ also to refer to the image $i_{\bd W}(\bd W) \subset W$, which is the boundary of $W$ as a topological manifold in the usual sense.

The following is an example in which $\bd^2 W$ immerses into $W$ in an interesting way.

\begin{example}\label{boundary}
	Consider the quotient $Q$ of $S^n \times \R^2_2$ by the diagonal $C_2$ action, where $C_2$ (the group of order 2) acts antipodally on $S^n$ and acts by permuting the two coordinates
	of $\R^2_2$.
	The projection of $Q$ onto $S^n / C_2 = \R P^n$ defines a fiber bundle with fiber $\R^2_2$.
	Local coordinates can then be used to endow $Q$ with the structure of a manifolds with corners.

	The subspace $S^1(Q)$ is diffeomorphic to $S^n \times (0,\infty)$, and $S^2(Q)$ is diffeomorphic to $\R P^n$.
	The boundary $\partial Q$ is the quotient of $S^n \times \partial \R^2_2$ by $C_2$, which is diffeomorphic to
	$S^n \times [0,\infty)$.
	Thus $\partial^2 Q$ is $S^n$, which maps to $S^2(Q)$ by the standard quotient by antipodal action.
\end{example}

In general, Proposition~2.9 of \cite{Joy12} identifies $\partial^k W$ with the set of points $(x,\bb_1,\ldots,\bb_k)$ with $x \in W$ and the $\bb_i$ providing an ordered $k$-tuple of distinct local boundary components of $W$.

\begin{comment}
	% , whose $\partial^3$ is two copies of the symmetric group on three letters.
	% $\partial^1(T \times [0,1])$..., $\partial^2(T \times [0,1])$..., $\partial^3 (T \times [0,1])$...
\end{comment}

Special cases of manifolds with corners, including
manifolds with faces or manifolds with embedded corners \cite{Joy12}, steer clear of the interesting boundary phenomena of \cref{boundary}.
Although, a more restrictive notion should suffice for our application, Lipyanskiy develops geometric cohomology in the current generality, and we appreciate Joyce's careful treatment of transversality in this category, so we use their definitions.

We conclude this section by observing that the product $V \times W$ of two manifolds with corners is naturally a manifold with corners with, by \cite[Proposition 2.12]{Joy12},
$$\bd (V \times W) = (\bd V \times W)\sqcup (V \times \bd W).$$

\subsection{Transversality}

Transversality of smooth maps will play a key role, as this is the condition that assures that intersections or, more generally, pullbacks of manifolds (with corners) are also manifolds (with corners).
Recall that in the classical setting if $f \colon V \to M$ and $g \colon W \to M$ are smooth maps of manifolds without boundary then we say that $f$ and $g$ are \textbf{transverse} if whenever $f(x) = g(y) = z$ for some $x \in V$, $y \in W$, and $z \in M$ we have $Df(T_xV)+Dg(T_yW) = T_z M$.
We here discuss the extension of transversality to manifolds with corners, though only in the case where $M$ is without boundary.
More general versions of transversality can be found in \cite[Section 6]{Joy12}.

\begin{definition}{\cite[Special case of Definition 6.1]{Joy12}}
	Let $f \colon V \to M$ and $g \colon W \to M$ be smooth maps of manifolds with corners to a manifold without boundary.
	We say $f$ and $g$ are \textbf{transverse} if whenever $x \in S^j(X)$ and $y \in S^k(Y)$ with $f(x) = g(y) = z$ then $Df|_{S^j(V)}(T_xS^j(V))+Dg|_{S^k(W)}(T_yS^k(W)) = T_zM$.
\end{definition}

While this particular formulation of transversality is given in terms of the behavior of $f$ and $g$ on the strata $S^j(V)$ and $S^k(W)$, it is sometimes useful to have a reformulation in terms of the boundaries $\bd^j V$ and $\bd^kW$.
This is the content of \cref{L: simple trans} below.

To establish notation, let $f \colon V \to M$ and $g \colon W \to M$ be maps from manifolds with corners to a manifold without boundary.
We say that $f$ and $g$ are \textbf{plainly transverse} if they are transverse as maps of manifolds in the classical sense, without special consideration of strata or boundaries.
To be explicit in the case that $x \in V-S^0(V)$ or $y \in W-S^0(W)$ with $f(x) = g(y)$, let $(U_x,\phi_x)$ and $(U_y,\phi_y)$ be charts with $\phi_x(0) = x$ and $\phi_y(0) = y$.
By definition of smoothness, there exist neighborhoods $N_x$ and $N_y$ of $0$ in $\R^v$ and $\R^w$, respectively, so that $f\phi_x$ and $g\phi_y$ extend to smooth maps $\psi_x \colon N_x \to M$ and $\psi_y \colon N_y \to M$.
Then $f$ and $g$ are plainly transverse at $x$ and $y$ if $D\psi_x(T_0 \R^v)+D\psi_y(T_0\R^w) = T_{f(x)}M$.
This property is independent of the involved choices as $D\psi_x(T_0 \R^v)$ is the limit of $D\psi_x(T_{a} \R^v)$ for $a$ taken along any smooth path in $U_x$, and this does not depend on the choice of $N_x$ or $\psi_x$, and similarly for $\psi_y$.

\begin{lemma}\label{L: simple trans}
	Let $f \colon V \to M$ and $g \colon W \to M$ be maps from manifolds with corners to a manifold without boundary.
	Then $f$ and $g$ are transverse if and only if $fi_{\bd^j} \colon \bd^jV \to M$ and $gi_{\bd^k} \colon \bd^kW \to M$ are plainly transverse for all $j,k$.
\end{lemma}

Note that, a priori, the latter is a stronger condition as it imposes conditions not just on the interior of strata but on their closures.

\begin{proof}
	First suppose $fi_{\bd^j}$ and $gi_{\bd^k}$ are plainly transverse for all $j,k$.
	Suppose $x \in S^j(V)$ and $y \in S^k(W)$ for some fixed $j,k$ and $f(x) = g(y)$.
	The preimage of $x$ under $i_{\bd^j}$ consists of $j!$ points in $S^0(\bd^jV)$, and $i_{\bd^j}$ maps a neighborhood of each such preimage point diffeomorphically to a neighborhood of $x$ in $S^j(V)$, and similarly for $y$.
	Let $\psi_x$ be the inverse diffeomorphism from a neighborhood of $x$ in $S^j(V)$ to a neighborhood of one of the preimages.
	Then $f|_{S^j(V)} = fi_{\bd^j}\psi_x$ in a neighborhood of $x$, and similarly for $y$.
	Since $fi_{\bd^j}$ and $gi_{\bd^k}$ are plainly transverse, and $\psi_x$ and $\psi_y$ are diffeomorphisms, it follows that $f|_{S^j(V)}$ and $g|_{S^k(W)}$ are transverse at $f(x) = g(y)$.

	Conversely, suppose $f|_{S^j(V)}$ and $g|_{S^k(W)}$ are transverse for all $j,k$, and suppose $x \in \bd^jV$ and $y \in \bd^kW$ for some fixed $j,k$ with $fi_{\bd^j}(x) = gi_{\bd^k}(y) = z$.
	Furthermore, suppose $i_{\bd^j}(x) \in S^a(V)$ and $i_{\bd^k}(y) \in S^b(W)$, which implies $x \in S^{a-j}(\bd^j V)$.
	By focusing on local charts, there is a neighborhood of $x$ in $\bd^j(V)$ whose intersection with $S^{a-j}(\bd^j V)$ maps diffeomorphically via $i_{\bd^j}$ onto a neighborhood of $i_{\bd^j}(x)$ in $S^a(V)$, and analogously for $y$.
	Thus, $f|_{S^a(V)}i_{\bd^j}|_{S^{a-j}(\bd^j V)}$ and the analogous $g|_{S^b(W)}i_{\bd^k}|_{S^{b-k}(\bd^k W)}$ are transverse at $f(x) = g(y)$ as they precompose transverse maps with local diffeomorphisms.
	But the image of $D_x(fi_{\bd^j})$ contains the image of $D_x(f|_{S^j(V)}i_{\bd^j}|_{S^{a-j}(\bd^j V)})$ and similarly for $y$, and thus the images of $D_x(fi_{\bd^j})$ and $D_y(gi_{\bd^k})$ must also span $T_{z}M$.
	Therefore, $fi_{\bd^j}$ and $gi_{\bd^k}$ are plainly transverse at $f(x)$.
\end{proof}

\subsubsection{Achieving transversality}

Throughout this text, we will need a series of increasingly more general results guaranteeing that we can make certain maps transverse to each other.
We begin here with a relatively simple case that will be first used in \cref{S: basic properties} to show that geometric cohomology is contravariantly functorial with respect to continuous maps of manifolds without boundary.
All of our transversality theorems will be built using some basic tools that can be found in \cite[Section 2.3]{GuPo74}.
In particular, we record the following results, referring the reader to \cite[Section 2.3]{GuPo74} for the proofs\footnote{We rephrase the statements of these theorems slightly to better fit our context and notation.}:

\begin{theorem}[Transversality Theorem]
	Suppose $F \colon X \times S \to Y$ is a smooth map of manifolds, where only $X$ has boundary, and let $Z$ be any boundaryless submanifold of $Y$.
	If both $F$ and $F|_{\bd X \times S}$ are transverse to $X$, then for almost every $s \in S$, both $F(-,s) \colon X \to Y$ and $F(-,s)|_{\bd X} \colon \bd X \to Y$ are transverse to $Z$.
\end{theorem}

\begin{theorem}[$\epsilon$-Neighborhood Theorem]
	For a compact boundaryless manifold $Y$ in $\R^M$ and a positive number $\epsilon$, let $Y_\epsilon$ be the open set of points in $\R^M$ with distance less than $\epsilon$ from $Y$.
	If $\epsilon$ is sufficiently small, then each point $w \in Y_\epsilon$ possesses a unique closest point in $Y$, denoted $\pi(w)$.
	Moreover, the map $\pi \colon Y_\epsilon \to Y$ is a submersion.
	When $Y$ is not compact, there still exists a submersion $\pi \colon Y_\epsilon \to Y$ that is the identity on $Y$, but now $\epsilon$ must be allowed to be a smooth positive function on $Y$, and $Y_\epsilon$ is defined as $\{w \in \R^m \mid |w-y|<\epsilon(y) \text{\ for some\ } y \in Y\}$.
\end{theorem}

These theorems are used in \cite{GuPo74} to prove the following basic transversality result:

\begin{theorem}[Transversality Homotopy Theorem]
	For any smooth map $f \colon X \to Y$ and any boundaryless submanifold $Z$ of the boundaryless manifold $Y$, there exists a smooth map $g \colon X \to Y$ homotopic to $f$ such that $g$ is transverse to $Z$ and $g|_{\bd X}$ is transverse to $Z$.
\end{theorem}

Among other generalizations as we progress, we will extend these results to maps of manifolds with corners.
We will also often require that the homotopies take a special form, i.e.\ that they are \textit{universal} homotopies as defined in \cref{S: transverse cochains}.
See, for example, \cref{P: perturb transverse to map}.

The following technical lemma will help us extend the results of \cite{GuPo74} from transversality with respect to submanifolds to transversality with respect to maps.

\begin{lemma}\label{L: all transversality is wrt embeddings}
	Let $f \colon V \to M$ and $g \colon W \to M$ be smooth maps from manifolds with corners to a manifold without boundary.
	Let $e \colon W \to M \times \R^n$ be an embedding such that $\pi e = g$, where $\pi$ is the projection $M \times \R^n \to M$.
	Then $f$ and $g$ are transverse if and only if $e$ is transverse to $f \times \id_{\R^n} \colon V \times \R^n \to M \times \R^n$.
\end{lemma}

\begin{proof}
	It suffices to assume that $V$ and $W$ are without boundary.
	Otherwise we can apply the following argument to each pair of strata of $V$ and $W$.

	Suppose that $f$ and $g$ are transverse, i.e.\ that if $f(v) = g(w)$ then $Df(T_vV)+Dg(T_wW) = T_{f(v)}M$.
	For each $w \in W$, we can write $e(w) = (g(w),e_\R(w)) \in M \times \R^n$.
	Now suppose $w \in W$ and $(v,z) \in V \times \R^n$ such that $e(w) = (f \times \id_{\R^n})(v,z)$.
	Then we have $(g(w),e_\R(w)) = (f(v),z)$.
	The image of the derivative of $f \times \id_{\R^n}$ at such a point will span $Df(T_vV) \oplus T_z(\R^{n}) = Df(T_vV) \oplus \R^{n}$, while the derivative of $e$ will take $a \in T_w(W)$ to $Dg(a)+ De_{\R}(a)$.
	But the image of $D(f \times \id_{\R^n})$ already includes $0 \oplus \R^{n}$, so subtracting off the second summand, $D(f \times \id_{\R^{n}})(T_{(v,z)}(V \times \R^n))+De(T_wW)$ contains $Dg(a)$.
	It follows that $D(f \times \id_{\R^{n}})(T_{(v,z)}(V \times \R^n))+De(T_wW)$ contains $Df(T_vV) \oplus 0$, $Dg(T_wW) \oplus 0$, and $0 \oplus \R^n$.
	Since $f$ and $g$ are transverse and $D(f \times \id_{\R^{n}})(T_{(v,z)}(V \times \R^n))+De(T_wW)$ is a vector space, it therefore contains all of $T_{f(v)}M \oplus \R^n = T_{e(w)}(M \oplus \R^n)$.
	So $f \times \id_{\R^n}$ and $e$ are transverse.

	Next suppose $f \times \id_{\R^n}$ and $e$ are transverse and that $f(v) = g(w) \in M$.
	Suppose $e(w) = (g(w),z)$.
	Then $e(w) = (f \times \id_{\R^n})(v,z)$.
	So, by definition and assumption,
	\begin{equation}\label{E: Quillen transverse}
		D(f \times \id_{\R^{n}})(T_{(v,z)}(V \times \R^n))+De(T_wW) = T_{e(w)}(M \times \R^n) = T_{f(v)}M \oplus \R^n.
	\end{equation}
	As $\pi$ is a submersion, the image of this tangent space under $D\pi$ is all of $T_{f(v)}M$.
	But $(D\pi)(De) = D(\pi e) = Dg$, so $(D\pi \circ De)(T_wW) = Dg(T_wW)$.
	Furthermore, letting $\pi_V \colon V \times \R^n \to V$ be the projection, we have $(D\pi)(D(f \times \id_{\R^{n}})) = D(\pi(f \times \id_{\R^{n}})) = D(f\pi_V) = (Df)(D\pi_V)$, so, as $D\pi_V \colon T_{(v,z)}(V \times \R^n) \to T_vV$ is surjective, we have $(D\pi)(D(f \times \id_{\R^{n}}))(T_{(v,z)}(V \times \R^n)) = Df(T_vV)$.
	So applying $D\pi$ to equation \eqref{E: Quillen transverse}, we get $Df(T_vV)+Dg(T_wW) = T_{f(v)}M$, and $f$ is transverse to $g$.
\end{proof}

\begin{theorem}\label{T: basic trans}
	Let $f \colon M \to N$ be a continuous map from a manifold with (possibly empty) boundary to a manifold without boundary, and let $g \colon V \to N$ be a smooth map from a manifold with corners.
	Then there is a homotopy $h \colon M \times I \to N$ such that $h(-,0) = f$, and $h(-,1)$ is smooth and transverse to $g$.
	If $f|_{\bd M}$ is already smooth and transverse to $g$, then we can find such an $h$ with $h(-,1) = f$ on $\bd M$.
\end{theorem}

\begin{proof}
	Let $j \colon V \to \R^k$ be an embedding (we recall that such an embedding always exists by our definition of manifolds with corners).
	Then $e \colon V \to N \times \R^k$ given by $e(x) = (g(x),j(x))$ is an embedding that satisfies the hypotheses of \cref{L: all transversality is wrt embeddings}, so by that lemma it suffices to find a homotopy such that $h(-,0) = f$ and $h(-,1) \times \id_{\R^k} \colon M \times \R^k \to N \times \R^k$ is smooth and
	transverse to $e(V)$.

	We first find a homotopy from $f$ to a smooth map that agrees with $f$ on $\bd M$ if $f|_{\bd M}$ is already smooth.
	This can be done by the smooth approximation theorem; see \cite[Theorem III.2.5]{Kos93}.
	So we assume for the rest of the proof that this first homotopy has been completed and that $f$ is a smooth map.
	We will first construct our homotopy to achieve transversality without consideration of whether or not $f$ is already transverse to $g$ on $\bd M$, and then we show how to modify the construction for that case.

	We can think of $N$ as properly embedded in some $\R^K$ by the Whitney Embedding Theorem \cite[Section 1.8]{GuPo74}, and we let $N_\epsilon$ be an $\epsilon$ neighborhood of $N$ in $\R^K$ with submersion $\pi \colon N_\epsilon \to N$.
	Let $D$ be the open unit ball in $\R^K$.
	We define a composite map $H \colon M \times D \to \R^K \to N$ by
	$$H(x,s) = \pi(f(x)+ \epsilon(x)s).$$
	We have $H(x,0) = \pi(f(x)) = f(x)$, and, since $\epsilon(x)>0$ for all $x$, the map $(x,s) \to f(x)+ \epsilon(x)s$ is a submersion onto its image in $\R^K$.
	As $\pi$ is also a submersion, so is $H$, and then $H \times \id_{\R^k} \colon M \times D \times \R^k \to N \times \R^k$ is a submersion.
	In particular, $H \times \id_{\R^k}$ is transverse to $e(V)$ and to $e(S^i(V))$ for each stratum $S^i(V)$ of $W$.
	We can now apply the Transversality Theorem (though with the parameter space $D$ in a slightly nontraditional location in the ordering of factors) to obtain that $H(-,s) \times \id_{\R^k}$ is transverse to $e(V) \times \R^k$ for almost every $s \in D$.
	If $W$ is a stratum of $V$, then this statement also holds replacing $V$ with $W$.
	It also holds, for the same reasons, replacing $M$ with $\bd M$.
	Since $V$ has a finite number of strata and $M$ and $\bd M$ are only two space, and since the intersection of a finite number of dense subsets is dense, it holds for almost every $s \in D$, that both $H(-,s) \times \id_{\R^k}$ and its restriction to $\bd M \times \R^k$ are transverse to every stratum of $V$.
	Now let $s_0$ be one such $s$, and let $h \colon M \times I \to N$ be given by $h(x,t) = H(x,ts_0)$.
	Then $h(x,0) = H(x,0) = \pi(f(x)) = f(x)$.
	While $h(x,1) = H(-,s_0)$ has the desired properties by construction.

	If $f|_{\bd M}$ is already transverse to $g$, we modify the construction as follows: Let $\rho \colon M \to [0,1]$ be a smooth function that is $0$ on $\bd M$ and $>0$ on $M-\bd M$.
	Then we define
	$$H(x,s) = \pi(f(x)+ \epsilon(x)\rho(x)s).$$
	Then when $x \in \bd M$, we have $H(x,s) = \pi(f(x)) = f(x)$, so $h$ is constant along $\bd M$.
	For $x\notin \bd M$, the argument goes through exactly as above.
\end{proof}

\subsection{Pullbacks and fiber products}

When two embedded submanifolds of a manifold meet transversely, their intersection is again a submanifold.
More generally, if two smooth maps of manifolds are transverse, we can form their pullback, also called their fiber product, which is again a manifold.
This construction extends to manifolds with corners mapping into a manifold without boundary.

\begin{definition}\label{D: top pullback}
	Let $f \colon V \to M$ and $g \colon W \to M$ be transverse smooth maps from manifolds with corners to a manifold without boundary.
	Define the \textbf{pullback} or \textbf{fiber product} $V \times _M W$ by
	$$V \times _M W = \{(x, y) \in V \times W \mid f(x) = g(y)\}.$$

	There are canonical maps from $V \times _M W$ to $V$, $W$, and $M$ that respectively take $(x,y)$ to $x$, $y$, and $f(x) = g(y)$.

	\begin{diagram}
		V \times _MW&\rTo^{g^*}& V\\
		\dTo^{f^*}&&\dTo_f\\
		W&\rTo^g& M
	\end{diagram}

	We typically suppress the maps from the notation, though we sometimes label them as in the diagram and sometimes write $f \times _M g \colon V \times _M W \to M$.
	We also sometimes write $f^*$ as $\pi_W$ and $g^*$ as $\pi_V$, as these maps are induced by restricting to $V \times _MW$ the projections from $V \times W$ to $V$ and $W$.

	We will generally use the term \textit{pullback} when we want to emphasize $V \times _M W$ with its map to $V$ or $W$, while the \textit{fiber product} is to be considered as mapping to $M$.
	When treating $V \times _M W$ as a pullback, we also sometimes use the notation $g^*V \to W$ or $f^*W \to V$; this notation is consistent with the analogous notation for pullbacks of fiber bundles, which is a special case.
\end{definition}

The following analysis of fiber products is standard -- see for example Proposition~7.2.7 of \cite{MaDo92}.

\begin{theorem}\label{pullback}
	Let $f \colon V \to M$ and $g \colon W \to M$ be transverse smooth maps from manifolds with corners to a manifold without boundary.
	Then $V \times _M W$ is a manifold with corners with
	\begin{equation*}
		S^i(V \times _M W) = \bigsqcup_{k + \ell = i} S^k(V) \times _M S^\ell(W).
	\end{equation*}
	Moreover, the maps from the fiber product to $V$, $W$, and $M$ are weakly smooth.
\end{theorem}

To generalize this theorem when $M$ is also a manifold with corners requires substantial additional hypotheses in the definition of transverse smooth maps.
Such a generalization is a central result in \cite{Joy12}.

There is a Leibniz rule for taking boundaries of fiber products of transverse maps \cite[Proposition~6.7]{Joy12}:
\begin{equation}\label{E: product boundary}
	\bd(V \times _M W) = (\bd V \times _M W)\sqcup (V \times _M \bd W),
\end{equation}
recalling that if $g \colon W \to M$ then we interpret $\bd W$ as equipped with the map $gi_{\bd W} \colon \bd W \to M$ and similarly for $V$.
We will see versions of this formula below that take into account orientations and co-orientations.

\subsubsection{Some further properties of transversality and fiber products}

In this section we collect some well known, though not always easy to find, results about transversality and fiber products.
We state these results mainly in the classical setting of manifolds without boundary, though they generally extend to the case of transverse maps of manifolds with corners mapping to a manifold without boundary, either by applying them to the pairwise transverse strata or by thinking in terms of ``plain transversality'' as defined above.

\begin{lemma}\label{L: tangent of pullbacks}
	Let $f \colon V \to M$ and $g \colon W \to M$ be transverse smooth maps of manifolds without boundary.
	Suppose $x \in V$ and $y \in W$ with $f(x) = g(y) = z$ so that $(x,y) \in V \times _MW$.
	Then the tangent space of $V \times _MW$ at $(x,y)$ as a subspace of $T_{(x,y)}(V \times W) = T_xV \oplus T_yW$ consists of those vectors $(\mathbf v,\mathbf w)$ such that $Df(\mathbf v) = Dg(\mathbf w) \in T_zM$.
	In other words,
	$$T_{(x,y)}(V \times _MW) = T_xV \times _{T_zM}T_yW,$$
	or ``the tangent space of the fiber product is the fiber product of the tangent spaces.''
\end{lemma}

A proof can be found in \cite[Theorem 5.47]{Wed16}.
Wedhorn proves this theorem for ``premanifolds,'' which are essentially manifolds minus the Hausdorff and second countability conditions.
These will be automatic in our setting, so Wedhorn's proof applies.
In many circumstances, this lemma allows us to reduce arguments about fiber products of maps of manifolds to arguments about fiber products of linear maps.

\begin{lemma}\label{L: normal pullback}
	Let $f \colon V \to M$ and $g \colon W \to M$ be transverse smooth maps of manifolds without boundary, and suppose $f$ is an embedding so that $V$ has a normal bundle $\nu V$ in $M$.
	Then the pullback map $g^*V = V \times _MW \to W$ is an embedding and the normal bundle of $g^*V$ in $W$ is isomorphic to the pullback of $\nu V$.
	In other words,
	$$g^*(\nu V) \cong \nu(g^*V).$$
	So ``the pullback of the normal bundle is the normal bundle of the pullback.''
	% anibal: in the previous lemma "or" was used instead of "So". Consider homogeneous style, maybe.

	In case $g$ is also an embedding, this allows us to identify the restriction of $\nu V$ to $V \cap W$ with a sub-bundle of $TW$ over $V \cap W$.
\end{lemma}

\begin{proof}
	See \cite[Proposition IV.1.4]{Kos93}.
\end{proof}

\begin{lemma}
	Let $f \colon V \to M$ and $g \colon W \to M$ be transverse smooth maps of manifolds without boundary, and suppose $f$ and $g$ are both embeddings.
	Then, identifying $V$ and $W$ as submanifolds of $M$, the fiber product $V \times _MW$ is simply the intersection $V \cap W$, which is a smooth submanifold of $M$.
\end{lemma}

\begin{proof}
	By \cref{L: normal pullback}, the pullback is a smooth submanifold of $W$, but $g$ is itself an embedding, so, identifying the manifolds with their embedded images, we obtain smooth submanifolds $V \times _MW \subset W \subset M$.
	That $V \times _MW = V \cap W$ in this situation follows immediately from the definition of the fiber product.
\end{proof}

The next lemma is a bit more specific but will be useful later in proving certain formulas.

\begin{lemma}\label{L: transverse to pullback}
	Let $f \colon V \to M$, $g \colon W \to M$, and $h \colon X \to W$ be smooth maps of manifolds without boundary, and suppose $f$ is transverse to $g$.
	Then $gh \colon X \to M$ is transverse to $f$ if and only $h$ is transverse to the pullback $\pi_W \colon g^*V \to W$.
\end{lemma}

\begin{proof}
	For simplicity of notation, let $P = g^*V = V \times _MW$, and let $\pi_V \colon P \to V$ and $\pi_W \colon P \to W$ be the maps induced by the projections from $V \times W$ to $V$ and $W$.
	So we have the diagram

	\begin{diagram}
		&&P = V \times _MW&\rTo^{\pi_V}& V\\
		&&\dTo^{\pi_W}&&\dTo_f\\
		X&\rTo^h&W&\rTo^g&M.
	\end{diagram}

	First suppose $h$ is transverse to the pullback $g^*V \to W$.
	Note that the existence of $g^*V$ uses the assumption that $f$ and $g$ are transverse.
	Suppose $x \in X$ and $y \in V$ with $gh(x) = f(y)$.
	Then $(y,h(x)) \in P$, and by assumption we have both $Df(T_yV)+Dg(T_{h(x)}W) = T_{f(y)}M$ and
	$Dh(T_xX)+D\pi_W(T_{(y,h(x))}P) = T_{h(x)}W$.
	Applying $Dg$ to the second formula, we have
	$$Dg(T_{h(x)}W) = D(gh)(T_xX)+D(g\pi_W)(T_{(y,h(x))}P).$$
	So from the first formula,
	$$T_{f(y)}M = Df(T_yV)+D(gh)(T_xX)+D(g\pi_W)(T_{(y,h(x))}P).$$
	But $g\pi_W = f\pi_V$, so $D(g\pi_W)(T_{(y,h(x))}P) = D(f\pi_V)(T_{(y,h(x))}P) \subset Df(T_yV)$.
	It follows that
	$$T_{f(y)}M = Df(T_yV)+D(gh)(T_xX),$$
	i.e.\ $f$ is transverse to $gh$.

	Next suppose $f$ is transverse to $gh$ and that $x \in X, p \in P$ with $h(x) = \pi_W(p)$.
	As $gh(x) = g\pi_W(p) = f\pi_V(p)$, we have $gh(x)$ in the image of $f$, so in particular $gh(x)$ is in the intersection of the images of $V$ and $W$ in $M$.
	Now suppose $\mathbf w \in T_{h(x)}W$.
	By assumption $T_{hg(x)}M = Df(T_{\pi_V(p)}V)+D(gh)(T_xX)$, so we can write $Dg(\mathbf w) = Df(\mathbf a)+Dgh(\mathbf b)$ for some $\mathbf a \in T_{\pi_V(p)}V$ and $\mathbf b \in T_x X$.
	Now consider $\mathbf z = \mathbf w-Dh(\mathbf b) \in T_{h(x)}W$.
	Applying $Dg$ to both sides, we have $Dg(\mathbf z) = Dg(\mathbf w)-Dgh(\mathbf b) = Df(\mathbf a)$.
	So $(\mathbf a,\mathbf z)$ is in the pullback $T_{\pi_V(p)}V \times _{T_{gh(x)}M} T_{h(x)}W$.
	But by \cref{L: tangent of pullbacks}, this is precisely the tangent space of $V \times _MW$ at $(h(x),\pi_V(p))$.
	Furthermore, we have $D\pi_W(\mathbf a,\mathbf z) = \mathbf z$.
	Thus $\mathbf w = \mathbf z+Dh(\mathbf b) = D\pi_W(\mathbf a,\mathbf z)+Dh(\mathbf b)$.
	As $\mathbf w$ was arbitrary, $W = D\pi_W(T_{(h(x),\pi_V(p))}P)+Dh(T_xX)$, as desired.
\end{proof}

\subsubsection{Fiber products with more than three inputs}

Finally, we briefly consider transversality of more than two maps. Remark \ref{R: multiproducts} highlights some of the issues and difficulties involved, while Proposition \ref{P: 3 out of 4 trans} shows that there are still some simplifications that can be observed.

\begin{remark}\label{R: multiproducts}
	In this long remark we briefly discuss transversality and fiber products that involve more than two maps.
	This is relevant, for example, when considering associativity of fiber products or pullbacks of fiber products.
	As transversality of maps of manifolds is defined in terms of the behavior of the maps of tangent spaces, it is useful to first recall some notions about transversality in the setting of linear maps of vector spaces.

	If $f \colon V \to M$ and $g \colon W \to M$ are linear maps, then transversality of $f$ and $g$ can be expressed in a number of equivalent ways \cite[Section 4.7]{RamBas09}:
	\begin{itemize}
		\item $f(V)$ and $g(W)$ span $M$,
		\item the map $\Delta \colon V \times W \to M$ given by $\Delta(v,w) = f(v)-g(w)$ is surjective,
		\item $\dim(V \times _MW) = \dim(V)+\dim(W)-\dim(M)$.
	\end{itemize}
	The last two formulations easily generalize to $n$-tuples of maps $f_i \colon V_i \to M$.
	Such an $n$-tuple is considered transverse (as an $n$-tuple) when either of the following equivalent conditions hold:
	\begin{itemize}
		\item the map $\Delta \colon \prod V_i \to M^{n-1}$ given by $\Delta(v_1,\ldots,v_n) = (f_2(v_2)-f_1(v_1),\ldots, f_n(v_n)-f_{n-1}(v_{n-1}))$ is surjective,
		\item the fiber product given by $\{(v_1,\ldots,v_n) \in V_1 \times \cdots \times V_n\mid f_1(v_1) = \cdots = f_n(v_n)\}$ has dimension $\sum_{i = 1}^n\dim(V_i) -n\dim(M)$.
	\end{itemize}
	This version of transversality behaves very well in that this ``$n$-transversality'' is equivalent to the iterated transversality conditions that are required when taking fiber products two at a time.
	In other words, an $n$-tuple is transverse if and only if for any $1\leq i<j\leq n$,
	\begin{itemize}
		\item the $j-i+1$-tuple $\{f_i,\ldots,f_j\}$ is transverse, and
		\item letting $P$ denote the fiber product of the maps $\{f_i,\ldots,f_j\}$, the $n-(j-i)$-tuple consisting of the fiber product map $P \to M$ and the $f_k$ with $k\notin\{i,\ldots,j\}$ is transverse.
	\end{itemize}
	Iterating this fact, we can see that this is equivalent to having $V_1$ transverse to $V_2$, then $V_3$ transverse to $V_1 \times _MV_2$ and so on.
	In particular, we can form the $n$-tuple fiber product if and only if we can form the \textit{transverse} iterated fiber products such as $(((V_1 \times _MV_2) \times _M V_3) \times _M\ldots) \times _M V_n$ as well as in any other order of association.
	See\footnote{Proposition 8-1 of \cite{RamBas09} actually concerns a more general situation of ``mixed associativity'' in which the data consists of zig-zags
	$$V_1\xr{f_1}S_1\xl{g_2}V_1\xr{f_2}\cdots\xr{f_{n-1}}S_{n-1}\xl{g_n}V_2.$$
	However, this reduces to our setting by taking $S_i = M$ and $g_i = f_i$ for all $i$.}
	\cite[Propositions~4-9 and 8-1]{RamBas09}.

	Unfortunately, in the setting of maps of manifolds (for the moment without corners), the situation is less well behaved.
	Let now $f_i \colon V_i \to M$ be an $n$-tuple of smooth maps of manifolds.
	We say that this $n$-tuple is transverse when for any $n$-tuple $(v_1,\ldots, v_n) \in V_1 \times \cdots \times V_n$ such that $f_1(v_1) = \cdots = f_n(v_n)$ the linear maps $D_{v_i}f_i$ are transverse.
	In this case, the fiber product $P = \{(v_1,\ldots, v_n) \in V_1 \times \cdots \times V_n\mid f_1(v_1) = \cdots = f_n(v_n)\}$ is a smooth manifold of dimension $\sum\dim(V_i)-(n-1)\dim(M)$.
	Furthermore, at any such point, this general transversality has implications such as those above for transversality of subcollections.

	However, the above relations between $n$-ary transversality and ``iterated transversality'' cannot hold in general, because the $n$-ary fiber product cannot necessarily know about all points in the iterated fiber product.
	For example, consider maps $f \colon V \to M$, $g \colon W \to M$, and $h \colon Z \to M$.
	Suppose that $f$ and $g$ are not transverse; in particular, we can suppose that the fiber product $V \times _MW$ is not a manifold.
	Further suppose that $h(Z)$ is disjoint from $f(V)$ and $g(W)$.
	In this case, the triple of maps is transverse (vacuously) and its 3-ary fiber product is well defined as $\emptyset$.
	However, the iterated fiber product $(V \times _MW) \times _M Z$ is not well defined in the category of smooth manifolds and maps.
	Given that we define fiber products only of transverse maps of smooth manifolds, in this case $V \times _M W$ is not properly defined in our category, and it is further impossible then to define $(V \times _MW) \times _M Z$ in this context.

	The upshot of all this is that when considering situations involving fiber products of more than two maps, we shall have to be careful about the transversality assumptions.
\end{remark}

Despite the preceding remark, the transversality conditions involved in associativity of fiber products are not completely independent, as the following proposition shows. In later sections, we will usually simply assume all needed transversality exists, but the following proposition can be useful in practice.

\begin{proposition}\label{P: 3 out of 4 trans}
	Let $f \colon V \to M$, $g \colon W \to M$, and $h:Z \to M$ be maps from manifolds with corners to a manifold without boundary. Suppose that $W$ is transverse to $Z$ and that $V$ is transverse to $W$ and to $W \times _MZ$. Then $V \times _MW$ is transverse to $Z$. In particular, if $V \times _M(W \times _MZ)$ and $V \times _MW$ are well defined, then so is $(V \times _MW) \times _M Z$.
\end{proposition}

\begin{proof}
	We must show that $V \times _MW$ is transverse to $Z$, so we consider points $(v,w) \in V \times _M W$ and $z \in Z$ such that $h(z)$ is equal to $(f \times _Mg)(v,w)$, which by definition is equal to $f(v) = g(w)$. In other words, we consider $(v,w,z) \in V \times W \times Z$ such that $f(v) = g(w) = h(z)$.

	So suppose $(v,w,z)$ is such a triple, and denote the common image by $m \in M$. By the transversality assumptions, we know that the images of $D_wg:T_wW \to T_mM$ and $D_zh:T_zZ \to T_mM$ span $T_mM$, i.e.\ that $D_wg$ and $D_zh$ are transverse as linear maps, and similarly that $D_{(w,z)}(g \times _M h):T_{(w,z)}(W \times _M Z) \to T_mM$ is transverse to $D_vf:T_vV \to T_mM$. Furthermore, by Lemma \ref{L: tangent of pullbacks}, the tangent space of a fiber product is the fiber product of the tangent spaces, so $T_{(w,z)}(W \times _M Z) = T_wW \times _{T_mM}T_zZ$ and $D_{(w,z)}(g \times _M h) = D_wg \times _{T_mM}D_zh$.

	Now by \cite[Propositions~4-9]{RamBas09}, the triple of linear maps $(D_vf,D_wg,D_zh)$ is transverse as a triple of maps, if and only if both $D_wg$ is transverse to $D_zh$ and $D_wg \times _{T_mM}D_zh$ is transverse to $D_vf$. As such statements are independent of how we order the terms, the transversality established in the preceding paragraph also implies that $D_vf$ and $D_wg$ are transverse (which already follows from the hypotheses of the proposition), and $D_vf \times _{T_mM}D_wg$ is transverse to $D_zh$. But this implies, again using Lemma \ref{L: tangent of pullbacks}, that $h$ is transverse to $f \times _Mg$, as desired.
\end{proof}

\begin{remark}
	The end of the preceding proof at first seems to imply that if $g$ and $h$ are transverse and $g \times _M h$ is transverse to $f$, then $f$ is transverse to $g$ and $f \times _Mg$ is transverse to $h$. Indeed, \cite[Propositions~4-9]{RamBas09} says this is the case for the linear maps of the tangent spaces. Unfortunately, however, as \cite[Propositions~4-9]{RamBas09} applies only to linear maps, we can apply it only at those points $(v,w,z) \in V \times W \times Z$ where we know that $f(v) = g(w) = h(z)$ so that all three tangent space maps are well defined. So such a result would hold if the only intersections among the maps were such triple intersections. However, as noted in Remark \ref{R: multiproducts}, there could be pairs $(v,w) \in V \times W$ with $f(v) = g(w)$, but with this common image in $M$ not in the image of $h$. At such points, \cite[Propositions~4-9]{RamBas09} cannot tell us anything about the transversality of $V$ and $W$, and so $V \times _MW$ might not be well defined due to failure of transversality, even if $V \times _M (W \times _M Z)$ is.
\end{remark} %
	% !TEX root = ../foundations.tex

\section{Orientations and co-orientations}\label{S: orientations and co-orientations}

Manifolds with corners are, in particular, topological manifolds (with boundary), and so they carry the standard notions of orientability and orientation.
As in singular or simplicial homology, orientations carry sign information in geometric versions of homology theory.
For geometric cohomology, however, it turns out that the natural structures to carry sign information are co-orientations, sometimes called orientations of maps.
Unlike orientations, co-orientation can be ``pulled back.''

Co-orientations are less familiar than orientations, so it is helpful to keep the following central example in mind: If $W \to M$ is an immersion of manifolds, a co-orientation is equivalent to an orientation of the normal bundle of the image; see \cref{normal co-or} below.
Notice that this condition does not require the orientability of either $W$ or $M$.
In fact, an important case is when neither $W$ nor $M$ is orientable but the monodromies of their orientation bundles around loops in $W$ are either both orientation-preserving or both orientation-reversing; it is in this sense that we have a \textit{co}-orientation.

The special case of (local) immersions is important both for intuition and in practice; for example, the only geometric cochains which evaluate nontrivially on fixed collections of chains through the intersection homomorphisms of \cite{FMS-flows} are local immersions.
However, for complete definitions that cover all cases of interest, it is critical to co-orient more general maps and to do so in a way that provides key properties when forming pullbacks, such as a Leibniz rule when taking boundaries, graded commutativity of cochains, and a simple expression when the maps are immersions.
While co-orientations can be found in many places in the literature, we could not find a careful treatment that guaranteed these key properties.
Therefore, we develop co-orientations in depth in this (long) section.
Some readers may prefer to skip ahead, either considering geometric cohomology only with mod~2 coefficients in a first reading, or perhaps take orientation of the normal bundle as a temporary partial definition, coming back later to understand the general setting.

\subsection{Orientations}\label{S: orientations}

If $W$ is a manifold with corners then it is a topological manifold with boundary and so in the interior of $W$, i.e.\ on $S^0(W)$, we can consider orientability and orientations of $W$ in the usual sense, from either the topological or smooth manifold points of view, which are equivalent \cite[Theorem VI.7.15]{Bred97}.
Following standard conventions, we typically refer to an oriented manifold with corners $W$ with the orientation tacit.
If an orientation on $W$ is understood, then $-W$ refers to $W$ with the opposite orientation.

When $W$ is orientable, so is its (topological) boundary \cite[Lemma 6.9.1]{Bred97}, and since $S^0(W) \cup S^1(W)$ is a smooth manifold with boundary, we can allow an orientation of $W$ to determine an orientation of $S^1(W)$ using standard smooth manifold conventions.
This gives an orientation for $\bd W$, as we can identify $S^1(W)$ with the interior of $\bd W$.
In particular, we choose the following convention, which agrees with that of Joyce \cite[Convention 7.2.a]{Joy12}:

\begin{convention}\label{Con: oriented boundary}
	For a smooth oriented manifold with boundary $N$, we orient $\bd N$ by stipulating that an outward normal vector followed by an oriented basis of $\bd N$ yields an oriented basis for $N$.

	When $W$ is an oriented manifold with corners, we identify $S^1(W)$ with the interior of $\bd W$, and this convention determines an orientation of $\bd W$.
\end{convention}

When we wish to work with orientations symbolically, the following interpretation will be extremely useful.

\begin{definition}\label{D: det bundle}
	Let $E \to B$ be a rank $d$ real vector bundle.
	Define the \textbf{determinant line bundle}
	$\Or(E)$ to be $\bigwedge^d E$.
	If $d = 0$ this is interpreted to be the trivial rank one ``bundle of coefficients.''
	We call the principal $O(1) \cong C_2$ bundle associated to $\Or(E)$ the \textbf{orientation cover} of $E$.

	An \textbf{orientation} of $E$, if it exists, is then a section of its orientation cover or, equivalently, an equivalence class of non-zero sections of $\Or(E)$ such that two sections are equivalent if they differ by multiplication by an everywhere positive scalar function.
\end{definition}

In particular, we thus think of orientations of a manifold $M^m$ as (equivalence classes of) non-zero sections of $\Or(TM)$.
We typically use the notation $\beta_M$ to stand for such a section, and, as we are only ever interested in such sections to represent orientations, we systematically abuse notation by not distinguishing between a section and its equivalence class.
Thus an expression such as $\beta_M = \beta_V \wedge \beta_W$ should be interpreted as an equality of equivalence classes.
Such formulas are also often meant to be interpreted locally over some particular point or subspace that will be understood from context, and in this context we refer to expressions such as $\beta_V$ as \textbf{local orientations}.
This notation turn out to be extremely useful in working with orientations, and we will use it frequently.
Whenever we form such wedge products, if one of the terms is an element of $\bigwedge^0 TV$ for some $V$ we treat that term as a scalar function and interpret $\wedge$ as the fiberwise scalar product.

Of course, over a point $x \in M$, we identify the oriented basis $(e_1,\ldots, e_m)$ of $T_xM$ with the section $e_1 \wedge \cdots \wedge e_m$ of $\Or(TM)$.
We also use the standard identification $\bigwedge (A\oplus B)=(\bigwedge A )\otimes (\bigwedge B)$, letting $\bigwedge$ always denote the top-dimensional exterior product.
So, for example, if $V$ and $W$ are two submanifolds of a manifold $M$ intersecting transversely at a point $x$, we would have $T_xM = T_xV\oplus T_xW$, and the local orientation formula $\beta_M = \beta_V \wedge \beta_W$ would indicate that concatenating an oriented basis for $T_xV$ with an oriented basis for $T_xW$, in that order, gives an oriented basis for $T_xM$.
More generally, if $E$ is an oriented bundle over $M$, we can write $\beta_E$ for the corresponding local orientation at a point or in the neighborhood of a point.
Any abuses of notation involved in this calculus are very much justified by its extreme usefulness in working with orientations, as we shall see.

\begin{example}
	We have said that for a smooth oriented manifold with boundary $N$, we orient $\bd N$ so that an outward normal vector followed by an oriented basis of $\bd N$ yields an oriented basis for $N$.
	In our notation, we can let $\nu$ denote the normal bundle to the boundary and $\beta_{\nu}$ the section of the normal bundle tangent bundle corresponding to the outward-pointing orientation.
	Then letting $\beta_N$ and $\beta_{\bd N}$ denote the local orientations of $N$ and $\bd N$ at a boundary point, we can write our boundary orientation convention by saying that at each boundary point $\beta_N=\beta_\nu\wedge \beta_{\bd N}$.

	Of course, the normal bundle is technically a quotient bundle, but we can use the splitting of exact sequence of vector bundles \cite[Theorem 3.9.6]{Hus75} to identify it with a subbundle of the tangent bundle of $N$.
	Such choices are not unique, but they are unique up to elements of $T(\bd N)$, so the (equivalence class of the) exterior product $\beta_\nu\wedge \beta_{\bd N}$ does not depend on such choices.
	Such identifications will be used regularly and tacitly when working with normal bundles.
\end{example}

\subsubsection{Orientations of fiber products}\label{S: orientation of fiber products}

If $V$ and $W$ are oriented manifolds, we orient $V \times W$ in the standard way by concatenating oriented bases of tangent spaces of $V$ with those of $W$.
More generally, if $f \colon V \to M$ and $g \colon W \to M$ are transverse maps with $V$, $W$, and $M$ all oriented and $M$ without boundary, Joyce defines an orientation on the pullback $V \times_M W$ as follows \cite[Convention 7.2b]{Joy12}.
Consider the short exact sequence of vector bundles over $P = V \times_M W$ given by
\begin{equation}\label{E: fiber orientation}
	0 \to TP \xr{D\pi_V \oplus D\pi_W} \pi_V^*(TV) \oplus \pi_W^*(TW) \xr{\pi_V^*(Df)-\pi_W^*(Dg)} (f\pi_V)^*TM \to 0.
\end{equation}
Here $\pi_V$ and $\pi_W$ are the projections of $V \times_M W$ to $V$ and $W$, and $D\pi_V$ is being treated as a map $TP \to \pi_V^*(TV)$ and similarly for $D\pi_W$.
Analogously, $\pi_V^*(Df)$ is the pullback of the map $Df \colon TV \to TM$ obtained first by treating it as a map $TV \to f^*(TM)$ and then pulling back functorially by $\pi_V^*$, and similarly for $\pi_W^*(Dg)$; note that $f\pi_V = g\pi_W$ at points of $P$.
By choosing a splitting, this sequence determines an isomorphism
\begin{equation*}
	TP \oplus (f\pi_V)^*TM\cong\pi_V^*(TV) \oplus \pi_W^*(TW).
\end{equation*}
The choices of orientations on $V$, $W$, and $M$ determine orientations on all summands in this expression except $TP$.
The orientation on $TP$ is then chosen so that the two direct sums differ in orientation by a factor of $(-1)^{wm}$, recalling that $w=\dim(W)$, etc.

Much more about the orientations of fiber products can be found in the technical report of Ramshaw and Basch \cite{RamBas09}.
While the focus there is on manifolds without boundary, and sometimes just fiber products of linear maps of vector spaces, the results about orientations extend to manifolds with corners by employing them on the top-dimensional stratum and utilizing their stability property, by which orientation properties of fiber products of linear maps extend to properties of fiber products of transverse manifolds (see \cite[Sections 6.3, 9.1.2, and 9.3]{RamBas09}).
Their orientation of fiber products agrees with Joyce's.
This can be checked directly from the definitions\footnote{Their multiplicative ``fudge factor'' in \cite[Theorem 9.14]{RamBas09} at first appears to be different from Joyce's, but this is only because their conventions utilize what in our notation would be the map $\pi_W^*(Dg)-\pi_V^*(Df)$ rather than $\pi_V^*(Df)-\pi_W^*(Dg)$.} or, as Joyce notes in \cite[Remark 7.6.iii]{Joy12}, axiomatically, as Ramshaw and Basch show that theirs is the unique choice of orientation convention satisfying certain basic expected properties.
It is these properties that determine the sign in the definition.
We state these properties in the following two propositions.

\begin{proposition}\label{P: oriented fiber product basic properties}
	Let $f \colon V \to M$ and $g \colon W \to M$ be transverse maps from oriented manifolds with corners to an oriented manifold without boundary.
	\begin{enumerate}
		\item When $M$ is a point, the oriented fiber product $V \times_M W$ is simply $V \times W$, and in this case the fiber product orientation is consistent with the basic concatenation rule for products.
		\item When one of the maps is the identity $\id_M \colon M \to M$, the projection maps to the other factors are orientation preserving diffeomorphisms
		\begin{equation*}
			M \times_M V = V\quad\text{and}\quad V \times_M M = V.
		\end{equation*}
	\end{enumerate}
\end{proposition}

\begin{proposition}\label{P: oriented fiber mixed associativity}
	Let $V$, $W$, and $Z$ be oriented manifolds with corners, and let $M$ and $N$ be oriented manifolds without boundary.
	Then the ``mixed associativity'' formula for oriented fiber products
	\begin{equation}\label{E: mixed associativity fiber orientation}
		(V \times_M W) \times_N Z = V \times_M (W \times_N Z)
	\end{equation}
	holds when given maps
	$$V \xr{f} M\xleftarrow{g} W \xr{h} N \xl{k} Z$$
	and assuming sufficient transversality for all the fiber products in \eqref{E: mixed associativity fiber orientation} to be well defined (see \cref{R: multiproducts}).
	In this case, the map $V \times_M W \to N$ is given by composing the projection from $V \times_M W$ to $W$ with $h$, and similarly for the map $W \times_N Z \to M$.
\end{proposition}

These propositions are evident at the level of spaces.
When taking orientations into account, the first property in \cref{P: oriented fiber product basic properties} is proven in \cite[Sections 9.3.9]{RamBas09} as the ``concatenation axiom,'' and the second is proven in \cite[Sections 9.3.5 and 9.3.6]{RamBas09} as the ``left and right identity axioms.''
The mixed associativity property is proven in \cite[Sections 9.3.7]{RamBas09}.
An important special case of this associativity that we will need below occurs when $M = N$ and $g = h$, so that our initial data is three maps all to $M$.
In this case we have the ordinary associativity
\begin{equation}\label{E: oriented fiber associativity}
	(V \times_M W) \times_M Z = V \times_M (W \times_M Z).
\end{equation}
That these properties determine the orientation rule for fiber products is the content of \cite[Theorem 9-10]{RamBas09}.
Technically, Ramshaw and Basch require for uniqueness two other properties: an Isomorphism Axiom, which says that the construction is consistent across oriented homeomorphisms, and a Stability Axiom, which implies that the orientation can be determined pointwise in a globally consistent manner.
These properties are both implicit in Joyce's global definition of the fiber product orientation.

There is also a commutativity rule proven in \cite[Sections 9.3.8]{RamBas09} that follows from the other properties:

\begin{proposition}\label{P: commute oriented fiber}
	Let $f \colon V \to M$ and $g \colon W \to M$ be transverse maps from oriented manifolds with corners to an oriented manifold without boundary.
	Then, as oriented manifolds,
	\begin{equation*}
		V \times_M W = (-1)^{(m-v)(m-w)}W \times_M V.
	\end{equation*}
\end{proposition}

Recalling that we write $\dim(M)=m$, etc., this means that the canonical diffeomorphism taking $(x,y) \in V \times_M W \subset V \times W$ to $(y,x) \in W \times_M V \subset W \times V$ takes a positively-oriented basis of the tangent space of $V \times_M W$ to a $(-1)^{(m-v)(m-w)}$-oriented basis of the tangent space of $W \times_M V$.
We note that these signs, while note quite in line with the Koszul conventions, agree with those for the intersection product of homology classes in Dold \cite[Section VIII.13]{Dol72}.

Furthermore, with our convention for oriented boundaries, one obtains the following useful identity; see \cite[Propositions 7.4 and 7.5]{Joy12}

\begin{proposition}\label{P: oriented fiber boundary}
	Let $f \colon V \to M$ and $g \colon W \to M$ be transverse maps from oriented manifolds with corners to an oriented manifold without boundary.
	Then, as oriented manifolds,
	\begin{equation*}
		\bd (V \times_M W) = (\bd V \times_M W) \sqcup (-1)^{m-v}(V \times_M \bd W).
	\end{equation*}
\end{proposition}

\subsubsection{Fiber products of immersions}

The special case of fiber products with $f \colon V \to M$ and $g \colon W \to M$ embeddings or, a bit more generally, immersions, is of particular interest, especially for developing intuition.
Once again, since we are concerned primarily with orientations in this section, it is sufficient to consider $V$ and $W$ to be manifolds without boundary, and then the results we obtain extend directly to manifolds with corners.
In the case of embeddings, $V \times_M W$ is simply the intersection of $V$ and $W$ as submanifolds of $M$ by \cref{L: fiber product of embeddings}, and in the immersed case this is true locally, i.e.\ restricting attention to submanifolds of $V$ and $W$ on which $f$ and $g$ are embeddings.
Let us try to understand the orientation of $V \times_M W$ determined by orientations of $V$, $W$, and $M$ in this setting.

For convenience of notation, let us assume we have transverse embeddings so that $V$ and $W$ are submanifolds of $M$.
Then, by \cref{L: fiber product of embeddings}, we know that $P = V \times_M W$ is just the intersection $V \cap W$.
Consider a point $x \in P$.
As $V$, $W$, and $P$, are all submanifolds, $T_xV$, $T_xW$, and $T_xP$ are all subspaces of $T_xM$. Furthermore, as orientations are defined via the tangent bundles, it suffices to consider the relation among the orientations of $T_x V$, $T_xW$, $T_xM$, and $T_xP$, and, as the tangent space of a fiber product is the fiber product of the tangent spaces by \cref{L: tangent of pullbacks}, we have $T_xP = T_xV \times_{T_xM} T_xW = T_xV \cap T_xW$.
For simplicity of notation in the following, we drop the ``$T_x$'' from the notation and treat the maps $f$ and $g$ as linear embeddings of vector spaces.
We can then work with orientations of vector spaces expressed in our exterior power notation.

So we have a vector space $M$ with subspaces $V$, $W$, and $P = V \times_M W = V \cap W$.
By the transversality assumption, $V$ and $W$ span $M$.
Let $\nu W \subset V$ be a complementary subspace to $P$ in $V$ so that $V = P \oplus \nu W$; the notation is meant to suggest that $\nu W$ is a choice of normal subspace to $W$ in $M$.
Similarly, let $\nu V \subset W$ be a complementary subspace to $P$ in $W$ so that $W = P \oplus \nu V$.
Then we have $M = \nu W \oplus P \oplus \nu V$. \greg{I think this would be a good place for a picture. Maybe have $V$ and $W$ be 2-dimensional subspaces of $\R^3$? In fact, see \cref{Ex: intersection orientation}.}

In our current context, the exact sequence \eqref{E: fiber orientation} becomes a sequence of vector spaces
\begin{equation*}
	0 \to P \to V \oplus W \xr{f-g} M \to 0
\end{equation*}
with the first non-trivial map being the direct sum of inclusions and with $f$ and $g$ the emeddings of $V$ and $W$ into $M$.
Using our direct sum decompositions of the preceding paragraph, we can choose a splitting $M \to V \oplus W$ as
\[
\nu W \oplus P \oplus \nu V \to P \oplus \nu W \oplus P \oplus \nu V,
\]
given by $(x,p,y) \mapsto (0, x, -p, -y)$.
The signs are necessary due to the sign in $f-g$.
With this splitting, our resulting isomorphism $P \oplus M \to V \oplus W$ can be written in block matrix form as
\begin{equation}\label{E: orientation matrix}
	\begin{pmatrix*}[r]
		I&0&0&0\\
		0&I&0&0\\
		I&0&-I&0\\
		0&0&0&-I
	\end{pmatrix*},
\end{equation}
which has determinant $(-1)^{w}$.

By definition, the orientation of $P$ determined by the orientations of $V$, $W$, and $M$ is the orientation so that this matrix takes the concatenation of the orientation of $P$ with an orientation of $M$ (i.e.\ an ordered basis representing this concatenation orientation) to $(-1)^{wm}$ times the concatenation of the orientations of $V$ and $W$ (i.e.\ an ordered basis of $V \oplus W$ that differs from the concatenation orientation by a permutation of sign $(-1)^{wm}$).

This call can be expressed via our calculus of local orientations as follows.

\begin{proposition}\label{P: orient intersection}
	Let $V$ and $W$ be transverse oriented submanifolds of the oriented manifold $M$.
	Let $\beta_V$, $\beta_W$, and $\beta_M$ be local orientations of $V$, $W$, and $M$ at a point of $P = V \times_M W = V \cap W$.
	Then, using our notation established just above, the fiber product orientation of $P$ is the unique orientation $\beta_P$ such that if we choose orientations $\beta_{\nu W}$ and $\beta_{\nu V}$ for $\nu W$ and $\nu V$ such that $\beta_P \wedge \beta_{\nu W} = \beta_V$ and $\beta_P \wedge \beta_{\nu V} = \beta_W$ then $\beta_{\nu W} \wedge \beta_P \wedge \beta_{\nu V} = (-1)^{w(m+1)}\beta_M$ or, alternatively, $$\beta_P \wedge \beta_{\nu V} \wedge \beta_{\nu W} = \beta_M.$$
\end{proposition}

\begin{proof}
	We may work with $V$, $W$, and $P$ as linear subspaces of a vector space $M$ as above. Note that if we replace $\beta_P$ with its opposite orientation, then this must also reverse the orientations $\beta_{\nu W}$ and $\beta_{\nu V}$ and hence altogether we get the opposite orientation for $\beta_P \wedge \beta_{\nu V} \wedge \beta_{\nu W}$.
	Thus there is a unique choice of orientation $\beta_P$ as described in the lemma, and we must show that this is the orientation as defined in \cref{S: orientation of fiber products}.

	It will be more convenient to prove the lemma in the first form, but the second form follows by observing that $$(-1)^{w(m+1)} = (-1)^{wm+w} = (-1)^{wm-w^2} = (-1)^{w(m-w)},$$
	and then
	\[
	\beta_{\nu W} \wedge \beta_P \wedge \beta_{\nu V} =
	(-1)^{w(m-w)} \beta_P \wedge \beta_{\nu V} \wedge \beta_{\nu W}
	\]
	as $\dim(P \oplus \nu V) = \dim(W) = w$ and $\dim(\nu W) = m-w$.

	To prove the first statement, let $(p_1,\cdots,p_a)$ be an ordered basis for $P$ consistent with the orientation described in the lemma; so we can write $\beta_P = p_1 \wedge\cdots\wedge p_a$.
	When we consider each $p_i$ as a vector in $V$, $W$, or $M$, we write $p_i^V$, $p_i^W$, or $p_i^M$.
	We employ a similar convention with the other bases we will consider.
	Let $(x_1,\cdots,x_b)$ and $(y_1,\cdots,y_c)$ be corresponding ordered bases for $\nu W$ and $\nu V$ as described in the lemma, and we can write $\beta_{\nu W}$ and $\beta_{\nu V}$ analogously. Recall that, by assumption, we have $\beta_P \wedge \beta_{\nu W} = \beta_V$, $\beta_P \wedge \beta_{\nu V} = \beta_W$, and $\beta_M = (-1)^{w(m+1)} \beta_{\nu W} \wedge \beta_P \wedge \beta_{\nu V}$, and we must show that this is consistent with the definition of the fiber product orientation for $P$.

	So our orientation of $P \oplus M$ obtained by concatenation is
	$$(-1)^{w(m+1)} p_1 \wedge\cdots\wedge p_a \wedge x^M_1 \wedge\cdots\wedge x^M_b \wedge p^M_1 \wedge\cdots\wedge p^M_a \wedge y^M_1 \wedge\cdots\wedge y^M_c.$$
	When we apply the matrix \eqref{E: orientation matrix}, we obtain the form in $V \oplus W$ given by
	$$(-1)^{w(m+1)} (p^V_1+p^W_1) \wedge\cdots\wedge (p^V_a+p^W_a) \wedge x^V_1 \wedge\cdots\wedge x^V_b \wedge (- p^W_1) \wedge\cdots\wedge (-p^W_a) \wedge (-y^W_1) \wedge\cdots\wedge(- y^W_c).$$
	As the number of terms with a negative sign is $w$, this expression simplifies to
	$$(-1)^{wm} p^V_1 \wedge\cdots\wedge p^V_a \wedge x^V_1 \wedge\cdots\wedge x^V_b \wedge p^W_1 \wedge\cdots\wedge p^W_a \wedge y^W_1 \wedge\cdots\wedge y^W_c.$$
	But this is now precisely $(-1)^{wm}$ times the concatenation orientation of $V \oplus W$ as desired for the definition of the fiber product orientation of $P$.
\end{proof}

\begin{example}\label{Ex: intersection orientation}
	As an example, let $M = \R^3$ oriented by the standard ordered basis $(e_x,e_y,e_z)$.
	Let $V$ be the $z = 0$ plane oriented by the ordered basis vectors $(e_x,e_y)$, and let $W$ be the $x = 0$ plane oriented by the ordered basis vectors $(e_y,e_z)$.
	The intersection $P$ is the $y$ axis.
	We claim that $P$ has the fiber product orientation by $-e_y$.
	Indeed, assuming so we have $\beta_V = e_x \wedge e_y = -e_y \wedge e_x = (-e_y) \wedge e_x$, so $\beta_{\nu W} = e_x$, and $\beta_W = e_y \wedge e_z = (-e_y) \wedge (-e_z)$, so $\beta_{\nu V} = -e_z$.
	And then $$\beta_P \wedge \beta_{\nu V} \wedge \beta_{\nu W} = -e_y \wedge (-e_z) \wedge e_x = e_x \wedge e_y \wedge e_z = \beta_M,$$
	as required.
\end{example}

\begin{corollary}\label{C: orient complementary intersection}
	Suppose $V$ and $W$ have complementary dimensions so that they intersect in a point.
	Then the fiber product orientation of the point is positive if and only if $\beta_{W} \wedge \beta_{V} = \beta_M$.
\end{corollary}

\begin{proof}
	In this case, $\nu W = V$, $\nu V = W$, and $\beta_P = \pm 1 \in \R$.
	If $\beta_P = 1$, then the formula from \cref{P: orient intersection} becomes exactly the formula of the corollary.
\end{proof}

\begin{remark}
The corollary shows that the fiber product orientation is \textit{not} necessarily the expected concatenation orientation in the case of transverse complementary embeddings.
\end{remark}

\begin{example}
	Let $M = \R^2$ with the standard orientation that we can write $e_x \wedge e_y$.
	Let $V$ be the $x$-axis with orientation $e_x$ and $W$ be the $y$-axis oriented by $e_y$.
	Then  $\beta_W \wedge \beta_V = e_y \wedge e_x$, while $\beta_M = e_x \wedge e_y = -e_y \wedge e_x$.
	So the fiber product orientation of the intersection point is the negative one.
	This runs against the standard convention for transverse intersections of manifolds of complementary dimension, but we nonetheless favor this overall convention for orienting fiber products due to the properties and uniqueness result of \cite{RamBas09}.
\end{example}

\subsection{Co-orientations}\label{S: co-orientations}

To define co-orientations, we recall our definition of an orientation of a bundle from \cref{D: det bundle} as an equivalence class, up to positive scalar multiplication, of an everywhere non-zero section of the top exterior power of the bundle.
This motivates the following.

\begin{definition}\label{D: co-orientations}
	A \textbf{co-orientation} $\omega_g$ of a \textit{continuous}\footnote{We will most often be interested in the case of $g$ smooth, but continuous co-orientable maps do come up, for example in \cref{S: basic properties} where we consider covariant functoriality of geometric cohomology with respect to continuous maps.} map $g \colon W \to M$ of manifolds with corners is an equivalence class, up to positive scalar multiplication, of a nowhere zero section of the line bundle $\Hom(\Or(TW), \Or(g^*TM)) \cong \Hom(\Or(TW), g^*\Or(TM))$.
	Equivalently, a co-orientation is a choice of isomorphism between the associated orientation cover $\Or(TW)$ and the pullback of the associated orientation cover $\Or(TM)$.

	In particular, a co-orientation exists if and only if $\Hom(\Or(TW), \Or(g^*TM))$ is a trivial line bundle, in which case we say that $g$ is \textbf{co-orientable}.
\end{definition}

Thus, if $W$ is connected and $g \colon W \to M$ is co-orientable, there are exactly two co-orientations, which are \textbf{opposite} to one another; we write the opposite of $\omega_g$ as $-\omega_g$.
In particular, for connected $W$ a choice of co-orientation at a single point determines a co-orientation globally when $g$ is co-orientable (analogously to orientations).
Also, just as most manifolds do not possess a preferred orientation, most maps $g \colon W \to M$ do not carry a natural choice of co-orientation, with the following notable exception.

\begin{definition}\label{D: tautological co-orientation}
	Suppose $g$ is a diffeomorphism, or more generally a codimension-0 immersion. In this case, the top exterior power of $Dg$,
	$$\textstyle{\bigwedge^w} Dg \colon \textstyle{\bigwedge^w} TW \to \textstyle{\bigwedge^w} g^*(TM),$$
	provides a \textbf{tautological co-orientation}.
\end{definition}

The local triviality of the determinant line bundle of a manifold means being able to choose a consistent basis vector over sufficiently small neighborhoods.
Again, we call such a choice of basis vector around a point in $W$ a \textbf{local orientation}, and, as for global orientations, often denote a local orientation by $\beta_W$.
Again, abusing notation, we also often allow $\beta_W$ to refer to its equivalence class up to multiplication by a positive scalar.
We identify $\beta_W$ with a local choice of (equivalence class of) non-zero section of\footnote{As usual, if $\dim(W) = 0$ we identify $\bigwedge^0 TW$ with the trivial $\R$ bundle, and, when forming exterior products, multiplication by a section is treated as scalar multiplication.} $\bigwedge^w TW$ in a neighborhood of a point $x$ in $W$ or, equivalently, a local smoothly varying ordered basis for the fibers of $TW$.
We typically do not specify the point $x$, though when necessary we write $\beta_{W,x}$.
We then use ordered-pair notation for co-orientation homomorphisms, with $\omega_g = (\beta_W, \beta_M)$ being the \textbf{local co-orientation} that sends the local orientation $\beta_W$ at $x \in W$ to the local orientation $\beta_M$ for $g^*(TM)$.
We will often further abuse notation by neglecting the pullback and treating $\beta_M$ as a local orientation at $g(x)$ in $M$.
We write the opposite co-orientation $(\beta_W,-\beta_M) = (-\beta_W,\beta_M)$ as $-(\beta_W,\beta_M)$.
As a co-orientation at a point completely determines the co-orientation of a co-orientable map for connected $W$, it is useful to cheat further and write $\omega_g = (\beta_W,\beta_M)$ for appropriate $\beta_W$ and $\beta_M$ with the chosen points $x$ implicit.

A manifold is orientable if and only if the orientation cover is trivial.
So if $M$ is orientable, $\Or(g^*(TM))$ is trivial, and co-orientability of $g \colon W \to M$ implies that $W$ is orientable.
Moreover, an orientation on $M$ along with a co-orientation of $g$ gives rise to an \textbf{induced orientation} of $W$.
Explicitly, if $\beta_M$ denotes the global orientation of $M$, then we orient $W$ at each point by the $\beta_W$ such that $\omega_g = (\beta_W,\beta_M)$.
Conversely, if $M$ and $W$ are both oriented, say by $\beta_M$ and $\beta_W$ respectively, we have the \textbf{induced co-orientation} given by $\omega_g = (\beta_W,\beta_M)$ at each point of $W$.
On the other hand, it is not true that if we have an orientation of $W$ and a co-orientation of $g \colon W \to M$ then we obtain an orientation of $M$.
For example, if $W$ is orientable, any constant map to $M$ is co-orientable, regardless of whether or not $M$ is orientable.

More generally, recall that the fundamental group of a manifold acts on classes of local orientations as the deck transformations of the orientation cover.
A map is co-orientable if it holds that a loop in $W$ acts nontrivially on a local orientation of $W$ if and only if its image in $M$ acts nontrivially on a local orientation of $M$.
Explicitly, if $g \colon W \to M$ is co-orientable, following a loop in $W$ may change the local orientation pair
$(\beta_W, \beta_M)$ to $(-\beta_W, -\beta_M)$, but these pairs define equivalent co-orientations.

Similarly, to compare local constructions at different points, it is useful to use paths.
Suppose $\gamma \colon I \to W$ is a path with $\gamma(0) = x$ and $\gamma(1) = y$.
We can choose a lift $\td \gamma$ of $\gamma$ to the complement of the $0$-section of $\Or(TW)$ such that $\td \gamma(0)$ is in the equivalence class of $\beta_{W,x}$.
We then define $\gamma_*\beta_{W}$ to be the equivalence class of $\td \gamma(1)$.
Likewise, we define $(g\gamma)_*\beta_M$ by a lift of $g\gamma$ to the complement of the zero section of $\Or(TM)$.
Then $\gamma_*\beta_{W}$ and $(g\gamma)_*\beta_M$ depend on $\gamma$, but if $g \colon W \to M$ is co-orientable the pair $(\gamma_*\beta_{W}, (g\gamma)_*\beta_M)$ is independent of $\gamma$ as this data also determines a non-vanishing lift of $\gamma$ in $\Hom(\Or(TW),g^*\Or(TM))$, which is trivial if $g$ is co-orientable.
In particular, if $g$ is co-oriented and $(\beta_{W}, \beta_M)$ represents the choice of co-orientation locally at $x$, then $\gamma_*(\beta_W,\beta_M) \defeq (\gamma_*\beta_{W}, (g\gamma)_*\beta_M)$ will represent the same co-orientation locally at $y$.
We refer to this as \textbf{transporting} the co-orientation from $x$ to $y$.

\begin{example}
	Let $g$ be any map $g \colon S^1 \to S^2$.
	As $S^1$ and $S^2$ are orientable, $g$ is co-orientable.
	If we choose a local orientation vector $e_{\theta}$ at any point $x \in S^1$ and latitude/longitude coordinates $\phi,\psi$ at $g(x)$ so that $e_\phi \wedge e_\psi$ is a local orientation in a neighborhood of $g(x)$, then the two possible co-orientations for $g$ can be written $(e_\theta, e_\phi \wedge e_\psi)$ and $-(e_\theta, e_\phi \wedge e_\psi) = (-e_\theta, e_\phi \wedge e_\psi) = (e_\theta,- e_\phi \wedge e_\psi)$.
	While the notation explicitly references a local orientation at a point, this is sufficient to determine the co-orientation globally.
	In what follows we will often demonstrate properties of co-orientations by showing that they hold locally at an arbitrary point but do not depend on the choice of point.

	As another example, consider the standard embedding $g \colon \R P^2 \into \R P^4$.
	Choosing local orientations $e_1 \wedge  e_2$ at some $x \in \R P^2$ and $f_1 \wedge f_2 \wedge f_3 \wedge f_4$ at $g(x)$, the two co-orientations are $(e_1 \wedge e_2, f_1 \wedge f_2 \wedge f_3 \wedge f_4)$ and its opposite.
	If $\gamma$ is a loop that reverses the orientation of $\R P^2$ then it also reverses the orientation of $\R P^4$, so $\gamma_*(e_1 \wedge e_2, f_1 \wedge f_2 \wedge f_3 \wedge f_4) = (-e_1 \wedge e_2,- f_1 \wedge f_2 \wedge f_3 \wedge f_4) = (e_1 \wedge e_2, f_1 \wedge f_2 \wedge f_3 \wedge f_4)$, reflecting that $g$ is co-orientable.

	By contrast, no embedding of the M\"obius strip in $\R^3$ is co-orientable.
\end{example}

\begin{remark}\label{R: cooriented composition}
	Co-oriented maps compose in an immediate way, forming a category.
	Namely, given $V \xr{f} W \xr{g} M$ and co-orientations $\Or(TV) \to \Or(f^*TW)$ and $\Or(TW) \to \Or(g^*TM)$, we simply compose the former with the pullback of the latter via $f^*$, recalling that $f^*(\Or(E)) = \Or(f^*E)$ in a natural way.
	We will refer to this simply as composing co-orientations and write the composition in symbols as $\omega_f*\omega_g$.
	Warning: note that we write the terms in the order $\omega_f*\omega_g$ for the map $g \circ f$.
	This is more convenient when writing out co-orientations using the local orientations as we obtain expressions such as $(\beta_V, \beta_W)*(\beta_W,\beta_M) = (\beta_V,\beta_M)$.
\end{remark}


\begin{notation}\label{N: implicit notation}
	It will be useful in notation to sometimes leave the maps, codomains, and co-orientations all implicit once they have already been established and just write $V$ to represent the co-oriented map $f \colon V \to M$.
	In this, case we write $-V$ to refer to the same map with the opposite co-orientation.
	\greg{Is there a better place to put this?}
\end{notation}

Just as is the case for orientations, when working with co-orientations it suffices to consider the interior of the manifold, in this case of the domain.

\begin{proposition}\label{P: interior co-orientation}
Let $g \colon W \to M$ be a map of manifolds with corners. Then $g$ is co-orientable if and only if its restriction to $S^0(W)$, the interior of $W$, is co-orientable.
\end{proposition}
\begin{proof}
It is clear from the definition that if $g$ is co-orientable then so is its restriction to any open set of $W$.

Conversely, suppose the restriction of $g$ to $S^0(W)$ is co-orientable. As $W$ is a topological manifold with boundary, which we will denote $bd(W)$, the collaring theorem tells us that $bd(W)$ possesses a collar \cite[Theorem 2]{Bro62}. Let $C$ be a closed collar of $bd(W)$. Then $C$ is homeomorphic to $bd(W)\times I$ with $bd (W)$ identified under the homeomorphism with $bd(W) \times 0$. Let $B$ be the image of $bd(W) \times 1$ in $W$, and let $int(C)$ be the image of $bd(W) \times [0,1)$. Since $g|_{S^0(W)}$ is co-orientable, any choice of co-orientation restricts to a bundle isomorphism $\bigwedge TW|_B \to \bigwedge g^*\Or(TM)|_B$. Due to the product structure of the collar, the restriction of $\bigwedge TW$ to $C$ is isomorphic to $\bigwedge TW|_B \times I$ and similarly for $\bigwedge g^*\Or(TM)$ \cite[Theorem 3.4.4]{Hus94}, and so the bundle isomorphism over $B$ can be extended to a bundle isomorphism over $C$. As we chose the bundle isomorphism over $B$ to be the restriction of an isomorphism over $g|_{S^0(W)}$, we can glue together the bundle isomorphisms over $C$ and over $W-int(C)$ to obtain an isomorphism over all of $W$.
\end{proof}

\subsection{Normal co-orientations of immersions and co-orientations of boundaries}\label{S: normal orientation}

Once again, a key example is when $g$ is an immersion, which is co-orientable if and only if its normal bundle is orientable\footnote{Recall that technically all bundles are over $W$, though our convention is to elide that in the notation.
Hence we can consider $W$ to have a normal bundle even if $g$ is merely an immersion and not actually an embedding.
The normal bundle can be identified with $g^*(TM)/TW$ after identifying $TW$ with a sub-bundle of $g^*(TM)$ using the differential.
In any case, locally in the neighborhood of any point of $W$ one has the usual identification of the normal bundle with a tubular neighborhood of the image, which suffices for our purposes here.}.
Specifically, if $g \colon W \to M$ is an immersion, letting $\nu W$ denote the normal bundle, we have $TW \oplus \nu W \cong g^*TM$.
So, taking $w = \dim(W)$ and $m = \dim(M)$, a co-orientation is a nowhere-zero map from $\bigwedge^w TW$ to $\bigwedge^m g^*TM = \bigwedge^m (TW \oplus \nu W) \cong \bigwedge^w TW \otimes \bigwedge^{m-w}\nu W$.
Such a nowhere-zero map exists if and only $\bigwedge^{m-w}\nu W$ is a trivial line bundle, i.e.\ if $\nu W$ is orientable.

Given a specific orientation of $\nu W$, we specify a standard choice of \textbf{normal co-orientation} for the immersion $g \colon W \to M$ by the following local construction:

\begin{definition}\label{normal co-or}
	Let $g \colon W \to M$ be an immersion with normal bundle $\nu$ locally oriented (at some point of $W$) by $\beta_\nu$.
	Define the \textbf{normal co-orientation} associated to $\beta_\nu$ locally by the pair $\omega_{\nu} = (\beta_W, \beta_W \wedge \beta_\nu)$, where $\beta_W$ is any choice of a local orientation of $W$.
\end{definition}

This construction is independent of the choice of $\beta_W$, as reversing the orientation of $\beta_W$ gives
$$(-\beta_W, -\beta_W \wedge \beta_\nu) = (\beta_W, \beta_W \wedge \beta_\nu).$$
If the normal bundle to $W$ is oriented globally on $W$ then the construction is also independent of the point at which it is carried out since if $\gamma$ is a path from $x$ to $y$ with $\beta_W$ and $\beta_\nu$ constructed at $x$ then
\begin{equation*}
	(\gamma_*\beta_{W,x}\, , (g\gamma)_* (\beta_{W,x} \wedge \beta_{\nu,x})) =
	(\gamma_*\beta_{W,x}\, , (\gamma_* \beta_{W,x}) \wedge (g\gamma)_*\beta_{\nu,x}) =
	(\gamma_*\beta_{W,x}\, , (\gamma_* \beta_{W,x}) \wedge \beta_{\nu,y}),
\end{equation*}
using that $\nu$ is assumed oriented and that, via the immersion, we can treat a path in $W$ as a path in $M$ and tangent vectors of $W$ as tangent vectors of $M$.
Now taking $\beta_{W,y}$ to be $\gamma_* \beta_{W,x}$, the expression on the right again has the prescribed form.
So if the normal bundle to $W$ is oriented, these local choices determine a global co-orientation of $W \to M$.
If the normal bundle is orientable, one can conversely orient the normal bundle via this formula if one is given a co-orientation: choose $\beta_\nu$ so that $(\beta_W, \beta_W \wedge \beta_\nu)$ is the co-orientation of the immersion.

As we shall see, signs in the theory of co-orientations are highly dependent on choices.
One such choice in this definition is whether to append the local normal orientation before or after the local tangent orientation.

\subsubsection{Quillen co-orientations}\label{S: Quillen}

There is a useful alternative, though equivalent, definition of co-orientations due to Quillen \cite{Quil71} that only involves orientations of normal bundles as in the preceding section.\footnote{Quillen's context was slightly different.
He assumed the normal bundles to have complex structures and so called these \textit{complex orientations}.}

\begin{lemma}\label{L: Quillen}
	A map $g \colon W \to M$ is co-orientable if and only if for some $N \in \Z_{\geq 0}$ it factors as the composition $W \into M \times \R^N \to M$ of an embedding and a projection such that the image of $W$ in $M \times \R^N$ has an orientable normal bundle.
\end{lemma}

\begin{proof}
	We first note that such a smooth factorization always exists by \cref{C: embed V}.

	Next we observe that $T(M \times \R^N) \cong \pi^*(TM) \oplus \underline{\R}^N$ where $\underline{\R}^N$ is the trivial $\R^N$ bundle over $M \times \R^N$.
	As $\Or( \underline\R^N)$ is a trivial line bundle,
	\begin{equation*}
		\Or(T(M \times \R^N)) \cong \Or(\pi^*(TM)) \otimes \Or(\underline\R^N) \cong \Or(\pi^*TM).
	\end{equation*}
	Thus $\pi$ is always co-orientable.
	Furthermore, we see that
	\begin{equation*}
		\Or(g^*TM) \cong \Or(e^*\pi^*TM) \cong e^*\Or(\pi^*TM) \cong e^*\Or(T(M \times \R^N)) \cong \Or(e^*T(M \times \R^N)).
	\end{equation*}
	So $e$ is co-orientable if and only if $g$ is co-orientable.
	But $e$ is an immersion, and so it is co-orientable if and only if the normal bundle of the image is orientable by the discussion preceding \cref{normal co-or}.
\end{proof}

\begin{definition}\label{D: Quillen normal or}
	If $(e_1, \ldots, e_n)$ is the standard ordered basis of $\R^N$, we denote the corresponding orientation from by $\beta_E = e_1 \wedge \cdots \wedge e_N$.
	Then the projection $\pi \colon M \times \R^N \to M$ has a canonical co-orientation $(\beta_M \wedge \beta_E, \beta_M)$ that is well defined as $\R^N$ is contractible.
	If $g \colon W \to M$ is co-oriented and $e \colon W \into M \times \R^N$ is an immersion with normal bundle $\nu$, we then define the \textbf{compatible normal orientation} or \textbf{Quillen normal orientation} of $\nu$ so that the composition of co-orientations $(\beta_W,\beta_W \wedge \beta_\nu)$ with $(\beta_M \wedge \beta_E,\beta_M)$ is the given co-orientation of $g$.
	In other words, if $(\beta_W,\beta_M)$ is the given co-orientation of $g$, then the compatible normal orientation of $\nu$ is such that $\beta_W \wedge \beta_\nu = \beta_M \wedge \beta_E$.

	We sometimes speak of the entire structure $W \into M \times \R^N \to M$ with an orientation of $\nu$ as a \textbf{Quillen co-orientation} for $W \to M$ or as a ``compatible Quillen co-orientation'' if we have already specified a co-orientation for $W \to M$ and we wish to choose the Quillen co-orientation that agrees with it.
\end{definition}

\begin{remark}\label{R: immersion}
	In particular, if $g \colon W \to M$ is an immersion, then, by taking $N = 0$, a co-orientation of $g$ is equivalent to a Quillen normal orientation of the normal bundle $\nu$ of $W$ in $M$.
	In particular, the co-orientation is given locally by $(\beta_W, \beta_M)$ if and only if $\nu$ is oriented so that $\beta_W \wedge \beta_\nu = \beta_M$.
	If $g$ is a codimension-$0$ immersion, then $\nu$ will be $0$-dimensional, and if the co-orientation is the tautological one then $\beta_\nu$ will be the positive orientation at each point.
\end{remark}

Lipyanskiy's definition of co-orientation in \cite{Lipy14} factors a proper map through a map which is surjective onto $TM$, rather than injective from $TW$ as in the Quillen approach.
We discuss Lipyanskiy's co-orientations in an appendix at the end of this section; see \cref{S: Lipyanskiy co-orientations}.

\subsubsection{Co-orientations of boundaries}

Given a co-oriented map $g \colon W \to M$ where $W$ is a manifold with corners, we can use a normal co-orientation of $\bd W$ in $W$ together with the composition of co-orientations noted in \cref{R: cooriented composition} to define ``boundary co-orientations'':

\begin{definition}\label{D: boundary co-orientation}
	Let $\nu$ be the normal bundle of $\bd W$ as an immersed with corners of $W$.
	The \textbf{standard co-orientation of a boundary immersion} $i_{\bd W} \colon \bd W \into W$ is the normal co-orientation (see \cref{normal co-or}) associated to the \textit{inward}-pointing\footnote{The outward-pointing normal would also work to provide a co-orientation convention for which the Leibniz formula of \cref{S: co-orient pullbacks} holds.
	However, using an outward normal is not consistent with the intersection map $\mc I$ of \cref{S: intersection map} being a chain map with our other conventions, while using the inward normal does make $\mc I$ a chain map.} orientation of $\nu$.

	If $g \colon W \to M$ is co-oriented, the \textbf{induced co-orientation} or \textbf{boundary co-orientation} of the composition $gi_{\bd W}$ is the composition of the standard co-orientation of $i_{\bd W}$ with the given co-orientation of $g \colon W \to M$.
	We write $\bd g \colon \bd W \to M$ to denote $gi_{\bd W}$ with its induced co-orientation.
\end{definition}

In \cref{S: geometric cochains}, we will use induced co-orientations on boundaries to define the differential in the geometric cochain complex.



\begin{example}\label{E: splitting example 1}
	Suppose $g \colon W \into M$ is the embedding of a codimension-$0$ submanifold of $M$.
	In this case $TW$ is the pullback of $TM$, and we have the tautological co-orientation of \cref{D: tautological co-orientation} that, slightly abusing notation, we can write as $(\beta_M,\beta_M)$.
	A particularly important pair of examples is given by the inclusions $g^- \colon (-\infty, 0] \into \R$ and $g^+ \colon [0,\infty) \into \R$, each tautologically co-oriented by $(e_1,e_1)$, where $e_1$ is the standard unit vector in $\R$.

	Next consider the submanifold consisting of the point $0 \in \R$.
	Its tangent space has trivial determinant line bundle, and we can choose the basis element to be the element $1 \in \bigwedge^0 T0 \cong \R$.
    If we give the normal bundle to $0$ in $\R$ the standard orientation in the positive direction, denoted by the standard basis vector $e_1$, then the corresponding normal orientation of the inclusion of $0$ into $\R$ will be $(1, 1 \wedge e_1) = (1, e_1)$.
	By \cref{normal co-or,D: boundary co-orientation}, the standard co-orientation of the boundary inclusion $\{0\} \into (-\infty, 0]$ is $(1, 1 \wedge -e_1) = (1, -e_1)$.
	The boundary co-orientation of the inclusion $\{0\} \to \R$ induced by the tautological co-orientation of the inclusion $g^- \colon (-\infty, 0] \to \R$ is then the composition of $(1, -e_1)$ with $(e_1, e_1)=(-e_1,-e_1)$, which is again $(1,-e_1) = -(1,e_1)$ (all bases interpreted in the appropriate spaces).
	As the inward normal to $[0,\infty)$ at $0$ is $e_1$, the inclusion $g^+ \colon [0,\infty) \to \R$ induces the opposite co-orientation $(1, e_1)$ on the inclusion $\{0\} \to \R$.
	So we see in this case different co-orientations of the same inclusion $0\into \R$ depending on which conventions and which contexts we choose to employ.
\end{example}

The following example will be very important later for breaking geometric chains and cochains into pieces.

\begin{example}\label{E: manifold decomposition}
	Suppose we have a smooth map $\varphi \colon W \to \R$ and that $0$ is a regular value of $\varphi$.
	Consider the spaces $W^0 = (\varphi)^{-1}(0)$, $W^- = (\varphi)^{-1}((-\infty,0])$, and $W^+ = (\varphi)^{-1}([0,\infty))$.
	We can identify $W^0$ with the fiber product $0 \times_{\R} W$ of $\varphi$ with the embedding of $0$ into $\R$, as our canonical realization of this fiber product is
	$$\{(0,x) \in 0 \times W \mid \varphi (x)= 0\}.$$
	Similarly, $W^-$ and $W^+$ are $(-\infty,0] \times_{\R} W$ and $[0,\infty) \times_{\R} W$,  
	as $$[0,\infty) \times_{\R} W = \{(t,x) \in [0,\infty) \times W \mid \varphi(x) = t\},$$
	which is simply the graph of $\varphi$ over $W^+$; 
	so there is a diffeomorphism between the fiber product and $W^+$ determined by the inverse maps $(t,x) \mapsto x$ and $x \mapsto (\varphi(x), x)$.
	The case of $W^-$ is analogous.
	That $0$ is a regular value of $\varphi$ implies that $\varphi$ is transverse to the embedding of $0$ into $\R$, and it follows by \cref{pullback} that $W^0$, $W^+$, and $W^-$ are manifolds with corners.
	In fact, $W^0$ is a submanifold (with corners) of $W$ by \cite[Proposition 4.2.9]{MaDo92}. 
	By the boundary formula of \cref{P: product boundary}, 
	$$\bd (W^-)= \left(0 \times_{\R} W\right) \sqcup \left((-\infty,0] \times_{\R} \bd W\right) = W^0 \sqcup (\bd W)^-,$$
	so $W^0$ is a boundary component of $W^-$ and similarly for $W^+$.

	Now, at least over its interior, the normal bundle of $W^0$ in $W$ is the pullback via $\varphi$ of the normal bundle of $0$ in $\R$ by \cref{L: normal pullback}. Giving the normal bundle of $0$ in $\R$ the standard orientation in the positive direction, denoted by the basis vector $e_1$, this orientation pulls back to a natural orientation of the normal bundle to the interior of $W^0$ in $W$. Applying \cref{normal co-or,P: interior co-orientation}, we obtain a normal co-orientation for $W^0$ in $W$, which we call the \textbf{co-orientation of $W^0$ in $W$ induced by $\varphi$}. This is the co-orientation $(\beta_{W^0}, \beta_{W^0} \wedge (\varphi)^*e_1)$.

	We also have the boundary co-orientations of the inclusions of $W^0$ into $W^{\pm}$.
	Analogously to \cref{E: splitting example 1}, the boundary co-orientation from the composite $W^0 \to W^- \into W$  is the opposite of the co-orientation of $W^0\into W$ induced by $\varphi$, while the boundary co-orientation from the composite $W^0 \to W^+ \into W$  agrees with the co-orientation of $W^0\into W$ induced by $\varphi$

	Next we suppose that $\varphi$ is the composition $\varphi = \phi g$ of two smooth maps $W \xr{g} M \xr{\phi}\R$, where $M$ is a manifold without boundary. 
	This may at first seem a bit artificial, but such a setting will arise in the creasing construction in \cref{S: creasing} for breaking up geometric chains and cochains.
	We further suppose $0$ is a regular value for $\phi$ and for $\phi g$ and note that $0$ being a regular value for $\varphi = \phi g$ is equivalent to $g$ being transverse to $M^0 = \phi^{-1}(0)$, which is a codimension-one submanifold of $M$ by the preceding discussion.
	In fact, by the applying the preceding discussion to $\phi$, we obtain spaces $M^0$ and $M^\pm$, and applying the discussion to $\varphi = \phi g$, we obtain spaces $W^0$ and $W^\pm$. 
	Furthermore, we notice for use later that $W^0$ and $W^\pm$ are diffeomorphic to the respective fiber products $M^0 \times_M W$ and $M^{\pm} \times_M W$ by an argument analogous to that above for the fiber products over $\R$; in particular, these are just the graphs of $g$ over $W^0$ and $W^\pm$.
	
	
	Now suppose that $g$ is co-oriented with local representatives $(\beta_W, \beta_M)$. 
	If we precompose this co-orientation of $g$ with the tautological co-orientations of the inclusions $W^\pm \into W$, we can co-orient $g|_{W^\pm}$ by $(\beta_W, \beta_M)$.
	This is simply the restriction to $W^\pm$ of the co-orientation of $g$.
	Similarly, by composing the co-orientation of $g$ with the $\varphi$-induced co-orientation of $W^0 \into W$, we obtain the \textbf{co-orientation of $g|_{W^0} \colon W^0 \to M$ induced by $\phi$}, which in symbols is just $(\beta_{W^0}, \beta_{W^0} \wedge (\phi g)^*e_1)*(\beta_W,\beta_M)$.
	Furthermore, as in \cref{E: splitting example 1}, the co-orientation of $g|_{W^0} \colon W^0 \to M$ induced by $\phi$ disagrees with its boundary co-orientation as a component of $\bd(g|_{W^-})$, while it agrees with its co-orientation as a component of $\bd(g|_{W^+})$.

	When $g$ is co-oriented, our co-orientation computations here for the restrictions of $g$ to $W^0$ and $W^\pm$ will later be seen to be consistent with the co-orientations we define for fiber products of co-oriented maps; see \cref{E: codim 0 and 1 co-or as fiber products}.
	In \cref{C: co-orient W0}, we will use that technology to see that $\bd (W^0)$ and $(\bd W)^0$ agree as spaces, but their maps to $M$ will have opposite co-orientations under our conventions, i.e.\ ``$\bd (W^0) = -(\bd W)^0$,'' eliding the maps.
\end{example}

\begin{comment}
	\red{TO DO SOMEWHERE: It will be convenient to show that $g \colon W^0 \to M$ with this co-orientation is the pullback of $M^0 \into M$ (co-oriented as defined here) and $g \colon W \to M$ and similarly for $g \colon W^\pm \to M$.
		Also need to rewrite things in other places as $M^0$, $M^\pm$, etc instead of always writing $\varphi^{-1}((\infty,0])$ etc.
		Also need to show that $\bd W^0 = -(\bd W)^0$, which should follow from the Leibniz rule and the first thing the previous sentence.}
\end{comment}

\subsubsection{Co-orientations of boundaries of boundaries}

In order to form a chain complex of geometric cochains in \cref{S: geometric cochains}, we will need a result about co-orientations of $\bd^2 W$.
Recall from \cref{S: boundaries} that Proposition 2.9 of \cite{Joy12} identifies $\bd^2 W$ with the set of points $(x,\bb_1,\bb_2)$ with $x \in W$ and the $\bb_i$ encoding distinct local boundary components.
The map $i_{\bd^2 W} \colon \bd^2 W \to W$ takes $(x,\bb_1,\bb_2)$ to $x$.
The manifold with corners $\bd^2 W$ is equipped with a canonical diffeomorphism $\rho$ defined by $(x,\bb_1,\bb_2) \mapsto (x,\bb_2,\bb_1)$.

\begin{lemma}\label{L: boundary2}
	Suppose $i_{\bd^2 W} \colon \bd^2 W \to W$ is co-oriented via the composition of boundary co-orientations $\bd^2 W \to \bd W \to W$, and suppose $\rho \colon \bd^2 W \to \bd^2 W$ is given its tautological co-orientation (see \cref{D: tautological co-orientation}).
	Then $i_{\bd^2 W}$ and $i_{\bd^2 W}\rho$ have opposite co-orientations.
\end{lemma}

\begin{proof}
	It suffices to consider points $(x,\bb_1,\bb_2) \in \bd^2 W$ with $x \in S^2(W)$, as such points fill out the interior of $\bd^2 W$.
	In $W$, such $x$ have neighborhoods of the form $[0,\infty)^2 \times \R^{w-2}$ with $x$ at the origin.
	We identify $[0,\infty)^2$ with the first quadrant of $\R^2$, letting $X$ and $Y$ denote the non-negative $x$ and $y$ axes.
	We let $\bb_X$ and $\bb_Y$ be the corresponding local boundary components.
	Then the preimage in $\bd^2 W$ of a small neighborhood $U$ of $x$ in $S^2(W)$ consists of two copies of $U$ that we can write $(U,\bb_X,\bb_Y)$ and $(U,\bb_Y,\bb_X)$.
	The notation indicates that we think of the first copy of $U$ as embedding into $X \times \R^{w-2} \subset \bd W$ and the second as embedding into $Y \times \R^{w-2} \subset \bd W$.
	The map $i_{\bd W} \colon \bd W \to W$ then identifies the two copies.
	The map $\rho$ simply interchanges them.\greg{Picture here?}

	Let $\beta_X$ and $\beta_Y$ correspond to the positively-directed tangent vectors in $X$ and $Y$, and let $\beta_U$ be an arbitrary local orientation of $U$.
	Abusing notation, we also write $\beta_U$ for the corresponding local orientations of $(U,\bb_X,\bb_Y)$ and $(U,\bb_Y,\bb_X)$.
	The induced co-orientation on $\rho$ can then be written $(\beta_U,\beta_U)$.
	Up to identifying neighborhoods in $W$ with their local models, the boundary co-orientation of $i_{\bd^2 W}$ on $(U,\bb_X,\bb_Y)$ comes from first mapping it into $X \times \R^{w-2}$ and then into $X \times Y \times \R^{w-2}$.
	So from the definition of boundary co-orientations, this co-orientation is $(\beta_U, \beta_U \wedge \beta_X \wedge \beta_Y)$.
	Analogously, the boundary co-orientation of $i_{\bd^2 W}$ on $(U,\bb_Y,\bb_X)$ is $(\beta_U, \beta_U \wedge \beta_Y \wedge \beta_X)$.
	By composition, the co-orientations of $i_{\bd^2 W}\rho$ on $(U,\bb_X,\bb_Y)$ and $(U,\bb_Y,\bb_X)$ are respectively $(\beta_U, \beta_U \wedge \beta_Y \wedge \beta_X)$ and $(\beta_U, \beta_U \wedge \beta_X \wedge \beta_Y)$ as first we interchange then embed.
	But $\beta_U \wedge \beta_X \wedge \beta_Y = -\beta_U \wedge \beta_Y \wedge \beta_X$, which establishes the lemma.
\end{proof}

\begin{remark}\label{R: bd2 oriented}
	A similar, though more familiar, argument using \cref{Con: oriented boundary} shows that if $W$ is oriented then $\bd^2 W$ possess an orientation reversing diffeomorphism.
	In this case, we observe that of our two copies of $U$, one is oriented by \cref{Con: oriented boundary} so that $\beta_X \wedge \beta_Y \wedge \beta_U$ is the local orientation of $W$ and the other is oriented so that $\beta_Y \wedge \beta_X \wedge \beta_U$ is the orientation of $W$.
	Thus the two copies of $U$ have opposite orientations, and again the diffeomorphism simply interchanges the two copies.
\end{remark}

\subsection{Co-orientation of homotopies}\label{S: co-oriented homotopy}

In this section we develop co-orientations related to homotopies.
As the product of two spaces is the same as their fiber product over a point, we have by \cref{P: product boundary} and rearranging the order of components:
\begin{equation*}
	\bd(W \times I) =
	(\bd W \times I) \sqcup (W \times \bd I) =
	(W \times 1) \sqcup (W \times 0) \sqcup (\bd W \times I).
\end{equation*}

Now recall that in general if we have a map $f \colon V \to M$ then we write $\bd f$ for the composition
$$\bd V \xr{i_{\bd V}} V \xr{f} M.$$
We make the following definition:

\begin{definition}\label{D: co-oriented homotopy}
	If $G \colon W \times I \to M$ is a co-oriented map, we say that $G$ is a \textbf{co-oriented homotopy} (or simply a \textbf{homotopy} when working with co-orientations is understood) from $g_0 \colon W \to M$ to $g_1 \colon W \to M$ if $\bd G = g_1 \amalg -g_0\amalg H$, where $g_1$, $-g_0$, and $H$ correspond respectively to the compositions of $G$ with the inclusions into $W \times I$ of $W \times 1$, $W \times 0$, and $\bd W \times I$, taking each with its boundary co-orientation as in \cref{D: boundary co-orientation}.
	In particular, then, $g_0$ is the composition
	$$W=W \times 0 \into W \times I \xr{G} M$$
	with the opposite of the boundary co-orientation coming from $G$.
\end{definition}

Note that, by analogy with homotopies involving oriented manifolds, a homotopy from $g_0$ to $g_1$ involves the oppositely co-oriented $-g_0$ in the boundary formula.
In the oriented case, this arises because if we orient $W$ by, say, $\beta_W$ then to orient $W \times I$ we consider $\beta_W \wedge \beta_I$.
Then at one end of the cylinder $\beta_W$ agrees with the boundary orientation of $\bd (W \times I)$ while at the other end it disagrees.
The situation for co-orientations is analogous.

Although we will not need it, we note that by employing appropriate smoothing near the boundaries in order to accomplish transitivity, co-oriented homotopy can be shown to be an equivalence relation on maps $W \to M$.

In our most common use of homotopies, we begin with a co-oriented map $g \colon W \to M$ and want to construct a homotopic co-oriented map.
For this the following lemma is useful.

\begin{lemma}\label{L: co-orientable homotopies}
	If $g \colon W \to M$ is co-orientable and $G \colon W \times I \to M$ is a homotopy with $g = G(-,t_0)$ for some $t_0 \in I$, then $G$ is co-orientable.
	Conversely, if $G \colon W \times I \to M$ is co-orientable, then so is $g = G(-,t_0) \colon W \to M$ for any $t_0 \in I$.
\end{lemma}

\begin{proof}
	Over $W \times t_0$, we have $\Or(T(W \times I)) \cong \Or(TW \oplus TI) \cong \Or(TW) \otimes \Or (TI) \cong \Or(TW)$, while the restriction of $G^*\Or(TM)$ over $W \times t_0$ is just $g^*\Or(TM)$.
	If $G$ is co-orientable then there is a nowhere-vanishing map of line bundles $\Or(T(W \times I)) \to G^*\Or(TM))$, so restricting to $W \times t_0$ and using the above identifications we obtain a nowhere-vanishing map of line bundles $\Or(TW) \to g^*\Or(TM))$ over $W \times t_0$, hence $g$ is co-orientable.
	Conversely,
	if $g$ is co-orientable, there is a nowhere-vanishing map of line bundles $\Or(TW) \to g^*\Or(TM)$ over $W \times t_0$.
	By general bundle theory \cite[Theorem 3.4.4]{Hus94}, any vector bundle $E$ over $W \times I$ is isomorphic to $E_{t_0} \times I$, where $E_{t_0}$ is the restriction of $E$ to $W \times \{t_0\}$.
	So our nowhere-vanishing map of line bundles over $W \times t_0$ extends to a nowhere vanishing map of line bundles $\Or(T(W \times I)) \to G^*\Or(TM))$ over $W \times I$.
	This implies the co-orientability of $G$.
\end{proof}

\begin{definition}\label{D: homotopy co-orientation}
	Suppose $g_0 \colon W \to M$ is co-oriented and $G \colon W \times I \to M$ is a smooth homotopy with $G(-,0) = g_0$.
	Then, by the above lemma, $G$ is co-orientable and clearly there is exactly one choice of co-orientation for $G$ for which the $W \times 0$ component of $\bd G$ is $-g_0$.
	We call this co-orientation the \textbf{co-orientation on $G$ induced by $g_0$}.
	The map $G$ then determines a co-oriented homotopy from $g_0$ to a co-oriented map $g_1 \colon W \to M$.
	We call this co-orientation on $g_1 = G(-,1)$ the \textbf{induced co-orientation on $g_1$.}
\end{definition}

\begin{remark}\label{R: stationary homotopy}
	In the above scenario, if $g_0$ is co-oriented locally at $x \in W$ by $(\beta_W,\beta_M)$, then at $(x,0) \in W \times I$, the corresponding local co-orientation of $G$ that yields $-g_0$ as a boundary component of $G$ is $(\beta_W \wedge -\beta_I, \beta_M)$, where $\beta_I$ corresponds to the standard orientation of $I$.
	This follows from $(\beta_W,\beta_W \wedge \beta_I)$ being the boundary co-orientation of $W \times 0 \into W \times I$ as $\beta_I$ corresponds to the inward pointing normal at $0 \in I$.
	As we can take $\beta_W \wedge -\beta_I$ to be a consistent orientation along the path given by $\gamma(t) = (x,t)$, we have, recalling the notation from \cref{S: co-orientations}, that $\gamma_*(\beta_W \wedge -\beta_I, \beta_M) = (\beta_W \wedge -\beta_I,\gamma_*\beta_M)$, and at this end of the homotopy the induced local co-orientation of $g_1$ at $x$ is $(\beta_W,\gamma_*\beta_M)$.
	If $G$ is stationary along $x \times I$, then the co-orientation for $g_1$ at $x$ is again $(\beta_W,\beta_M)$ so that the co-orientations of $g_0$ and $g_1$ agree at $x$.
	This observation will be useful below in showing that pullback co-orientations are well defined.
\end{remark}

\begin{comment}
	If $\beta_{W,x}$ is a local orientation of $W$ at a point $x \in W$ and $\gamma$ is a path in $W$ with $\gamma(0) = x$, then $\gamma$ determines a local orientation $\gamma_*\beta_{W,x}$ of $W$ at $\gamma(1)$ via any lift of $\gamma$ to the complement of the zero section of $\Or(TW)$.
	Similarly, given a map $g \colon W \to M$ and a local orientation $\beta_{M,g(x)}$ of $M$ at $g(x)$, the path $g\gamma$ determines a local orientation $(g\gamma)_*\beta_{M,g(x)}$ at $g\gamma(1)$.
	Of course $\gamma_*\beta_{W,x}$ and $(g\gamma)_*\beta_{M,g(x)}$ depend on $\gamma$, but the condition that $g$ be co-orientable is precisely the condition that the pair $(\gamma_*\beta_{W,x}, (g\gamma)_*\beta_{M,g(x)})$ be independent of $\gamma$.

	We use an analogous construction when we have a homotopy $G: W \times I \to M$ and a co-orientation of $g_0 = G(-,0) \colon W \to M$ and wish to define a co-orientation of $g_1 = G(-,1) \colon W \to M$.
	Explicitly, if the pair $(\beta_{W,x},\beta_{M,g(x)})$ is a local co-orientation for $g_0$ at $x \in W$, then we define the \emph{induced co-orientation} for $g_1$ at $x$ to be $(\beta_{W,x},G(x,-)_*\beta_M)$.
	Any $g_t = G(-,t)$ can be co-oriented analogously.

	\begin{definition}
		If $g \colon W \to M$ is co-oriented and $G \colon W \times I \to M$ is a homotopy with $G|_{W \times \{t_0\}} = g$, we define the \textbf{homotopy co-orientation on $G \colon W \times I \to M$ induced by $g$} so that if $(\beta_W,\beta_M)$ is a local co-orientation of $g$ at $x \in W$ then $(\beta_W \wedge \beta_{e_1},\beta_M)$ co-orients $G$ at $(x,t_0)$, where $\beta_{e_1}$ is the local orientation of $I$ corresponding to the standard positively-oriented tangent vector.
		As $G$ is co-orientable, this determines a co-orientation for the whole map $G$.
	\end{definition}

	\begin{lemma}
		Let $g_0 \colon W \to M$ be co-oriented, and let $G \colon W \times I \to M$ be a homotopy with $G(-,0) = g_0$.
		Let $G$ and $g_1 = G(-,1)$ have the induced co-orientations.
		Furthermore, let $G_\bd:(\bd W) \times I \to M$ be the composition $(\bd W) \times I \xr{i_{\bd W} \times \id_I} W \times I \xr{G}M$, co-oriented by taking the homotopy orientation induced from $g_0i_{\bd W} \colon \bd W \to M$ with its standard boundary co-orientation obtained from the co-orientation of $g_0$.
		Then $$\bd G = g_1-g_0-G_\bd.$$
	\end{lemma}
\end{comment}

\begin{lemma}\label{L: co-oriented homotopy}
	Suppose $G \colon W \times I \to M$ is a co-oriented homotopy from $g_0$ to $g_1$ so that $\bd G = g_1 \amalg -g_0\amalg H$ as in \cref{D: co-oriented homotopy}.
	Then $H$ is a homotopy from $-\bd g_0$ to $-\bd g_1$.
\end{lemma}

\begin{proof}
	By definition, $H$ is co-oriented as a boundary component of the co-oriented map $G$, so it remains to check that the induced co-orientations of the ends of $H$ have the expected signs.
	This could be done directly, but rather we use \cref{L: boundary2}, noting that each copy of $\bd W$ (at the top and bottom of the cylinder) can be considered to be a piece of $\bd^2(W \times I)$.
	In particular, applying \cref{L: boundary2} we see the maps $\bd W \to M$ take opposite co-orientations depending on whether we think of them as first mapping $\bd W$ into $W$ and then identifying $W$ as one end of the cylinder versus first including $\bd W$ into $(\bd W) \times I$ and then mapping this to $W \times I$.
	In both cases, we follow with the map $G$.
	As we think of $g_1$ as defined on $W \subset W \times I$ and let $\bd g_1$ denote its boundary, we see that the corresponding map from the top of the cylinder $(\bd W) \times I$ must be $-\bd g_1$.
	Similarly, the map at the bottom of the cylinder is $\bd g_0$.
	The lemma now follows from \cref{D: co-oriented homotopy}.
\end{proof}

\begin{comment}
	\begin{proof}
		We will always denote the local co-orientation of $g_0$ by $(\beta_W,\beta_M)$.
		Recall that the standard co-orientation for the boundary inclusion $i_{\bd V} \colon \bd V \to V$ of a manifold with corners $V$ is $(\beta_{\bd V},\beta_{\bd V} \wedge \beta_\nu)$, where $\nu$ is an outward pointing normal vector.
		If $f \colon V \to M$ is co-oriented, then the composite $\bd V \xr{i_{\bd V}}V \xr{f}M$ is co-oriented by composing the co-orientations of the component maps.

		Now consider $W \times 1 \subset W \times I$.
		At $(x,1)$, the induced homotopy orientation is $(\beta_W \wedge \beta_{e_1},\gamma_*\beta_M)$, where $\gamma$ is the path $I \to G(x,-)$.
		Composing with the standard boundary co-orientation $(\beta_W,\beta_W \wedge \beta_\nu)$ gives $(\beta_W,\gamma_*\beta_M)$, as we can take $\beta_\nu = \beta_{e_1}$.
		This is the the induced co-orientation for $g_1$.

		On the other hand, as the positively-oriented tangent vector to $I$ points inward at $0$, the standard boundary co-orientation of $W \times 0 \to W \times I \to M$ is the composition of $(\beta_W, \beta_W \wedge \beta_\nu) = -(\beta_W, \beta_W \wedge \beta_{e_1})$ with $(\beta_W \wedge \beta_{e_1},\beta_M)$.
		Thus as a piece of the boundary, $G \circ i_{W \times 0} = -g_0$.

		Finally, consider a point $(x,0) \in (\bd W) \times I$ and let $\nu_\bd$ be an outward pointing normal to $W$ at $i_{\bd W}(x)$.
		Then the standard co-orientation of $\bd W$ in $W$ is $(\beta_{\bd W},\beta_{\bd W} \wedge \beta_\nu)$ and the standard co-orientation of the composite $\bd g_0 = g_0i_{\bd W} \colon \bd W \to M$ is the composite of $(\beta_{\bd W},\beta_{\bd W} \wedge \beta_\nu)$ with $(\beta_W,\beta_M)$.
		If $\beta_W = \beta_{\bd W} \wedge \beta_{\nu_\bd}$, this composite co-orientation is $(\beta_{\bd W},\beta_M)$, and otherwise it is
		$-(\beta_{\bd W},\beta_M)$.
		The induced homotopy co-orientation of $G_\bd \colon \bd W \times I \xr{i_{\bd W} \times \id}W \times I \to M$ is then $\pm(\beta_{\bd W} \wedge \beta_{e_1},\beta_M)$, as $\beta_W = \beta_{\bd W} \wedge \beta_{\nu_\bd}$ or not.

		On the other hand, consider $(\bd W) \times I$ as part of the boundary of $W \times I$.
		The standard boundary co-orientation for $(\bd W) \times I$ in $W \times I$ is $(\beta_{(\bd W) \times I},\beta_{(\bd W) \times I}\wedge\beta_{\nu_\bd})$.
		This is independent of the choice of $\beta_{(\bd W) \times I}$, so we may take $\beta_{(\bd W) \times I} = \beta_W \wedge \beta_{e_1}$, where $\beta_{e_1}$ is positively-directed in $I$.
		Then
		$$(\beta_{(\bd W) \times I},\beta_{(\bd W) \times I}\wedge\beta_{\nu_\bd}) = (\beta_{\bd W} \wedge \beta_{e_1},\beta_{\bd W} \wedge \beta_{e_1}\wedge\beta_{\nu_\bd}) = -(\beta_{\bd W} \wedge \beta_{e_1},\beta_{\bd W} \wedge \beta_{\nu_\bd}\wedge\beta_{\nu_I}).$$
		Composing with the induced co-orientation $(\beta_W \wedge \beta_{e_1},\beta_M)$ of $G$ gives the composite $(\bd W) \times I \xr{i_{(\bd W) \times I}}W \times I \xr{G} M$ the co-orientation $-(\beta_{\bd W} \wedge \beta_{e_1},\beta_M)$ if $\beta_W = \beta_{\bd W} \wedge \beta_{\nu_\bd}$ and $(\beta_{\bd W} \wedge \beta_{e_1},\beta_M)$ otherwise.
		Thus, altogether, the $(\bd W) \times I$ boundary component of $W \times I$ with its boundary co-orientation is $-G_\bd$.
	\end{proof}
\end{comment}

\subsection{Co-orientations of pullbacks and fiber products}\label{S: co-orient pullbacks}

In this section, we define a convention for co-orientations of pullbacks and fiber products.
More specifically, if $f \colon V \to M$ and $g \colon W \to M$ are transverse smooth maps from manifolds with corners to a manifold without boundary and $f$ is co-oriented, we define a co-orientation of the pullback $f^* \colon V \times_M W \to W$.
This does not require $g$ to be co-oriented, but if it is, we can compose with $g$ to also get a co-orientation of the fiber product $V \times_M W \to M$.
Ultimately, this will allow us to define cup products of geometric cochains, and the various properties we will demonstrate for co-orientations of fiber products will be reflected in the standard properties for cup products. 

Recall that our canonical realization of the topological pullback $P = V \times_M W$ is defined to be $P = \{(x,y) \in V \times W \mid f(x) = g(y)\}$.
By Joyce \cite[Section 6]{Joy12}, the projections $P \to V$ and $P \to W$ are smooth, and hence so is $f \times_M g \colon P \to M$ given by $(x,y) \to f(x) = g(y)$.
It is not obvious how to define the co-orientations of pullbacks and fiber products, and any such definition will depend on choices of convention.
Our goal in this section is to provide a definition such that co-orientations of fiber products of co-oriented maps possess the following desirable properties:

\begin{enumerate}
	\item Embedding property: If $f \colon V \to M$ and $g \colon W \to M$ are transverse co-oriented embeddings, then their fiber product is just the (embedding of the) intersection of the images of $V$ and $W$ in $M$.\greg{Is this always an embedding? Do we care?}
	If $f$ and $g$ are normally co-oriented (see \cref{normal co-or}), then the intersection should be normally co-oriented with the orientation of the normal bundle of the intersection given by concatenating the orientation for the normal bundle of $V$ followed by the orientation for the normal bundle of $W$.

	\item Associativity: If $V$, $W$, and $X$ are all manifolds with corners mapping to the manifold without boundary $M$ and if all the required transversality assumptions hold for the following statement to involve only well-defined fiber products, then, using Notation \ref{N: implicit notation} to allow the domain to represent also the co-oriented map, we should have 
	$$(V \times_M W) \times_M X = V \times_M (W \times_M X).$$

	\item Graded commutativity: We should have 
	$$V \times_M W = (-1)^{(m-v)(m-w)}W \times_M V$$
	as fiber products, using Notation \ref{N: implicit notation}.
	See \cref{R: precise commutativity}, below, for further remarks on how to interpret this formula.

	\item Leibniz rule: We should have
	$$\bd (V \times_M W) = (\bd V \times_M W)\amalg (-1)^{m-v}(V \times_M \bd W),$$
	 again using Notation \ref{N: implicit notation}.
	This formula will hold for pullbacks as well as fiber products.
\end{enumerate}
Of course the latter two properties will later correspond to the graded commutativity and boundary formulas for geometric cup products.

Before getting into the specifics of the construction, we conclude this introductory section with some important general observations.
In the following subsections, we first show that pullbacks of co-orientable maps are co-orientable; the proof will then become the roadmap for defining a specific pullback co-orientation convention.
We then demonstrate that our convention yields pullback and fiber product co-orientations satisfying a number of desireable properties, including the above Leibniz rule.
Further properties of co-orientations of fiber products, including associativity and graded commutativity, will be proven in \cref{S: exterior products}, using a co-orientation we will define there for direct (exterior) products of co-oriented maps.

\begin{remark}\label{R: pullback representative}
	While our canonical pullback $P$ has been defined as $V \times_M W = \{(x,y) \in V \times W \mid f(x) = g(y)\}$, categorically the pullback $P$ is technically only well defined up to canonical diffeomorphisms.
	In particular, if $P$ and $P'$ are two specific representatives of the pullback categorically, we have commutative diagrams
	\begin{equation*}
		\begin{tikzcd}[column sep=small]
			P \arrow[rr, "\cong"] \arrow[dr] & & P' \arrow[dl] \\
			& W. &
		\end{tikzcd}
	\end{equation*}
	But, as we have observed in \cref{D: tautological co-orientation}, diffeomorphisms come equipped with tautological co-orientations, and so a co-orientation of $P \to M$ determines a unique co-orientation of $P' \to M$ by composition and vice versa.
	Thus, when working with co-orientations of pullbacks, we typically think of selecting a fixed representative $P \to W$ to work with for computations, though not always the canonical one.
	This observation shows that we are free to do so, and typically we will do so tacitly.
	This foreshadows the notion of isomorphic representatives of geometric chains and cochains; see \cref{D: equiv triv and small}.
\end{remark}

\begin{remark}\label{R: precise commutativity}
	This is also a good place to point out exactly what we mean by writing $V \times_M W = (-1)^{(m-v)(m-w)}W \times_M V$ in our graded commutativity statement, as, using the canonical pullbacks, $V \times_M W \subset V \times W$ and $W \times_M V \subset W \times V$ are different spaces, though canonically identified via the map $\tau \colon V \times W \to W \times V$ that switches the coordinates.
	This map fits into a commutative diagram of fiber products
	\[
	\begin{tikzcd}[column sep=tiny]
		V \times_M W \arrow[rr, "\tau"] \arrow[dr] & & W \times_M V \arrow[dl] \\
		& M. &
	\end{tikzcd}
	\]
	Again, $\tau$ has a tautological co-orientation from being a diffeomorphism, and so the statement means that co-orientation of the fiber product $V \times_M W \to M$ and the composite co-orientation of $\tau$ and then the fiber product co-orientation of $W \times_M V \to M$ should differ by the sign $(-1)^{(m-v)(m-w)}$.
\end{remark}

\begin{comment}
	Achieving all of these properties requires a number of non-obvious choices of conventions, as we shall see.
	We begin by providing some general perspective before proceeding to dive in to the general definition of induced co-orientation of a fiber product.

	At a high level, induced co-orientations of fiber products of transverse co-oriented maps will arise from a common alternative description of $P = V \times_M W$ as the preimage of the diagonal $\Delta M = \{(x,x) \in M \times M\}$ under the product map $f \times g \colon V \times W \to M \times M$.
	In other words, $P = (f \times g)^{-1}(\Delta M)$; the reader can easily verify that this is equivalent to the preceding definition.
	Such an
	identification gives rise to the following exact sequences of bundles over $P$, leaving the pullbacks of the bottom row to $P$ implicit in the notation:

	\begin{equation}\label{pullback exact}
		\begin{tikzcd}
			0 \arrow[r] & TP \arrow[r] \arrow[d] & T (V \times W) \arrow[r] \arrow[d] & \nu_{P \subset V \times W} \arrow[r] \arrow[d,"\cong", "i"'] & 0 \\
			0 \arrow[r] & TM \arrow[r,"D\Delta"] & T (M \times M) \arrow[r] & \nu_{\Delta M \subset M \times M} \arrow[r] & 0.
		\end{tikzcd}
	\end{equation}
	Here the two bundles labeled with $\nu$ are normal bundles, and we use the general fact that, given a smooth map of manifolds $h \colon A \to B$ with $C$ immersed in $B$ and $h$ transverse to $C$, the normal bundle of $h^{-1}(C)$ in $A$ is the pullback of the normal bundle of $C$ in $B$.
	In the case at hand, $P = (f \times g)^{-1}(\Delta M) \subset V \times W$ and the transversality of $f$ and $g$ implies that $f \times g$ is transverse to $\Delta M$, and so $\nu_{P \subset V \times W}$ is the pullback of $\nu_{\Delta M \subset M \times M}$; we label this isomorphism $i$.

	\red{D: we should have some more extensive differential topology discussion, pulling from Joyce's oeuvre.
		In particular, the general fact above
		should be developed in the manifolds with corners section, having a ``differential topology'' subsection.
	}
	\red{GBF: Should we do this????}

	Next we recall that if the sequence of vector bundles $0 \to K \to A \to C \to 0$ 	is exact, then we have a splitting $A \cong K \oplus C$ so that $\Or(A) \cong \Or(K) \otimes \Or(C)$.
	Such an
	isomorphism is not canonical, though it can be made concrete by, for example, taking the image in $A$ of an oriented basis of $K$ and following it by the
	preimage in $A$ of an oriented basis of $C$ to choose a representative oriented basis of $A$.
	But in our present development we will not fix an isomorphism in such a way, using only that they are isomorphic and
	later defining such isomorphisms implicitly.

	Applying these ideas to the exact sequences of Equation~\eqref{pullback exact},
	we have an isomorphism
	$\Or(TP) \otimes \Or(\nu_{P \subset V \times W}) \cong \Or (T(V \times W)) \cong \Or(TV) \otimes \Or(TW)$ and similarly for the second exact sequence.
	These fit into a not-necessarily-commutative square
	\begin{equation}\label{co-or stuff}
		\begin{tikzcd}
			\Or (TP) \otimes \Or(\nu_{P \subset V \times W}) \arrow[r, "\cong"] \arrow[d, "\gamma \otimes \Or(i)"] & \Or (T V) \otimes \Or (TW) \arrow[d] \\
			\Or (TM) \otimes \Or( \nu_{\Delta M \subset M \times M}) \arrow[r, "\cong"] & \Or (T M) \otimes \Or (TM).
		\end{tikzcd}
	\end{equation}
	%where $\gamma$ is induced by the standard map from the pullback to $M$
	%\red{[GBF: I do not think we can say this - isn't the whole point of this section is to define $\gamma$ - if there were a standard way to induce this we wouldn't need this whole diagram.]}.
	Note that the vertical maps of this diagram are not in general induced by the maps in Diagram \eqref{pullback exact}, just as in general a map $W \to M$ does not determine its co-orientation.
	However, we can choose the horizontal isomorphisms by our choices of splittings of the short exact sequences, we can let the vertical map on the right be the tensor product of the co-orientations of $V$ and $W$, and we can let
	$\Or(i)$ be determined by the canonical identification of $\nu_{P \subset V \times W}$ with the pullback of $\nu_{\Delta M \subset M \times M}$.
	Such choices will then determine a $\gamma$ making the diagram commute, and this will be our co-orientation of $P \to M$.
	Our choices of horizontal isomorphisms are essentially ``sign conventions.''
	We could for example set the top and bottom isomorphisms
	by the sort of ``basis of kernel followed by basis of cokernel'' convention mentioned above, but
	these ``obvious'' choices would not result in our three desired properties.

	BCOMMENT
	\red{Again, isn't the point that we do not know $\gamma$, so how can we fix the diagram to commute and then use that to determine $\gamma$? I think the idea is that we really need to say that the first diagram somehow determines this diagram via some conventions (what are those?).
		Then we know what the maps on the right are because that's just the tensor product of co-orientations of $V$ and $W$.
		On the left we know $i$ since that's canonical somehow (we still need to look up a good reference for that), and then all these other things determine a unique $\gamma$ so that the diagram commutes.
		This $\gamma$ is our co-orientation for $P$.
		So I think this all needs to be clarified.}
	Any such set of choices then yields
	a definition of pullback co-orientation through a diagram chase.
	In concrete terms, fix a local orientation $\beta_M$ of $M$, and then
	use the co-orientations of $f$ and $g$ to identify compatible local orientations $\beta_V$ of $V$ and $\beta_W$ of $W$.
	A fixed identification
	of the normal bundle of $\Delta M$ with the tangent bundle of $M$ then gives a $\beta_{\nu P \subset V \times W}$ which corresponds to $\beta_M$.
	The pullback co-orientation of the map $P \to M$ can then be defined pair $\beta_M$ with a
	local orientation $\beta_P$ of $P$ so that $\beta_P \otimes \beta_{\nu P \subset V \times W}$ maps to $\beta_V \otimes \beta_W$ under
	the top horizontal isomorphism of Equation~\ref{co-or stuff}.
	ECOMMENT

	In order to obtain these properties we will develop additional structure to control the
	isomorphisms in Diagram~\eqref{co-or stuff}.
	We do this first by working at the level of vector spaces and linear maps over a point before expanding to local definitions and then
	back to the global level.
\end{comment}

\subsubsection{Co-orientability of pullbacks and fiber products}

Before defining pullback and fiber product co-orientations, we first want to ensure that pullbacks and fiber products of co-orientable maps are themselves co-orientable.
The following argument about co-orientability will also provide a roadmap to defining such co-orientations.
We also take the opportunity to observe that pullbacks of proper maps are proper.

\begin{lemma}\label{L: co-orientable pullback}
	Suppose $f \colon V \to M$ and $g \colon W \to M$ are transverse maps of manifolds with corners to a manifold without boundary.
	Then:
	\begin{enumerate}
		\item If $f$ is co-orientable, the pullback $f^* \colon P = V \times_M W \to W$ is co-orientable.
		\item If $f$ is proper, the pullback $f^* \colon P = V \times_M W \to W$ is proper.
	\end{enumerate}
\end{lemma}

Note that $g$ need not be co-orientable or proper for this lemma to apply.

\begin{proof}
	We first show that the pullback of a proper map is proper.
	Recall that if we use the canonical version of $V \times_M W$ as $P = \{(x,y) \mid f(x) = g(y )\}$ then the pullback map $f^*:V\times_MW\to W$ can be identified with the restriction to $V \times_M W$ of the projection $\pi_W: V\times W\to W$. Let us write $f^*=\pi_W$ so that we label our maps
	\[
	\begin{tikzcd}
		P \arrow[r, "\pi_V"] \arrow[d, "\pi_W"] & V \arrow[d, "f"] \\
		W \arrow[r, "g"] & M.
	\end{tikzcd}
	\]
	Suppose $K \subset W$ is compact.
	We have
	\begin{align*}
		\pi_W^{-1}(K)& = \{z \in P \mid \pi_W(z) \in K\}\\
		& \subset \{z \in P \mid g\pi_W(z) \in g(K)\} \\
		& = \{z \in P \mid f\pi_V(z) \in g(K)\} \\
		& = \{z \in P \mid \pi_V(z) \in f^{-1}(g(K))\}.
	\end{align*}
	So $\pi_W^{-1}(K) \subset K \times f^{-1}(g(K)) \subset V \times W$.
	But this is a product of compact sets as $f$ is proper.
	So $\pi_W$ is proper.

	For co-orientability, it suffices by \cref{P: interior co-orientation} to consider the restriction to the interior of $V \times_M W$. Equivalently, we assume that $V$ and $W$ are manifolds without boundary for the remainder of the argument.
	By \cref{L: Quillen}, it suffices to utilize Quillen's approach to co-orientability as discussed in \cref{S: Quillen}.
	We factor $f$ as $V \xhookrightarrow{e} M \times \R^N \to M$, and then we have the pullback diagram
	\begin{equation}\label{D: pullback}
		\begin{tikzcd}
			P \arrow[r, "\pi_V"] \arrow[d] & V \arrow[hookrightarrow, d, "e"] \\
			W \times \R^N \arrow[r, "g \times \id"] \arrow[d] & M \times \R^N \arrow[d, "\pi_M"] \\
			W \arrow[r, "g"] & M.
		\end{tikzcd}
	\end{equation}

	The bottom square is evidently a pullback.
	Thus by elementary topology the top square is a pullback diagram if and only if the composite rectangle is a pullback diagram.
	So by choosing $P$ so that the top square is a pullback diagram, we obtain also the pullback of $W \xr{g} M\xleftarrow{f} V$.

	Since $f$ is transverse to $g$, we have $g \times \id$ transverse to $e$ by \cref{L: all transversality is wrt embeddings}.
	As $e$ is an embedding, it follows that $P = (g \times \id)^{-1}(e(V))$ is a submanifold of $W \times \R^N$ by \cite[Proposition IV.1.4]{Kos93}.
	Furthermore, by \cref{L: Quillen}, $e(V)$ has an orientable normal bundle in $M \times \R^N$, and since the pullback of the normal bundle is the normal bundle of the pullback, again by \cite[Proposition IV.1.4]{Kos93}, it follows that the normal bundle of $P$ in $W \times \R^N$ is also orientable.
	Applying \cref{L: Quillen} again, the map $f^* = \pi_W \colon P \to W$ is co-orientable.
\end{proof}

\begin{remark}\label{R: pullback representative 2}
	As foreshadowed in \cref{R: pullback representative}, we here use a different realization of $P = V \times_M W$.
	Thinking of the top square of the diagram as a pullback square, this $P$ is concretely the subset $\{(v,(w,z)) \in V \times (W \times \R^N) \mid e(v) = (g(w),z)\}$.
	If $\pi_1 \colon M \times \R^N \to M$ and $\pi_2 \colon M \times \R^N \to \R^N$ are the projections, we know by definition that $\pi_1(e(v)) = f(v)$, and this is also $g(w)$, so the points in this realization of $P$ also satisfy $f(v) = g(w)$.
	In fact, there is a canonical diffeomorphism between this realization of $P$ and our standard realization $\{(v,w) \in V \times W \mid f(v) = g(w)\}$ given by $(v,(w,z)) \mapsto (v,w)$ with inverse given by $(v,w) \mapsto (v,(w,\pi_2(e(v))))$.

	We also already observed in the above proof that $P$ can be identified with $(g \times \id)^{-1}(e(V)) \subset W \times \R^N$, though here we need to be a bit more careful as it is not completely clear that $(g \times \id)^{-1}(e(V))$ has the structure of an embedded submanifold with corners.
	In fact, the notions of immersion and embedding are complex when working with manifolds with corners; for example, see \cite[Chapter 3]{MaDo92}.
	However, we know by \cref{L: normal pullback} that $(g|_{S^0(W)} \times \id)^{-1}(e(S^0(V)))$ will be an embedded submanifolds in $S^0(W)\times \R^N$, and due to \cref{pullback,P: interior co-orientation} this will generally suffice for working with co-orientations.
	So, unless noted otherwise, we will below assume such restrictions, or simply that we are working with manifolds without boundary in the first place, whenever we identify $P$ with $(g \times \id)^{-1}(e(V))$.
	Otherwise, we assume that $P$ has its structure as a manifold with corners as a pullback as given in \cite{Joy12}.
	Of course when we do identify $P$ with $(g \times \id)^{-1}(e(V))$, it is via the map $\{(v,(w,z)) \in V \times (W \times \R^N) \mid e(v) = (g(w),z)\} \to W \times \R^N$ given by $(v,(w,z)) \mapsto (w,z)$.
\end{remark}

\begin{corollary}
	If $f \colon V \to M$ and $g \colon W \to M$ are transverse and co-orientable, their fiber product $V \times_M W \to M$ is also co-orientable.
\end{corollary}

\begin{proof}
	By the preceding lemma, the pullback $V \times_M W \to W$ is co-orientable, and the map $g \colon W \to M$ is co-orientable by assumption.
	Now choose co-orientations and compose to get a co-orientation of $P \to M$.
\end{proof}

\subsubsection{Co-orientations of pullbacks and fiber products}\label{S: co-orientation of pullbacks}

The construction in the proof of \cref{L: co-orientable pullback} provides a roadmap to define the co-orientations of pullbacks and fiber products.
For the following definition, refer again to Diagram \eqref{D: pullback}. Again thanks to \cref{P: interior co-orientation}, we can assume that our spaces are all smooth manifolds.

\begin{definition}\label{D: pullback coorient}
	Suppose $f \colon V \to M$ and $g \colon W \to M$ are transverse maps from manifolds without boundary to a manifold without boundary such that $f$ is co-oriented and the normal bundle $\nu V$ of $e(V) \subset M \times \R^N$ is given its Quillen normal orientation as defined in \cref{D: Quillen normal or}.
	Then the pullback $P = V \times_M W = (g \times \id_{\R^N})^{-1}(e(V)) \subset W \times \R^N$ has an oriented normal bundle that is the pullback of $\nu V$, which, by abuse of notation, we also label $\nu V$.
	Let $\beta_P$ and $\beta_W$ be local orientations of $P$ and $W$, and let $\beta_E$ be the standard orientation of $\R^N$.
	Define the \textbf{pullback co-orientation} on $P \to W$ to be the composition of the normal co-orientation $(\beta_P,\beta_P \wedge \beta_{\nu V})$ with the canonical co-orientation $(\beta_W \wedge \beta_E,\beta_W)$.
	In other words, the pullback co-orientation is $(\beta_P,\beta_W)$ if $\beta_P \wedge \beta_{\nu V} = \beta_W \wedge \beta_E$ and $-(\beta_P,\beta_W)$ otherwise.
  Equivalently, the co-orientation of the pullback is the one for which $\beta_{\nu V}$ is the compatible Quillen normal orientation.

	If $V$ and $W$ are manifolds with corners, then we define the \textbf{pullback co-orientation} on $P \to W$ by applying the above construction to the restrictions of $f$ and $g$ to $S^0(V)$ and $S^0(W)$; this is sufficient to determine the co-orientation by \cref{pullback,P: interior co-orientation}.

	Following \cref{D: top pullback}, we sometimes write $f^* \colon P \to W$.
	We also sometimes write $P = g^*V$ to emphasize that $P$ is the pullback of $V$ by $g$ to a manifold over $W$.

	If $g$ is co-oriented, define the \textbf{fiber product co-orientation} on $P \to M$ as the composition of the pullback co-orientation with the co-orientation of $g \colon W \to M$.
\end{definition}

In the definition, note that the Quillen orientation of $\nu V$ is determined by the co-orientation of $f$, and the orientation $\beta_E$ of $\R^N$ is taken to be canonically fixed across all instances.
The other orientations appearing in the definition are $\beta_P$ and $\beta_W$, but the co-orientation of the pullback $f^* \colon P \to W$ does not depend on the particular choices.
For example, suppose we choose $\beta_P$ and $\beta_W$ so that $\beta_P \wedge \beta_{\nu V} = \beta_W \wedge \beta_E$ and hence the pullback co-orientation is $(\beta_P,\beta_W)$.
If we replace $\beta_P$ with $\beta_P' = -\beta_P$, then
$\beta_P' \wedge \beta_{\nu V} = -\beta_P \wedge \beta_{\nu V} = -\beta_W \wedge \beta_E$, so the co-orientation is $-(\beta_P',\beta_W) = -(-\beta_P,\beta_W) = (\beta_P,\beta_W)$.
So the co-orientation is unchanged.
Similarly, the definition is independent of our choice of $\beta_W$.
We will show just below that the definition is independent of $N$ and $e$, as well.

\begin{remark}\label{R: co-or restriction or switch}
	It follows from the definition that reversing the co-orientation of $f \colon V \to M$ reverses the co-orientation of $f^* \colon V \times_M W \to W$.
	Furthermore, if $g \colon W \to M$ is co-oriented, then reversing the co-orientation of either $f$ or $g$ reverses the co-orientation of the fiber product $f \times_M g \colon V \times_M W \to M$.

	It is also clear that the definition is consistent under restrictions to open sets.
	In other words if $x \in V$, $y \in W$ with $f(x) = g(y)$, then replacing $V$, $W$, and $M$ with neighborhoods of $x$, $y$, and $f(x) = g(y)$ yields a co-orientation of the restriction of $f^*$ to a neighborhood of $(x,y) \in V \times_M W$ that is consistent with the co-orientation of all of $f^*$, at least so long as we use the same $N$ and a restriction of $e$, though we will now show independence of these choices as well.
\end{remark}

\begin{lemma}\label{L: pullback co well defined}
	The pullback and fiber product co-orientations do not depend on the choices of $N$, $e$, or the choices of local orientations of $P$, $V$, $W$, or $M$ used in the definition (note that local orientations of $V$ and $M$ are implicit in choosing a Quillen normal orientation for $\nu V$).
\end{lemma}

\begin{remark}
	The pullback co-orientations \textit{do} depend on such choices as the choice to use the standard orientation for $\R^N$ and the choice for the co-orientation of the projection $M \times \R^N \to M$ to be $(\beta_M \wedge \beta_E, \beta_M)$, but these are universal choices that we make once and for all.
	The point is that the pullback and fiber product co-orientations only depend on $f$, $g$, and their co-orientations, after fixing such universal choices that do not depend on $f$ or $g$.
\end{remark}

\begin{comment}
\greg{I've significantly updated the proof of this lemma. The first part of the proof is the same, but the old ending didn't make sense to me anymore, so I wrote a new ending, split off into its own second lemma below. It makes sense to me, and I think it's correct, but it was hard to get the ideas down concretely. So someone else should read through the whole proof and double check it, both for correctness and perhaps to improve the exposition. One very good thing about the new proof is that it's completely organic here - the old version made a forward reference to the Leibniz formula, whose proof can be taken to be independent to this by using a fixed map $e$, but it wasn't a great way to proceed. On the other hand, there was a certain elegance to using a higher level result instead of the hands-on mucking around that's in the proof now.}
\end{comment}

\begin{proof}[Proof of \cref{L: pullback co well defined}]
	It suffices to assume that $P$, $V$, $W$, and $M$ are all manifolds without boundary.
	As the local orientations of $P$, $V$, $W$, and $M$ used in the construction all come in pairs (e.g.\ $\beta_P$ in $(\beta_P,\beta_P \wedge \beta_{\nu V}))$, the construction is independent of those choices.

	Next, suppose we are given an embedding $e \colon V \into M \times \R^N$ and extend it to $e' = (e,0) \colon V \into M \times \R^N \times \R^n$.
	In the construction involving $e$, if we choose $\beta_V$, $\beta_M$ so that $(\beta_V,\beta_M)$ is the co-orientation for $f$, then by the definition of the Quillen orientation, $\nu V$ will be such that $\beta_V \wedge \beta_{\nu V} = \beta_M \wedge \beta_E$.
	If we now increase the dimension of the Euclidean factor to $\R^{N+n}$ and write its canonical local orientation as $\beta_{E^N} \wedge \beta_{E^n}$ while extending $e$ to $e'$, we see that $\nu V$ becomes $\nu V \oplus \underline{\R}^n$ so that $\beta_{\nu V}$ becomes $\beta_{\nu V} \wedge \beta_{E^n}$.
	Pulling back over $W$ we obtain the pullback co-orientation $(\beta_P,\beta_P \wedge \beta_{\nu V} \wedge \beta_{E^n})*(\beta_W \wedge \beta_{E^N} \wedge \beta_{E^n},\beta_W)$.
	This is $(\beta_P,\beta_W)$ if and only if $\beta_P \wedge \beta_{\nu V} \wedge \beta_{E^n} = \beta_W \wedge \beta_{E^N} \wedge \beta_{E^n}$, but this condition is equivalent to having $\beta_P \wedge \beta_{\nu V} = \beta_W \wedge \beta_{E^N}$.
	So the pullback co-orientation is unchanged.

	Now suppose that $e_0 \colon V \to M \times R^{N_0}$ and $e_1 \colon V \to M \times R^{N_1}$ are any two embeddings over $f$.
	By the preceding paragraph, by adding Euclidean factors we can assume $N_0 = N_1 = N$ for some sufficiently large $N$ without changing the pullback co-orientations associated to $e_0$ and $e_1$.
	Let $\pi \colon M \times \R^N \to M$ be the projection to $M$.
	As $\pi e_0 = \pi e_1$ by assumption, the maps $e_0$ and $e_1$ are homotopic over $f$, say by linear homotopies in the Euclidean fibers.
	Let $H \colon V \times I \to M \times \R^N$ be the chosen homotopy.
	Next, by the same argument by which embeddings such as $e_0$ and $e_1$ exist (see the proof of \cref{L: Quillen}), there is an embedding $\td H \colon V \times I \into M \times \R^N \times \R^Q$ for some $Q \geq 0$ so that if $\td \pi \colon M \times \R^N \times \R^Q \to M \times \R^N$ is the projection then $\td \pi \td H = H$.
	If we let $\td e_0, \td e_1$ denote respectively $\td H(-,0), \td H(-,1) \colon V \to M \times \R^N \times \R^Q$, then $\td \pi \td e_0 = e_0$, $\td \pi \td e_1 = e_1$, and $\td H$ is a homotopy from $\td e_0$ to $\td e_1$.
	If then $(e_0,0) \colon V \to M \times \R^N \times \R^Q$ denotes the map $x \mapsto (e_0(x),0)$, then there is also a homotopy from $\td e_0$ to $(e_0,0)$; in fact as $e_0$ is an embedding and $\td \pi \td e_0 = e_0$, we can let these homotopies be linear in the $\R^Q$ factor and constant in the other factors and this homotopy will be an embedding of $V \times I$ into $M\times \R^N \times \R^Q$.
	We can define $\td e_1$, $(e_1,0)$, and an embedded homotopy between them similarly.

	So we have a sequence of three embedded homotopies, say $F_0$, $\td H$, and $F_1$, respectively from $(e_0,0)$ to $\td e_0$, from $\td e_0$ to $\td e_1$, and from $\td e_1$ to $(e_1,0)$.
	Additionally, as for $\td H$, we have $\pi \td \pi F_j(x,t) = f(x)$, so each of the three homotopies is constant in $I$ when projected down to $M$, and in particular each of $(e_0,0)$, $(e_1,0)$, $\td e_0$, and $\td e_1$ is an embedding $V \into M \times \R^{N+Q}$ over $f \colon V \to M$.
	We know from above that the pullback co-orientation obtained from using $(e_0,0)$ and $(e_1,0)$ agree with those obtained from $e_0$ and $e_1$.
	So it suffices to use these homotopies to show successively that $(e_0,0)$, $\td e_0$, $\td e_1$, and $(e_1,1)$ all provide the same pullback co-orientation of $f^*$. This is the content of the following lemma.
\end{proof}

\begin{lemma}\label{L: homotopy pullback independence}
	Suppose that $f \colon V \to M$ and $g \colon W \to M$ are transverse with $f$ co-oriented.
	Let $\pi \colon M \times \R^N \to M$ be the projection, and let $h \colon V \times I \to M \times \R^N$ be an embedding such that $\pi h(x,t) = f(x)$.
	Say that $h$ is a homotopy between the embedding $h_0$ and $h_1$.
	Then the co-orientations of $V \times_M W$ obtained from the above pullback co-orientation construction, using $h_0$ and $h_1$ as the embeddings of $V$, are the same.
\end{lemma}

\begin{proof}
	The composition $\pi h \colon V \times I \to M$ is the constant homotopy with $\pi h(x,t) = f(x)$, so $\pi h$ is also transverse to $g$, and
	\begin{align*}
		(V \times I) \times_M W
		&= \{ (x,t,y)\in V \times I \times M \mid \pi h(x,t) = g(y) \} \\
		&= \{ (x,t,y)\in V \times I \times M \mid f(x) = g(y) \} \\
		&= (V \times_M W) \times I.
	\end{align*}
	So, within the setup of the lemma, we have $V \times I$ embedded in $M \times \R^N$ and $(V \times_M W) \times I$ embedded in $W \times \R^N$ with $g \times \id_{\R^N}$ mapping the latter to the former.
	Let us fix through the proof a particular pair $(x,y)$ with $f(x) = g(y)$.
	Then identifying the spaces with their embeddings, the map $g \times \id$ takes $(x,y,t) \subset (V \times_M W) \times I$ to $(x,t) \subset V$; in fact, as $\pi h$ is a constant homotopy, these correspond to the same path in the fiber $\R^N$.
	To simplify notation, we write $x_0 = (x,0) \in V_0= V \times 0$ and $x_1 = (x,1) \in V_1 = V \times 1$ and identify these points and spaces also with their images under $h$.

	Suppose $\beta_V$ is a local orientation at $x$ and $\beta_M$ is a local orientation at $f(x)$ such that $(\beta_V,\beta_M)$ is the given co-orientation of $f$.
	We can also think of these as co-orientations of the restrictions of $\pi h$ to each $V \times t \subset V \times I$.
	We co-orient the composite $\pi h \colon V \times I \to M$ with the co-orientation $(\beta_V \wedge \beta_I, \beta_M)$ at each $(x,t)$, with $\beta_I$ corresponding to the standard positive orientation of the interval.
	So then if we give each inclusion $V \to V \times t \subset V \times I$ the normal co-orientation $(\beta_V, \beta_V \wedge \beta_I)$ at $x$, the composition of this normal co-orientation with the co-orientation of $\pi h$ is the co-orientation of $f$.

	Now, considering $V \times I$ as embedded via $h$ in $M \times I$, we have the Quillen orientation $\beta_{\nu(V \times I)}$ of the normal bundle $\nu(V \times I)$.
	If we let $\beta_{V \times I} = \beta_V \wedge \beta_I$ at some $(x,t)$, then by definition the Quillen orientation of the normal bundle is the one such that
	\[
	(\beta_V \wedge \beta_I, \beta_V \wedge \beta_I \wedge \beta_{\nu(V \times I)}) * (\beta_M\wedge \beta_E,\beta_M) =
	(\beta_V\wedge \beta_I,\beta_M),
	\]
	i.e.\ such that $\beta_V \wedge \beta_I \wedge \beta_{\nu(V \times I)} = \beta_M \wedge \beta_E$.
	Similarly, if $\nu V_0$ is the normal bundle to $V_0$ at $x_0$, then its Quillen orientation is such that
	\[
	(\beta_V, \beta_V \wedge \beta_{\nu V_0})*(\beta_M \wedge \beta_E,\beta_M) =
	(\beta_V, \beta_M),
	\]
	i.e.\  $\beta_V \wedge \beta_{\nu V_0} = \beta_M \wedge \beta_E$.
	But on $V_0$, the normal bundle $\nu V_0$ is just the sum of $\nu(V \times I)$ and a line bundle tangent to $V \times I$ in the $I$ direction, which we write as $\nu V_0 = \nu(V \times I) \oplus TI$.
	And since the above computations imply $\beta_V \wedge \beta_I \wedge \beta_{\nu(V \times I)} = \beta_V \wedge \beta_{\nu V_0}$ at $x_0$, the relation among the orientations is that $\beta_I \wedge \beta_{\nu(V \times I)} = \beta_{\nu V_0}$.
	By the same argument, we have the equivalent relation at $x_1 \in V_1$.
	The point of all this is that we have now related the orientations of the Quillen normal bundles at $x_0 \in V_0$ and $x_1 \in V_1$ to the orientations there of the single oriented bundle $\nu(V \times I)$.

	Now, following the recipe for the pullback co-orientation of $(V \times I) \times_M W \cong (V \times_M W) \times I$, we pull $V \times I \subset M \times \R^N$ back by $g \times \id$, and the oriented normal bundles of $(V \times_M W) \times I$, $V_0 \times_M W$, and $V_1 \times_M W$ will be the pullbacks of the normal bundles of $V \times I$, $V_0$, and $V_1$.
	Again, we abuse notation and use the same notations for the pulled back bundles and their orientations.
	As we have noted that the path $(x,y) \times I \in (V \times_M W) \times I$ maps to the path $x \times I \in V \times I$, preserving the orientation of $I$, the identification $\nu V_0 = \nu(V \times I) \oplus TI$ continues to hold under the pullback, and similarly at the other end of the homotopy.

	At $(x_0,y) \in (V \times_M W) \times I$, let us now choose a local orientation of the form $\beta_P \wedge \beta_I$ with $\beta_P$ a local orientation of $V \times_M W$.
	By definition, the pullback co-orientation of the projection $(V \times I) \times_M W \to W$ at this point will be $(\beta_P \wedge \beta_I, \beta_P \wedge \beta_I \wedge \beta_{\nu(V \times I)})* (\beta_W \wedge \beta_E,\beta_W)$.
	For convenience, let use suppose we choose $\beta_W$ so that $\beta_P \wedge \beta_I \wedge \beta_{\nu(V \times I)} = \beta_W \wedge \beta_E$, in which case the co-orientation of $(V \times I) \times_M W \to W$ is $(\beta_P \wedge \beta_I, \beta_W)$.
	Similarly, continuing to use the same $\beta_P$ for $V \times_M W$, the co-orientation at $(x_0,y)$ of $V_0 \times_M W \to W$ will be $(\beta_P,\beta_P \wedge \beta_{\nu V_0})* (\beta_W \wedge \beta_E,\beta_W)$.
	As the relationship $\beta_I \wedge \beta_{\nu(V \times I)} = \beta_{\nu V_0}$ is maintained under the pullback, this latter co-orientation is
	$(\beta_P,\beta_P \wedge \beta_I \wedge \beta_{\nu(V \times I)}) * (\beta_W \wedge \beta_E, \beta_W)$, which is then $(\beta_P,\beta_W)$ by our preceding assumption that $\beta_P \wedge \beta_I \wedge \beta_{\nu(V \times I)} = \beta_W \wedge \beta_E$.
	So the pullback co-orientation of $V_0 \times_M W \to W$ is just $(\beta_P, \beta_W)$.
	But now the computations concerning $V_1$ are equivalent.
	Furthermore, as the pullback map $(V \times_M W) \times I \to W$ is also a constant homotopy, the choice of $\beta_W$ so that $\beta_P \wedge \beta_I \wedge \beta_{\nu(V \times I)} = \beta_W \wedge \beta_E$ will be the same: as we transport the orientations $\beta_P$, $\beta_I$, and $\beta_{\nu(V \times I)}$ along $(x,y) \times I$ in $(V \times_M W) \times I$, the corresponding $\beta_W$ will stay constant.
	In particular, the entire equality $\beta_P \wedge \beta_I \wedge \beta_{\nu(V \times I)} = \beta_W \wedge \beta_E$ transports along the image of $(x,y) \times I$ in $(V \times_M W) \times I$, so the co-orientation of $V_1 \times_M W$ at $(x_1,y)$ is also $(\beta_P, \beta_W)$.
\end{proof}

\begin{comment}
	%By \cref{L: co-orientable homotopies,D: homotopy co-orientation}, we can use the embeddings $F_j$ to %co-orient each of our constant homotopies.
	Now, using $F_0$, $\td H$, and $F_1$ respectively in place of $e$ in \cref{D: pullback coorient}, we obtain three co-orientations of the pullbacks $(V \times I) \times_M W \to W$ with the map $V\times I \to M$ being given by $(x,t) \mapsto f(x)$ in all three cases.
	Note that such a map is certainly transverse to $g \colon W \to M$ if $f$ is.
	For specificity, let us focus on the homotopy $F_0$. As we know that the restrictions of $F_0$ to $V \times 0$ and $V \times 1$ are co-orientable, so is $F_0$ by \cref{L: co-orientable homotopies}. If we choose any co-orientation for $F_0$, then by

	We will see below in \cref{leibniz}, whose proof is independent of this one, that when accounting for co-orientations, pullback co-orientations satisfy a Leibniz rule of the following form, again allowing spaces to stand also for their maps:
	$$\bd (V \times_M W) = (\bd V) \times_M W \bigsqcup (-1)^{m-v} V \times_M (\bd W).$$
	Applying this in our current setting, each of our three homotopies will include boundary components of the form $\left(\bd (V\times I) \right)\times_M W$, which by   and, in particular, two of the signed boundary components of each co-oriented $(V \times I) \times_M W \to W$ will be $(V \times \{0\}) \times_M W \to W$ and $(V \times \{1\}) \times_M W \to W$, occurring with opposite signs.
	In other words, with appropriate choices on the co-orientations of the homotopies, by \cref{D: co-oriented homotopy}, we obtain three sequential co-oriented (constant) homotopies from $f$ to itself.
	It now follows by applying \cref{R: stationary homotopy} sequentially that all four copies of $f$ must have the same co-orientation.
	In particular, this is the case for the co-orientations of $f$ obtained from the embeddings $e_0$ and $e_1$.
\end{comment}

\begin{remark}\label{R: local pullback co-orientations}
	The pullback co-orientation is determined locally in the sense that if $U$ is an open subset of $M$ then the pullback co-orientation of $f^{-1}(U) \times_U g^{-1}(U) \to g^{-1}(U)$ will just be the restriction of the pullback co-orientation of $V \times_M W \to W$.
	This is clear from the construction if we co-orient the local pullback using the Quillen co-orientation of $f^{-1}(U) \to U$ obtained from $f^{-1}(U) \xhookrightarrow{e|_{f^{-1}(U)}} U \times \R^N \to U$, the restriction of the Quillen co-orientation for $f$  obtained using the map $V\xhookrightarrow{e}M \times \R^N \times M$.
	But \cref{L: pullback co well defined} says that we are free to make such a choice.
\end{remark}

\begin{remark}\label{R: what products exist}
	We have just shown that, after choosing conventions, the fiber product of two transverse co-oriented maps is co-oriented, and this will eventually lead us to the cup product of geometric cochains.
	Analogously, if $f \colon V \to M$ is co-oriented and $W$ is oriented, then the pullback co-orientation $f^* \colon P = V \times_M W \to W$ provides a way to orient $P$, namely if $\beta_W$ is the given globally-defined orientation of $W$ then we can choose $\beta_P$ so that $(\beta_P, \beta_W)$ is the co-orientation of $f^*$ (this is just the induced orientation discussed in \cref{S: co-orientations}).
	This observation will be utilized below in our construction of the cap product.
	However, somewhat surprisingly, the fiber product of two maps with oriented domains cannot necessarily be oriented, and so there is in general no product of geometric chains and hence, in general, no homology product.
	Such oriented fiber products can be formed if the the codomain $M$ is oriented, as in this case there is an equivalence between orientations of domains and co-orientations of maps.
	But this is not always possible when $M$ is not orientable.
	For example, we recall that the intersection of two orientable $\R P^3$s in the non-orientable $\R P^4$ can be a non-orientable $\R P^2$.
\end{remark}

\subsubsection{Functoriality of pullbacks}

The co-oriented pullback construction is functorial in the following sense.

\begin{proposition}\label{P: pullback functoriality}
	Suppose $f \colon V \to M$ is a co-oriented map from a manifold with corners to a manifold without boundary.
	Then the pullback of $f$ by the identity $\id_M \colon M \to M$ is (diffeomorphic to) $f \colon V \to M$ with the same co-orientation.

	Suppose further that $X$ is a manifold with corners, that $W$ is a manifold without boundary, that $g \colon W \to M$ is transverse to $f$ and that $h \colon X \to W$ is transverse to $V \times_M W \to W$ (or, equivalently by \cref{L: transverse to pullback}, that $gh$ is transverse to $f$).
	Then $(gh)^*V \cong h^*g^*V$ as co-oriented manifolds over $X$.
\end{proposition}

\begin{proof}
	By \cref{pullback,P: interior co-orientation} it suffices to assume that all manifolds are without boundary.

	We first note that there is a diffeomorphism between $V$ and $V \times_M M = \{(v,x) \in V \times M \mid f(v) = x\}$ given by $v \mapsto (v,f(v))$ with inverse $(v,x) \mapsto v$.
	Then, given a compatible Quillen co-orientation of $f$, we can form the pullback diagram as
	\[
	\begin{tikzcd}[column sep=large]
		V \arrow[r, "\id_V"] \arrow[d,"e"] & V \arrow[d,"e"] \\
		M \times \R^N \arrow[r, "\id_{M \times \R^N}"] \arrow[d,"\pi_M"] & M \times \R^N \arrow[d,"\pi_M"] \\
		M \arrow[r, "\id_M"] & M,
	\end{tikzcd}
	\]
	and the conclusion is evident.

	For the second claim, there is a diffeomorphism between $V \times_M X = \{(v,x) \in V \times X \mid f(v) = g(h(x))\}$ and $(V \times_M W) \times_W X = \{((v,w),x) \in (V \times_M W) \times X \mid w = h(x)\}$ given by $(v,x) \mapsto (v,h(x),x)$ and $((v,w),x) \mapsto (v,x)$.
	To see that the last map is well defined notice that $f(v) = g(h(x))$ as $h(x) = w$, and $f(v) = g(w)$ from the assumption $(v,w) \in V \times_M W$.
	Alternatively, these two pullbacks must be diffeomorphic by general category theory, as our pullbacks are pullbacks in the category of manifolds with corners by \cite[Section 6]{Joy12}.

	Compatibility of the co-orientations now follows by considering the following diagram.
	We may assume $V$ and $W$ are manifolds without boundary by \cref{P: interior co-orientation}, and then the map labeled $e^*$ is an embedding, as recalled in the proof of \cref{L: co-orientable pullback}.
	We then note that it is equivalent to pull back the normal bundle $\nu V$ to $X \times \R^N$ either in two steps or all at once.
	\[
	\begin{tikzcd}[column sep=large]
		(V \times_M W) \times_W X \arrow[r, "\pi_{V \times_M W}"] \arrow[d] & V \times_M W \arrow[r, "\pi_V"] \arrow[d,"e^*"] & V \arrow[d, "e"] \\
		X \times \R^N \arrow[r, "h \times \id"] \arrow[d] & W \times \R^N \arrow[r, "g \times \id"] \arrow[d] & M \times \R^N \arrow[d, "\pi_M"] \\
		X \arrow[r, "h"] & W \arrow[r, "g"] & M.
	\end{tikzcd}
	\]
\end{proof}

\subsubsection{Fiber products of immersions}\label{S: co-or product immersion}

Pullbacks have particularly nice descriptions when one or both of the maps are embeddings or immersions.
In addition, these special cases are good for building intuition about the more general situation.

\begin{example}\label{E: V embedded}
	When $f \colon V \to M$ is a co-oriented embedding, the pullback co-orientation is particularly easy to describe. Again, for co-orientation purposes, we can restrict to considering the case where $V$ and $W$ are manifolds without boundary.

	We know from \cref{S: normal orientation} that in this case a co-orientation is equivalent to an orientation $\beta_{\nu V}$ of the normal bundle to $V$ in $M$.
	Then, as $f$ is already an embedding, we can take $N = 0$ in \cref{D: pullback coorient}.
	So the pullback $V \times_M W$ is just the submanifold $g^{-1}(V) \subset W$, co-oriented by $(\beta_P,\beta_W)$, where $\beta_P \wedge \beta_{\nu V} = \beta_W$, the $\nu V$ here being the pullback of the normal bundle to $g^{-1}(V)$ in $W$.
	In other words, the co-orientation of the pullback is just the normal co-orientation corresponding to the pulled back orientation of $\nu V$.

	The case where $g$ is an embedding instead also has a nice description but requires some more technology.
	We will discuss that case below in \cref{E: embedded}.
\end{example}

In the key example when both $f \colon V \to M$ and $g \colon W \to M$ are immersions, we know by \cref{L: fiber product of embeddings,S: normal orientation} that the co-orientations correspond locally to orientations of the normal bundles $\nu V$ and $\nu W$ and the fiber product $V \times_M W \to M$ corresponds locally to the intersection of the images of $V$ and $W$.
In this case our fiber product co-orientation of $V \times_M W \subset M$ is easily determined in terms of the orientations of $\nu V$ and $\nu W$. 
Again, \cref{pullback,P: interior co-orientation} allow us to focus on the case where $V$ and $W$ lack boundaries.

\begin{proposition}\label{P: normal pullback}
	Let $f \colon V \to M$ and $g \colon W \to M$ be transverse co-oriented immersions from manifolds without boundary to a manifold without boundary.
	Let $\nu V$ and $\nu W$ denote the respective normal bundles.
	Choose local Quillen orientations $\beta_{\nu V}$ and $\beta_{\nu W}$ so that the normal co-orientations $(\beta_V, \beta_V \wedge \beta_{\nu V})$ and $(\beta_W, \beta_W \wedge \beta_{\nu W})$ agree with the given co-orientations of $f$ and $g$.
	Then, decomposing the normal bundle of the fiber product $P = V \times_M W \to M$ at any point as $\nu V \oplus \nu W$ and giving it the orientation $\beta_{\nu V} \wedge \beta_{\nu W}$, the fiber product co-orientation agrees with the normal co-orientation, i.e.\
	$$\omega_{f \times_M g} = (\beta_P,\beta_P \wedge \beta_{\nu V} \wedge \beta_{\nu W}).$$
\end{proposition}

That is, if one orients the normal bundle of the intersection by following an oriented basis of the normal bundle of $V$ with an oriented basis of the normal bundle of $W$, the associated normal co-orientation is the fiber product co-orientation.

\begin{proof}
	It suffices to demonstrate this property in the neighborhood of any intersection point, so we may assume that $f$ and $g$ are embeddings of manifolds without corners and consider $x \in V$, $y \in W$ with $f(x) = g(y) = z \in M$.
	Locally, for our Quillen co-orientation of $f$ we can apply the definition of the pullback co-orientation with $N = 0$ and the embedding $e \colon V \into M \times \R^N$ being simply $f$ itself.
	As $N = 0$, in this case $\nu V$ is itself the oriented normal bundle of $e(V) = f(V)$ in $M \times \R^N = M$.
	Pulling back via $g$ to $W$, we obtain the oriented pullback of $\nu V$ (which we also call $\nu V$) as the normal bundle of $P = g^{-1}(V)$ in $W$.
	By definition, the co-orientation of $P \to W$ is then the composition of $(\beta_P,\beta_P \wedge \beta_{\nu V})$ with the standard co-orientation of the projection $W \times \R^N$ to $W$, which in this case is the identity.
	The co-orientation of the fiber product is thus the composition of $(\beta_P,\beta_P \wedge \beta_{\nu V})$ with the co-orientation $(\beta_W, \beta_W \wedge \beta_{\nu W})$ of $g$.
	But this last co-orientation is independent of the choice of $\beta_W$, so we can take $\beta_W = \beta_P \wedge \beta_{\nu V}$.
	Thus we see that the fiber product co-orientation of $P \to M$ is $(\beta_P, \beta_P \wedge \beta_{\nu V} \wedge \beta_{\nu W})$, as desired.
\end{proof}

\subsubsection{The Leibniz rule}

We now verify the Leibniz rule.
We recall from \cref{D: boundary co-orientation} that if $g \colon W \to M$ is co-oriented, then the boundary co-orientation of the composite $\bd W \xr{i_{\bd W}} W \xr{g} M$ is the composite of the boundary co-orientation $(\beta_{\bd W}, \beta_{\bd W} \wedge \beta_{\nu \bd W})$, with $\beta_{\nu \bd W}$ corresponding to the inward pointing normal vector, and the co-orientation of $g$.
Recall also that we often abuse notation by letting $W$ stand for the co-oriented map $g$, and in this case we write $\bd W$ to stand for the co-oriented composite.
We also write $-W$ for $g$ with the opposite co-orientation.
This notation makes the statement of the Leibniz rule, stated just below, comprehensible.
Establishing the rule directly for immersions, for which we can use the normal co-orientations, is a quick exercise; the general case requires more care.


\begin{proposition}[Leibniz rule]\label{leibniz}
	Let $f \colon V \to M$ and $g \colon W \to M$ be transverse maps from manifolds with corners to a manifold without boundary, and suppose $f$ co-oriented.
	Let $V \times_M W \to W$ be the co-oriented pullback.
	Then
	$$\bd (V \times_M W) = \left((\bd V) \times_M W\right) \bigsqcup (-1)^{m-v} \left(V \times_M (\bd W)\right),$$
	interpreting each of these pullback spaces as representing its co-oriented map to $W$; see Notation \ref{N: implicit notation}. Here we interpret $V \times \bd W \to W$ as the composition of the co-oriented pullback $V \times \bd W \to \bd W$ with the boundary immersion $\bd W \to W$ with its boundary co-orientation. 

	If $g$ is also co-oriented then this formula also holds as fiber products mapping to $M$.
\end{proposition}


We first need a lemma.

\begin{lemma}\label{L: pullback boundary normal}
	Let $V$ be a smooth manifold with boundary embedded in a manifold without boundary $M$, and suppose $g \colon W \to M$ is a map from a manifold without boundary and transverse to $V$.
	In this case we know by \cref{L: immersion pullback} that $V \times_M W \to W$ is a local embedding onto $g^{-1}(V)$ and the pullback of the normal bundle $\nu V$ of $V$ in $M$ is the normal bundle of $g^{-1}(V)$ in $W$, and similarly replacing $\bd V$ with $V$.
	Let $w\in \bd (V \times_M W) = (\bd V) \times_M W$, which we identify as a subset of $W$.
	Then there is a vector $b \in T_w(V \times_M W)$, not contained in $T_w((\bd V) \times_M W)$, such that $Dg(b) \in T_{g(w)}V$ but $Dg(b) \notin T_{g(w)}(\bd V)$. Furthermore, both $b$ and $Dg(b)$ can be taken to be inward pointing, toward $V \times_M W$ and $V$, respectively.
	So, roughly speaking, there is a correspondence between the normal direction to $(\bd V) \times_M W$ in $V \times_M W$ and the normal direction to $\bd V$ in $V$ (up to the usual ambiguities in the choices of splittings for normal bundles).
\end{lemma}
\begin{proof}
Let $v = g(w)$, and let $a \in T_v V \subset T_v M$ such that $a \notin T_v (\bd V)$.
As $g$ is transverse to $\bd V$, there are vectors $b \in T_w W$ and $c \in T_v(\bd V)$ such that $a= Dg(b) + c$.
Now rewriting as $Dg(b) = a - c$, the righthand side is contained in $T_v(V)$, but not in $T_v(\bd V)$ or else $a$ would be in $T_v(\bd V)$.
As the tangent space of the pullback at $w$ is the pullback of the tangent spaces of $T_v V$ and $T_w W$ mapping to $T_v M$ by \cref{L: tangent of pullbacks}, we see that the pair $(a-c, b) \in T_v V \times_{T_v M} T_w W$ is in the tangent space of the pullback, and as the derivative of the pullback map $\pi_W \colon V \times_M W \to W$ is just projection in the $W$ co-ordinate, we see that $b$ must be a tangent vector to at $w$ mapping to $Dg(b)$ via $Dg$ as desired.
Note that $b$ cannot be in $T_w((\bd V) \times_M W)$ as such a vector would map to $T_v (\bd V)$.
It is clear that $Dg$ must take inward pointing vectors to inward pointing vectors.
\end{proof}

The lemma will let us identify a normal direction to $\bd V$ in $V$ with a normal direction to $(\bd V) \times_M W$ in $V \times_M W$ in the following argument.

\begin{proof}[Proof of \cref{leibniz}]
	The statement at the level of underlying manifolds with corners is \cite[Proposition 6.7]{Joy12}, so we focus on co-orientations.
	The second statement follows from the first by composing each map with the co-oriented map $g \colon W \to M$ and taking the composite co-orientations.
	We will write $\bd P$ when considering the boundary of the pullback map $P = V \times_M W \to W$ with its boundary co-orientation, and we write $(\bd V) \times_M W$ or $V \times_M (\bd W)$ when considering the maps from these boundary components with their pullback co-orientations, and in general we speak respectively of the ``boundary''  and ``pullback'' co-orientations.
	However, in both cases we write $\beta_{\bd P}$ when speaking of local orientations to simplify the notation.
	In the following arguments, it suffices to consider points in the interiors of $\bd V$ or $\bd W$, as knowing a co-orientation at one such point of each component determines it globally; in other words, we can avoid corners.
	So working locally and separating the two cases, we can assume for each case that only one of $V$ or $W$ has a boundary.

	We first consider the boundary co-orientation.
	By \cref{D: pullback coorient}, at a point of $P = V \times_M W$, the co-orientation $\omega_{f^*}$ of $P \to W$ is $(\beta_P,\beta_W)$ if and only if $\beta_P$ and $\beta_W$ are chosen so that $\beta_P \wedge \beta_{\nu V} = \beta_W \wedge \beta_E$, where $\nu V$ is the pullback to $P$ of the Quillen-oriented normal bundle of $e(V)$ in $M \times \R^N$ for some appropriate embedding $e$ that we fix throughout the following.
	The normal bundle $\nu V$ is given its Quillen orientation corresponding to the co-orientation of $f$.
	By \cref{D: boundary co-orientation}, if $\nu (\bd P)$ is an inward pointing normal of $\bd P$ in $P$ then the boundary co-orientation $\bd P \to W$ is obtained by composing the normal co-orientation of the boundary immersion $(\beta_{\bd P},\beta_{\bd P} \wedge \beta_{\nu (\bd P})$ with $\omega_{f^*}$.
	If we choose $\beta_W$ and $\beta_P$ to satisfy the condition above that  $\beta_P \wedge \beta_{\nu V} = \beta_W \wedge \beta_E$ and then choose $\beta_{\bd P}$ so that $\beta_P = \beta_{\bd P} \wedge \beta_{\nu(\bd P)}$, then we have that $\omega_{\bd P \to W}$ is $(\beta_{\bd P},\beta_W)$.
	We assume in what follows that we have made such choices.

	Now, consider a point in the interior of $(\bd V) \times_M W \subset V \times_M W = P$.
	In the construction of the pullback co-orientation for $(\bd V) \times_M W$, we can take $e \colon \bd V \to M \times \R^N$ to be the restriction to $\bd V$ of our fixed embedding $e$.
	As there are two objects we could reasonable notate $\nu (\bd V)$, throughout the proof we will use $\nu (\bd V)$ for the normal bundle of $\bd V$ in $V$, and we will write $\nu^s(\bd V)$ ($s$ for stable) for the normal bundle of $e(\bd V)$ in $M \times \R^N$.
	Analogously, $\nu(\bd W)$ and $\nu(\bd P)$ will be the normal bundles as submanifolds of $W$ and $P$, respectively.
	Note that $\nu^s(\bd V) \cong \nu(\bd V) \oplus \nu V$.
	Furthermore, by employing \cref{L: pullback boundary normal}, when we pull back $\nu^s(\bd V)$ to the normal bundle of $(\bd V) \times_M W$ in $W \times \R^N$, we maintain this decomposition, identifying the pullback of $\nu V$ with the normal bundle to $V \times_M W$, as usual, and the pullback of $\nu(\bd V)$ with $\nu (\bd P)$, the normal to $\bd P$ in $P$ at our point.
	Employing our standard abuses of notation, we thus write $\beta_{\nu(\bd P)} =  \beta_{\nu(\bd V)}$, identifying the inward pointing normal orientations.

	Next, recall the Quillen orientation $\beta_{\nu V}$ was chosen so that if $(\beta_V, \beta_M)$ is the co-orientation of $V$ (at an appropriate point) then $\beta_V \wedge \beta_{\nu V} = \beta_{M} \wedge \beta_E$.
	Let us fix such $\beta_V$ and $\beta_M$.
	Then if we choose $\beta_{\bd V}$ so that $\beta_V = \beta_{\bd V} \wedge \beta_{\nu(\bd V)}$ then the boundary co-orientation of $\bd V \to M$ will be $(\beta_{\bd V}, \beta_{\bd V} \wedge \beta_{\nu(\bd V)}) * (\beta_V, \beta_M) = (\beta_{\bd V},\beta_M)$, so we can then perform the pullback co-orientation construction of \cref{D: pullback coorient} using this co-orientation of $\bd V$.
	Note that as $\beta_{\nu V}$ is chosen so that $\beta_V \wedge \beta_{\nu V} = \beta_M \wedge \beta_E$, we will have also $\beta_M \wedge \beta_E = \beta_{\bd V} \wedge \beta_{\nu(\bd V)} \wedge \beta_{\nu V}$ and so the corresponding Quillen orientation of $\nu^s(\bd V)$ is $\beta_{\nu^s(\bd V)} = \beta_{\nu(\bd V)} \wedge \beta_{\nu V}$.

	So now applying \cref{D: pullback coorient}, the pullback co-orientation of $(\bd V) \times_M W \to W$ is $(\beta_{\bd P},\beta_W)$ (for our previously chosen $\beta_{\bd P}, \beta_W$) if and only if $\beta_{\bd P} \wedge \beta_{\nu^s(\bd V)} = \beta_W \wedge \beta_E$.
	But $\beta_{\bd P} \wedge \beta_{\nu^s(\bd V)} = \beta_{\bd P} \wedge \beta_{\nu(\bd V)} \wedge \beta_{\nu V}$.
	So the pullback co-orientation is $(\beta_{\bd P},\beta_W)$ if and only if $\beta_{\bd P} \wedge \beta_{\nu(\bd V)} \wedge \beta_{\nu V} = \beta_W \wedge \beta_E$.
	But we have previously identified $\beta_{\nu(\bd V)}$ with $\beta_{\nu(\bd P)}$, and $\beta_{\bd P} \wedge \beta_{\nu(\bd P)} = \beta_P$ by assumption.
	So this condition reduces to $\beta_P \wedge \beta_{\nu V} = \beta_W \wedge \beta_E$, which also holds by previous assumption.
	So the boundar and pullback co-orientations agree at points of $(\bd V) \times_M W$.

	Next, consider a point $x$ in $V \times_M \bd W$.
	As our pullback $V \times_M W$ is embedded neatly in $W \times \R^N$ \cite[Proposition IV.1.4]{Kos93}, it is immediate that, at $x$, the normal bundle of $V \times_M \bd W = (g \times \id)|_{(\bd W) \times \R^N}^{-1}(V)$ in $W \times \R^N$ can be decomposed into the direct sum of the pullback of $\nu V$, which can be identified with a subbundle of $T_x((\bd W) \times \R^N)$, and a 1-dimensional summand that is normal to $(\bd W) \times \R^N$. By projection, we can identify this summand with a normal direction to $\bd W$ in $W$ and again write $\beta_{\nu(\bd P)}=\beta_{\nu(\bd W)}$ for the orientation determined by these inward pointing normals.


	Now, we continue to assume that $(\beta_P, \beta_W)$ is the co-orientation of $V \times_M W \to W$ (and so $\beta_P \wedge \beta_{\nu V} = \beta_W \wedge \beta_E$).
	We choose $\beta_M$ so that $(\beta_W,\beta_M)$ is the co-orientation of $g$ and $\beta_{\bd W}$ so that $(\beta_{\bd W},\beta_W)$ co-orients $i_{\bd W}$, which implies $\beta_W = \beta_{\bd W} \wedge \beta_{\nu(\bd W)}$.
	We also continue to choose $\beta_{\bd P}$ so that $\beta_P = \beta_{\bd P} \wedge \beta_{\nu(\bd P)} = \beta_{\bd P} \wedge \beta_{\nu(\bd W)}$ and hence the boundary co-orientation of $\bd P \to W$ is $(\beta_{\bd P}, \beta_W)$.

	Using these local orientations and applying \cref{D: pullback coorient} to $\bd g \colon \bd W \to M$, the co-orientation of the pullback $V \times_M \bd W \to \bd W$ is $(\beta_{\bd P},\beta_{\bd W})$ (and so the composite co-orientation to $W$ is $(\beta_{\bd P}, \beta_W)$) if and only if $\beta_{\bd P} \wedge \beta_{\nu V} = \beta_{\bd W} \wedge \beta_E$ as local orientations at the image of $x$ in $\bd W \times \R^N$.
	Considering $\bd W \times \R^N \subset W \times \R^N$, this condition holds if and only if $\beta_{\bd P} \wedge \beta_{\nu V} \wedge \beta_{\nu(\bd W)} = \beta_{\bd W} \wedge \beta_E \wedge \beta_{\nu(\bd W)}$ in $W \times \R^N$.
	But as the dimensions as bundles are $\dim(\nu(\bd W)) = 1$ and $\dim(\nu V) = m+N-v$,
	$$\beta_{\bd P} \wedge \beta_{\nu V} \wedge \beta_{\nu(\bd W)} = (-1)^{m+N-v}\beta_{\bd P} \wedge \beta_{\nu(\bd W)} \wedge \beta_{\nu V} = (-1)^{m+N-v}\beta_{\bd P} \wedge \beta_{\nu(\bd P)} \wedge \beta_{\nu V} = (-1)^{m+N-v}\beta_P \wedge \beta_{\nu V} ,$$
	and
	$$\beta_{\bd W} \wedge \beta_E \wedge \beta_{\nu(\bd W)} = (-1)^N\beta_{\bd W} \wedge \beta_{\nu(\bd W)} \wedge \beta_E = (-1)^N\beta_{W} \wedge \beta_E.$$
	So the pullback co-orientation of $V \times_M \bd W \to W$ is $(\beta_{\bd P},\beta_W)$ if and only if $\beta_P \wedge \beta_{\nu V} = (-1)^{m-v}\beta_{W} \wedge \beta_E$.
	But in defining the boundary co-orientation we assumed that $\beta_P \wedge \beta_{\nu V} = \beta_{W} \wedge \beta_E$, so two co-orientations agree or disagree according to the sign $(-1)^{m-v}$ as claimed.
\end{proof}


\begin{comment}
\subsubsection{Graded commutativity}

We demonstrate here graded commutativity of fiber product co-orientations.
We do so only for proper maps, but this will suffice for the purposes of geometric chains and cochains below.
The reader should recall \cref{R: precise commutativity} for a precise explanation of the statement of the proposition.

\begin{proposition}\label{P: graded comm}
	Suppose $f \colon V \to M$ and $g \colon W \to M$ are transverse proper co-oriented maps from manifolds with corners to a manifold without boundary.
	Then as co-oriented fiber products over $M$ we have $\omega_{g \times_M f} = (-1)^{(m-v)(m-w)} \omega_{f \times_M g}$, or, using Notation \ref{R: precise commutativity},
	$$V \times_M W = (-1)^{(m-v)(m-w)} W \times_M V.$$
\end{proposition}
\greg{Can we remore the proper assumption?}

Before proving the proposition, we put it to use in the following example.

\begin{example}\label{E: embedded}
	In \cref{E: V embedded}, we considered pullback co-orientations $V \times_M W \to W$ when $V \into M$ was embedded.
	In this example, we discuss the case where $W$ is embedded, assuming both maps are proper.

	Let $f \colon V \to M$ and $g \colon W \to M$ be transverse proper maps from manifolds with corners to a manifold without boundary.
	Suppose $f$ is co-oriented and $g \colon W \to M$ is an embedding, which we use throughout the example to identify $W$ with its image.
	Let $(x,y) \in V \times_M W$.
	For the remainder of the argument, we fix a local orientation $\beta_M$ at $f(x) \in M$, and we choose $\beta_V$ at $x$ so that $(\beta_V,\beta_M)$ is the co-orientation of $V$ at $x$.
	Furthermore, even though $g$ might not be co-oriented, let us choose a Euclidean neighborhood $U$ of $y$ in $W$ and an arbitrary co-orientation $(\beta_U,\beta_M)$ on the restriction of $g$ to $U$.
	This can be done as $U$ is contractible and so $g|_U$ is co-orientable.


	Although we are interested in $V \times_M U$, the definition of the pullback co-orientation makes it easier to work with $U \times_M V$ when $U$ is embedded; see \cref{E: V embedded}.
	As we have chosen a co-orientation for $U \into M$ and as $V \to M$ comes with a co-orientation, we can consider the fiber product $P_U = U \times_M V \to M$.
	As $U$ is embedded in $M$, we have $P_U = f^{-1}(U) \subset V$.
	If we choose an orientation $\beta_{\nu U}$ of the normal bundle to $U$ in $M$ at $y$ so that $\beta_U \wedge \beta_{\nu U} = \beta_M$, then by \cref{E: V embedded} and the definition of fiber product co-orientation, the map $P_U \xr{f} M$ is co-oriented by $(\beta_P,\beta_M)$, where $\beta_P$ is chosen so that $\beta_P \wedge \beta_{\nu U} = \beta_V$, as usual letting $\nu U$ here also stand for its pullback as a normal bundle of $P_U$ in $V$.
	We note that if we had chosen the opposite co-orientation for $U \into M$ then, as we have fixed $\beta_V$ and $\beta_M$, the result would be to replace our current $\beta_U$ with $-\beta_U$, which would result in also reversing the signs of $\beta_{\nu U}$ and $\beta_P$.
	In particular, the fiber product would have the opposite co-orientation.

	Next, we will apply \cref{P: graded comm}.
	Technically we need $g|_U$ to be proper, but as $W$ is embedded in $M$, we can achieve this by assuming that $U$ is the intersection of $W$ with some Euclidean neighborhood $\mc U$ of $y$ in $M$.
	Then we can replace the co-oriented embedding $U \into M$ with the proper co-oriented embedding $U \into \mc U$ and $f$ with its restriction to $f^{-1}(\mc U)$.
	As pullback co-orientations are determined locally, as noted in \cref{R: local pullback co-orientations}, this will suffice.
	For simplicity, though, we maintain our original notations.

	By \cref{P: graded comm}, we have $U \times_M V = (-1)^{(m-v)(m-w)}V \times_M U$ as co-oriented fiber products over $M$.
	By \cref{R: precise commutativity}, this means that the fiber product $V \times_M U \to M$ with its fiber product co-orientation corresponds to the fiber product $U \times_M V = f^{-1}(U) \xr{f} M$ with co-orientation at $(x,y)$ given by $(-1)^{(m-v)(m-w)}(\beta_P, \beta_M)$.
	But decomposing the fiber product as the pullback $V \times_M U \to U$ and the inclusion $U \into M$, we write $(-1)^{(m-v)(m-w)}(\beta_P, \beta_M)$ as the composite co-orientation
	$$(-1)^{(m-v)(m-w)}(\beta_P, \beta_M) = (-1)^{(m-v)(m-w)}(\beta_P, \beta_U)*(\beta_U, \beta_M).$$
	As $(\beta_U, \beta_M)$ is our chosen co-orientation for $g|_U$, the pullback co-orientation of $V \times_M U \to U$ must be $(-1)^{(m-v)(m-w)}(\beta_P, \beta_U)$.
	But we have already observed that if we had chosen the opposite co-orientation for $g|_U$, that would reverse the signs of both $\beta_U$ and $\beta_P$, so in either case we obtain the same pullback co-orientation for $V \times_M U \to U$.
	In other words, this description of the co-oriented pullback $V \times_M U \to U$ is independent of our choice of co-orientation for $g|_U$, and so it extends globally to give our pullback co-orientation of $V \times_M W \to W$.

	Summarizing then, as a space we have $P = V \times_M W = f^{-1}(W)$, and the pullback and fiber product are the first map and composite of $f^{-1}(W) \xr{f} W \into M$ (analogously to the case where $f$ was an embedding in \cref{E: V embedded}).
	Furthermore, fixing the co-orientation of $f$ as $(\beta_V, \beta_M)$, the co-orientation of the pullback $V \times_M W \to W$ is by $(-1)^{(m-v)(m-w)}(\beta_P, \beta_W)$, where, if we choose any local orientation $\beta_{\nu W}$ for the normal bundle of $W$ in $M$, then $\beta_P \wedge \beta_{\nu W} = \beta_V$ and $\beta_W \wedge \beta_{\nu W} = \beta_M$.

	It is a nice exercise to confirm that this agrees with the computation of \cref{P: normal pullback} when $V$ and $W$ are both embedded.
\end{example}

\cref{P: graded comm} will be proven through a sequence of lemmas.
One common theme is to first show that the statement will be true if it is true at any one point in each connected component of $P$.
As fiber product co-orientations are well defined globally, if they agree or disagree at any one point then they must agree or disagree on the entire connected component.
By \cref{R: precise commutativity} we can also generally identify $V \times_M W$ with $W \times_M V$ as spaces in what follows and focus only on how the two constructions specify co-orientations.

Again, by applying \cref{pullback,P: interior co-orientation}, we may typically focus on interior points of $V$ and $W$, though we cannot assume that the interiors of $V$ and $W$ map properly to $M$, so we will need to be careful at the point where properness comes in.

\begin{lemma}\label{L: im/im}
	If there exists $x \in S^0(V)$ and $y \in S^0(W)$ such that $f(x) = g(y)$ and $f$ and $g$ are respectively immersions at $x$ and $y$, then the theorem holds for the connected component containing $(x,y) \in V \times_M W$.
\end{lemma}

\begin{proof}
	As observed just above, it suffices to prove the desired identity at $(x,y)$.
	By \cref{P: normal pullback}, we have $\omega_{f \times_M g} = (\beta_P, \beta_P \wedge \beta_{\nu V} \wedge \beta_{\nu W})$, noting in the second term that $\beta_P$ is technically the image of the local orientation $\beta_P$ of $P$ now considered as a submanifold of $M$ via the embedding.
	Similarly, \cref{P: normal pullback} gives
	$\omega_{g \times_M f} = (\beta_P,\beta_P \wedge \beta_{\nu W} \wedge \beta_{\nu V})$.
	So the two fiber products differ by $(-1)^{\dim(\nu W)\dim(\nu V)} = (-1)^{(m-v)(m-w)}$ as required.
\end{proof}

\begin{lemma}\label{L: im/sub}
	If there exists $x \in S^0(V)$ and $y \in S^0(W)$ such that $f(x) = g(y)$, $f$ is a submersion at $x$, and $g$ is an immersion at $y$, then the theorem holds for the connected component containing $(x,y) \in V \times_M W$.
\end{lemma}

\begin{proof}
	By \cref{R: co-or restriction or switch} and the above observations, it suffices to consider a neighborhood of $x$ and a chart $U$ around $f(x)$ with respect to which $f$ agrees locally with the projection $U \times \R^N \to U$.
	For simplicity of notation, we can take all of $M$ to be this chart and replace $V$ and $W$ by $f^{-1}(U)$ and $g^{-1}(U)$, respectively.
	For the remainder of the proof, we assume we have done so and return to writing $W$, $M$, and $V = M \times \R^N$.
	We can choose the ordering of the coordinates in $\R^N$ so that $(\beta_M \wedge \beta_E,\beta_M)$ agrees with the co-orientation for $f$.

	We first consider the fiber product co-orientation for $V \times_M W$.
	In this case, we can let our usual embedding $e \colon V \into M \times \R^N$ be the diffeomorphism realizing $V$ as $M \times \R^N$.
	Then as we have written the co-orientation of $f$ as $(\beta_M \wedge \beta_E,\beta_M)$, the normal bundle $\nu V$ is $0$-dimensional and positively oriented.
	The pullback $(g \times \id)^{-1}(V)$ is then all of $W \times \R^N$, and if we also choose $\beta_W$ at $y$ so that $\omega_g = (\beta_W,\beta_M)$, then by \cref{D: pullback coorient} the fiber product co-orientation for $f \times_M g$ is $(\beta_P,\beta_M)$ if and only if $\beta_P = \beta_W \wedge \beta_E$.

	Now consider instead the fiber product co-orientation construction for $W \times_M V$.
	In this case we take $e = g \colon W \to M \times \R^0$.
	The map $f$ is still our projection $M \times \R^N \to M$, and so the pullback is again $f^{-1}(W) = W \times \R^N$, as expected.
	We identify this $W \times \R^N$ with the copy from the preceding paragraph and assign it the same local orientation $\beta_P$ at the corresponding points.
	Note that we can assume the identification to be the identity on the $\R^N$ factors.
	In fact, in both constructions we obtain an embedding of $W \times \R^N$ into $M \times \R^N$ that is the identity on the $\R^N$ factors, and the fiber product is the composition of this embedding with the projection to $M$.
	As usual we let $\nu W$ denote the normal bundle of $W$ in $M$ and its pullback normal bundle to $W \times \R^N$ in $M \times \R^N$.
	We continue to assume coordinates so that $f$ is co-oriented by $(\beta_M \wedge \beta_E, \beta_M)$.
	By \cref{D: pullback coorient} the fiber product co-orientation for $g \times_M f$ is $(\beta_P,\beta_M)$ if and only if $\beta_P \wedge \beta_{\nu W} = \beta_V$.
	But if we continue to let $\beta_P = \beta_W \wedge \beta_E$ and $\beta_V = \beta_M \wedge \beta_E$ as above, then we compute in $M \times \R^N$ that
	\begin{multline*}
		\beta_P \wedge \beta_{\nu W} =
		\beta_W \wedge \beta_E \wedge \beta_{\nu W} =
		(-1)^{N\dim(\nu W)} \beta_W \wedge \beta_{\nu W} \wedge \beta_E \\ =
		(-1)^{N\dim(\nu W)} \beta_M \wedge \beta_E =
		(-1)^{N\dim(\nu W)} \beta_V.
	\end{multline*}
	Here we have used that $\beta_M = \beta_W \wedge \beta_{\nu W}$ by the choice of Quillen local orientation for $\nu W$ in \cref{D: pullback coorient}, taking into account $N = 0$.
	As $\dim(\nu W) = m-w$ and $N = v-m$, the lemma is follows.
\end{proof}

\begin{lemma}\label{L: sub/sub}
	If there exist $x \in S^0(V)$ and $y \in S^0(W)$ such that $f(x) = g(y)$ and $f$ and $g$ are respective submersions at $x$ and $y$, then the theorem holds for the connected component containing $(x,y) \in V \times_M W$.
\end{lemma}

\begin{proof}
	As in the proof of the preceding lemma, we can restrict to a small neighborhood of $f(x) = g(y)$ and relabel to assume that $f$ is the projection $M \times \R^N \to M$ and $g$ is the projection $M \times \R^n \to M$.
	Let $\beta_E$ and $\beta_F$ be local orientations of $\R^N$ and $\R^n$ such that $\omega_f = (\beta_V,\beta_M) = (\beta_M \wedge \beta_E, \beta_M)$ and $\omega_g = (\beta_W,\beta_M) = (\beta_M \wedge \beta_F, \beta_M)$.
	To compute $\omega_{f \times_M g}$, we identify $V$ and $M \times \R^N$ as in the proof of the preceding lemma, so the Quillen normal bundle is trivial and positive.
	The pullback is $W \times \R^N \cong M \times \R^n \times \R^N$.
	And by \cref{D: pullback coorient}, taking $\beta_P = \beta_W \wedge \beta_E$, we have $\omega_{f \times_M g} = (\beta_W \wedge \beta_E, \beta_M) = (\beta_M \wedge \beta_F \wedge \beta_E, \beta_M)$.
	Analogously, $\omega_{g \times_M f} = (\beta_V \wedge \beta_F, \beta_M) = (\beta_M \wedge \beta_E \wedge \beta_F, \beta_M)$.
	The lemma follows as $N = v-m$ and $n = w-m$.
\end{proof}

\begin{corollary}\label{C: if full}
	If there exists $x \in S^0(V)$ and $y \in S^0(W)$ such that $f(x) = g(y)$ and if $f$ and $g$ are of full rank at $x$ and $y$ then \cref{P: graded comm} holds for the connected component containing $(x,y) \in V \times_M W$.
\end{corollary}

\begin{proof}
	If $f \colon V \to M$ is of maximal rank at $x \in V$, it is an immersion or submersion at that point, and similarly for $g$.
	Thus the corollary follows directly from \cref{L: im/im,L: im/sub,L: sub/sub}.
	Note that while the statement of \cref{L: im/sub} assumes $f$ is the submersion and $g$ the immersion, we obtain the opposite case by reversing the roles of $g$ and $f$ in the statement of \cref{P: graded comm}.
\end{proof}

BEGIN OLD COMMENT
	If they are both immersions, apply Lemma \ref{L: im/im}.
	If $f$ is a submersion, it is also a submersion in a neighborhood of $U$ and so it is a submersion on a neighborhood $U$ of $x$.
	Let $A$ be the intersection of $U$ with the interior of $V$.
	By the transversality of $f$ and $g$, if $g(y) = f(x)$, there must be points in a neighborhood of $B$ of $y$ in $W$ that map to $f(A)$, and as a map is a submersion or immersion on an open set, there is a $y'$ in the interior of $B$ that maps to $f(A)$.
	Taking $x'$ in $f^{-1}(y')$, if $g$ is an immersion we apply Lemma \ref{L: im/sub} to $x', y'$, and if $g$ is a submersion we apply Lemma \ref{L: sub/sub}.
	If $f$ is an immersion and $g$ a submersion, we reverse the roles in the argument.
END OLD COMMENT

We now show that for arbitrary proper transverse intersecting $f$ and $g$ there are always homotopies that maintain these properties while taking $f$ and $g$ each to maps of full rank at a given intersection point.
Note that this is where we use the properness assumption in \cref{P: graded comm}.

\begin{lemma}\label{L: make full}
	Let $f \colon V \to M$ and $g \colon W \to M$ be transverse proper maps from manifolds with corners to a manifold without boundary.
	Suppose $x \in S^0(V)$ and $y \in S^0(W)$ such that $f(x) = g(y)$.
	Then there is a smooth homotopy $F \colon V \times I \to M$ such that $F(-,0) = f$, $F(x,t) = f(x)$ for all $t \in I$, $F(-,t)$ is transverse to $g$ for all $t \in I$, and $D(F(-,1))$ has maximal rank at $x$.
\end{lemma}

\begin{proof}
	If $Df$ already has maximal rank at $x$ we can take $F(-,t) = f(-)$.
	So suppose $Df$ does not have maximal rank at $x$.
	We will construct a homotopy fixed outside a small neighborhood of $x$, so we work in local charts, identifying neighborhoods of $x$ and $f(x)$ with $\R^v$ and $\R^m$ so that $x$ and $f(x)$ are at the respective origins.
	By \cite[Sections 1.3-1.4]{GuPo74}, we may also choose the charts so that $D_xf(e_i) = e_i$ for $1 \leq i \leq k$ for some $k<\dim(V) = v$ and $D_xf(e_i) = 0$ for $i>k$.
	Let $\eta \colon \R^v \to \R$ be a smooth function that is $0$ outside a compact neighborhood of the origin and $1$ on a neighborhood of the origin.
	Let $z = (z_1,\ldots,z_v)$ be the coordinates of $\R^v$ and $(z_1, \ldots, z_m)$ the coordinates for $\R^m$.
	Using the charts to identify these Euclidean spaces with their corresponding subsets of $V$ and $M$, we define a homotopy $H \colon \R^v \times I \to \R^m$ by
	$$H(z_1,\ldots,z_v,t) = f(z) + t\eta(z)\left(\sum_{i = k+1}^{\min(v,m)} z_i\right).$$
	Then $H(z,t) = f(z)$ outside a compact neighborhood of the origin in $\R^v$, and so $H$ extends to a homotopy defined on all of $V \times I$.
	Furthermore, $H(0,t) = f(0)$, i.e.\ $H(x,t) = f(x)$.
	Also, for each fixed $t > 0$, we have that
	$(DH(-,t))_0e_i = e_i$ for $i \leq k$ and $(DH(-,t))_0e_i = te_i$ for $k<i \leq \min(v,m)$.
	Thus $DH(-,t)$ has full rank at $x$ for all $t>0$.
	Finally, as $f$ and $g$ are proper and transverse, there is an $\epsilon$ so that $H(-,t)$ and $g$ will be transverse for $0 \leq t \leq \epsilon$ by stability of transversality\footnote{See \cite[Theorem 1.6]{GuPo74} for the case where one manifold is compact and the other is embedded as a closed submanifold. A proof for the current situation is given within our proof of \cref{P: perturb transverse to map}, below.}.\greg{Note to self: Pull this out as a corollary of the proof to refer back to when I get there in proofreading.\label{stability pageref}}
	Now define $F(-,t) = H(-,\epsilon t)$.
\end{proof}

\begin{proof}[Proof of \cref{P: graded comm}]
	As noted above, it suffices to verify the claim at one point of each connected component of $P$.
	As $f$ and $g$ are transverse, if a component $P_0$ of $V \times_M W$ is nonempty, by \cref{pullback} we can find $x \in S^0(V)$ and $y \in S^0(W)$ so that $(x,y) \in P_0$.
	By \cref{L: make full}, we can first perform a homotopy $F \colon V \times I \to M$ of $f$ to $f' \colon V \to M$ and then a homotopy $G \colon W \times I \to M$ of $g$ to $g' \colon W \to M$ so that $f'$ and $g'$ have full rank at $x$ and $y$ respectively and the images of $x$ and $y$ are fixed throughout the homotopies.

	First consider $F$.
	As $f$ is co-oriented, $F$ is co-orientable by \cref{L: co-orientable homotopies}, and we co-orient it via \cref{D: homotopy co-orientation}.
	Then $(V \times I) \times_M W$ will have its fiber product co-orientation, and by \cref{P: product boundary} two of its boundary components will be (topologically) $f \times_M g \colon (V \times 0) \times_M W \to M$ and
	$f' \times_M g \colon (V \times 1) \times_M W \to M$ with appropriate boundary co-orientations.
	Similarly, two of the boundary components of $W \times_M (V \times I)$ will be $g \times_M f: W \times_M (V \times 0) \to M$ and
	$g \times_M f' \colon W \times_M (V \times 1) \to M$ with appropriate co-orientations.
	But now we note that $(V \times I) \times_M W$ and $W \times_M (V \times I)$ are diffeomorphic as spaces mapping to $M$ (see \cref{R: pullback representative}), and so their co-orientations either agree or disagree, but the relations between the co-orientations of these spaces and the co-orientations of their boundary components will be the same, and so the co-orientations of the boundary components will agree or disagree as the co-orientations of $(V \times I) \times_M W$ and $W \times_M (V \times I)$ agree or disagree.
	In other words, identifying these fiber products as in Diagram \eqref{D: comm triangle}, we see that regardless of the actual co-orientations of $F \times_M g$ and $g \times_M F$ we have $\omega_{f \times_M g} = (-1)^{(m-v)(m-w)}\omega_{g \times_M f}$ if and only if $\omega_{f' \times_M g} = (-1)^{(m-v)(m-w)}\omega_{g \times_M f'}$.
	Analogously, using $G$, we have $\omega_{f' \times_M g} = (-1)^{(m-v)(m-w)}\omega_{g \times_M f'}$ if and only if $\omega_{f' \times_M g'} = (-1)^{(m-v)(m-w)}\omega_{g' \times_M f'}$.
	But this last equality is true by \cref{C: if full}.
\end{proof}

BEGIN OLD COMMENT
	OLD PROOF
	As noted, it suffices to verify the claim at one point of each connected component of $P$.
	As $f$ and $g$ are transverse, if a component $P_0$ of $V \times_M W$ is nonempty, by Theorem \ref{pullback} we can find $x \in V$, $y \in W$, each point in the interior, so that $(x,y) \in P_0$.
	By Lemma \ref{L: make full}, we can first perform a homotopy $F \times I \colon V \times I \to M$ of $f$ to $f' \colon V \to M$ and then a homotopy of $G \colon W \times I \to M$ of $g$ to $g' \colon W \to M$ so that $f'$ and $g'$ have full rank at $x$ and $y$ respectively.
	By the construction of Lemma \ref{L: make full}, these properties will continue to hold if we take the homotopies constructed there to be of arbitrarily short duration, and we can assume the homotopies are constant outside of compact neighborhoods of $x$ and $y$ in the interiors of $V$ and $W$.
	Considering $F$ first, by doing so, we can assure that, as spaces but ignoring co-orientations, the pullback $(V \times I) \times_M W$ will have the form of a cylinder $P \times I$ with $F \times_M g$ being a homotopy from $f \times_M g$ to $f' \times_M g$.
	As $f \times_M g$ is co-orientable, so is this homotopy, and, now considering co-orientations, there is a co-orientation for this homotopy making it a co-oriented homotopy from $f \times_M g$ to $f' \times_M g$ (recall Definition \ref{D: co-oriented homotopy} and Section\label{S: co-oriented homotopy} in general).
	Similarly, $g \times_M F$ can be co-oriented as a co-oriented homotopy from $g \times_M f$ to $g \times_M f'$.
	Identifying these pullback as in Diagram \eqref{D: comm triangle}, we see that $\omega_{f \times_M g} = (-1)^{(m-v)(m-w)}\omega_{g \times_M f}$ if and only if $\omega_{f' \times_M g} = (-1)^{(m-v)(m-w)}\omega_{g \times_M f'}$.
	Analogously, this holds if and only if $\omega_{f' \times_M g'} = (-1)^{(m-v)(m-w)}\omega_{g' \times_M f'}$, which holds by Corollary \ref{C: if full}.
END OLD COMMENT

BEGIN OLD COMMENT
	\begin{corollary}
		Let $f \colon V \to M$ and $g \colon W \to M$ be transverse proper maps from manifolds with corners to a manifold without boundary.
		Suppose $x \in V$ and $y \in W$ such that $f(x) = g(y)$.
		Then there are smooth homotopies $F \colon V \times I \to M$ and $G \colon W \times I \to M$ such that $F(-,0) = f$, $G(-,0) = g$, $F(x,t) = f(x) = g(y) = G(y,t)$ for all $t \in I$, $DF(-,1)$ and $DG(0,1)$ have maximal rank at $x$ and $y$ respectively, and $F(-,t)$ is transverse to $G(-,t)$ for all $t \in I$.
	\end{corollary}
	\begin{proof}
		We can apply Lemma \ref{L: make full} twice in succession with time rescalings, once with a homotopy of $f$ that holds $g$ fixed for $t\in[0,1/2]$ and then with a homotopy of $g$.
	\end{proof}

END OLD COMMENT

\end{comment}

\subsubsection{Codimension $0$ and $1$ pullbacks}\label{S: codim 0 and 1 co-or}

The results in this section should be compared with, and justify, the discussion and choices in \cref{E: splitting example 1,E: manifold decomposition}.
They will be useful when working with the creasing construction for geometric cochains introduced in \cref{S: creasing}.

\begin{proposition}\label{P: codim 0 pullback}
	Let $V$ be an embedded codimension $0$ submanifold with corners in the manifold without boundary $M$, and let $f \colon V \to M$ be the embedding, co-oriented by the tautological co-orientation.
	Let $W$ be a manifold with corners and suppose $g \colon W \to M$ is transverse to $f$.
	Then the co-oriented pullback $V \times_M W \to W$ is the inclusion of the codimension $0$ manifold with corners $g^{-1}(V) \into W$, co-oriented by the tautological co-orientation.
	Consequently, if $g$ is co-oriented, the co-oriented fiber product $V \times_M W \to M$ is just the restriction of the co-oriented map $g$ to $g^{-1}(V)$.
\end{proposition}

\begin{proof}
	It is clear topologically that the pullback is $g^{-1}(V)$, and it must be a codimension $0$ manifold with corners in $W$ by Joyce \cite[Theorem 6.4]{Joy12}.
	So we consider co-orientations, for which we can assume $V$ and $W$ are without boundary by  \cref{pullback,P: interior co-orientation}.
	As $f$ is an embedding we may choose $N = 0$ and $e = f$ in \cref{D: pullback coorient}.
	As $V \to M$ is tautologically co-oriented, we can identify $\beta_V$ with $\beta_M$ at any point via the embedding.
	The Quillen normal bundle of $V$ in $M$ is then the positively oriented $\R^0$-bundle.
	So then the definition says that the pullback co-orientation at any point is $(\beta_P, \beta_W)$ when $\beta_P = \beta_W$, i.e.\ the pullback co-orientation is $(\beta_P, \beta_P)$, the tautological co-orientation.
\end{proof}

\begin{corollary}\label{C: cup with identity}
	Let $f \colon V \to M$ be a co-oriented map from a manifold with corners to a manifold without boundary, and let $\id_M \colon M \to M$ be the identity.
	Then both co-oriented fiber products $V \times_M M \to M$ and $M \times_M V \to M$ are again $f \colon V \to M$ with the given co-orientation.
\end{corollary}

\begin{proof}
	The case of $M \times_M V \to M$ follows from the preceding lemma, and the other is the first statement of \cref{P: pullback functoriality}.
\end{proof}

\begin{example}\label{E: codim 0 and 1 co-or as fiber products}
	Let $\phi \colon M \to \R$ be a smooth function from a manifold without boundary to $\R$ that is transverse to $0$.
	Then, as $[0, \infty)$, $(-\infty, 0]$ and $0$ are all embedded manifolds with corners in $\R$, we know (ignoring co-orientations for the moment) that $[0,\infty) \times_\R M = \phi^{-1}([0,\infty))$, that $(-\infty,0] \times_\R M = \phi^{-1}((-\infty,0])$, and that $0 \times_\R M = \phi^{-1}(0)$.
	As in \cref{E: manifold decomposition}, we denote these respectively as $M^+$, $M^-$, and $M^0$.
	By \cite[Proposition 4.2.9]{MaDo92}, the inclusion of $M^0$ into $M$ is a closed embedding and by \cite[Proposition 6.7]{Joy12}, we have $\bd M^\pm = M^0$.

	Suppose the embeddings $M^\pm \into M$ are given the tautological co-orientations of \cref{D: tautological co-orientation}, and let $g \colon W \to M$ be transverse to $M^\pm$, which in this case is equivalent to being transverse to $M^0$, which is also equivalent to $\phi g$ being transverse to $0$ in $\R$.
	Then let $W^\pm = M^\pm \times_M W$, which map onto $g^{-1}(M^\pm)$, diffeomorphically on their interiors.
	By \cref{P: codim 0 pullback}, the pullback co-orientations of $W^\pm \into W$ are again the tautological co-orientations of the embeddings.
	If $g$ is co-oriented, the compositions $W^\pm \into W \to M$ are then the fiber products $M^\pm \times_M W \to M$, and by \cref{P: codim 0 pullback} their co-orientations are just the restrictions of the co-orientation of $g$ to $W^\pm$.
	This agrees with the co-orientations in \cref{E: manifold decomposition}.
\end{example}

\begin{proposition}\label{P: codim 1 co-orient}
	Suppose $V \subset M$ is a closed codimension $1$ submanifold without boundary in the manifold without boundary $M$.
	Further suppose $V$ has oriented normal bundle $\nu$.
	Let the embedding $f \colon V \into M$ be co-oriented by the normal co-orientation $(\beta_V, \beta_V \wedge \beta_\nu)$.
	Let $W$ be a manifold without corners and suppose $g \colon W \to M$ is transverse to $f$.
	Then $W^0 \defeq g^{-1}(V)$ is a codimension $1$ submanifold with corners with oriented pullback normal vector bundle $\nu_0$, and the co-oriented pullback $V \times_M W \to W$ is the embedding $W^0 \into W$, co-oriented by $(\beta_P, \beta_P \wedge \beta_{\nu_0})$.
\end{proposition}

\begin{proof}
	Since the normal bundle of $V$ is $1$-dimensional and oriented, it is the trivial line bundle.
	Embedding the normal bundle as a tubular neighborhood, we can then construct a map $\phi: M \to \R$ so that $V = \phi^{-1}(0)$.
	Then $g$ being transverse to $V$ is equivalent to $\phi g$ being transverse to $0$, so by  \cite[Proposition 4.2.9]{MaDo92} the pullback $g^{-1}(V) = W^0$ is a codimension $1$ submanifold with corners of $W$.
	For the co-orientation, if we take $N = 0$ and $e = f$ in \cref{D: pullback coorient}, then $\nu$ is just our normal bundle $\nu V$ and $\nu_0$ is simply the pullback.
	Then, by definition, the pullback co-orientation is $(\beta_P, \beta_W)$ (at interior points) if and only if $\beta_P \wedge \beta_{\nu_0} = \beta_W$, as claimed.
\end{proof}

\begin{example}\label{E: codim 1 pullbacks}
	We continue with the assumptions and notation of \cref{E: codim 0 and 1 co-or as fiber products}, but now let $V = M^0 = \phi^{-1}(0)$ with normal bundle oriented by the pullback of the standard (positive-direction) orientation of the normal bundle of $0 \in \R$.
	This determines the normal co-orientation of the embedding $M^0 \to M$.
	Then, by \cref{P: codim 1 co-orient}, the pullback co-orientation of $W^0 = M^0 \times_M W \into W$ agrees with the $\phi$-induced co-orientation of $W^0$ defined in \cref{E: manifold decomposition}.
	We can also confirm now, using the Leibniz rule and that the codimension of $M^-$ in $M$ is $0$, that as spaces with co-oriented maps to $W$ we have
	\begin{multline*}
		\bd(W^-) = \bd(M^- \times_M W) = \left( (\bd (M^-)) \times_M W \right) \bigsqcup \left( M^- \times_M \bd W \right)\\
		= \left( -(M^0) \times_M W \right) \bigsqcup \left( M^- \times_M \bd W \right)= -(W^0) \bigsqcup (\bd W)^-.
	\end{multline*}
	Here we also use that the orientation of the normal bundle to $M^0$ is outward pointing from $M^-$ and so disagrees with the inward-pointing normal used to co-orient the boundary inclusion; hence $\bd(M^-) = -(M^0)$.
	We also note that, by the preceding, $(\bd W)^- \to W$ is co-oriented by the tautological co-orientation $(\bd W)^- \to \bd W$ followed by the boundary co-orientation of $i_{\bd W} \colon \bd W \to W$.

	Analogously,
	$$\bd (W^+) = W^0 \bigsqcup (\bd W)^+,$$
	using that the orientation of the normal bundle to $M^0$ is inward pointing for $M^+$.
\end{example}

\begin{comment}
	\begin{proof}
		Again it is clear that $f \times_M g = g|_{W^0}$ as maps.
		Suppose given structural co-orientations.
		As $f$ is an embedding, we have $K_f = 0$, while $V^\perp$ is spanned by $\nu$; note that as $\nu_0$ maps to $\nu$, we write simply $\nu$ in the local decomposition of $TW$.
		The structural co-orientation of $f$ is $(\beta_{W^\perp} \wedge \beta_I, \beta_{W^\perp} \wedge \beta_I \wedge \beta_\nu)$, which agrees with the assumed co-orientation for $f$.
		So the co-orientation of the pullback will be the structural orientation or not according to whether the structural co-orientation of $g$ agrees with the given co-orientation of $g$ or not.
		The structural co-orientation of $g$ is $(\beta_{K_g} \wedge \beta_I \wedge \beta_\nu, \beta_{W^\perp} \wedge \beta_I \wedge \beta_\nu)$, while the structural co-orientation of the pullback is $(\beta_{K_g} \wedge \beta_I, \beta_{W^\perp} \wedge \beta_I \wedge \beta_\nu)$.
		If the given co-orientation for $g$ agrees with the structural orientation, then the claimed co-orientation for $g|_{W^0}$ is the composition of the structural co-orientation for $g$ with $(\beta_{W^0},\beta_{W^0} \wedge \beta_{\nu})$.
		In this last expression we are free to choose any $\beta_{W^0}$ we like, so we can let $\beta_{W^0} = \beta_{K_g} \wedge \beta_I$.
		Then the claimed composite co-orientation is $(\beta_{K_g} \wedge \beta_I, \beta_{W^\perp} \wedge \beta_I \wedge \beta_\nu)$, which agrees with the pullback co-orientation as claimed.
		If the given co-orientation of $g$ disagrees with the structural co-orientation, this changes the sign of both the pullback co-orientation and of the representative of the co-orientation of $g$ used in our composite but not the sign of $(\beta_{W^0},\beta_{W^0} \wedge \beta_{\nu})$.
		So again the pullback co-orientation agrees with the claimed composite.
	\end{proof}
\end{comment}

We can now prove the claim from the end of \cref{E: manifold decomposition}.
We express the following corollary using Notation \ref{N: implicit notation}.

\begin{corollary}\label{C: co-orient W0}
	Suppose the hypotheses and notation of \cref{P: codim 1 co-orient} and suppose $V$ is without boundary.
	Then $(\bd W)^0 = -\bd (W^0)$ as co-oriented maps to $W$, with $W^0$ and $(\bd W)^0 = (gi_{\bd W})^{-1}(V)$ co-oriented as in \cref{P: codim 1 co-orient} as the pullbacks $V \times_M W \to W$ and $V \times_M \bd W \to W$.
\end{corollary}

\begin{proof}
	By \cref{leibniz} and the preceding examples, we have that
	$$\bd (W^0) = \bd (V \times_M W) = (-1)^{m-v} V \times_M \bd W = -V \times_M \bd W = -(\bd W)^0$$
	as spaces mapping to $W$.
\end{proof}

\subsection{Exterior products and their relations with fiber products}\label{S: exterior products}
In this section we consider products of maps that will eventually become the exterior products in geometric homology and cohomology, as well as their relations to fiber products.
While fiber products, which will eventually be used to define cup and intersection products, require special transversality conditions in order to be defined, exterior products are always fully defined.
Of course products of oriented manifolds are familiar objects, so we treat them only briefly in the next section.
Then we consider products of co-oriented maps of manifolds.
In \cref{S: product relations}, we show that the co-oriented fiber product is the pullback by the diagonal map of the co-oriented exterior product, foreshadowing the classical cohomology relation between exterior products and cup products.
This will also allow us to prove associativity of fiber products.

\subsubsection{Exterior products of oriented manifolds}

Recall that we defined the oriented fiber product of oriented manifolds with corners in \cref{S: orientation of fiber products}.
In particular, if $V$ and $W$ are oriented manifolds, we saw in \cref{P: oriented fiber product basic properties} that the oriented fiber product of the maps from $V$ and $W$ to a point is just the standard product $V \times W$ oriented with the usual concatenation convention.
In other words, if $V$ and $W$ are oriented by $\beta_V$ and $\beta_W$, then $V \times W$ is oriented at any point by $\beta_V \wedge \beta_W$.

We observe the following interplay between fibered and exterior products of maps of oriented manifolds more generally.
In our notational shorthand (see \cref{N: implicit notation}), if $f \colon V \to M$ and $h \colon X \to N$, then we let $V \times X$ represent the product map $f \times h \colon M \times N$.

\begin{proposition}\label{P: oriented interchange}
	Suppose $f \colon V \to M$ and $g \colon W \to M$ are transverse maps of oriented manifolds with corners to an oriented manifold without boundary and similarly for $h \colon X \to N$ and $k \colon Y \to N$.
	Then
	$$(V \times X)\times_{M \times N} (W \times Y) = (-1)^{(m-w)(n-y)}(V \times_M W) \times (X \times_N Y)$$
	as oriented manifolds.
\end{proposition}

\begin{proof}
	We first note that the transversality assumptions ensure also that $f \times h$ will be transverse to $g \times k$.
	It is straightforward to verify that these are diffeomorphic spaces, so we focus on the orientations.
	For simplicity, let us write
	$P = (V \times X)\times_{M \times N} (W \times Y)$ and $P' = (V \times_M W) \times (X \times_N Y)$ as oriented manifolds.
	We then write local orientations symbolically as $\beta_P$, etc.
	By the construction of fiber product orientations in \cref{S: orientation of fiber products}, and omitting the pullbacks from the notation, $P$ is oriented so that
	$$\beta_P \wedge \beta_{M \times N} = (-1)^{(w+y)(m+n)}\beta_{V \times X} \wedge \beta_{W \times Y},$$
	or, as (non-fiber) products are oriented by concatenation, we have
	$$\beta_P \wedge \beta_M\wedge\beta_N = (-1)^{(w+y)(m+n)}\beta_V \wedge \beta_X \wedge \beta_W \wedge \beta_Y.$$
	Recall that here we identify $T(M \times N)$ as a summand of $T(V \times X \times W \times Y)$ over $P$ by splitting the derivative $D(f \times h) - D(g \times k)$.
	Similarly, for $V \times_M W$ and $X \times_N Y$ we have
	\begin{align*}
		\beta_{V \times_M W} \wedge \beta_M & = (-1)^{wm}\beta_V \wedge \beta_W \\
		\beta_{X \times_N Y} \wedge \beta_N & = (-1)^{yn}\beta_X \wedge \beta_Y,
	\end{align*}
	using the splittings of $Df-Dg$ and $Dh-Dk$.
	We note that the signs of $Df$, $Dg$, $Dh$, and $Dk$ in all the splitting formulas are consistent in computing the orientations for $P$ and $P'$.

	As $\beta_{P'} = \beta_{V \times_M W} \wedge \beta_{X \times_N Y}$, we have
	\begin{align*}
		\beta_{P'} \wedge \beta_M \wedge \beta_N
		&= \beta_{V \times_M W} \wedge \beta_{X \times_N Y} \wedge \beta_M \wedge \beta_N\\
		& = (-1)^{m(x+y-n)}\beta_{V \times_M W} \wedge \beta_M \wedge \beta_{X \times_N Y} \wedge \beta_N\\
		& = (-1)^{m(x+y-n)+wm+ny}\beta_V \wedge \beta_W \wedge \beta_X \wedge \beta_Y \\
		& = (-1)^{m(x+y-n)+wm+ny+xw}\beta_V \wedge \beta_X \wedge \beta_W \wedge \beta_Y.
	\end{align*}
	So $\beta_P$ differs from $\beta_{P'}$ by $-1$ to the power
	$$m(x+y-n) + wm + ny + xw - (w+y)(m+n).$$
	An elementary computation now shows that this is $(m-w)(n-x)$ as desired.
\end{proof}

\subsubsection{Exterior products of co-oriented maps}

Next we define and study a co-oriented exterior product for co-oriented maps.
In the next subsection, we will see that such products are intimately related to fiber products, and this will allow us to easily prove some properties about fiber products that we have delayed.

\begin{comment}
Before getting to co-orientations, we first show that a product of proper maps is proper.

\begin{lemma}\label{L: proper product}
	If $f \colon V \to M$ and $g \colon W \to N$ are proper maps of spaces then the product map $f \times g \colon V \times W \to M \times N$ is proper.
\end{lemma}

\begin{proof}
	Let $\pi_M,\pi_N$ be the projections of $M \times N$ to $M$ and $N$, and similarly for $\pi_V, \pi_W$.
	Let $K$ be a compact subspace of $M \times N$.
	Then $\pi_M(K)$ and $\pi_N(K)$ are compact, and hence so is $\pi_M(K) \times \pi_N(K) \subset M \times N$, and this set contains $K$.
	So
	$$(f \times g)^{-1}(K) \subset (f \times g)^{-1}(\pi_M(K) \times \pi_N(K)) = f^{-1}(\pi_M(K)) \times g^{-1}(\pi_N(K)).$$
	But now $f^{-1}(\pi_M(K))$ and $g^{-1}(\pi_N(K))$ are compact as $f$ and $g$ are proper and so $(f \times g)^{-1}(K)$ is a closed subset of a compact set, hence compact.
\end{proof}

\end{comment}

\begin{lemma}
	If $f \colon V \to M$ and $g \colon W \to N$ are co-orientable maps of manifolds with corners then the product map $f \times g \colon V \times W \to M \times N$ is co-orientable.
\end{lemma}

\begin{proof}
	We recall that, by definition, a co-orientation of $f$ is equivalent to a choice of isomorphism between the orientation cover $\Or(TV)$ and the pullback $f^*\Or(TM)$ of the orientation cover $\Or(TM)$, and similarly for $g$.

	If we let $\pi_V, \pi_W$ denote the projections of $V \times W$ to $V$ and $W$, then $T(V \times W) \cong \pi_V^*(TV) \oplus \pi_W^*(TW)$, and so $$\Or(T(V \times W)) \cong \Or(\pi_V^*(TV))\otimes\Or(\pi_W^*(TW)) \cong \pi_V^*\Or(TV)\otimes\pi_W^*\Or(TW).$$ Similarly
	\begin{multline*}(f \times g)^*T(M \times N) \cong (f \times g)^*(\pi_M^*(TM) \oplus \pi_N^*(TN))\\
		 \cong (f \times g)^*\pi_M^*(TM) \oplus (f \times g)^*\pi_N^*(TN)) \cong \pi_V^*f^*(TM) \oplus \pi_W^*g^*(TN),
	\end{multline*}
	using that $\pi_M(f \times g) = f\pi_V \colon V \times W \to M$ and $\pi_N(f \times g) = g\pi_W \colon V \times W \to N$.
	So
	\begin{multline*}
		(f \times g)^*\Or(T(M \times N)) \cong \Or((f \times g)^*T(M \times N))\\ \cong \Or(\pi_V^*f^*TM) \otimes \Or(\pi_W^*g^*TN) \cong \pi_V^*f^*\Or(TM) \otimes \pi_W^*g^*\Or(TN).
	\end{multline*}
	Thus if $\Or(TV) \cong f^*\Or(TM)$ and $\Or(TW) \cong g^*\Or(TN)$, we can construct an isomorphism $\Or(T(V \times W)) \cong (f \times g)^*\Or(T(M \times N))$.
\end{proof}

\begin{definition}\label{D: co-oriented exterior}
	If $f \colon V \to M$ and $g \colon W \to N$ are co-oriented maps of manifolds with corners with co-orientations given by isomorphisms $\phi \colon \Or(TV) \to f^*\Or(TM)$ and $\psi \colon \Or(TW) \to g^*\Or(TN)$, we define the \textbf{product co-orientation} of $f \times g \colon V \times W \to M \times N$ by the isomorphism $(-1)^{(m-v)w}\pi_V^*\phi \otimes \pi_W^*\psi$.
	In particular, if at $x \in V$ the co-orientation of $f$ is given locally by $(\beta_V,\beta_M)$ and at $y \in W$ the co-orientation of $g$ is given locally by $(\beta_W,\beta_N)$, then the product co-orientation is locally represented at $(x,y)$ by $$(-1)^{(m-v)w}(\beta_V \wedge \beta_W,\beta_M \wedge \beta_N).$$

	Following our standard convention from \cref{N: implicit notation}, we often write simply $V \times W$ to represent the co-oriented product. 
\end{definition}

\begin{remark}
	The sign in the definition is not at first obvious, though it will be justified in the following lemmas.
	One way to think of it is as follows: If we we take $V$ and $W$ as immersed submanifolds co-oriented by orienting their normal bundles as in \cref{normal co-or}, then $V \times W$ is also immersed, and at an image point we have $T(M \times N) \cong TV \oplus \nu V \oplus TW \oplus \nu W$, letting $\nu V$ and $\nu W$ stand for the normal bundles of $V$ and $W$ in $M$ and $N$, respectively.
	The sign $(-1)^{(m-v)w}$ is the sign needed in the local orientation to permute this to $TV \oplus TW \oplus \nu V \oplus \nu W$ so that we can properly utilize the normal co-orientation for $\nu(V \times W) \cong \nu V \oplus \nu W$.
	While this argument is essentially heuristic, it is borne out in the computations below.
\end{remark}

The following example will be useful in the proof of \cref{T: intersection is cup product}.

\begin{example}\label{E: sphere product}
	Let $S^p$ and $S^q$ be oriented spheres with $p,q>0$.
	Let $V = W = pt$, and let $f \colon V \to S^p$ and $g \colon W \to S^q$ be embeddings to points $x \in S^p$, $y \in S^q$.
	Let $f$ be co-oriented by $(1,\beta_{S^p})$; in other words $V$ is normally co-oriented by the orientation of its normal bundle that agrees with the orientation of $S^p$.
	Let $g$ be co-oriented similarly.
	Then $V \times W$ is represented by the embedding of the point to $(x,y) \in S^p \times S^q$ with normal bundle oriented consistently with the product orientation of $S^p \times S^q$.
	There is no extra sign in this case as $\dim(W) = 0$.
\end{example}

\begin{proposition}\label{P: co-oriented exterior unit}
	Let $f \colon V \to M$ be a co-oriented map of manifolds with corners, and let $g:pt \to pt$ be the unique map with the canonical co-orientation.
	Then $f \times g \colon V \times pt \to M \times pt$ and $g \times f:pt \times V \to pt \times M$ are each isomorphic as co-oriented maps of manifolds with corners to $f \colon V \to M$.
\end{proposition}

\begin{proof}
	This is obvious ignoring co-orientations.
	Considering co-orientations, if $f$ is co-oriented at a point by $(\beta_V,\beta_M)$, then the co-orientation of $f \times g$ is simply $(\beta_V \wedge 1,\beta_M \wedge 1) = (\beta_V,\beta_M)$, noting that
	the sign $(-1)^{(m-v)\cdot 0} = 1$ in this case.
	The case $g \times f$ is similar, though due to the transposition the sign is now $(-1)^{(0-0)v}$, which is again $1$.
\end{proof}

\begin{proposition}\label{P: boundary of exterior product}
	Let $f \colon V \to M$ and $g \colon W \to N$ be co-oriented maps of manifolds with corners and suppose $f \times g \colon V \times W \to M \times N$ is given the product co-orientation.
	Then the boundary co-orientation of $V \times W$ as co-oriented maps to $M \times N$ is $$\bd(V \times W) = (\bd V) \times W \bigsqcup (-1)^{m-v}V \times \bd W.$$
\end{proposition}

\begin{proof}
	We know that this expression is an identity ignoring co-orientations, so we must establish the agreement of the co-orientations for each component.
	As usual, it suffices to consider points in the top dimensional strata of $\bd(V \times W)$.
	In what follows, we fix $\beta_V$, $\beta_W$, $\beta_M$, and $\beta_N$, so that $(\beta_V,\beta_M)$, $(\beta_W,\beta_N)$, and $(-1)^{(m-v)w}(\beta_V \wedge \beta_W,\beta_M \wedge \beta_N)$ denote the co-orientations of $V$, $W$, and $V \times W$ at the point under consideration.

	Let $\nu$ denote an inward pointing normal to $V \times W$ at such a point.
	Then the inclusion $\bd(V \times W) \to V \times W$ is co-oriented at that point by $(\beta_{\bd(V \times W)},\beta_{\bd(V \times W)} \wedge \beta_\nu)$ for any $\beta_{\bd(V \times W)}$.
	If we choose $\beta_{\bd(V \times W)}$ so that $(\beta_{\bd(V \times W)}\wedge\beta_\nu,\beta_M \wedge \beta_N)$ represents the co-orientation of $V \times W \to M \times N$, then from the definition of the boundary co-orientation, the boundary $\bd(V \times W) \to M \times N$ is co-oriented by $(\beta_{\bd(V \times W)},\beta_M \wedge \beta_N)$.
	We fix such a choice in what follows.

	Now suppose our point is more specifically in the top-dimensional stratum of $(\bd V) \times W$.
	If we choose $\beta_{\bd V}$ so that $\beta_{\bd V} \wedge \beta_\nu = \beta_V$, then $\bd V \to M$ is co-oriented by $(\beta_{\bd V},\beta_M)$ and so $(\bd V) \times W \to M \times N$ is co-oriented by $(-1)^{(m-v+1)w}(\beta_{\bd V} \wedge \beta_W,\beta_M \wedge \beta_N)$.
	On the other hand, the co-orientation of $V \times W$ can then be written $(-1)^{(m-v)w}(\beta_{\bd V} \wedge \beta_\nu \wedge \beta_W,\beta_M \wedge \beta_N) = (-1)^{(m-v)w+w}(\beta_{\bd V} \wedge \beta_W \wedge \beta_\nu,\beta_M \wedge \beta_N)$, so the boundary co-orientation of $\bd(V \times W)$ is $$(\beta_{\bd V} \wedge \beta_W,\beta_{\bd V} \wedge \beta_W \wedge \beta_\nu)*(-1)^{(m-v)w+w}(\beta_{\bd V} \wedge \beta_W \wedge \beta_\nu,\beta_M \wedge \beta_N) = (-1)^{(m-v)w+w}(\beta_{\bd V} \wedge \beta_W,\beta_M \wedge \beta_N),$$ 
	which agrees with our co-orientation for $(\bd V) \times W$.

	Next consider a point in the top-dimensional stratum of $V \times \bd W$.
	If we choose $\beta_{\bd W}$ so that $\beta_{\bd W} \wedge \beta_\nu = \beta_W$ then we have $\bd W$ co-oriented by $(\beta_{\bd W},\beta_N)$ and so $V \times \bd W$ is co-oriented by $(-1)^{(m-v)(w-1)}(\beta_{V} \wedge \beta_{\bd W},\beta_M \wedge \beta_N)$.
	On the other hand, the co-orientation of $V \times W$ can now be written $(-1)^{(m-v)w}(\beta_{V} \wedge \beta_{\bd W} \wedge \beta_\nu,\beta_M \wedge \beta_N)$, so the boundary co-orientation of $\bd(V \times W)$ is $(-1)^{(m-v)w}(\beta_{V} \wedge \beta_{\bd W},\beta_{M} \wedge \beta_N)$, which differs from that of $V \times \bd W$ by a factor of $(-1)^{m-v}$.
\end{proof}

\begin{proposition}\label{P: exterior associativity}
	Let $f \colon V \to M$, $g \colon W \to N$, and $h \colon X \to Q$ be co-oriented maps of manifolds with corners.
	Then the co-orientations of $(V \times W) \times X \to M \times N \times Q$ and $V \times (W \times X) \to M \times N \times Q$ agree.
	In other words, forming co-oriented products is associative.
\end{proposition}

\begin{proof}
	If $f,g,h$ are co-oriented by $(\beta_V,\beta_M)$, etc., then both products are co-oriented up to sign by $(\beta_V \wedge \beta_W \wedge \beta_X,\beta_M \wedge \beta_N \wedge \beta_P)$.
	In forming $(V \times W) \times X$ we first have the sign $(-1)^{(m-v)w}$ from $V \times W$, then taking the product with $X$ on the right multiplies by $(-1)^{(m+n-v-w)x}$.
	So the total sign is $(-1)^{(m-v)w+(m+n-v-w)x}$.
	Alternatively, forming $W \times X$ has the sign $(-1)^{(n-w)x}$ and then taking the product with $V$ on the left contributes $(-1)^{(m-v)(w+x)}$.
	So the total sign is $(-1)^{(n-w)x+(m-v)(w+x)}$.
	One readily verifies that these signs agree.
\end{proof}

\begin{comment}
	Dev and Anibal, please check the following arguments carefully as I'm not 100\% confident in it.
	It gives the ``right'' answer but I'm a little uncomfortable divorcing the order of the orientation terms from the order of the manifold terms.
	Of course this happens all the time - even if we think of $\R^2$ as $\R_x \oplus \R_y$ we can still think about the two-form $y \wedge x$, but I'm still a little nervous about maybe having missed a sign somewhere.
	I'm also a little nervous about my trick of taking $a$ and $b$ to be even so that they won't contribute, but the earlier work says that this should be allowable.
	Presumably if I didn't do this there would be a bunch of extra signs that miraculous cancel out, but I'm not so excited about trying that out in detail to see.
\end{comment}

The following lemma provides a nice description of the Quillen co-orientation of a product of co-oriented maps.
Among other things, it will help us to next demonstrate a commutativity property for exterior products of co-oriented maps.
We assume for convenience that our Euclidean factors are even dimensional, which simplifies the computations and will be sufficient for what follows; of course if $V$ embeds in $M \times \R^n$, it also embeds in $M \times \R^{n+1}$, and we have shown that our fiber product co-orientations do not depend on such choices. 

\begin{lemma}\label{L: Quillen product co-orientation}
	Let $f \colon V \to M$ and $g \colon W \to N$ be co-oriented maps from manifolds with corners to manifolds without boundary.
	Consider Quillen co-orientations representing $f$ and $g$ via embeddings $e_V \colon V \into M \times \R^a$ and $e_W \colon W \into N \times \R^b$ with $a$ and $b$ even.
	Denote the normal bundles of $V$ and $W$ in $M \times \R^a$ and $N \times \R^b$ by $\nu V$ and $\nu W$.
	Let $T \colon M \times \R^a \times N \times \R^b \to M \times N \times \R^{a+b}$ be the diffeomorphism that interchanges the middle two factors.
	Then
	$T(e_V \times e_W)$ gives an embedding $V \times W \to M \times N \times \R^{a+b}$ with normal bundle isomorphic to the sum of the pullbacks of $\nu V$ and $\nu W$ by the projections of $V \times W \to M \times N \times \R^a \times \R^b$ to either the first and third factor or the second and fourth factors.
	For simplicity, we simply write $\nu V \oplus \nu W$.

	Then the Quillen normal orientation of the normal bundle of $f \times g \colon V \times W \to M \times N$ is given by $$\beta_{\nu V \oplus \nu W} = \beta_{\nu V} \wedge \beta_{\nu W},$$
	suitably interpreting the relevant coordinates in $M \times N \times \R^a \times \R^b$.
\end{lemma}

The last line simply means that if an element of $\nu V$ has coordinates $(x,y)$ in $M \times \R^a$, then this corresponds to a normal vector with coordinates $(x,0,y,0)$ in $M \times N \times \R^a \times \R^b$ for which we do not create a new notation, and similarly for $\nu_W$. 

\begin{proof}
	Let $\beta_a$ and $\beta_b$ denote the standard orientations for $\R^a$ and $\R^b$.
	By definition, $\nu V$ and $\nu W$ are oriented so that $\beta_V \wedge \beta_{\nu V} = \beta_M \wedge \beta_a$ and $\beta_W \wedge \beta_{\nu W} = \beta_N \wedge \beta_b$.

	By definition, the Quillen orientation of $\nu V \oplus \nu W$ corresponding to the product co-orientation of $V \times W$ is the local orientation $\beta_{\nu V \oplus \nu W}$ such that
	$$(\beta_{V \times W}, \beta_{V \times W} \wedge \beta_{\nu V \oplus \nu W})*(\beta_{M \times N} \wedge \beta_{a+b},\beta_{M \times N}) = (-1)^{(m-v)w}(\beta_V \wedge \beta_W,\beta_M \wedge \beta_N).$$
	Taking $\beta_{M \times N} = \beta_M \wedge \beta_N$ and $\beta_{V \times W} = \beta_V \wedge \beta_W$ and noting $\beta_{a+b} = \beta_a \wedge \beta_b$, this formula becomes
	$$(\beta_V \wedge \beta_W, \beta_V \wedge \beta_W \wedge \beta_{\nu V \oplus \nu W})*(\beta_M \wedge \beta_N \wedge \beta_a \wedge \beta_b,\beta_M \wedge \beta_N) = (-1)^{(m-v)w}(\beta_V \wedge \beta_W,\beta_M \wedge \beta_N).$$
	We also have $$\beta_M \wedge \beta_N \wedge \beta_a \wedge \beta_b = \beta_M \wedge \beta_a \wedge \beta_N \wedge \beta_b,$$ as $a$ is even, so using $\beta_V \wedge \beta_{\nu V} = \beta_M \wedge \beta_a$ and $\beta_W \wedge \beta_{\nu W} = \beta_N \wedge \beta_b$, we require
	$$(\beta_V \wedge \beta_W, \beta_V \wedge \beta_W \wedge \beta_{\nu V \oplus \nu W})*(\beta_V \wedge \beta_{\nu V}\wedge\beta_W \wedge \beta_{\nu W} ,\beta_M \wedge \beta_N) = (-1)^{(m-v)w}(\beta_V \wedge \beta_W,\beta_M \wedge \beta_N).$$
	But now $$\beta_V \wedge \beta_{\nu V}\wedge\beta_W \wedge \beta_{\nu W} = (-1)^{(m-v)w} \beta_V \wedge \beta_{W}\wedge\beta_{\nu V} \wedge \beta_{\nu W},$$
	so, after all that, we see that the Quillen orientation of the normal bundle to $V \times W$ is simply $$\beta_{\nu V \oplus \nu W} = \beta_{\nu V} \wedge \beta_{\nu W}.$$
\end{proof}

\begin{proposition}\label{P: exterior commutativity}
	Let $f \colon V \to M$ and $g \colon W \to N$ be co-oriented maps from manifolds with corners to manifolds without boundary and suppose $f \times g \colon V \times W \to M \times N$ is given the product co-orientation.
	Let $\tau \colon N \times M \to M \times N$ be the diffeomorphism that interchanges coordinates.
	Denote the pullback of $f \times g$ by $\tau$ as $\tau^*(V \times W) \to N \times M$. Then this pullback is diffeomorphic, as co-oriented maps, to $(-1)^{(m-v)(n-w)}W \times V \to N \times M$. In other words,
	$$\tau^*(V \times W) = (-1)^{(m-v)(n-w)}W \times V.$$
\end{proposition}
We assume $M$ and $N$ to be without corners so that we can properly use the pullback construction, which requires transversality, in the proposition and its proof.
However, the pullback is by a diffeomorphism, so this result should extend without problem to more general settings.
\begin{proof}
	This is clear at the level of spaces, so we focus on co-orientations. 
	Let $(\beta_V,\beta_M)$ and $(\beta_W,\beta_N)$ be local representations of the co-orientations at some points.
	The product co-orientation of $V \times W \to M \times N$ is $(-1)^{(m-v)w}(\beta_V \wedge \beta_W,\beta_M \wedge \beta_N)$.

	As in \cref{L: Quillen product co-orientation}, we consider Quillen co-orientations representing $f$ and $g$ via embeddings $e_V \colon V \into M \times \R^a$ and $e_W \colon W \into N \times \R^b \to N$ with $a$ and $b$ even.
	This is sufficient as we know that the pullback construction is independent of $a$ and $b$ for sufficiently large dimensions by \cref{L: pullback co well defined}.
	Assuming the other notation from \cref{L: Quillen product co-orientation}, that lemma tells us that the normal co-orientation of $V \times W$ in $M \times N \times \R^a \times \R^b$ is $$\beta_{\nu V \oplus \nu W} = \beta_{\nu V} \wedge \beta_{\nu W}.$$

	Now using this Quillen co-orientation for $f \times g$, we pull back by the diffeomorphism $\tau \colon N \times M \to M \times N$, obtaining the composition we can write $W \times V \into N \times M \times \R^a \times \R^b \to N \times M$.
	The pulled back normal bundle is still oriented in each fiber as $\beta_{\nu V} \wedge \beta_{\nu W}$ (though of course the order of actual local coordinates have now been jumbled around).
	By definition, the pullback co-orientation is $(\beta_W \wedge \beta_V,\beta_N \wedge \beta_M)$ if and only if $$\beta_W \wedge \beta_V \wedge \beta_{\nu V} \wedge \beta_{\nu W} = \beta_N \wedge \beta_M \wedge \beta_{a+b},$$
	and as $a$ and $b$ are even this last expression is equal to
	$\beta_N \wedge \beta_b \wedge \beta_M \wedge \beta_{a}.$ But by the previous choices, $\beta_V \wedge \beta_{\nu V} = \beta_M \wedge \beta_a$ and $\beta_W \wedge \beta_{\nu W} = \beta_N \wedge \beta_b$.
	So
	\begin{align*}
		\beta_N \wedge \beta_b \wedge \beta_M \wedge \beta_{a}
		& = \beta_W \wedge \beta_{\nu W} \wedge \beta_V \wedge \beta_{\nu V} \\
		& = (-1)^{v(n+b-w)}\beta_W \wedge \beta_V \wedge \beta_{\nu W} \wedge \beta_{\nu V} \\
		& = (-1)^{v(n+b-w)+(m+a-v)(n+b-w)}\beta_W \wedge \beta_V \wedge \beta_{\nu V} \wedge \beta_{\nu W} \\
		& = (-1)^{v(n-w)+(m-v)(n-w)}\beta_W \wedge \beta_V \wedge \beta_{\nu V} \wedge \beta_{\nu W}, \\
	\end{align*}
	where again we use that $a$ and $b$ are even.

	Therefore, the pullback co-orientation is $(-1)^{v(n-w)+(m-v)(n-w)}(\beta_W \wedge \beta_V,\beta_N \wedge \beta_M)$, which is $(-1)^{(m-v)(n-w)}$ times the product co-orientation of $W \times V$, as claimed.
\end{proof}

The next result concerns co-oriented products in which one map is the identity.
We show that such products are simply pullbacks by projections.

\begin{proposition}\label{P: projection pullbacks}
	Let $f \colon V \to M$ be a co-oriented map from a manifolds with corners to a manifold without boundary, and let $\id_N \colon N \to N$ be the identity map of a manifold with corners with the tautological co-orientation.
	\begin{enumerate}
		\item The co-oriented pullback of $V$ by the projection $\pi_1 \colon M \times N \to M$ is $f \times \id_N \colon V \times N \to M \times N$ with its product co-orientation, i.e.\ $\pi_1^*V = V \times N$.
		\item The co-oriented pullback of $V$ by the projection $\pi_2 \colon N \times M \to M$ is $\id_N \times f \colon N \times V \to N \times M$ with its product co-orientation, i.e.\ $\pi_2^*V = N \times V$.
	\end{enumerate}
\end{proposition}

\begin{proof}
	As the projections are submersions, the required transversality conditions to ensure the existence of the pullbacks are met.
	These claims are then clear concerning maps of topological spaces, so we need only verify the co-orientations.

	As in the preceding argument, we start again with an embedding $e \colon V \into M \times \R^a$ to establish a Quillen co-orientation for $f$.
	We again may assume $a$ to be even for simplicity.
	We write the co-orientation of $f$ as $(\beta_V,\beta_M)$, and we let $\nu V$ denote the normal to $e(V)$ and orient $\nu V$ so that $\beta_V \wedge \beta_{\nu V} = \beta_M \wedge \beta_a$, writing $\beta_a$ for the standard orientation of $\R^a$.

	For the second statement, the product co-orientation of $\id_N \times f \colon N \times V \to N \times M$ is $(\beta_N \wedge \beta_V,\beta_N \wedge \beta_M)$, as the domain and codomain of $\id_N$ have the same dimension.
	The pullback by the projection $N \times M \to M$ gives us the embedding/projection sequence $N \times V\xhookrightarrow{\id_N \times e} N \times M \times \R^a \to N \times M$, and the orientations of the pullback of the normal bundle $\nu V$ by $\pi_2 \times \id_{\R^a}$ is again $\beta_{\nu V}$ at each point of $N \times e(V)$.
	So now from \cref{D: pullback coorient}, the pullback has the product co-orientation if and only if $\beta_N \wedge \beta_V \wedge \beta_{\nu V} = \beta_N \wedge \beta_M \wedge \beta_a$.
	But $ \beta_V \wedge \beta_{\nu V} = \beta_M \wedge \beta_a$ by assumption, so this holds.

	For the first statement, the product co-orientation of $f \times \id_N \colon V \times N \to M \times N$ is $(-1)^{(m-v)n}(\beta_V \wedge \beta_N,\beta_M \wedge \beta_N)$.
	The pullback by the projection $M \times N \to M$ gives us an embedding/projection sequence $V \times N \into M \times N \times \R^a \to M \times N$ (where the first arrow is the composition of $e \times \id_N$ with a permutation of coordinates), and the orientation of the pullback of the normal bundle $\nu V$ by $\pi_1 \times \id_{\R^a}$ is again $\beta_{\nu V}$.
	So now from the definition, the pullback has the product co-orientation if and only if $\beta_V \wedge \beta_N \wedge \beta_{\nu V} = (-1)^{(m-v)n}\beta_M \wedge \beta_N \wedge \beta_a$.
	But $ \beta_V \wedge \beta_{\nu V} = \beta_M \wedge \beta_a$ by assumption, so
	\begin{align*}
		\beta_V \wedge \beta_N \wedge \beta_{\nu V}& = (-1)^{(m+a-v)n}\beta_V \wedge \beta_{\nu V} \wedge \beta_N\\
		& = (-1)^{(m+a-v)n}\beta_M \wedge \beta_a \wedge \beta_N\\
		& = (-1)^{(m-v)n}\beta_M \wedge \beta_N \wedge \beta_a.\qedhere
	\end{align*}
\end{proof}

The next proposition shows that the exterior product construction is natural.

\begin{proposition}\label{P: natural exterior}
	Let $f \colon V \to M$ and $g \colon W \to N$ be co-oriented maps of manifolds with corners with $M$ and $N$ having no boundaries.
	Let $h \colon X \to M$ and $k \colon Y \to N$ be maps of manifolds with corners that are transverse to $f$ and $g$ respectively.
	Then $(h \times k)^*(V \times W) = h^*V \times k^*W$ as spaces with co-oriented maps to $X \times Y$.
\end{proposition}

\begin{proof}
	It is easy to show that $h \times k$ is transverse to $f \times g$, so we focus on co-orientation.
	As in the preceding proofs, we write the Quillen orientation of the normal bunle to the embedded image of $V \times W \into M \times N \times \R^a \times \R^b$ as $\beta_{\nu V \oplus \nu W} = \beta_{\nu V} \wedge \beta_{\nu W}$.
	Then the pullback $P = (h \times k)^*(V \times W)$ is co-oriented by $(\beta_P,\beta_{X \times Y})$ if and only if we choose $\beta_P$ and $\beta_{X \times Y}$ so that
	$$\beta_{P} \wedge \beta_{\nu V} \wedge \beta_{\nu W} = \beta_{X \times Y} \wedge \beta_{a+b}.$$

	On the other hand, we know $h^*V$ is co-oriented by $(\beta_{h^*V},\beta_X)$ if an only if $\beta_{h^*V} \wedge \beta_{\nu V} = \beta_X \wedge \beta_a$, and $k^*W$ is co-oriented by $(\beta_{k^*W},\beta_Y)$ if an only if $\beta_{k^*W} \wedge \beta_{\nu W} = \beta_Y \wedge \beta_b$.
	Assuming these hold, then $h^*V \times k^*W$ is co-oriented by $$(-1)^{(x-(v+x-m))(w+y-n)}(\beta_{h^*V} \wedge \beta_{k^*W},\beta_X \wedge \beta_Y) = (-1)^{(m-v)(w+y-n)}(\beta_{h^*V} \wedge \beta_{k^*W},\beta_X \wedge \beta_Y).$$

	Now, continuing to assume the equalities of the last paragraph and taking $a$ and $b$ even as usual, we have
	\begin{align*}
		\beta_{h^*V} \wedge \beta_{k^*W} \wedge \beta_{\nu V} \wedge \beta_{\nu W}
		& = (-1)^{(w+y-n)(m-v)}\beta_{h^*V} \wedge \beta_{\nu V} \wedge \beta_{k^*W} \wedge \beta_{\nu W}\\
		& = (-1)^{(w+y-n)(m-v)}\beta_X \wedge \beta_a \wedge \beta_Y \wedge \beta_b \\
		& = (-1)^{(w+y-n)(m-v)}\beta_X \wedge \beta_Y \wedge \beta_a \wedge \beta_b.
	\end{align*}
	So if we take $\beta_P = (-1)^{(w+y-n)(m-v)}\beta_{h^*V} \wedge \beta_{k^*W}$ and $\beta_{X \times Y} = \beta_X \times \beta_Y$, then this also gives us $\beta_{P} \wedge \beta_{\nu V} \wedge \beta_{\nu W} = \beta_{X \times Y} \wedge \beta_{a+b}$.
	Therefore,
	$(h \times k)^*(V \times W)$ is also co-oriented by
	$$(\beta_P,\beta_X \wedge \beta_Y) = ((-1)^{(w+y-n)(m-v)}\beta_{h^*V} \wedge \beta_{k^*W},\beta_X \wedge \beta_Y).$$ We conclude $(h \times k)^*(V \times W) = h^*V \times k^*W$.
\end{proof}

\subsubsection{Applications of co-oriented exterior products to co-oriented fiber products}\label{S: product relations}

Having established some elementary properties for our exterior product, we can now relate co-oriented exterior products to co-oriented fiber products.
These relationships correspond to those in singular cohomology between the exterior and cup products of cochains, though one very nice feature is that in our context these relationships all hold ``on the nose'' at the cochain level and can be proven without any need for Alexander-Whitney maps or any other approximations to the diagonal.
This will also be useful for proving some properties of the co-oriented fiber product that we have deferred so far, inclusing associativity and graded commutativity.

We start with the following version of a well-known fact. 


\begin{lemma}\label{L: alternative transversality}
	Suppose $f \colon V \to M$ and $g \colon W \to M$ are co-oriented maps from manifolds with corners to a manifold without boundary.
	Let $\diag \colon M \to M \times M$ be the diagonal map $\diag(x) = (x,x)$.
	Then $f$ and $g$ are transverse if and only if $f \times g$ is transverse to $\diag$.
\end{lemma}
\begin{proof}
	As transversality is determined stratum by stratum, it suffices to suppose $V$ and $W$ are manifolds without boundary. 
	We briefly recall the argument in this case.
	
	First suppose $f$ and $g$ are transverse and that $f(x) = g(y) = z \in M$.
	Then $Df(T_xV)+Dg(T_yW) = T_zM$.
	Now suppose $(a,b) \in T_{(z,z)}(M \times M) \cong T_zM \oplus T_zM$.
	Write $a = v_a+w_a$ with $v_a \in Df(T_xV)$ and $w_a \in Dg(T_yW)$.
	Similarly, write $v = v_b+w_b$.
	Then
	\begin{align*}
		(a,b)& = (v_a+w_a,v_b+w_b)\\
		& = (v_a-v_b+v_b+w_a, v_b+w_a-w_a+w_b)\\
		& = (v_a-v_b,-w_a+w_b)+(v_b+w_a, v_b+w_a),
	\end{align*}
	which is in $D(f \times g)(T_xV \times T_yW)+D\diag(T_zM)$.
	So $f \times g$ is transverse to $\diag$.

	Conversely, suppose $f \times g$ is transverse to $\diag$, and continue to assume that $f(x) = g(y) = z \in M$ so that $(f \times g)(x,y) = \diag(z)$. 
	Let $v \in T_zM$.
	Then there are $a\in T_x V$, $b \in T_y W$, and $c \in T_zM$ such that $$(v,0) = D(f \times g)(a,b) + D\diag(c) = (Df(a), Dg(b)) + (c,c).$$
	Then $Df(a) + c = v$ and $Dg(b) + c = 0$, so $Df(a) - Dg(b) = Df(a) + Dg(-b) = v$.
	So $f$ and $g$ are transverse.
\end{proof}

\begin{proposition}\label{P: cross to cup}
	Suppose $f \colon V \to M$ and $g \colon W \to M$ are transverse co-oriented maps from manifolds with corners to a manifold without boundary.
	Then the pullback of $V \times W \to M \times M$ by $\diag$ is the co-oriented fiber product $V \times_M W \to M$, i.e.\ $$V \times_M W = \diag^*(V \times W).$$
\end{proposition}

\begin{proof}
	By \cref{L: alternative transversality}, we know $f \times g$ is transverse to $\diag$, so the pullback is defined, and it is a manifold with corners by \cref{pullback}.
	This pullback by $\diag$ is $$P' = \{(v,w,z) \in V \times W \times M \mid (f(v),g(w)) = \diag(z) = (z,z)\},$$ which is diffeomorphic to $$P = V \times_M W = \{(v,w) \in V \times W \mid f(v) = g(w)\}$$ via the projection $(v,w,z) \mapsto (v,w)$ with inverse $(v,w) \mapsto (v,w,f(v))$.
	So we consider the co-orientations, for which we may assume as usual that all manifolds are without boundary by \cref{pullback,P: interior co-orientation}.

	By \cref{L: Quillen product co-orientation}, now with $M = N$, if $f$ and $g$ are co-oriented (at appropriate points) by $(\beta_V,\beta_M)$ and $(\beta_W,\beta_M)$ and if we we take Quillen co-orientations coming from $V \xhookrightarrow{e_V} M \times \R^a \to M$ and $W \xhookrightarrow{e_W} M \times \R^b \to M$ (with $a$ and $b$ assumed even) by orienting $\nu V$ and $\nu W$ in $M \times \R^a$ and $M \times \R^b$, then the co-orientation of the product $f \times g$ has Quillen co-orientation with the normal bundle to $V \times W$ in $M \times M \times \R^{a+b}$ oriented by $\beta_{\nu V} \wedge \beta_{\nu W}$.
	Recall that we interpret this expression so that $\nu V$ is our standard normal bundle to $V$ in $M \times \R^a$ embedded into the first and third factors of $M \times M \times \R^a \times \R^b$ and analogous for $\nu W$.

	Pulling back by the diagonal, we thus obtain from \cref{D: pullback coorient} that the pullback co-orientation is $(\beta_{P'},\beta_M)$ if and only if $\beta_{P'} \wedge \beta_{\nu V} \wedge \beta_{\nu W} = \beta_M \wedge \beta_{a+b}$.
	On the other hand, using the same Quillen co-orientation for $V$, the co-orientation of the fiber product $V \times_M W \to W \to M$ is $(\beta_P,\beta_M)$ if and only if $\beta_P \wedge \beta_{\nu V} = \beta_W \wedge \beta_a$.

	Consider now the following computation, where we use that $\beta_W \wedge \beta_{\nu W} = \beta_{M\times \R^b} = \beta_M \wedge \beta_b$ from the definition of the Quillen orientation of the normal bundle:
	\begin{align*}
		\beta_P \wedge \beta_{\nu V} \wedge \beta_{\nu W}& = \beta_W \wedge \beta_a \wedge \beta_{\nu W}\\
		& = \beta_W \wedge \beta_{\nu W} \wedge \beta_a\\
		& = \beta_M \wedge \beta_b \wedge \beta_a \\
		& = \beta_M \wedge \beta_{a+b}\\
		& = \beta_{P'} \wedge \beta_{\nu V} \wedge \beta_{\nu W}.
	\end{align*}
	This appears to demonstrate that, if we fix $\beta_M$, we get corresponding $\beta_P$ and $\beta_P'$. 
	The trouble with this argument, however, is that we've been very free about what spaces exactly all these various local orientations live over.
	Even with our given identification of $P$ and $P'$, the symbol $\nu V$ here represents a normal bundle in at least two or three different spaces. 
	To solidify the argument, we need a bit more care.

	First, we recall that our model of $P = V \times_M W$ when computing pullback co-orientations is technically $V \times_{M \times \R^a} (W \times_M \R^a)$, identified with $V \times_M W$ in the obvious way, and similarly for $P'$; see \cref{R: pullback representative 2}. 
	It will be convenient here to use these representations, so, relabeling for convenience and letting $e_V(v) = (f(v),e_a(v))$ and $e_W(w) = (g(w),e_b(w))$, we let
	\begin{align*}
		P &= \{(v,w,s)\in V\times W\times \R^a\mid (g(w),s)=e_V(v)\}\\
		P' &= \{(v,w,z,s,t)\in V\times W\times M \times \R^a \times \R^b \mid (z,z,s,t)=(f(v),g(w),e_a(v),e_b(w))\}.
	\end{align*}
	Then $P$ is diffeomorphic to our standard $V \times_M W$ via $(v,w) \leftrightarrow (v,w,e_a(v))$, and $P'$ is diffeomorphic to $(V \times W) \times_{M \times M} M$ via $(v,w,z) \leftrightarrow (v,w,z,e_a(v),e_b(w))$. Furthermore, $P \cong P'$ via $(v,w,s)\leftrightarrow (v,w,g(w),s,e_b(w))$. 

	We now consider the following diagram, in which the righthand horizontal maps are those occuring in the pullback co-orientation construction:
	\begin{diagram}
		P&\rInto^{\pi_{W\times \R^a}}& W \times \R^a &\rTo^{g \times \id_{\R^a}}& M\times \R^a\\
		\dTo^\cong && \dInto^{h} &&\uTo^{\Pi}\\
		P'&\rInto^{\pi_{M\times \R^a \times \R^b}}& M \times \R^a \times \R^b &\rInto^{\diag \times \id_{\R^a}\times \id_{\R^b}}& M \times M\times \R^a \times \R^b.
	\end{diagram}	
	The lefthand vertical map is our diffeomorphism $P \cong P'$. We let $h$ be the map $h(w,s) = (g(w),s,e_b(w))$, and we let $\Pi$ be the projection onto the first and third factors. 
	One readily checks that this diagram commutes and that the maps with hooks are embeddings. 
	
	Let $p = (v,w,s)\in P$, let $p' = (v,w,z,s,t)$ be the corresponding point in $P'$, so in particular  $z = f(v) = g(w)$. 
	Then taking derivatives at these points and their images, we obtain the diagram

	\begin{diagram}
		T_pP&\rInto^{D\pi_{W\times \R^a}}& T_wW \oplus T_s\R^a &\rTo^{D(g \times \id_{\R^a})}& T_zM \oplus T_s\R^a\\
		\dTo^\cong && \dInto^{Dh} &&\uTo^{\Pi}\\
		T_{p'}P'&\rInto^{D\pi_{M\times \R^a \times \R^b}}& T_zM \oplus T_s\R^a \oplus T_t\R^b &\rInto^{D(\diag \times \id_{\R^a}\times \id_{\R^b})}& T_zM \oplus T_zM\oplus T_s\R^a \oplus T_t\R^b.\\
	\end{diagram}

	Now the normal bundle $\nu V$ to $e_V(V) \subset M \times \R^a$ is our standard normal bundle for the construction of a co-orientation of $V \times_M W$, and we know it pulls back to the normal bundle of $P$, identified with $(g \times \id_{\R^a})^{-1}(e_V(V)) \subset W \times \R^a$. 
	
	By construction, we typically think of choosing a splitting so that we can identify a normal space $\nu V$ of $P$ in $W \times \R^a$ at $p$ with a subspace $N$ of $T_{w,s}(W \times \R^a)=T_wW \oplus T_s\R^a$ such that the restriction to $N$ of the composition of $D(g \times \id_{\R_a})$ with the quotient map to $T_{(g(w),s)}(M \times \R^a)/ T_{(g(w),s)}V$ is an isomorphism. 
	With such a choice, it follows that the composition the other way around the righthand square, followed by the quotient map, is also an isomorphism onto $T_{(g(w),s)}(M \times \R^a)/ T_{(g(w),s)}V$, and so we can identify the image of $N$ under $D(\diag \times \id_{\R^a}\times \id_{\R^b}) Dh$ with the fiber of the pullback of $\nu V$ by $\Pi$. But now we observe that if $(\xi,\eta) \in T_wW \oplus T_s\R^a$, then
	$$D(\diag \times \id_{\R^a}\times \id_{\R^b}) Dh(\xi,\eta) = (Dg(\xi),Dg(\xi), \eta, De_b(\xi)).$$
	But $(Dg(\xi),De_b(\xi)) = De_W(\xi)$, so $(0, Dg(\xi),0, De_b(\xi))$ is in the tangent space of our chosen embedding of $V \times W$ into $M \times M \times \R^a \times \R^b$. 
	So, the image $D(\diag \times \id_{\R^a}\times \id_{\R^b})Dh(N)$ represents our standard choice of a $\nu V$ summand in the normal bundle to $V \times W$, it simply corresponds to a different choice of splitting. 
	In particular, this image and the $\nu V$ summand from \cref{L: Quillen product co-orientation} project to isomorphic images in $T(M \times M \times \R^a \times \R^b)/T(V \times W)$. 

	The upshot is that we can thus identify our versions of $\nu V$ in $W \times \R^a$ and $M \times \R^a \times \R^b$ via $Dh$, and similarly identify $T_pP$ and $T_{p'}P'$ via the diagram. In fact, the lefthand square consist entirely of embeddings, we can now properly interpret the above computation as a computation entirely with elements of $T_zM \oplus T_s\R^a \oplus T_t\R^b$ with $\nu W$ being the normal from the embedding of $W$ into the first and third coordinates of $M \times \R^a \times \R^b$.
\end{proof}

\begin{corollary}\label{C: fiber natural pullback}
	Suppose $f \colon V \to M$ and $g \colon W \to M$ are transverse co-oriented maps of manifolds with corners to a manifold without boundary, that $N$ is a manifold with corners, and that $h \colon N \to M$ is transverse to $f$, $g$, and $f \times_M g \colon V \times_M W \to M$.
	Then
	$$h^*(V \times_M W) = h^*V \times_N h^*W$$
	as manifolds with co-oriented maps to $N$.
\end{corollary}

\begin{proof}
	Let $\diag_M \colon M \to M \times M$ and $\diag_N \colon N \to N \times N$ denote the diagonal maps.
	Using \cref{P: cross to cup,P: pullback functoriality,P: natural exterior}, and that $\diag_M h = (h \times h) \circ \diag_N$,
	we compute
	\begin{align*}
		h^*(V \times_M W)& = h^*\diag_M^*(V \times W)\\
		& = (\diag_M h)^*(V \times W)\\
		& = ((h \times h) \circ \diag_N)^*(V \times W)\\
		& = \diag_N^*(h \times h)^*(V \times W)\\
		& = \diag_N^*(h^*V \times h^*W)\\
		& = h^*V \times_N h^*W.
	\end{align*}
	To apply \cref{P: pullback functoriality} in the second line, we note that $h$ is transverse to $\diag_M^*(V \times W) = V \times_M W$ by assumption.
	For the fourth line we observe that $h \times h$ is transverse to $f \times g$ because $h$ is transverse to $f$ and $g$, and the composite $(h \times h) \circ \diag_N = \diag_M h$ is transverse to $f \times g$ by our assumptions and \cref{L: transverse to pullback}.
\end{proof}

\begin{corollary}[Associativity of co-oriented fiber products]\label{C: fiber assoc}
	Suppose $f \colon V \to M$, $g \colon W \to M$, and $h \colon X \to M$ are co-oriented maps from manifolds with corners to a manifold without boundary such that the following pairs are transverse (see \cref{R: multiproducts}): $(V,W)$, $(W,X)$, $(V \times_M W,X)$, and $(V,W \times_M X)$.
	Then $$(V \times_M W) \times_M X = V \times_M (W \times_M X)$$ as co-oriented fiber products mapping to $M$.
\end{corollary}

\begin{proof}
	We compute using \cref{P: cross to cup,P: pullback functoriality,P: exterior associativity,P: natural exterior} and that $(\id_M \times \diag)\diag = (\diag \times \id_M)\diag$:
	\begin{align*}
		(V \times_M W) \times_M X& = \diag^*(\diag^*(V \times W) \times X)\\
		& = \diag^*(\diag \times \id_M)^*((V \times W) \times X)\\
		& = ((\diag \times \id_M)\diag)^*((V \times W) \times X)\\
		& = ((\id_M \times \diag)\diag)^*(V \times (W \times X))\\
		& = \diag^*(\id_M \times \diag)^*(V \times (W \times X))\\
		& = \diag^*(V \times \diag^*(W \times X))\\
		& = V \times_M (W \times_M X).\qedhere
	\end{align*}
\end{proof}


\begin{proposition}[Graded commutativity of co-oriented fiber products]\label{P: graded comm}
	Suppose $f \colon V \to M$ and $g \colon W \to M$ are transverse co-oriented maps from manifolds with corners to a manifold without boundary.
	Then as co-oriented fiber products over $M$ we have $\omega_{g \times_M f} = (-1)^{(m-v)(m-w)} \omega_{f \times_M g}$, or, using Notation \ref{R: precise commutativity},
	$$V \times_M W = (-1)^{(m-v)(m-w)} W \times_M V.$$
\end{proposition}

\begin{proof}		
	By \cref{P: cross to cup}, the fiber products $V \times_M W \to M$ and $W \times_M V \to M$ are respectively the pullbacks of $f \times g \colon V \times W \to M \times M$ and $g \times f \colon W \times V \to M \times M$ by the diagonal map $\diag \colon M \to M \times M$. 
	Meanwhile, by \cref{P: exterior commutativity}, we have 
	$$\tau^*(V \times W) = (-1)^{(m-v)(m-w)}W \times V,$$
	where $\tau \colon M \times M \to M \times M$ is the map that interchanges the factors. 
	We observe that $\tau \diag = \diag$, so using the functoriality of pullbacks from \cref{P: pullback functoriality} we can compute
	\begin{align*}
	W \times_M V &= \diag^*(W \times V)\\
	&= (-1)^{(m-v)(m-w)} \diag^* \tau^*(V \times W)\\
	&= (-1)^{(m-v)(m-w)}(\tau \diag)^*(V \times W)\\
	&= (-1)^{(m-v)(m-w)}(\diag)^*(V \times W)\\
	&= (-1)^{(m-v)(m-w)} V \times_M W.\qedhere
	\end{align*} 
\end{proof}

\begin{example}\label{E: embedded}
	In \cref{E: V embedded}, we considered pullback co-orientations $V \times_M W \to W$ when $V \into M$ was embedded.
	In this example, we discuss the case where $W$ is embedded.

	Let $f \colon V \to M$ and $g \colon W \to M$ be transverse maps from manifolds with corners to a manifold without boundary.
	Suppose $f$ is co-oriented and $g \colon W \to M$ is an embedding, which we use throughout the example to identify $W$ with its image.
	Let $(x,y) \in V \times_M W$.
	For the remainder of the argument, we fix a local orientation $\beta_M$ at $f(x) \in M$, and we choose $\beta_V$ at $x$ so that $(\beta_V,\beta_M)$ is the co-orientation of $V$ at $x$.
	Furthermore, even though $g$ might not be co-oriented, let us choose a Euclidean neighborhood $U$ of $y$ in $W$ and an arbitrary co-orientation $(\beta_U,\beta_M)$ on the restriction of $g$ to $U$.
	This can be done as $U$ is contractible and so $g|_U$ is co-orientable.


	Although we are interested in $V \times_M U$, the definition of the pullback co-orientation makes it easier to work with $U \times_M V$ when $U$ is embedded; see \cref{E: V embedded}.
	As we have chosen a co-orientation for $U \into M$ and as $V \to M$ comes with a co-orientation, we can consider the fiber product $P_U = U \times_M V \to M$.
	As $U$ is embedded in $M$, we have $P_U = f^{-1}(U) \subset V$.
	If we choose an orientation $\beta_{\nu U}$ of the normal bundle to $U$ in $M$ at $y$ so that $\beta_U \wedge \beta_{\nu U} = \beta_M$, then by \cref{E: V embedded} and the definition of fiber product co-orientation, the map $P_U \xr{f} M$ is co-oriented by $(\beta_P,\beta_M)$, where $\beta_P$ is chosen so that $\beta_P \wedge \beta_{\nu U} = \beta_V$, as usual letting $\nu U$ here also stand for its pullback as a normal bundle of $P_U$ in $V$.
	We note that if we had chosen the opposite co-orientation for $U \into M$ then, as we have fixed $\beta_V$ and $\beta_M$, the result would be to replace our current $\beta_U$ with $-\beta_U$, which would result in also reversing the signs of $\beta_{\nu U}$ and $\beta_P$.
	In particular, the fiber product would have the opposite co-orientation.

	Next, we will apply graded commutativity.
	By \cref{P: graded comm}, we have $U \times_M V = (-1)^{(m-v)(m-w)}V \times_M U$ as co-oriented fiber products over $M$.
	By \cref{R: precise commutativity}, this means that the fiber product $V \times_M U \to M$ with its fiber product co-orientation corresponds to the fiber product $U \times_M V = f^{-1}(U) \xr{f} M$ with co-orientation at $(x,y)$ given by $(-1)^{(m-v)(m-w)}(\beta_P, \beta_M)$.
	But decomposing the fiber product as the pullback $V \times_M U \to U$ and the inclusion $U \into M$, we write $(-1)^{(m-v)(m-w)}(\beta_P, \beta_M)$ as the composite co-orientation
	$$(-1)^{(m-v)(m-w)}(\beta_P, \beta_M) = (-1)^{(m-v)(m-w)}(\beta_P, \beta_U)*(\beta_U, \beta_M).$$
	As $(\beta_U, \beta_M)$ is our chosen co-orientation for $g|_U$, the pullback co-orientation of $V \times_M U \to U$ must be $(-1)^{(m-v)(m-w)}(\beta_P, \beta_U)$.
	But we have already observed that if we had chosen the opposite co-orientation for $g|_U$, that would reverse the signs of both $\beta_U$ and $\beta_P$, so in either case we obtain the same pullback co-orientation for $V \times_M U \to U$.
	In other words, this description of the co-oriented pullback $V \times_M U \to U$ is independent of our choice of co-orientation for $g|_U$, and so it extends globally to give our pullback co-orientation of $V \times_M W \to W$.

	Summarizing then, as a space we have $P = V \times_M W = f^{-1}(W)$, and the pullback and fiber product are the first map and composite of $f^{-1}(W) \xr{f} W \into M$ (analogously to the case where $f$ was an embedding in \cref{E: V embedded}).
	Furthermore, fixing the co-orientation of $f$ as $(\beta_V, \beta_M)$, the co-orientation of the pullback $V \times_M W \to W$ is by $(-1)^{(m-v)(m-w)}(\beta_P, \beta_W)$, where, if we choose any local orientation $\beta_{\nu W}$ for the normal bundle of $W$ in $M$, then $\beta_W \wedge \beta_{\nu W} = \beta_M$ in $M$ and $\beta_P \wedge \beta_{\nu W} = \beta_V$ after pulling back to $V$.

	It is a nice exercise to confirm that this agrees with the computation of \cref{P: normal pullback} when $V$ and $W$ are both embedded.
\end{example}


\begin{corollary}\label{C: criss cross}
	Suppose $f \colon V \to M$ and $g \colon W \to M$ are transverse co-oriented maps from manifolds with corners to a manifold without boundary and similarly for $h \colon X \to N$ and $k \colon Y \to N$.
	Then$$(V \times X)\times_{M \times N} (W \times Y) = (-1)^{(m-w)(n-x)} (V \times_M W) \times (X \times_N Y) $$
	as co-oriented fiber products over $M \times N$.
%	Add a version for pullbacks (as opposed to fiber products)? Might have some extra signs to figure out.
%	Not really needed anywhere I do not think.}
\end{corollary}

\begin{proof}
	With our given transversality assumptions, $f \times h$ is transverse to $g \times k$, so the expression on the left is well defined.
	We then compute using \cref{P: cross to cup,P: pullback functoriality,P: exterior associativity,P: exterior commutativity,P: natural exterior} and letting $\tau$ here be the interchange of the interior $N$ and $M$ in the quadruple product:
	\begin{align*}
		(V \times X)\times_{M \times N} (W \times Y)& = \diag_{M \times N}^*(V \times X \times W \times Y)\\
		& = \diag_{M \times N}^*(\id \times \tau^* \times \id)^*(-1)^{(n-x)(m-w)}(V \times W \times X \times Y)\\
		& = (\diag_M \times \diag_N)^*(-1)^{(n-x)(m-w)}(V \times W \times X \times Y)\\
		& = (-1)^{(n-x)(m-w)}\diag_M^*(V \times W) \times \diag_N^*( X \times Y)\\
		& = (-1)^{(m-w)(n-x)} (V \times_M W) \times (X \times_N Y).
	\end{align*}
	For the third equality, we use that $\diag_M \times \diag_N = (\id_M \times \tau \times \id_N)\diag_{M \times N}$.
\end{proof}

\begin{corollary}\label{C: cross is cup}
	Let $V \to M$ and $W \to N$ be maps from manifolds with corners to manifolds without boundary.
	Let $\pi_M \colon M \times N \to M$ and $\pi_N \colon M \times N \to N$ be the projections.
	Then $$V \times W = \pi_M^*(V)\times_{M \times N}\pi_N^*(W)$$ as co-oriented manifolds mapping to $M \times N$.
\end{corollary}

\begin{proof}
	By \cref{P: projection pullbacks}, $\pi_M^*(V) = V \times N$ and $\pi_N^*(W) = M \times W$, so these are transverse as spaces mapping to $M \times N$.
	Then by \cref{C: criss cross,C: cup with identity}, we have
	\begin{align*}
		\pi_M^*(V)\times_{M \times N}\pi_N^*(W)& = (V \times N)\times_{M \times N} (M \times W)\\
		& = (V \times_M M) \times (N \times_N W)\\
		& = V \times W.\qedhere
	\end{align*}
\end{proof}

\subsection{Properties mixing orientations and co-orientations}

In this section we study properties that involve both orientations and co-orientations.
In particular, we are mostly interested in the pullback of a co-oriented map $V \to M$ by a map $W \to M$ with $W$ oriented, in which case the co-orientation of the pullback $V \times_M W \to W$ together with the orientation of $W$ produces an induced orientation on $V \times_M W$ as described in \cref{S: co-orientations}.
As $V \times_M W \to M$ with this orientation will eventually correspond to the cap product when we get to geometric homology and cohomology, we will here refer to this orientation as the \textbf{cap orientation}.

The following results all concern cap orientations on $V \times_M W$.
We note that, by construction, this oriented manifold comes equipped with a map to $W$ and, by composing with the given map $W \to M$, a map to $M$.

We start with the next result, which is yet another Leibniz formula. 
This will lead us eventually to a boundary formula for the cap product of geometric chains and cochains that agrees with that for the boundary of cap products of singular chains and cochains in Spanier \cite[Section 5.6.15]{Span81} and Munkres \cite[Section 66]{Mun84}.
It will also be used in \cref{S: intersection map} to demonstrate that our intersection map $\mc I$, relating geometric cochains to cubical cochains of a cubulation, is a chain map.
This map is a critical component in relating geometric cohomology to other cohomology theories and is also central to the main result about cup products in \cite{FMS-flows}.

\begin{proposition}\label{P: Leibniz cap}
	Let $f \colon V \to M$ and $g \colon W \to M$ be transverse maps of manifolds with corners to a manifold without boundary.
	Suppose $f$ is co-oriented and $W$ is oriented.
	Then $$\bd(V \times_M W) = \left[(-1)^{v+w-m} (\bd V) \times_M W\right] \bigsqcup (V \times_M \bd W)$$
	as oriented manifolds, giving $V \times_M W$, $(\bd V) \times_M W$, and $V \times_M \bd W$ each their cap orientations.
	In the last case, this is the cap orientation of the pullback of $V$ by $\bd W \xr{i_{\bd}} W \xr{g} M$, with $\bd W$ having its boundary orientation as the boundary of $W$.
\end{proposition}

\begin{proof}
	We compute and compare these orientations by first considering the pullback co-orientations as defined in \cref{D: pullback coorient}.
	We proceed by analogy to the proof of the Leibniz rule for the pullback of co-oriented maps in \cref{leibniz}, utilizing the computations already performed there. 
	Recall that, as needed, we can assume by working locally that $V$ and $W$ are manifolds without boundary or manifolds without strata of greater depth.

	Recall, in brief, from \cref{D: pullback coorient} that to co-orient the pullback $P = V \times_M W \to W$ we first construct a composition $V \xhookrightarrow{e} M \times \R^N \to M$ and find a Quillen orientation for the normal bundle $\nu V$ of $e(V) \subset M \times \R^N$ as determined by the co-orientation of $V \to M$.
	Then we pull back via $W \times \R^N \to M \times \R^N$ to obtain a normal bundle, also labeled $\nu V$, of $P \subset W \times \R^N$.
	Then we co-orient $P \to W$ locally by $(\beta_P,\beta_W)$ so that $\beta_P \wedge \beta_{\nu V} = \beta_W \wedge \beta_E$, where $\beta_E$ represents the standard orientation of $\R^N$.
	In the case at hand, we can assume $\beta_W$ to represent the global orientation of $W$, and then $\beta_P$ becomes the global cap orientation for $P = V \times_M W$.

	Let $\nu\bd P$ denote an outward pointing normal vector in the tangent bundle to $P$ at a boundary point of $P$, and let $\beta_{\nu\bd P}$ denote the corresponding orientation.
	Then, by definition, $\bd P$ is oriented at that point by $\beta_{\bd P}$ so that $\beta_{\nu\bd P} \wedge \beta_{\bd P} = \beta_P$.
	In other words, with $\beta_P$, $\beta_W$, $\beta_{\nu V}$, and $\beta_E$ given as above, $\bd P$ is oriented by $\beta_{\bd P}$ such that $\beta_{\nu\bd P} \wedge \beta_{\bd P} \wedge \beta_{\nu V} = \beta_W \wedge \beta_E$.

	Now, recall that $\bd(V \times_M W) = (\bd V) \times_M W \bigsqcup V \times_M \bd W$ as spaces and consider a point in $(\bd V) \times_M W$.
	By \cref{leibniz}, at such a point the pullback co-orientation of $(\bd V) \times_M W \to W$ (as the pullback of $\bd V \to M$ by $g$) agrees with boundary co-orientation of the pullback $P = V \times_M W \to W$.
	So continuing to let $(\beta_P,\beta_W)$ denote the pullback co-orientation of $P \to W$ and recalling that the boundary co-orientation utilizes the \textit{inward} normal, the boundary co-orientation of $(\bd V) \times_M W \to W$ is the composite $(\beta_{\bd P}, \beta_{\bd P} \wedge -\beta_{\nu\bd P})*(\beta_P,\beta_W)$ for any $\beta_{\bd P}$.
	But if we choose $\beta_{\bd P}$ to represent the orientation of $\bd P$ found above by orienting $P$ and then taking its boundary orientation, we have $\beta_P = \beta_{\nu\bd P} \wedge \beta_{\bd P} = (-1)^{\dim(\bd P)}\beta_{\bd P} \wedge \beta_{\nu\bd P}$.
	So the boundary co-orientation of $(\bd V) \times_M W \to W$ is the composite
	$$(\beta_{\bd P}, \beta_{\bd P} \wedge -\beta_{\nu\bd P})*((-1)^{\dim(\bd P)}\beta_{\bd P} \wedge \beta_{\nu\bd P},\beta_W) = (-1)^{\dim(\bd P)+1}(\beta_{\bd P},\beta_W).$$
	Thus the resulting cap orientation of $(\bd V) \times_M W$ is $(-1)^{\dim(P)}\beta_{\bd P}$, which is $(-1)^{\dim(P)} = (-1)^{v+w-m}$ times the orientation of $\bd P$ obtained by taking the oriented boundary of $V \times_M W$.

	Next we consider a point in $V \times_M \bd W$.
	Note that by \cref{L: normal pullback} we can consider our outward pointing normal vector $\nu\bd P$ at a point of $\bd P$ to also be an outward pointing normal of $\bd W$; below we write $\nu \bd P = \nu \bd W$.
	Again from the definition of pullback co-orientations, the co-orientation of the pullback $V \times_M \bd W \to \bd W$ is $(\beta_{\bd P},\beta_{\bd W})$ when $\beta_{\bd P} \wedge \beta_{\nu V} = \beta_{\bd W} \wedge \beta_E$. 
	If we here give $\bd W$ its boundary orientation, this determines the cap orientation as $\beta_{\bd P}$.
	Wedging with $\beta_{\nu\bd P}$ and using the definition of boundary orientation of $W$, we thus have 
	$$\beta_{\nu\bd P} \wedge \beta_{\bd P} \wedge \beta_{\nu V} = \beta_{\nu\bd P} \wedge \beta_{\bd W} \wedge \beta_E =  \beta_{\nu\bd W} \wedge \beta_{\bd W} \wedge \beta_E = \beta_W \wedge \beta_E.$$
	But this was exactly our condition above for $\beta_{\bd P}$ to be the boundary orientation of $P$ when $P$ is given its cap orientation. 
\end{proof}

Next we describe the cap orientation when $V \to M$ and $W \to M$ are embeddings.
As we've observed in the cases where either both maps are oriented or both maps are co-oriented, this is often an instructive and important example.

\begin{proposition}\label{P: cap of immersions}
	Let $f \colon V \to M$ and $g \colon W \to M$ be transverse embeddings from manifolds with corners to a manifold without boundary.
	Suppose $f$ is co-oriented and $W$ is oriented.
	Then $P = V \times_M W$ is just the intersection of $V$ and $W$ in $M$.
	If $\beta_W$ is the orientation of $W$ and $\beta_{\nu V}$ is the Quillen orientation of the normal bundle to $V$ in $M$, which at points of $P$ we can identify\footnote{See \cref{L: normal pullback}.} with the normal bundle to $P$ in $W$, then the cap orientation $\beta_P$ of $P$ satisfies $\beta_P \wedge \beta_{\nu V} = \beta_W$.
	If $f$ and $g$ are immersions, then this description holds locally.
\end{proposition}

\begin{proof}
	As $f$ is an embedding, we can take $N = 0$ in the definition of the pullback co-orientation, \cref{D: pullback coorient}.
	Then the pullback is just the inclusion of $P = g^{-1}(V) = V \cap W$ into $W$, and by definition the pullback co-orientation has the form $(\beta_P,\beta_W)$, where $\beta_P \wedge \beta_{\nu V} = \beta_W$ and $\nu V$ here is the pullback of the normal bundle of $V$ in $M$ to be the normal bundle of $V \cap W$ in $W$.
	Furthermore, if we take $\beta_W$ to be the given orientation of $W$, then $\beta_P$ is the cap orientation on the intersection by definition.
	The last statement about immersions follows as we can compute the co-orientations locally.
\end{proof}

The following corollary is particularly important and follows immediately from \cref{P: cap of immersions}.

\begin{corollary}\label{C: complementary cap}
	Let $f \colon V \to M$ and $g \colon W \to M$ be transverse embeddings from manifolds with corners to a manifold without boundary.
	Suppose $f$ is co-oriented, $W$ is oriented, and $\dim(V) + \dim(W) = \dim(M)$.
	Then $V \times_M W$ is the union of intersection points of $V$ and $W$.
	Such a point $x \in V \cap W$ is positively oriented if and only if the Quillen orientation of the normal bundle $\nu V$ of $V$ at $x$ agrees with the orientation of $W$ at $x$, identifying the fiber of $\nu V$ at $x$ with $T_xW$.
	If $f$ and $g$ are immersions, then this description holds locally.
\end{corollary}

The next two propositions will eventually correspond to the unital identities for the cap product for geometric chains and cochains.
The analogues for singular chains and cochains are the cap product with the cochain $1$ and the cap product with a chain representing the fundamental class, though in this case our underlying spaces do not need to be compact.

\begin{proposition}\label{P: cap with 1}
	Let $g \colon W \to M$ be a map from an oriented manifold with corners to a manifold without boundary, and consider $M \to M$ as the identity with the tautological co-orientation.
	Then $M \times_M W = W$ as oriented manifolds.
\end{proposition}

\begin{proof}
	By definition, there is a Quillen co-orientation for $M$ consisting of the sequence of identity maps $M \into M \to M$ with the normal bundle to $M$ in itself being the $0$-dimensional vector bundle, which we consider to have positive orientation at each point.
	It follows from the definition of the pullback that the corresponding Quillen co-orientation for $M \times_M W$ comes from the sequence $g^{-1}(M) = W \into W \to W$ with $W$ also having a $0$-dimensional positively-oriented normal bundle in itself.
	Consequently, the pullback co-orientation for $W \to W$ is the tautological one, and so the induced orientation on $W$ is the given one.
\end{proof}

\begin{proposition}\label{P: cap with identity M}
	Let $f \colon V \to M$ be a co-oriented map from a manifold with corners to an oriented manifold without boundary, and consider $M$ equipped with its identity map $M \to M$.
	Then $V \times_M M$ is $V$ with its induced orientation.
\end{proposition}

\begin{proof}
	This follows directly from \cref{C: cup with identity} and the definitions.
\end{proof}

The next property relates products of pullbacks with pullbacks of products. Again, our signs agree with the cap product formulas in Spanier, in this case \cite[Section 5.6.21]{Span81}.

\begin{proposition}\label{P: cap cross}
	Let $f \colon V \to M$ and $g:X \to N$ be co-oriented maps from manifolds with corners to manifolds without boundary, and let $h \colon W \to M$ and $k \colon Y \to N$ be maps with $W$ and $Y$ oriented manifolds with corners.
	Suppose that $V$ is transverse to $W$ and that $X$ is transverse to $Y$.
	Then,
	$$(V \times X)\times_{M \times N} (W \times Y) = (-1)^{(x+y-n)(m-v)} (V \times_M W) \times (X \times_N Y),$$
	as oriented manifolds with the pullbacks given their cap orientations.
\end{proposition}

\begin{proof}
	Let $\beta_W$ and $\beta_Y$ denote the orientations of $W$ and $Y$.
	Then $W \times Y$ is oriented by $\beta_W \wedge \beta_Y$.

	Now let $P = V \times_M W$ and $P' = X \times_N Y$.
	By definition, $P$ and $P'$ are oriented by the orientations $\beta_P$ and $\beta_{P'}$ such that $(\beta_P,\beta_W)$ and $(\beta_{P'},\beta_Y)$ are the pullback co-orientations for $P \to W$ and $P' \to Y$.
	Then $P \times P'$ is oriented by $\beta_P \wedge \beta_{P'}$.

	Furthermore, using our construction of pullback co-orientations, $\beta_P$ and $\beta_{P'}$ are such that $\beta_P \wedge \beta_{\nu V} = \beta_W \wedge \beta_a$ and $\beta_{P'} \wedge \beta_{\nu X} = \beta_Y \wedge \beta_b$, where $\beta_a$ and $\beta_b$ are the standard orientations of the Euclidean spaces $\R^a$ and $\R^b$ and we are free to take $a$ and $b$ to be even integers.

	By \cref{L: Quillen product co-orientation}, we have that the Quillen co-orientation of $V \times X \to M \times N$ is represented by an embedding $V \times X \into M \times N \times \R^a \times \R^b$ with normal bundle $\nu V \oplus \nu X$ suitably interpreted.
	So letting $Q = (V \times X)\times_{M \times N} (W \times Y)$, the orientation $\beta_Q$ is the one such that $(\beta_Q,\beta_W \wedge \beta_Y)$ is the pullback co-orientation, i.e.\ the one such that $\beta_Q \wedge \beta_{\nu V} \wedge \beta_{\nu X} = \beta_W \wedge \beta_Y \wedge \beta_a \wedge \beta_b$.
	But then we compute, using $a$ and $b$ even,
	\begin{align*}
		\beta_W \wedge \beta_Y \wedge \beta_a \wedge \beta_b& = \beta_W \wedge \beta_a \wedge \beta_Y \wedge \beta_b\\
		& = \beta_P \wedge \beta_{\nu V} \wedge \beta_{P'} \wedge \beta_{\nu X}\\
		& = (-1)^{|P'||\nu V|}\beta_P \wedge \beta_{P'} \wedge \beta_{\nu V} \wedge \beta_{\nu X}\\
		& = (-1)^{(x+y-n)(m-v)}\beta_P \wedge \beta_{P'} \wedge \beta_{\nu V} \wedge \beta_{\nu X}.
	\end{align*}
	So $\beta_Q = (-1)^{(x+y-n)(m-v)}\beta_P \wedge \beta_{P'} = (-1)^{(x+y-n)(m-v)}\beta_{P \times P'}$.
\end{proof}

The following technical lemma will be used to prove \cref{P: OC mixed associativity}, which will eventually become the associativity relation among cup and cap products, i.e.\ $(a \smile b) \frown x = a \frown (b \frown x)$.

\begin{lemma}\label{L: same induced}
	Let $f \colon V \to M$ and $g \colon W \to M$ be transverse maps from manifolds with corners to a manifold without boundary.
	Suppose that $f$ is co-oriented and that $W$ and $M$ are oriented, with respective (global) orientations $\beta_W$ and $\beta_M$.
	Suppose we co-orient $g$ by $(\beta_W,\beta_M)$.
	Then the cap orientation of $V \times_M W$ (i.e.\ that induced from the the pullback co-orientation of $V \times_M W \to W$ and the orientation of $W$) is the same as the orientation induced on $V \times_M W$ by the fiber product co-orientation of $V \times_M W \to M$ and the orientation of $M$.
	In particular, this orientation does not depend on the orientation of $M$.
\end{lemma}

\begin{proof}
	By definition, the orientation of $V \times_M W$ induced from the orientation of $W$ and the pullback co-orientation of $V \times_M W \to W$ is the orientation $\beta_P$ such that $(\beta_P,\beta_W)$ is the pullback co-orientation.
	But then the fiber product co-orientation is the composite $(\beta_P,\beta_W)*(\beta_W,\beta_M) = (\beta_P,\beta_M)$.
	So the orientation induced by the orientation of $M$ and the composite co-orientation is again $\beta_P$.
\end{proof}

\begin{proposition}[Mixed associativity]\label{P: OC mixed associativity}
	Let $f \colon V \to M$ and $g \colon W \to M$ be co-oriented maps from manifolds with corners to a manifold without boundary.
	Let $h \colon Z \to M$ be a map with $Z$ an oriented manifold with corners.
	Then, assuming sufficient transversality for all terms to be defined (see \cref{R: multiproducts}),
	$$(V \times_M W) \times_M Z = V \times_M (W \times_M Z),$$
	as oriented manifolds.
	Here, on the left, $V \times_M W$ has its fiber product co-orientation, so we can form the cap orientation of the pullback over $Z$.
	On the right we first give $W \times_M Z$ its cap orientation as a pullback over $Z$ and then use that to form the cap orientation of $V \times_M (W \times_M Z)$.
\end{proposition}


\begin{proof}
	First suppose $M$ is orientable and that we have given it an arbitrary, but fixed, orientation.
	Then applying \cref{L: same induced}, the cap orientation of $(V \times_M W) \times_M Z$ is the same as the orientation induced by the orientation of $M$ and the fiber product co-orientation $(V \times_M W) \times_M Z \to M$, after co-orienting $Z \to M$ with the co-orientation induced by the orientations of $Z$ and $M$.
	Similarly, applying \cref{L: same induced} twice, the cap orientation of $V \times_M (W \times_M Z)$ is the same as that induced by the orientation of $M$ and the iterated fiber product co-orientation $V \times_M (W \times_M Z) \to M$ again coming from the canonical co-orientation of $Z \to M$.
	But now these co-oriented fiber products are the same by \cref{C: fiber assoc}.

	Next, suppose $M$ is not necessarily orientable.
	We know from their constructions that $(V \times_M W) \times_M Z$ and $V \times_M (W \times_M Z)$ are oriented manifolds, and it is not difficult to see that they are diffeomorphic, both being diffeomorphic to $\{(v,w,z) \in V \times W \times Z \mid f(v) = g(w) = h(z)\}$.
	So it suffices to consider these as identical spaces and to show that their induced orientations agree at any arbitrary point.
	If $(v,w,z)$ is such a point, consider its image $a = f(v) = g(w) = h(z) \in M$.
	Let $U$ be a Euclidean neighborhood of $a$, and consider the restrictions of $f$, $g$, and $h$ to $f^{-1}(U)$, $g^{-1}(U)$, and $h^{-1}(U)$.
	The resulting products over $U$ give us the pieces of $(V \times_M W) \times_M Z$ and $V \times_M (W \times_M Z)$ over $U$, and the resulting orientations will be compatible with those of the full manifolds $(V \times_M W) \times_M Z$ and $V \times_M (W \times_M Z)$, as orientations and co-orientations of fiber products are determined locally (see \cref{R: local pullback co-orientations} and the construction of fiber product orientations).
	But as $U$ is orientable, the preceding argument shows that these orientations must agree with each other.
\end{proof}

The following property will eventually manifest itself in geometric (co)homology as the familiar naturality formula for cap products $f_*(f^*(\alpha)\frown x)) = \alpha\frown f_*(x)$.

\begin{proposition}\label{P: natural cap}
	Let $f \colon V \to M$ and $h \colon N \to M$ be transverse maps with $f$ co-oriented, $V$ a manifold with corners and $M$ and $N$ manifolds without boundary.
	Furthermore, let $g \colon W \to N$ be a map from an oriented manifold with corners that is transverse to the co-oriented pullback $V \times_M N \to N$.
	Then the cap orientation induced on $(V \times_M N) \times_N W$ by pulling back the co-oriented map $V \times_M N \to N$ over $W \to N$
	is the same as the cap orientation obtained by pulling back $V \to M$ by the composite $hg \colon W \to M$.
	In other words, $(V \times_M N) \times_N W = V \times_M W$ as oriented manifolds.
\end{proposition}

\begin{proof}
	Note that $V$ is transverse to $hg \colon W \to M$ by \cref{L: transverse to pullback}, so both expressions are defined.
	It follows directly from \cref{P: pullback functoriality} that the two pullback co-orientations we have described for $(V \times_M N) \times_N W \to W$ agree.
	Therefore, the induced cap orientations agree.
\end{proof}

\subsubsection{Comparing the oriented and co-oriented fiber products}

Suppose $f \colon V \to M$ and $g \colon W \to M$ are two transverse co-oriented maps from manifolds with corners to a manifold without boundary.
Further, suppose $M$ oriented.
Then we know from the discussion in \cref{S: co-orientations} that there is a bijection between co-orientations of $f$ and orientations of $V$; an orientation of $V$ induces a co-orientation of $f$ and vice versa.
Of course the same is true of $W$ and $g$.
In this scenario, we have two different ways to orient $V \times_M W$, depending on whether we start by thinking of $V$ and $W$ as oriented or by thinking of $f$ and $g$ as co-oriented.
If we think of $V$ and $W $as oriented, we have the fiber product orientation of $V \times_M W$
discussed in \cref{S: orientations}.
Alternatively, if we think of $f$ and $g$ as co-oriented, we can form the fiber product co-orientation of $V \times_M W \to M$ as in \cref{S: co-orientation of pullbacks} and then consider the induced orientation given the orientation of $M$.

Our goal in this section is to compare these two orientations on $V \times_M W$.
To attempt to avoid confusion, we will write $V \times_M ^oW$ for the fiber product orientation of \cref{S: orientations} and $V \times_M ^cW$ for the co-oriented fiber product or, equivalently, the resulting induced orientation.

The reader might have noticed that a third way to orient $V \times_M W$ is to consider $f$ to be co-oriented and $W$ to be oriented and then form the cap orientation that we studied in detail in the preceding section.
However, we already know this to be identical to $V \times_M ^cW$ by \cref{L: same induced}.
By contrast, these are not always the same as $V \times_M ^oW$.

When we move on to geometric homology, $V \times_M ^oW$ will correspond to the classical intersection product of homology classes, as described for example in \cite[Section VI.11]{Bred97}, while $V \times_M ^cW$ will correspond to the cup product of cohomology classes.
When $M$ is closed and oriented, switching between thinking of $V$ and $W$ as oriented vs. co-oriented will be precisely the Poincar\'e duality isomorphism.
So this proposition will ultimately demonstrate that the intersection product is Poincar\'e dual to the cup product, up to a sign; see \cref{S: PD}.

\begin{proposition}\label{P: compare cup and intersection orientations}
	Let $f \colon V \to M$ and $g \colon W \to M$ be transverse co-oriented maps from manifolds with corners to an oriented manifold without boundary or, equivalently, suppose $V$, $W$, and $M$ all oriented.
	Then $$V \times_M ^oW = (-1)^{(m-v)(m-w)} V \times_M ^cW$$ as oriented manifolds with corners.
\end{proposition}

The proof will take a bit of work.
Our strategy will be as follows. 
We recall by \cref{pullback,P: interior co-orientation} that co-orientations are determined entirely by what happens on the pullbacks of the interiors, and this is also a standard fact for orientations of topological manifolds with boundary. 
So it suffices in what follows to consider only manifolds without boundary.	

First, we will prove in \cref{L: compare cup and intersection for immersions} that the result holds when $f$ and $g$ are immersions.
Then we will show, first for co-orientations and then for orientations, that we can replace the fiber product of
$$V \xr{f} M \xleftarrow{g} W$$
with the fiber product
$$V \times \R^b \xhookrightarrow{e \times \id_{\R^b }} M \times \R^a  \times \R^b  \hookleftarrow W \times \R^a ,$$
where $a$ and $b$ are even, the map $e$ is the embedding $V \into M \times \R^a $ of a Quillen co-orientation of $f$, and the leftward arrow is the identity between the $\R^a $ factors and takes the $W$ factor into $M \times \R^b $ by the embedding map of a Quillen co-orientation of $g$.
By ``replace,'' we mean in the oriented case that the two fiber products are canonically oriented diffeomorphic.
In the co-oriented case we mean that we have a canonical oriented diffeomorphism
between the domains of the two co-oriented fiber products, oriented with their induced orientations coming respectively from the orientation of $M$ and from the concatenation orientation of $M \times \R^a  \times \R^b $ using the standard orientations of the Euclidean terms.
The maps of this second fiber product are embeddings for which the proposition holds by \cref{L: compare cup and intersection for immersions}, and so the general case will follow.
The even dimensions of the Euclidean factors are chosen to avoid some extraneous signs in the arguments below.

Before proceeding, let us explain in more detail what we mean by ``canonical'' here and below.
Recall that, as a space, $P = V \times_M W$ can be identified with $\{(v,w) \in V \times W \mid f(v) = g(w)\}$.
Below we will see various fancier embeddings of $P$ in spaces of the form $V \times X \times W \times Y$, with $X$ and $Y$ Euclidean or $I$.
For each such embedding, the projection to $V \times W$ will take the embedding of $P$ back to $P$.
In this way, all versions of $P$ can be canonically identified, and it is these identifications that will yield our orientation preserving diffeomorphisms.
Such identifications have already been discussed in \cref{R: pullback representative,R: pullback representative 2}.
In the latter remark, we provide exactly such a canonical identification between our standard realization of $V \times_M W$ as a subset of $V \times W$ and the version used for co-orienting pullbacks and fiber products.

\begin{comment}
	is already contained in the construction of the co-oriented fiber product, where $V\times^c_MW$ can be considered to be contained in $V \times W \times \R^N$.
	Explicitly we constructed $P = (g \times \id_{\R^N})^{-1}(e(V))$, but as $e$ is an embedding, the $V$ factor of points in $P$ is implicitly determined by $(w,z) \in P \subset W \times \R^N$ as $e^{-1}((g(w),z))$.
	So $P = (g \times \id_{\R^N})^{-1}(e(V))$ is canonically identified with $\{(v,w,z) \in V \times W \times \R^N \mid e(v) = (g(w),z)\}$.
	And as $e$ is part of a Quillen co-orientation, by projecting $M \times \R^N$ to $M$, we have of course $g(w) = f(v)$, so there is a canonical map $\{(v,w,z) \in V \times W \times \R^N \mid e(v) \to \{(v,w) \in V \times W \mid f(v) = g(w)\}$ given by projection to the first two factors.
	This is a diffeomorphism, whose inverse takes $(v,w)$ to $(v,w,z)$ such that $e(v) = (g(w),z)$.

	\red{Put some of this as a remark in the section where pullback of co-orientation is defined.}
\end{comment}

\medskip

We begin with the case where $V$ and $W$ are immersions and then work toward the general case.

\begin{lemma}\label{L: compare cup and intersection for immersions}
	If $f$ and $g$ are transverse immersions of oriented manifolds without boundary into an oriented manifold without boundary, then $V \times_M ^oW = (-1)^{(m-v)(m-w)} V \times_M ^cW$.
\end{lemma}

\begin{proof}
	It suffices to consider small neighborhoods of interior points on which $f$ and $g$ are embeddings.
	Let $\beta_V$, $\beta_W$, and $\beta_M$ denote the local orientations of $V$, $W$, and $M$, respectively, at such a point.
	Assuming $f$ and $g$ to be co-oriented with the compatible co-orientations $(\beta_V,\beta_M)$ and $(\beta_W,\beta_M)$, we have the resulting Quillen orientations $\beta_{\nu V}$ and $\beta_{\nu W}$ of the normal bundles of $V$ and $W$.
	Recall that these are defined so that $(\beta_V,\beta_V \wedge \beta_{\nu V})$ and $(\beta_W,\beta_W \wedge \beta_{\nu W})$ are the co-orientations of $f$ and $g$, respectively.
	In this scenario, with $\beta_V$ and $\beta_W$ fixed as the orientations of $V$ and $W$, this is equivalent to requiring $\beta_V \wedge \beta_{\nu V} = \beta_M$ and $\beta_W \wedge \beta_{\nu W} = \beta_M$.
	Again, to keep the contexts clear, for the remainder of the argument we will write $\beta^c_{\nu V}$ and $\beta^c_{\nu W}$ for the Quillen orientations of $\nu V$ and $\nu W$.

	By \cref{P: normal pullback}, the co-orientation of the fiber product $V \times_M^c W$ is $(\beta_P,\beta_P \wedge \beta^c_{\nu V} \wedge \beta^c_{\nu W})$.
	In particular, the induced orientation of $V\times_m^c W$ is the orientation $\beta_P^c$ such that $\beta_P^c \wedge \beta^c_{\nu V} \wedge \beta^c_{\nu W} = \beta_M$.

	On the other hand, in \cref{P: orient intersection}, $\beta^o_P$ is such that if $\beta^o_P \wedge \beta^o_{\nu W} = \beta_V$ and $\beta^o_P \wedge \beta^o_{\nu V} = \beta_W$ then $\beta^o_P \wedge \beta^o_{\nu V} \wedge \beta^o_{\nu W} = \beta_M.$ A priori these may be different orientations of $\nu V$ and $\nu W$ than those of the preceding paragraph, so we use these alternate labels.
	In fact, let us suppose $\beta^o_P$, $\beta^o_{\nu V}$, and $\beta^o_{\nu W}$ chosen so that these expressions all hold.
	Then we have
	$$\beta_M = \beta^o_P \wedge \beta^o_{\nu V} \wedge \beta^o_{\nu W} = \beta_{W} \wedge \beta^o_{\nu W},$$
	so
	$$\beta_{\nu W}^o = \beta_{\nu W}^c.$$
	Similarly,
	$$\beta_M = \beta^o_P \wedge \beta^o_{\nu V} \wedge \beta^o_{\nu W} = (-1)^{(m-v)(m-w)}\beta^o_P \wedge \beta^o_{\nu W} \wedge \beta^o_{\nu V} = (-1)^{(m-v)(m-w)}\beta^o_V \wedge \beta^o_{\nu V},$$
	so
	$$\beta^0_{\nu V} = (-1)^{(m-v)(m-w)}\beta^c_{\nu V}.$$

	Thus
	$$\beta_M = \beta^o_P \wedge \beta^o_{\nu V} \wedge \beta^o_{\nu W} = (-1)^{(m-v)(m-w)}\beta^o_P \wedge \beta^c_{\nu V} \wedge \beta^c_{\nu W}.$$

	We conclude that $\beta^c_P = (-1)^{(m-v)(m-w)}\beta^o_P$.
\end{proof}

Now we show how to replace a general co-oriented fiber product with a co-oriented fiber product whose maps are embeddings.
In our remaining constructions in this section we take all introduced Euclidean spaces to be even-dimensional to simplify the signs in our computations.

\begin{lemma}
	Let $f \colon V \to M$ and $g \colon W \to M$ be transverse co-oriented maps from manifolds without boundary to an oriented manifold without boundary.
	Let $V\xhookrightarrow{e}M \times \R^a  \to M$ be a Quillen co-orientation for $f$ with $a$ even.

	Then $V \times_M W$ (with its orientation induced by the fiber product co-orientation and the orientation of $M$) is canonically oriented diffeomorphic to $V\times_{M \times \R^a }(W \times \R^a )$ (with its orientation induced by the fiber product co-orientation and the orientation of $M \times \R^a $).
	Here the maps for the second fiber product are $e \colon V \to M \times \R^a $ and $g \times \id_{\R^a } \colon W \times \R^a  \to M \times \R^a $.
	As in the construction of the Quillen co-orientation (\cref{D: Quillen normal or}), we assume $e \colon V \into M \times \R^a $ to be co-oriented so that its composition with the canonical co-orientation $(\beta_{M}\wedge\beta_a,\beta_M)$ is the co-orientation of $f$.
	We also take $g \times \id_{\R^a }$ to be co-oriented by the product co-orientation $(\beta_W \wedge \beta_a,\beta_M \wedge \beta_a)$ if $(\beta_W,\beta_M)$ is the co-orientation of $g$.
	Finally, $M \times \R^a $ is given the product orientation with $\R^a $ having the standard orientation.
\end{lemma}

\begin{proof}
	By the definitions of the induced orientation and the fiber product co-orientation, the co-orientation of the fiber product is obtained by identifying $V \times_M W$ with $(g\times \id_{\R^a})^{-1}(e(V)) \subset W \times \R^a$, and then induced orientation is $\beta_P$, where $\beta_P \wedge \beta_{\nu V} = \beta_W \wedge \beta_a$.
	We recall that the $\nu V$ in this formula is actually the pullback of the normal bundle of $e(V) \subset M \times \R^a $ via the map $g \times \id_{\R^a }: W \times \R^a  \to M \times \R^a $.
	But this is also exactly the description of the induced fiber product orientation from the co-oriented fiber product $V\times_{M \times \R^a }(W \times \R^a )$, treating $e \colon V \into M \times \R^a $ as its own Quillen co-orientation with $N = 0$ in \cref{D: pullback coorient}.
	In fact, the co-orientation of \cref{D: pullback coorient} is obtained precisely by identifying these two forms of the pullback.
\end{proof}

\begin{corollary}\label{C: co-oriented full transition to embedded}
	Let $f \colon V \to M$ and $g \colon W \to M$ be transverse co-oriented maps from manifolds without boundary to an oriented manifold without boundary.
	Let $V \xhookrightarrow{r} M \times \R^a  \to M$ and $W\xhookrightarrow{s} M \times \R^b  \to M$ be Quillen co-orientations compatible with $f$ and $g$.
	Then the fiber product $V \times_M W$ (with its orientation induced from the co-oriented fiber product) is canonically oriented diffeomorphic to the fiber product $(V \times \R^b ) \times_{M \times \R^a  \times \R^b }(W \times \R^a )$ (with its orientation induced from the co-oriented fiber product), in which the first map is $r \times \id_{\R^b }$ and the second map takes $(w,z) \in W \times \R^a $ to $s(w)$ in the first and third coordinates and $z$ in the second coordinate.
\end{corollary}

\begin{proof}
	By the preceding lemma we have $V \times_M W$ canonically oriented diffeomorphic to $V\times_{M \times \R^a }(W \times \R^a )$ with $V \to M \times \R^a $ an embedding.
	By \cref{P: graded comm}, the transposition map from this space to $(W \times \R^a ) \times_{M \times \R^a }V$ is $(-1)^{(m-v)(m-w)}$-orientation preserving, using that all our Euclidean spaces are taken even-dimensional.
	We next observe the map described for $W \times \R^a  \to M \times \R^a  \times \R^b $ is an embedding whose composition with the projection to $M \times \R^a $ gives a Quillen co-orientation for $g \times \id_{\R^a }$.
	Now we can apply the lemma again and then transpose again with the same sign, so that the signs cancel out.
\end{proof}

The preceding lemma and corollary concerned orientations obtained from co-orientations. 
Next we consider oriented manifolds and maps that are not necessarily co-oriented and show how to replace the oriented fiber product with an oriented fiber product whose maps are embeddings.
Again, we take all introduced Euclidean spaces to be even-dimensional to simplify the signs.

\begin{lemma}
	Let $f \colon V \to M$ and $g \colon W \to M$ be transverse maps from oriented manifolds without boundary to an oriented manifold without boundary.
	Let $V \xhookrightarrow{e}M \times \R^a  \to M$ be a factorization of $f$ with $e$ an embedding.
	Then $V \times_M W$ is canonically oriented diffeomorphic to the oriented fiber product $V\times_{M \times \R^a } (W \times \R^a )$ of $e \colon V \to M \times \R^a $ and $g \times \id_{\R^a } \colon W \times \R^a  \to M \times \R^a $.
\end{lemma}

\begin{proof}
	We will first show that $V \times_M W$ is canonically oriented diffeomorphic to the fiber product $V\times_{M \times \R^a } (W \times \R^a )$ with the map $V \to M \times \R^a $ being the composition of $f$ with the inclusion $M = M \times \{0\} \into M \times \R^a $.
	We will write this composite as $f_0$.
	The maps $f_0$ and $g \times \id_{\R^a }$ are transverse as $f$ and $g$ are transverse in $M$ and the $\id_{\R^a }$ factor takes care of the $\R^a $ factor of the tangent spaces.
	We also observe that the two fiber products are canonically the same as spaces, as $V \times_M W = \{(v,w) \in V \times W \mid f(v) = g(w)\}$, while the other fiber product is $\{(v,w,0) \in V \times W \times \R^a \mid f(v) = g(w)\}$.

	Now we consider the orientations. 
	Let us choose a point $p = (v,w) \in P = V \times_M W$ and let $z=f(v)=g(w)\in M$.
	As the tangent space of the pullback is the pullback of the tangent spaces by \cref{L: tangent of pullbacks}, we have $T_pP = T_vV \times_{T_z(M)} T_wW \subset T_vV \oplus T_wW$.
	By the definition of the fiber product orientation in \cref{S: orientation of fiber products}, we consider the map
	\begin{equation*}
		\Phi \colon T_pP \oplus T_zM \to T_vV \oplus T_wW
	\end{equation*}
	given by $\Phi((a,b),c) =(a,b)+s(c)$, where $s$ is a splitting of the map $T_vV \oplus T_wW \to T_zM$ that takes $(a,b)$ to $Df(a)-Dg(b)$.
	We recall from \cref{S: orientation of fiber products} that the orientation of $P$ is chosen so that $\Phi$ is an orientation preserving isomorphism up to a sign of $(-1)^{wm}$.

	In the case of $V\times_{M \times \R^a } (W \times \R^a )$, if we write $P'$ for the fiber product, 
	we have $T_{(p,0)}P' = T_vV \times_{T_z(M)} T_{(w,0)}(W \oplus \R^a) \subset T_vV \oplus T_{(w,0)}(W \oplus \R^a)$.
	As $Df_0 = (Df,0)$, we have $T_{(p,0)}P' = \{(a,b,0) \in T_vV \oplus T_{(w,0)}(W \oplus \R^a) \mid Df(a)=Dg(b)\}$, so $T_{(p,0)}P'$ is canonically isomorphic to $T_pP =  \{(a,b) \in T_vV \oplus T_wW  \mid Df(a)=Dg(b)\}$.
	In this case, we consider the map 
	\begin{equation}\label{E: fiber plus euclidean}
		\Psi: T_{(p,0)}P' \oplus T_zM \oplus T_0\R^a  \to T_vV \oplus T_wW \oplus T_0\R^a ,
	\end{equation}
	with the restriction of $\Psi$ to $T_{(p,0)}P'$ again being the projection maps, while the restriction to $T_zM \oplus T_0\R^a$ must be a splitting of the map $\Upsilon: T_vV \oplus T_{(w,0)}(W \oplus \R^a)  \to T_{(z,0)}(M \oplus \R^a)$ that is $Df_0$ on the first factor and $-D(g \oplus \id_{\R^a })$ on the last two factors.
	We claim that we can take $\Psi((a,b,0),c,u) = (a,b,0)+(s(c),0)-(0,0,u)$, with the splitting map $s$ as above.
	This is certainly correct on the $P$ factor.
	For the $M \oplus \R^a $ factor, we must show $\Upsilon \Psi(0,c,u) = (c,u)$.
	We have $\Psi(0,c,u) = (s(c),0)-(0,0,u)$, noting that $s(c) \in T_vV \oplus T_wW$.
	If we write $s(c) = (s_V(c),s_W(c))$, then by definition $Df(s_V(c))-Dg(s_W(c)) = c$.
	So we have
	\begin{align*}
		\Upsilon \Psi(0,c,u) & =
		\Upsilon((s(c),0)-(0,0,u))\\& =
		\Upsilon(s_V(c),s_W(c),-u)\\& =
		Df_0(s_V(c))-(Dg(s_W(c)),-u)\\& =
		(Df(s_V(c)),0)-(Dg(s_W(c)),-u)\\& =
		(Df(s_V(c))-Dg(s_W(c)),u)\\& =
		(c,u).
	\end{align*}
	So our definition of $\Psi$ suffices for determining the orientation of the fiber product.

	As $\Phi$ and $\Psi$ agree in the first two factors (identifying $T_pP$ and $T_{(p,0)}P'$ in the obvious way) and the dimension of $\R^a $ is even, we see that $\Psi$ is orientation-preserving if and only if $\Phi$ is.
	Furthermore, we have $(-1)^{w(m+a)} = (-1)^{wm}$, so the two fiber product orientations agree in this case.

	Next we must generalize from $f_0 \colon V \to M \times \R^a $ to the general case of $e \colon V \to M \times \R^a $.
	By assumption, $f \colon V \to M$ is the composition of $e$ with the projection $M \times \R^a  \to M$, so we may write $e(v) = (f(v), e_{\R}(v))$, and there is a fiberwise homotopy $H \colon V \times I \to M \times \R^a $ from $e$ to $f_0$ given by $H(v,t) = (f(v), te_{\R}(v))$.
	We note that $H$ and $e$ are each transverse to $g \times \id_{\R^a }$.
	Indeed, if $e(v) = (g \times \id_{\R^a })(w,u)$, then $(f(v),e_{\R}(v)) = (g(w),u)$.
	The image of the derivative of $g \times \id_{\R^a }$ at such a point will span $Dg(T_wW) \oplus T(\R^a )$, while the derivative of $e$ will have the form $(Df,De_{\R})$.
	But the image of $D(g \times \id_{\R^a })$ already spans $0\oplus \R^a $, so $De(T_vV)+D(g \times \id_{\R^a })T_{(w,u)} = (Df,0)T_vV+(Dg,0)T_wW+(0,\id_{\R^a })T_u\R^a = T_{(z,u)}(M \times \R^a )$.
	The same argument holds for $H(-,t)$ for any fixed $t$, replacing $De_{\R}$ with $tDe_{\R}$.
	But if each $H(-,t)$ is transverse to $g \times \id_{\R^a }$ then so is $H$.

	It follows that we can form the oriented fiber product of $H$ and $g \times \id_{\R^a }$ over $M \times \R^a $.
	In fact, this fiber product is diffeomorphic to $P \times I$: Noting that we have $H(v,t) = (f(v),te_{\R}(v)) = (g(w),u)$ if and only if $f(v) = g(w)$ and $te_{\R}(v) = u$, we obtain a diffeomorphism $P \times I \to (V \times I)\times_{M \times \R^a }(W \times \R^a )$ given by $((v,w),t) \mapsto ((v,t),(w, te_{\R}(v))$ with inverse $((v,t),(w,u)) \mapsto ((v,w),t)$.
	So this space is a cylinder and the two ends correspond to our two versions of $V\times_{M \times \R^a } (W \times \R^a )$, one mapping $V$ by $f_0$ and the other by $e$.
	Since we have a cylinder, these two end spaces are oriented diffeomorphic, and canonically so by our construction.

	Putting this oriented diffeomorphism together with the one constructed above gives the desired oriented canonical diffeomorphism with the original $V \times_M W$.
\end{proof}

\begin{corollary}\label{C: oriented full transition to embedded}
	Let $f \colon V \to M$ and $g \colon W \to M$ be transverse maps from oriented manifolds without boundary to an oriented manifold without boundary.
	Let $V\xhookrightarrow{r} M \times \R^a  \to M$ and $W\xhookrightarrow{s} M \times \R^b  \to M$ be factorizations of $f$ and $g$ with $r$ and $s$ embeddings.
	Then the oriented fiber product $V \times_M W$ is canonically oriented diffeomorphic to the fiber product $(V \times \R^b )\times_{M \times \R^a  \times \R^b }(W \times \R^a )$, in which the first map is $r \times \id_{\R^b }$ and the second map takes $(w,z) \in W \times \R^a $ to $s(w)$ in the first and third coordinates and $z$ in the second coordinate.
\end{corollary}

\begin{proof}
	As in the proof of \cref{C: co-oriented full transition to embedded},
	we apply the preceding lemma to get $V \times_M W$ canonically oriented diffeomorphic to $V\times_{M \times \R^a }(W \times \R^a )$.
	Then we use the graded commutativity rule for oriented fiber product, as given by \cref{P: commute oriented fiber},
	by which
	the transposition map to $(W \times \R^a )\times_{M \times \R^a }V$ is $(-1)^{(m-v)(m-w)}$-orientation preserving, using that all our Euclidean spaces are taken even-dimensional.
	Then we observe the map described for $W \times \R^a  \to M \times \R^a  \times \R^b $ is an embedding whose composition with the projection to $M \times \R^a $ is $g \times \id_{\R^b }$.
	Now we apply the lemma again and then transpose again.
\end{proof}

\begin{proof}[Proof of \cref{P: compare cup and intersection orientations}]
	By \cref{C: co-oriented full transition to embedded,C: oriented full transition to embedded}, the proposition reduces to \cref{L: compare cup and intersection for immersions}.
\end{proof}

\subsection{Appendix: Lipyanskiy's co-orientations}\label{S: Lipyanskiy co-orientations}

In \cite{Lipy14}, Lipyanskiy uses a different notion of co-orientation from the one we have used to define geometric cochains.
We here discuss Lipyanskiy's co-orientations, which he initially refers to as \textit{orientations of maps}, and show that for a smooth map $f \colon M \to N$, his definition is equivalent to our definition, up to possible sign conventions.
In other words, we show that a smooth map is co-orientable in our sense if and only if it is co-orientable in Lipyanskiy's sense.
We will not explore the precise differences between the specific co-orientation conventions.
We will also see that this alternative framework is in some sense dual to Quillen's approach to co-orientations that we presented in \cref{S: Quillen}: while Quillen's formulation involves replacing arbitrary maps with embeddings, the formulation here involves replacing arbitrary bundle maps with surjective bundle maps.

To define co-orientations, Lipyanskiy utilizes the determinant line bundles of Donaldson and Kronheimer in \cite[Section 5.2.1]{DoKr90}.
A key point throughout our discussion will be the following lemma, which is \cite[Lemma 5.2.2]{DoKr90}.
Donaldson and Kronheimer actually state (without proof) a stronger version of the lemma---that the isomorphism is canonical---but we will not need that for our purposes.
For the statement of the lemma, recall our definition of $\Or(V)$ in \cref{D: det bundle}.

\begin{lemma}\label{L: det sequence}
	Given an exact sequence of vector bundles
	\[
	0 \to V_1 \xr{d_1} \cdots \xr{d_{m-1}} V_m \to 0,
	\]
	there is an isomorphism
	\[
	\bigotimes_{i\ \text{odd}} \Or(V_i) \ \cong
	\bigotimes_{i\ \text{even}} \Or(V_i).
	\]
\end{lemma}

\begin{proof}
	It is implicit in the hypothesis of exactness that the kernels and image of the maps in the sequence are well-defined vector bundles so that it makes sense to say $\im(d_{i-1}) = \ker(d_{i})$ as objects in the category of vector bundles.
	We thus have for each $i$ a short exact sequence
	\[
	0 \to \ker(d_i) \to V_i \to \im(d_i) \to 0,
	\]
	and since short exact sequences of vector bundles split \cite[Theorem 3.9.6]{Hus75}, we have $V_i \cong \ker(d_i) \oplus \im(d_i) = \im(d_{i-1}) \oplus \im(d_i)$.
	Consequently,
	\[
	\bigotimes_{i\ \text{odd}} \Or(V_i)\, \cong
	\bigotimes_{i\ \text{odd}} \Or(\im(d_{i-1}) \oplus \im(d_i))\, \cong
	\bigotimes_{i\ \text{odd}} \Or(\im(d_{i-1})) \otimes \Or(\im(d_i))\, \cong
	\bigotimes_{\text{all}\ i} \Or(\im(d_{i})),
	\]
	and similarly for the other tensor product.
\end{proof}

We can now define the Donaldson-Kronheimer determinant line bundles as in \cite[Section 5.2.1]{DoKr90}.
Donaldson and Kronheimer work in a more general setting, but we will confine ourselves to considering a map of vector bundles $F \colon E \to E'$ over $M$.
At first, we also assume that $\ker(F)$ and $\cok(F)$ are well-defined vector bundles.
Then the Donaldson-Kronheimer determinant line bundle is defined to be
$$\Or(\ker(F)) \otimes \Or(\cok(F))^*,$$
where the $*$ over $\Or(\cok(F))$ denotes the dual line bundle.
Below we will consider that $\ker(F)$ and $\cok(F)$ are not always vector bundles, but for now we see that the determinant bundle is morally related to the index of an operator.
We refer to \cite[Section 5.2.1]{DoKr90} for a more precise statement of the relationship.

To relate the Donaldson-Kronheimer determinant line bundle to our notion of co-orientation, consider the exact sequence of vector bundles
\begin{equation*}
	0 \to \ker(F) \to E \to E' \to \cok(F) \to 0.
\end{equation*}
Applying \cref{L: det sequence}, we have $\Or(\ker(F)) \otimes \Or(E') \cong \Or(E) \otimes \Or(\cok(F))$.
Next we use that for a line bundle $L$ we have $L \otimes L^* \cong \underline{\R}$, the trivial line bundle.
So multiplying both sides by $\Or(\cok(F))^*$ and $\Or(E')^*$, we get
$$\Or(\ker(F)) \otimes \Or(\cok(F))^* \cong \Or(E) \otimes \Or(E')^*.$$
The latter is isomorphic to $\Hom(\Or(E'), \Or(E))$, which is dual to $\Hom(\Or(E), \Or(E'))$.
In particular, $\Hom(\Or(E), \Or(E'))$ is trivial, and so admits a non-zero section, if and only if the Donaldson-Kronheimer determinant bundle $\Or(\ker(F)) \otimes \Or(\cok(F))^*$ is trivial.

In the setting of a smooth map $f \colon M \to N$, we can think of the derivative $Df$ as a map $Df \colon TM \to f^*(TN)$, and then the above demonstrates that $\Hom(\Or(TM), \Or(f^*(TN)))$ is trivial if and only if the determinant bundle $\Or(\ker(Df)) \otimes \Or(\cok(Df))^*$ is trivial.
We recall that the triviality of
\[
\Hom(\Or(TM), \Or(f^*(TN)))
\]
is the condition for co-orientability of $f$ in the sense of \cref{D: co-orientations}.
Our co-orientations in this setting are equivalence classes of non-zero sections of $\Hom(\Or(TM), \Or(f^*(TN)))$ up to positive scalars or, equivalently, orientations of this line bundle.
Lipyanskiy's co-orientations are orientations of $\Or(\ker(Df)) \otimes \Or(\cok(Df))^*$.
As orientations of line bundles exist if and only if the line bundle is trivial, the two notions of co-orientability coincide.
We leave it to the reader to define the isomorphisms in sufficient detail to carry a particular co-orientation as defined in \cref{S: co-orientations} to one of Lipyanskiy's co-orientations.

The problem with the preceding analysis is that in general $\ker(Df)$ and $\cok(Df)$ do not necessarily have the same dimensions from fiber to fiber, and so $\ker(Df)$ and $\cok(Df)$ are not necessarily well defined as vector bundles.
The solution is to reframe the definition of the determinant line bundle as in \cite{DoKr90} so that it is always well defined and such that it is isomorphic to $\Or(\ker(F)) \otimes \Or(\cok(F))^*$ when it is also well defined.

For this, let $\underline{\R}^n$ be the trivial $\R^n$ bundle over $M$, and suppose we have a map $\psi \colon \underline{\R}^n \to E'$ such that $F \oplus \psi \colon E \oplus \underline{\R}^n \to E'$ is surjective\footnote{Donaldson and Kronheimer work with complex vector bundles, so \cite{DoKr90} features $\underline{\C}^n$ rather than $\underline{\R}^n$.}; here we write $F \oplus \psi$ for the map $(x,y) \to F(x) + \psi(y)$. 
Such a map will always exist in our setting, as we defined manifolds with corners to be embedded in finite dimensional Euclidean space.
Hence tangent bundles are subbundles of trivial bundles and so also the images of projections of trivial bundles (or, up isomorphism, quotients of the trivial bundle by their orthogonal complements after endowing the trivial bundle with a Riemannian structure).
The gain is that $F \oplus \psi$ then has trivial cokernel and a kernel that is a vector bundle, as now the fibers of the kernel have a fixed dimension.
We then define the determinant line bundle to be
$$\mathscr L = \Or(\ker(F \oplus \psi)) \otimes \Or(\underline{\R}^n)^* \cong \Or(\ker(F \oplus \psi)).$$

In the case where $\ker(F)$ and $\cok(F)$ were already vector bundles, $\mathscr L$ is isomorphic to the earlier Donaldson-Kronheimer determinant line bundle using \cref{L: det sequence} and the following lemma:

\begin{lemma}
	If $F \colon E \to E'$ and $\psi \colon \underline{\R}^n \to E'$ are bundle maps with $F \oplus \psi \colon E \oplus \underline{\R}^n \to E'$ surjective and $\ker(F)$ and $\cok(F)$ well-defined vector bundles, then the following sequence is exact\footnote{This exact sequence appears incorrectly in \cite{DoKr90} with the $\psi$ in place of $F$ in the first and last terms.}:
	\[
	\begin{tikzcd}[column sep=small]
		0 \arrow[r] & \ker(F) \arrow[r] & \ker(F \oplus \psi) \arrow[r] & \underline{\R}^n \arrow[r] & \cok(F) \arrow[r] & 0.
	\end{tikzcd}
	\]
\end{lemma}

\begin{proof}
	This exact sequence is simply the snake lemma exact sequence obtained from the commutative diagram of exact sequences
	\[
	\begin{tikzcd}
		0 \arrow[r] & E  \arrow[r] \arrow[d, "F"] & E \oplus \underline{\R}^n \arrow[r] \arrow[d, "F \otimes \psi"] & \underline{\R}^n \arrow[r] \arrow[d] & 0 \\
		0 \arrow[r] & E' \arrow[r, "="] & E' \arrow[r] & 0 \arrow[r] & 0.
	\end{tikzcd}
	\]
	The category of vector bundles over a space is not technically an abelian category, but one can check by hand for this diagram that, with our assumptions, all the maps of the exact sequence are well defined and the exactness then holds fiberwise by the classical snake lemma.
	In particular, the map $\underline{\R}^n \to \cok(F)$ is the composition of the splitting map $\underline{\R}^n \to E \oplus \underline{\R}^n$, the map $F \oplus \psi$, and the projection $E'$ to $\cok(F)$.
\end{proof}

Combining this lemma with \cref{L: det sequence} gives us an isomorphism
$$\Or(\ker(F)) \otimes \Or(\underline{\R}^n) \cong \Or(\ker(F \oplus \psi)) \otimes \Or(\cok(F)).$$
Multiplying both sides by $\Or(\underline{\R}^n)^* \otimes \Or(\cok(F))^*$ and using again that for a line bundle $L$ we have $L \otimes L^* \cong \underline{\R}$, we obtain
$$\Or(\ker(F)) \otimes \Or(\cok(F))^* \cong \Or(\ker(F \oplus \psi)) \otimes \Or(\underline{\R}^n)^*.$$
So, as promised, the two definitions of the Donaldson-Kronheimer determinant line bundle agree (up to canonical isomorphisms) when $\ker(F)$ and $\cok(F)$ are defined.

Finally, we should observe that the construction of $\mathscr L$ is independent, at least up to isomorphism, of the choice of $\psi$ and $n$.
Clearly $\Or(\underline{\R}^n) \cong \Or(\underline{\R}^n)^* \cong \underline{\R}$ for all $n$, so we must only show that if $\psi_1 \colon \underline{\R}^n \to E'$ and $\psi_2 \colon \underline{\R}^m \to E'$ are two maps satisfying the requirement of the definition then $\Or(\ker(F \oplus \psi_1)) \cong \Or(\ker(F \oplus \psi_2))$.
Adapting an argument in \cite[Section 5.1.3]{DoKr90}, we note that the bundle maps
\begin{align*}
	F \oplus \psi_1 \oplus 0 \colon &E \oplus \underline{\R}^{n+m} \to E'\\
	F \oplus 0 \oplus \psi_2 \colon &E \oplus \underline{\R}^{n+m} \to E'
\end{align*}
are each homotopic through fiberwise linear homotopies to $F \oplus \psi_1 \oplus \psi_2$, and so they are homotopic to each other.
Furthermore, these are homotopies through surjective bundle maps, so we can write the homotopy between the two maps as a surjective bundle map $(E \oplus \underline{\R}^{n+m}) \times I \to E' \times I$ over $M \times I$.
Thus we have a well-defined kernel bundle over $M \times I$, which implies that the kernel bundles over $M \times \{0\}$ and $M \times \{1\}$ are isomorphic.
But, up to reordering the summands, these are, respectively, $\ker(F \oplus \psi_1) \oplus \underline{\R}^m$ and $\ker(F \oplus \psi_2) \oplus \underline{\R}^n$.
Therefore,
$$\Or(\ker(F \oplus \psi_1) \oplus \underline{\R}^m) \cong \Or(\ker(F \oplus \psi_2) \oplus \underline{\R}^n).$$
But
$$\Or(\ker(F \oplus \psi_1) \oplus \underline{\R}^m) \cong \Or(\ker(F \oplus \psi_1)) \otimes \Or(\underline{\R}^m) \cong \Or(\ker(F \oplus \psi_1)),$$
and similarly for the other bundle.
So $\Or(\ker(F \oplus \psi_1)) \cong \Or(\ker(F \oplus \psi_2))$. %
	% !TEX root = ../foundations.tex

\section{Geometric chains and cochains}\label{S: geometric cochains}

Geometric homology and cohomology are homology/cohomology theories for smooth manifolds defined through submanifolds or, more generally, maps from manifolds with corners.
They agree with singular homology and cohomology, but having different representatives
at the chain/cochain level, they provide geometric approaches to both theory and calculations.
They are thus akin to de Rham theory in the sense that chains and cochains are defined through the smooth structure rather than continuous maps or open sets, but they are defined over the integers and not just the real numbers.

Our definitions of geometric chains and cochains will be a modification of those given by Lipyanskiy in \cite{Lipy14}, though his results will continue to hold with our modified definitions.
As Lipyanskiy primarily focuses on geometric chains and geometric homology, our focus, where the theories diverge, will primarily be on geometric cochains and cohomology, for which Lipyanskiy's account is much less complete despite the cohomological setting having its own subtleties.
We also take the opportunity to fill in some of the details missing from Lipyanskiy's account more generally, especially utilizing the more thorough foundations on manifolds with corners provided by \cite{Joy12,MaDo92}.

\subsection{Preliminary definitions}

We first identify certain types of ``manifolds over $M$.'' In future sections $M$ will most typically be a manifold without boundary, as that is the case where we obtain agreement between geometric (co)homology and singular (co)homology, but in this section we allow it to be a manifold with corners anywhere that the extra generality may be useful but so long as it does not create more technical work.


\begin{definition}\label{V: maps are co-oriented}
	Let $M$ be a smooth manifold, possibly with corners. A \textbf{manifold over $\mathbf{M}$} is a manifold with corners $W$ with a smooth map $r_W \colon W \to M$,
	called the \textbf{reference map}.
	We freely and almost always abuse notation
	by using the domain $W$ to refer to the manifold over $M$, not $r_W$ or some other symbol, letting
	context determine whether we are referring to the entire data or the domain.

	We say a manifold over $M$, is:
	\begin{itemize}
		\item \textbf{compact} if its domain is compact,
		\item \textbf{proper} if its reference map is proper (i.e. the preimage of compact subsets is compact),
		\item \textbf{oriented} if its domain is oriented, and
		\item \textbf{co-oriented} if its reference map is co-oriented.
	\end{itemize}
	If $W$ is a manifold over $M$ then so is $\bd W$ using the reference map $r_W \circ i_{\bd W} \colon \bd W \to M$.
	If $W$ is oriented or co-oriented, then $\bd W$ inherits an orientation or co-orientation as in \cref{Con: oriented boundary} or \cref{D: boundary co-orientation}.
	By \cite[Lemma 2.8]{Joy12}, $i_{\bd W}$ is proper, so, as the composition of proper maps is proper, if $W$ is a proper manifold over $M$ then so is $\bd W$.

	If $W$ is oriented or co-oriented, we write $-W$ to refer to the manifold over $M$ with the opposite orientation or co-orientation; it should always be clear from context which structure we are replacing with its opposite.
\end{definition}

%The most important case of manifolds over $M$ are submanifolds.
%At times we also employ the notation $W \to M$, to address general maps but keep the emphasis on the case of embeddings.

Geometric chains and cochains will be equivalence classes of manifolds over $M$ under an equivalence relation we define using the following concepts, which are taken from or modify
the definitions of \cite{Lipy14}.

\begin{definition}\label{D: equiv triv and small}
	Let $V, W$ be manifolds over $M$ with reference maps $r_V$ and $r_W$.
	\begin{itemize}
		\item Suppose $V$ and $W$ are co-oriented and $\phi \colon W \to V$ is a diffeomorphism such that $r_V \circ \phi = r_W$.
		We say that $\phi$ is \textbf{co-orientation preserving (or co-orientation reversing)} if the composition of the tautological co-orientation of $\phi$ (\cref{D: tautological co-orientation}) with the co-orientation of $r_V$ agrees with (or disagrees with) the co-orientation of $r_W$.
		\item If $V$ and $W$ are oriented, we say they are \textbf{(oriented) isomorphic} if there is an orientation-preserving diffeomorphism $\phi \colon W \to V$ such that $r_V \circ \phi = r_W$.
		\item If $V$ and $W$ are co-oriented, we say they are \textbf{(co-oriented) isomorphic} if there is a co-orientation-preserving diffeomorphism $\phi \colon W \to V$.
		\item If $W$ is oriented then $W$ is \textbf{(oriented) trivial} if there is an orientation-reversing diffeomorphism $\rho \colon W \to W$ such that $r_W \circ \rho = r_W$.
		\item If $W$ is co-oriented then it is \textbf{(co-oriented) trivial} if there is a co-orientation-reversing diffeomorphism $\rho \colon W \to W$.
		\item $W$ has \textbf{small rank} if the differential $D r_W$ is everywhere less than full rank, in other words if the rank of $D_w r_W$ is less than $\dim(W)$ for all $w\in W$.
		\item If $W$ is (co\nobreakdash-)oriented, then it is \textbf{((co\nobreakdash-)oriented) degenerate} if it has small rank and ${\bd W}$ is the disjoint union of a trivial (co\nobreakdash-)oriented manifold over $M$ and one with small rank.
	\end{itemize}
	We declare the empty map $\emptyset \to M$ to be both trivial and degenerate.
\end{definition}

Note that while we can speak of a diffeomorphism being orientation-preserving or -reversing without it necessarily commuting with the reference maps, the notion of being co-orientation-preserving or -reversing is only defined when $r_V \circ \phi = r_W$.

We observe that isomorphism is an equivalence relation and that it preserves triviality and degeneracy.

Rather than small rank, Lipyanskiy uses the condition of \textbf{small image}.
In our notation, a map $r_W \colon W \to M$ has small image if there is another map $r_T \colon T \to M$ such that $r_T(T)\supset r_W(W)$ but $\dim(T)<\dim(W)$.
However, the small rank condition turns out to be more manageable for our purposes while still providing geometric homology and cohomology theories that are equivalent to singular homology and cohomology.
Roughly speaking, the geometric chains and cochains of a manifold $M$ will consist of isomorphism classes of oriented or proper co-oriented manifolds with corners over $M$ modulo the trivial and degenerate chains or cochains.

The most obvious use of the notion of triviality will be to ensure that $\bd^2 = 0$ so that we have a chain complex.
The degeneracy comes into play in ensuring that geometric homology and cohomology satisfy the dimension axiom; see \cref{E: dimension} and \cref{R: degen1}.

\begin{example}
	Let $S^1$ be the unit circle in the plane with the standard counterclockwise orientation.
	Let $\pi \colon S^1 \to \R$ be the projection of the circle onto the $x$-axis.
	This map is trivial.
	Indeed, if $\rho$ is the reflection of the circle across the $x$-axis, then $\rho$ is orientation-reversing and $\pi \rho = \pi$.

	Alternatively, let $\pi \colon S^1 \to \R$ be co-oriented by $(e_\theta,e_x)$, where $e_x$ is the standard positively-directed unit vector in $\R$ and $e_\theta$ is the counterclockwise tangent vector in $S^1$.
	This map is trivial as a co-oriented map, again via the reflection $\rho$ across the $x$-axis.
	Indeed, we still have $\pi \rho = \pi$, and the tautological co-orientation of $\rho$ is $(e_\theta,-e_\theta)$ so that the composite co-orientation of $\pi\rho$ is $-(e_\theta,e_x)$.
\end{example}

\begin{example}
	If $r_V \colon V \to M$ is any oriented or co-oriented map and $-r_V \colon V \to M$ is the same map with the opposite orientation or co-orientation, then $r_V \sqcup -r_V \colon V \sqcup V \to M$ is trivial, taking $\rho$ to be the map that switches the two copies of $V$.

	To be technically accurate, the space $V \sqcup V$ is not clearly defined as a manifold with corners, as we have defined our manifolds with corners in \cref{D: MWC} to be submanifolds of some $\R^N$. We solve this problem below by working with isomorphism classes of maps.
	In this case we could take $V \sqcup V$ up to isomorphism to be $V \sqcup W$ where $V$ and $W$ are isomorphic as manifolds over $M$.
	See \cref{D: prechain sum} below for precise details.
\end{example}

\begin{example}
	Any co-oriented map of the interval to a point has small rank, but its boundary does not have small rank; it is nonetheless (co-oriented) degenerate because its boundary is trivial.
\end{example}

\begin{example}\label{E: projected triangle}
	Let $V$ be the $2$-simplex in $\R^2$ with vertices at $(1,0)$, $(-1,0)$, and $(0,1)$, and let $\pi \colon V \to \R$ be the projection to the $x$-axis.
	This map has small rank, but the boundary does not have small rank and is not trivial.
\end{example}

\begin{example}
	Let $V = W = M = \R^1$.
	Let $r_W \colon W \to M$ be the identity map of $\R^1$ with its tautological co-orientation, which we can write $(e_1,e_1)$, letting $e_1$ be a positively-oriented tangent vector to $\R^1$.
	Let $r_V \colon V \to M$ be given by $r_V(t) = -t$ with co-orientation $(-e_1,e_1)$.
	Let $\phi \colon W \to V$ be given by $\phi(t) = -t$.
	Then the tautological co-orientation of $\phi$ is $(e_1,-e_1)$.
	So $r_V\phi = r_W$ as co-oriented maps, and $\phi$ provides a co-oriented isomorphism between $r_W$ and $r_V$ even though they are very different maps.
\end{example}

The notions of orientation-preserving and -reversing diffeomorphisms are fairly standard, and oriented isomorphisms and triviality of oriented manifolds over $M$ are simply such diffeomorphisms that commute with the reference maps.
By contrast, the corresponding notions for co-orientation are less familiar, but the following statement shows how the situation for co-orientations is locally similar to that for orientations.

\begin{lemma}\label{L: co-or preserving/reversing}
	Let $r_V \colon V \to M$ and $r_W \colon W \to M$ be co-oriented manifolds over $M$, and let $\phi \colon W \to V$ be a diffeomorphism such that $r_W = r_V \phi$.
	For $x \in W$, let $(\beta_{W,x},\beta_{M,r_W(x)})$ be a local representation of the co-orientation of $r_W$ and let $(\beta_{V, \phi(x)},\beta_{M,r_W(x)})$ be a local representation of the co-orientation of $r_V$, noting $r_V\phi(x)=r_W(x)$.
	Then $\phi$ provides a co-oriented isomorphism if and only if for all $x\in W$ the derivative $D_x\phi$ takes the orientation $\beta_{W,x}$ of $T_xW$ to the orientation $\beta_{V, \phi(x)}$ of $T_{\phi(x)}V$.
	Similarly, the map $\phi$ is a co-orientation-reversing diffeomorphism if and only if for all $x\in W$ the derivative $D_x\phi$ takes the orientation $\beta_{W,x}$ of $T_xW$ to the orientation $-\beta_{V, \phi(x)}$ of $T_{\phi(x)}V$.
\end{lemma}

In the statement of the lemma, by $D_x\phi$ taking an orientation of $T_xW$ to an orientation of $T_{\phi(x)}V$, we mean that $D_x\phi$ takes an ordered basis consistent with the given orientation of $T_xW$ to an ordered basis consistent with the given orientation of $T_{\phi(x)}V$.
We could equivalently phrase this in terms of the map $\textstyle{\bigwedge^w}D_x\phi$ as in \cref{D: tautological co-orientation}.

\begin{proof}[Proof of \cref{L: co-or preserving/reversing}]
By definition, $\phi$ is an isomorphism if and only if $\omega_{r_W} = \omega_{\phi} * \omega_{r_V}$, where $\omega_\phi$ is the tautological co-orientation of $\phi$.
This will be the case if and only if for each $x$ we have $\omega_\phi = (\beta_{W,x}, \beta_{V, \phi(x)})$.
But by the definition of the tautological co-orientation, if $e_1 \wedge \cdots \wedge e_w$ represents an oriented basis for $T_xW$, then $\omega_\phi$ is represented at $x$ by the pair $(e_1 \wedge \cdots \wedge e_w, D_x\phi(e_1) \wedge \cdots \wedge D_x\phi(e_w))$.
So $\phi$ is an isomorphism if and only if $D_x\phi$ takes an ordered basis representing the local orientation $\beta_{W,x}$ to an ordered basis representing the local orientation $\beta_{V, \phi(x)}$.

The orientation-reversing case is equivalent, considering instead the requirement $\omega_{r_W} = - \omega_{\phi} * \omega_{r_V}$.
\end{proof}

The following corollary is now immediate from \cref{L: co-or preserving/reversing} and the definitions.

\begin{corollary}\label{C: co-or preserving is or preserving}
	Let $r_V \colon V \to M$ and $r_W \colon W \to M$ be co-oriented manifolds over the oriented manifold $M$, and let $\phi \colon W \to V$ be a diffeomorphism such that $r_W = r_V \phi$. Then $\phi$ is co-orientation preserving (reversing) if and only if it is orientation preserving (reversing) with respect to the orientations of $V$ and $W$ induced by the co-orientations and the orientation of $M$ (\cref{S: co-orientations}).
\end{corollary}



\subsection{Geometric homology and cohomology}

We can now move toward defining geometric homology and cohomology, starting with ``prechains'' and ``precochains.''

\begin{definition}
	Let $M$ be a smooth manifold with corners.
	Denote by $PC^\Gamma_*(M)$ the set of oriented isomorphism classes of compact oriented manifolds over $M$, $r_W \colon W \to M$,
	graded by dimension $\dim(W)$.
	Denote by $PC_\Gamma^*(M)$ the set of co-oriented isomorphism classes of proper co-oriented manifolds over $M$, $r_W \colon W \to M$,
	graded by \textbf{codimension} $\dim(M) - \dim(W)$.
	We declare the empty manifold of each dimension to be orientable and the empty maps from the empty manifolds to $M$ to be co-orientable.

	As per \cref{V: maps are co-oriented}, we will often write $W \in PC^\Gamma_*(M)$ or $W \in PC_\Gamma^*(M)$, letting $W$ represent both its reference map and its isomorphism class as a manifold over $M$.
	In these respective cases we write $-W$ for $W$ with the opposite orientation or co-orientation.

	We will sometimes refer to elements of $PC^\Gamma_*(M)$ or $PC_\Gamma^*(M)$ as \textbf{prechains} or \textbf{precochains}, respectively.
\end{definition}

When necessary, as in the following definition, we may write prechains or precochains as $[W]$ to emphasize that they are isomorphism classes.


\begin{definition}\label{D: prechain sum}
	If $V,W$ are compact oriented manifolds over $M$ (respectively proper co-oriented manifolds over $M$), then define $[V] \sqcup [W] = [V' \sqcup W']$, where on the right $\sqcup$ denotes disjoint union and $V'$ and $W'$ are compact oriented manifolds over $M$ (respectively proper co-oriented manifolds over $M$) such $[V]=[V']$, $[W]=[W']$, and $V'$ and $W'$ are disjoint in $\R^\infty$ (cf.
	\cref{D: MWC}); note that the images of $V'$ and $W'$ are not necessarily disjoint in $M$.
\end{definition}

The last clause in the definition is due to our requirement that all manifolds with corners be subsets of $\R^\infty$.
As given, the definition allows constructions like $[W]\sqcup[W]$ to be well defined.
We observe that this definition is itself well defined, since if $V''$ and $W''$ are two other disjoint manifolds over $M$ isomorphic to $V$ and $W$ (in the appropriate sense), then we can compose the diffeomorphisms $V'\xleftarrow{\phi'}V \xr{\phi''} V''$ and $W'\xleftarrow{\psi'}W \xr{\psi''} W''$ to obtain an isomorphism $\phi''(\phi')^{-1} \sqcup \psi''(\psi')^{-1} \colon V' \sqcup W' \to V'' \sqcup W''$.

With $\sqcup$, $PC^\Gamma_*(M)$ and $PC_\Gamma^*(M)$ become commutative monoids in each degree with the empty maps $r_\emptyset \colon \emptyset \to M$ as the identities.

We now return to denoting isomorphism classes by their representatives, noting again that triviality and small rank are properties of the isomorphism classes.

\begin{definition}
	Let $Q_*(M) \subset PC_*(M)$ denote the set of (isomorphism classes of) compact oriented manifolds over $M$ of the form $V \sqcup W$ with $V$ trivial and $W$ degenerate.
	Let $Q^*(M) \subset PC^*(M)$ denote the set of (isomorphism classes of) proper co-oriented manifolds over $M$ of the form $V \sqcup W$ with $V$ trivial and $W$ degenerate.
	In either case $V$ or $W$ may be empty.

	We will sometimes write $W \in Q(M)$ to mean $W \in Q_*(M)$ or $W \in Q^*(M)$ for arguments that are analogous in the two cases.
	When we do so, we assume a consistent choice of $Q_*(M)$ or $Q^*(M)$ throughout the discussion; see for instance \cref{L: bd defined} and its proof.
\end{definition}

Our first lemma is immediate:

\begin{lemma}\label{L: sum of trivial/degenerate}
If $V$ and $W$ are trivial, then $V \sqcup W$ is trivial.
If $V$ and $W$ are degenerate, then $V \sqcup W$ is degenerate.
If $V,W\in Q_*(M)$ (or $Q^*(M)$), then $V \sqcup W \in Q_*(M)$ (or $Q^*(M)$).
\end{lemma}

The following useful basic properties are proven in \cite{Lipy14}; for completeness we provide versions of the arguments here, occasionally augmenting those of \cite{Lipy14}.
In each case, ``isomorphic,'' ``trivial,'' or ``degenerate'' should be read consistently to refer to the compact oriented case or the proper co-oriented case.
Lipyanskiy's proofs assume small image rather than small rank, but the proofs are equivalent.



\begin{lemma}[Lipyanskiy Lemma 10]\label{L: Lip L10}
	If $V$ is trivial and $V \sqcup W$ is trivial, then $W$ is trivial.
\end{lemma}

\begin{proof}
	We can write $W$ as the disjoint union of a (possibly infinite) number of isomorphism classes of connected components and then group the isomorphism classes together up to (co\nobreakdash-)orientation as $W = W_1 \sqcup W_2 \sqcup \cdots$ so that all connected components of each $W_i$ are isomorphic (ignoring (co\nobreakdash-)orientations) for each $i$ and so that no connected component of $W_i$ is isomorphic to a connected component of $W_j$ for $i\neq j$.
	As either $W$ is compact or $r_W$ is proper, each $W_i$ is the union of a finite number of connected components; if not, then there would be a $W_i$ with an infinite number of connected components and, since the components of $W_i$ are isomorphic to each other, a point $x\in M$ such that $r_W^{-1}$ has infinite components, violating that $r_W$ is proper.
	Any automorphism of $W$ preserves the decomposition into $W_i$, and so $W$ is trivial if and only if for all $i$ either $W_i$ has zero components when counting with (co\nobreakdash-)orientation or each component of $W_i$ has a (co\nobreakdash-)orientation reversing automorphism.

	Similarly, since $V$ is trivial, $V$ can be decomposed into unions of isomorphism classes, up to (co\nobreakdash-)orienta\-tion, of connected components, each with zero components when counting with sign or with all components having (co\nobreakdash-)orientation reversing automorphisms.
	In particular, forming $V \sqcup W$ adds to $W$ zero components when counting with sign or components with (co\nobreakdash-)orientation reversing automorphisms, so if $V \sqcup W$ is trivial, $W$ must have already been trivial.
\end{proof}

\begin{lemma}[Lipyanskiy Lemma 11]\label{L: bd defined}
	If $W$ is in $Q(M)$ then so is $\bd W$.
\end{lemma}

\begin{proof}
	We first check that if $W$ is trivial then so is its boundary $r_{\bd W} = r_Wi_{\bd W} \colon \bd W \to M$.
	If $\rho \colon W \to W$ is (co\nobreakdash-)orientation reversing, then we can consider $\rho_\bd$ as defined in \cref{R: bd diff}.
	As $r_W\rho = r_W$ we also have $r_Wi_{\bd W} = r_W\rho i_{\bd W} = r_W i_{\bd W} \rho_\bd$.
	Thus we only need see that $\rho_\bd$ is (co\nobreakdash-)orientation reversing.
	It is sufficient to consider what happens at points on the interior of $\bd W$ so working locally we may identify such points of $\bd W$ with points of $W$ itself and similarly identify $\rho_\bd$ with $\rho$.
	In the oriented case, the orientation of $W$ determines orientations of $T_xW$ and $T_{\rho(x)}W$, and by assumption $D\rho: T_xW \to T_{\rho(x)}W$ takes the orientation of $T_xW$ to the opposite of the orientation of $T_{\rho(x)}W$.
	But also $D\rho$ must preserve inward/outward pointing vectors.
	Thus $D\rho$ must restrict to a map $T_x (\bd W) \to T_{\rho(x)}(\bd W)$ that also reverses the orientation.
	The co-oriented situation is analogous using local orientation pairs $\left(\beta_{W,x}, \beta_{M,r_W(x)}\right)$ and $\left(\beta_{W,\rho(x)}, \beta_{M,r_W\rho(x)}\right)$ and noting $\beta_{M,r_W\rho(x)} = \beta_{M,r_W(x)}$ as $\rho$ is a diffeomorphism over $M$.

	Now suppose $W$ degenerate.
	By definition $\bd W = A \sqcup B$ with $A$ trivial and $B$ of small rank.
	Then $\bd^2 W = \bd A \sqcup \bd B$.
	As noted, $\bd A$ is trivial, and $\bd^2 W$ is trivial for all $W$ by \cref{L: boundary2}.
	It follows from \cref{L: Lip L10} that $\bd B$ is trivial, and so $B$ is degenerate.
	Thus $\bd W$ is degenerate.
\end{proof}

\begin{lemma}[Lipyanskiy Lemma 12]\label{L: Lipy12}
	If both $V$ and $V \sqcup W$ are in $Q(M)$ then so is $W$.
\end{lemma}

\begin{proof}
	As in the proof of \cref{L: Lip L10}, decompose $W$ as $W = W_1 \sqcup W_2 \sqcup \cdots$ and $V$ as $V = V_1 \sqcup V_2 \sqcup \cdots$.
	As $V$ and $V \sqcup W$ are in $Q(M)$, each $V_i$ has small rank or, also as in the proof of \cref{L: Lip L10}, $V_i$ is trivial, and similarly for $V \sqcup W$, from which it follows again by counting with signs in the trivial components that each $W_i$ is either trivial or has small rank.
	By grouping terms of the decomposition we can write $W = A \sqcup B$ with $A$ trivial and $B$ of small rank.

	We then have $\bd V$ and $\bd (V \sqcup W) = \bd V \sqcup \bd A \sqcup \bd B$ in $Q(M)$ by \cref{L: bd defined}, and also $\bd A$ is trivial as the boundary of a trivial manifold over $M$ by the proof of \cref{L: bd defined}.
	Now by the same argument as in the previous paragraph, replacing $V$ with $\bd V \sqcup \bd A$ and $W$ with $\bd B$, we have that $\bd B$ can be decomposed into the disjoint union of a trivial manifold over $M$ and one with small rank.
	But this shows $B$ is degenerate, so $W \in Q(M)$.
\end{proof}

\begin{lemma}[Lipyanskiy Lemma 13]\label{L: cancel Q}
	The relation given by $V\sim W$ if $V \sqcup -W$ is in $Q_*(M)$ (respectively $Q^*(M)$) is an equivalence relation on $PC^\Gamma_*(M)$ (respectively $PC_\Gamma^*(M)$).
\end{lemma}

\begin{proof}
	Reflexivity: For any $W$, we have $W \sqcup -W$ trivial via the map that interchanges the two copies of $W$.

	Symmetry: If $V \sqcup -W$ is the union of trivial and degenerate manifolds over $M$ then certainly so is $-(V \sqcup -W) = W \sqcup -V$.

	Transitivity: If $V \sqcup -W$ and $W \sqcup -U$ are in $Q_*(M)$ (or $Q^*(M)$), then so is $V \sqcup -W \sqcup W \sqcup -U \cong V \sqcup -U \sqcup W \sqcup -W$.
	We know $W \sqcup -W$ is trivial, and so $V \sqcup -U$ is in $Q_*(M)$ (or $Q^*(M)$) by \cref{L: Lipy12}.
\end{proof}

These lemmas allow us to follow Lipyanskiy in defining geometric chains and cochains.
We will show that the claims of the following definition hold in \cref{L: co/chains well defined} just below.

\begin{definition}\label{D: chains and cochains}
	The \textbf{geometric chains} of $M$, denoted $C^\Gamma_*(M)$, are the $\sim$ equivalence classes in $PC^\Gamma_*(M)$.
	The \textbf{geometric cochains} of $M$, denoted $C_\Gamma^*(M)$, are the $\sim$ equivalence classes in $PC_\Gamma^*(M)$.
	In either case, we denote the equivalence class of $W$ by $\uW$.

	These are chain complexes with group operation $\uV + \uW = \underline{V \sqcup W}$ and boundary map $\bd \uW = \underline{\bd W}$.
	Furthermore, $\uW = 0$ in $C^\Gamma_*(M)$ (respectively $C_\Gamma^*(M)$) if and only if $W$ is in $Q_*(M)$ (respectively $Q^*(M)$), and in either case, the inverse of $\uW$ is $-\uW = \underline{-W}$.
	In particular, the empty manifold $\emptyset$ represents $0$ in $C^\Gamma_*(M)$ and the empty map $\emptyset \to M$ represents $0$ in $C^*_\Gamma(M)$.

	We define the \textbf{geometric homology} of $M$, written $H_*^\Gamma(M)$, to be $H_*(C^\Gamma_*(M))$, and
	we define the \textbf{geometric cohomology} of $M$, written $H^*_\Gamma(M)$ to be $H^*(C_\Gamma^*(M))$.
\end{definition}

We note that the formula $\uV + \uW = \underline{V \sqcup W}$ implies that every geometric chain or geometric cochain can be represented by a single map $r_W \colon W \to M$ for appropriate (possibly disconnected) $W$.


\begin{lemma}\label{L: co/chains well defined}
	The chain complexes $C_\Gamma^*(M)$ and $C_\Gamma^*(M)$ are well defined with the properties claimed in the definition.
	In particular, $\uW = 0$ in $C^\Gamma_*(M)$ (respectively $C_\Gamma^*(M)$) if and only if $W$ is in $Q_*(M)$ (respectively $Q^*(M)$), and the additive inverse of $\uW$ is $\underline{-W}$.
\end{lemma}

\begin{proof}
	We will apply the preceding lemmas.
	For simplicity we work with $Q_*(M)$ and $C^\Gamma_*(M)$, but the identical arguments hold with $Q^*(M)$ and $C_\Gamma^*(M)$.

	Suppose $\uV,\uW \in C^\Gamma_*(M)$ with $V$ and $V'$ in the class $\uV$ and $W$ and $W'$ in the class $\uW$.
	Then the sum $\uV+\uW$ is well defined as the class $\underline{V \sqcup W}$ because $(V \sqcup W) \sqcup -(V' \sqcup W') = (V \sqcup -V') \sqcup (W \sqcup -W')$, and $V \sqcup -V'$ and $W \sqcup -W'$ are both in $Q_*(M)$ by assumption.

	The identity in each degree is represented by $\emptyset$ with the unique empty map to $M$ (with either orientation or co-orientation).
	In fact, every element of $Q_*(M)$ represents $0$ in $C^\Gamma_*(M)$ as elements of $Q_*(M)$ are all equivalent to $\emptyset$.
	Conversely, if $\uW = 0$ then $W \in Q_*(M)$, as if $\uV+\uW = \uV$ then $V \sqcup W \sqcup -V \in Q_*(M)$.
	But $V \sqcup -V \in Q_*(M)$, so by \cref{L: Lipy12}, $W \in Q_*(M)$.
	We also see that the additive inverse of $\uW$ is $\underline{-W}$, as $W \sqcup -W$ is trivial.

	That the boundary map is well defined with $\bd \uW = \udW$ is due to \cref{L: bd defined} and \cite[Lemma 2.8]{Joy12}, which implies that $\bd W$ is proper.
	That $\bd^2 = 0$ in the case of cochains follows from \cref{L: boundary2}, which shows that $\bd^2 W$ is always trivial.
	Similarly, to obtain $\bd^2 = 0$ for chains see \cref{R: bd2 oriented}.
\end{proof}

\begin{comment}
	Thom deeply considered the interplay between manifolds
	and homology \cite{Thom54}, and the cohomology classes we produce for submanifolds are equal to the pullbacks of Thom classes by Thom collapse maps.
	Similar objects were defined by Quillen \cite{Quil71} as an immediate translation of the data which encodes cobordism generalized cohomology theories.
	Thus, we name the QT-objects which represent equivalence classes of geometric cochains, defined below, in honor of Quillen and Thom.
\end{comment}

\begin{comment}
	If $i \colon W \to M$ is an embedding of a submanifold,
	we abuse notation by referring to the cochain as $\tau_W$, suppressing the reference map from the notation.
	Indeed, the image of any $\tau_i$ under our comparison map to cubical cochains will be the same.
\end{comment}

\begin{comment}
	%Geometric reasoning is aided by focussing on the image of the map rather than the map itself.
\end{comment}

\begin{comment}
	Having given our version of geometric chains and cochains, for the remainder of this section we work exclusively with cochains, referring to \cite{Lipy14}for (or leaving as exercises for the interested reader) the analogous properties of geometric chains.

	\begin{lemma}
		Geometric chains and cochains on $M$ form chain complexes --- that is, $d^\Gamma$ and $d_\Gamma$ square to zero.
	\end{lemma}

	This lemma is essentially Lemma~13 of \cite{Lipy14}.
	The key is that while $\bd^2 W$ is not empty,

	BCOMMENT
	, nor is it zero at the level of pre-cochains
	ECOMMENT
	it is trivial.
	That is, it has a $C_2$-action, permuting the local boundary components attached to points in $S^2(W)$.
	Moreover, under our co-orientation conventions, the two vectors appended to the co-orientation of $W$ to obtain one for $\bd^2 W$ over the same point in
	$S^2(W)$ differ by a transposition, so this $C_2$-action is co-orientation reversing.
	This fact about $\bd^2$ not only eventually shows that $d_\Gamma^{\,2} = 0$, but is first needed to show that the boundary of a degenerate map is degenerate; \red{see the proof of \cite[Lemma 11]{Lipy14}}.
\end{comment}

\begin{comment}
	\red{Reflecting on all of these definitions, an illustrative example of a manifold over $M$ defined by a proper map
		is given by embedding the positive $x$-axis within $\R^3 \setminus \{0\}$, \red{together with a choice of orientation of its normal bundle}, which represents a generator of the second cohomology [GBF: I modified this example to make it clearer that we index by codimension, since none of the previous examples really showed that off.]}.
	Another illustrative example is given by a linear embedding of $\R P^2$ into $\R P^4$.
	The domain manifold is not orientable, but the map is co-orientable and represents a nonzero class in $H^2(\R P^4; \Z)$.
	\red{[GBF: How do we know this class is non-zero? Can we prove that?]}
	We also recall that Schubert subvarieties of Grassmannians are not smooth, but do have standard
	smooth resolutions with reference maps to Grassmannians which can used to represent cohomology.
\end{comment}

\begin{comment}
	In Section 6 of \cite{Lipy14}, Lipyanskiy shows that the homology of $C_\Gamma^*(M)$, which we denote by $H_\Gamma^*(M)$,
	agrees with singular cohomology
	through the verification of homotopy and excision axioms.
	We find Mayer--Vietoris better for our applications, and we review both its verification
	and that of homotopy invariance as we need details about such constructions in our work.
\end{comment}

\begin{remark}\label{R: cycles and boundaries}
	Suppose $W \in PC_*^{\Gamma}(M)$. It follows directly from \cref{{D: chains and cochains},L: co/chains well defined} that $W$ represents a cycle in $C_*^{\Gamma}(M)$, i.e.\ $\bd \uW=0$, if and only if $\bd W \in Q_*(M)$.
	We also observe that $W$ represents a boundary precisely when there is some $Z \in PC_*^\Gamma(M)$ such that $\bd \underline{Z} = \underline{\bd Z} = \uW$, which translates to $(\bd Z) \sqcup -W \in Q_*(M)$.
	The corresponding statements for cohomology are analogous.
\end{remark}

In Section 6 of \cite{Lipy14}, Lipyanskiy shows that the homology theory based on geometric chains satisfies some of the Eilenberg-Steenrod axioms.
This is enough to state in Section 10 of \cite{Lipy14} that geometric homology is isomorphic to singular homology on the fixed manifold $M$, though we provide our own proofs below in \cref{T: geometric is singular,T: hom iso map}.
Unfortunately, Lipyanskiy does not provided a detailed treatment of geometric cohomology, which is different from geometric homology in several respects, though we will also show that it is isomorphic to singular cohomology in \cref{T: geometric is singular,T: intersection qi}.

We next present some immediate examples and observations. The first is that geometric homology and cohomology satisfy a very strong form of Poincar\'e duality.

\begin{theorem}[Poincar\'e Duality]\label{T: PD}
	If $M$ is closed and oriented then tautologically $C_*^\Gamma(M) = C_\Gamma^{m-*}(M)$.
	Consequently, also $H_*^\Gamma(M) = H_\Gamma^{m-*}(M)$.
\end{theorem}
\begin{proof}
	This following directly from the definitions using that the domain of a proper map to a compact space must be compact and that there is an induced orientation on the domain of a co-oriented map with oriented codomain; see the discussion following \cref{D: tautological co-orientation}.
\end{proof}



\begin{example}\label{E: first examples}
	We will see in the next chapter that geometric homology and cohomology are isomorphic to ordinary singular homology and cohomology with $\Z$-coefficients.
	So the following calculations are not surprising, though we hope they may be illuminating and help build some intuition.
	We begin by considering homology and cohomology classes represented by maps from $0$-manifolds.

	For any manifold without boundary $M$, an element of $PC_0^\Gamma(M)$ is a map to $M$ from a finite disjoint union of signed points.
	As disjoint union in $PC_0^\Gamma(M)$ translates to addition in $C_0^\Gamma(M)$ the map of any oriented point to $M$ represents a generator of $C_0^\Gamma(M)$.
	Analogously to the computation for singular homology, any two points with the same orientation mapping to the same component of $M$ represent the same element of $H_0^\Gamma(M)$ as can be seen by joining them with a smooth path and applying our boundary conventions.
	On the other hand, the boundary of such an interval mapping to $M$ consists of two points with opposite orientations, and the only compact $1$-dimensional manifolds with corners consist of closed intervals and circles, which have no boundaries.
	So it follows that a positively-oriented point mapping to $M$ generates an infinite cyclic subgroup of $H_0^\Gamma(M)$.
	Altogether, $H_0^\Gamma(M) \cong \oplus \Z$, where the sum is taken over the connected components of $M$.

	Next, suppose $M$ is a closed, connected manifold. The considerations for $H^m_\Gamma(M)$ are similar to those for $H_0^\Gamma(M)$, noting that elements of $PC^m_\Gamma(M)$ are represented by proper co-oriented maps of $0$-manifolds to $M$.
	However, in this case if we are given a map $r \colon [0,1] \to M$ with $r(0) = r(1) = z \in M$, the signs of the boundary components will depend on whether or not the loop determined by $r$ preserves or reverse the orientation of $M$.
	More specifically, let $r_0$ denote the map that takes the point to $z \in M$ with the co-orientation  $(1, \beta_M)$.
	Treating $r$ as a homotopy, by \cref{R: stationary homotopy} the boundary of $r$ will be the disjoint union of $-r_0$  and $r_1$, where $r_1$ takes the point to $z$ with co-orientation $(1, r_*\beta_M)$. Recall $r_*\beta_M$ is the orientation of $M$ at $z$ obtained by starting with the orientation $\beta_M$ at $z$ and traveling around the loop $r$; see \cref{S: co-orientations}.
	If $r_*\beta_M = \beta_M$, then $r_1 = r_0$ and we have $\bd r = r_1 - r_0 = 0$, which is analogous to the homology computation.
	But if $r_*\beta_M = -\beta_M$, then $r_1 = -r_0$ and we have $\bd r = r_1 - r_0 = -2 r_0$, so twice $r_0$ is a boundary.
	Furthermore, for any two maps $r_0$ and $s_0$ taking the point to the same component of $M$ (with either co-orientation), by joining them with a smooth path, we see that the maps will be cohomologous up to sign.
	So we conclude that if $M$ is connected then
	\[H^m_\Gamma(M) =
		\begin{cases}
		\Z, & M \text{ is orientable},\\
		\Z_2, & M \text{ is not orientable}.
		\end{cases}
	\]



	On the other hand, if $M$ is connected, without boundary, and not compact, then any co-oriented map $r_0$ from the point to $M$ represents $0$ in $H_\Gamma^m(M)$ because any proper smooth path $(-\infty,0] \to M$ with the restriction to $0$ being $r_0$ gives a null-cohomology of $r_0$ with an appropriate choice of co-orientation.

	Altogether, if $M$ is a manifold without boundary, not necessarily connected, then $H^m_\Gamma(M)$ is the direct product of $\Z$ and $\Z_2$ factors contributed respectively by the closed orientable and closed non-orientable components of $M$.

	\begin{comment}
		In $\R^3 \setminus \{0\}$, the inclusion of the positive $x$-axis with either co-orientation is a generator of $H_\Gamma^2(M)$, while the embedding the oriented unit 2-sphere containing the origin is a generator of $H_2^\Gamma(M)$.
	\end{comment}

	At the opposite end of the spectrum, the groups $H_m^\Gamma(M)$ and $H^0_\Gamma(M)$ are represented by maps of $m$-dimensional manifolds with corners.
	If $M$ is connected, then $H^0_\Gamma(M) \cong \Z$ with a generator represented by the identity map $M \to M$ with its tautological co-orientation.
	This will follow most readily from \cref{E: coho 0 generator,T: intersection qi}, below.
	Consequently, if $M$ is closed and oriented, then  \cref{T: PD} implies $H_m^\Gamma(M) \cong \Z$ generated by the identity map.
\end{example}

\begin{example}\label{E: dimension range}
	If $\dim(M) = m$, then $H_i^\Gamma(M) = H^i_\Gamma(M) = 0$ if $i < 0$ or $i > m$.
	In fact, $C_i^\Gamma(M) = 0$ when $i<0$ and $C^i_\Gamma(M) = 0$ when $i > m$, as in these cases the only elements of $PC_i^\Gamma(M)$ or $PC^i_\Gamma(M)$ are respectively the empty manifold $\emptyset$ and the empty map $\emptyset \to M$.
	For homology when $i>m+1$, every element of $PC_i^\Gamma(M)$ must have small rank and its boundary must also have small rank, so it is in $Q_i(M)$ and hence $C_i^\Gamma(M) = 0$.
	When $i = m+1$, $C_{m+1}^\Gamma(M)$ may be non-zero, but every element will have small rank.
	If $r_W \colon W \to M$ represents a degree $m+1$ cycle, then its boundary must be in $Q_m(M)$ by \cref{R: cycles and boundaries}.
	So $W$ is degenerate and represents $0$ in $C_{m+1}^\Gamma(M)$ and hence in $H_{m+1}^\Gamma(M)$.
	The argument for $H^i_\Gamma(M)$ for $i<0$ is the same.
\end{example}



\begin{example}[Dimension axiom]\label{E: dimension}
	By \cref{E: dimension range}, $H_i^\Gamma(pt) = H^i_\Gamma(pt) = 0$ unless $i = 0$.
	When $i = 0$, we have $H^0_\Gamma(M) \cong H_0^\Gamma(M) \cong \Z$ by the computations of \cref{E: first examples}.
	In fact, this is an equality by \cref{T: PD}.
\end{example}

\begin{remark}\label{R: degen1}
	It is in these examples that we most obviously see the need to include degenerate chains and cochains in $Q(M)$.
	On the other hand, the formulation of degeneracy as given will create some difficult for us in \cref{S: products} when it comes to consider notions of transversality for geometric chains and cochains.
	The reason is that degeneracy causes much of the problematic ambiguity in choosing representatives for chains and cochains.
	For example, consider a connected prechain $V \in PC_*(M)$ with small rank but a boundary that is not in $Q_*(M)$.
	If $V'$ is any other such prechain with small rank and $\bd V = \bd V'$, then $V \sqcup -V' \in Q_*(M)$ since it will have small rank and trivial boundary.
	So $V$ and $V'$ represent the same chain but could behave wildly differently aside from their boundaries.
	By contrast, we will see in \cref{S: products} that trivial chains are less of an issue (they can generally be ignored) and that non-trivial components of prechains without small rank are ``essential'' in a sense we will make precise in \cref{D: essential}.
	In particular, essential components appear in any representative of the same geometric chain or cochain.

	Given the headaches thus caused by the degenerate chains and cochains, it is tempting to ask for a simpler definition of degeneracy.
	One variant that comes to mind would be defining degeneracy so that each individual component must have a boundary consisting of trivial and small rank (co)chains.
	This would seem to be sufficient for the dimension axiom and would eliminate the difficulty described above.
	Unfortunately, with such an alternative definition of degeneracy, it will not generally be true that if $V \in Q(M)$ then fiber products $V \times_M W$ are also in $Q(M)$.
	This is an important property that will arise in the next section and then be needed both to construct cup and cap products and to prove the existence of Mayer--Vietoris sequences.
	See \cref{R: degen2} for further discussion of this point.
\end{remark}

The following algebraic property will be useful below as we consider homological algebra with geometric cochains:

\begin{lemma}
	Each $C_i^\Gamma(M)$ or $C_\Gamma^i(M)$ is torsion-free and hence flat as a $\Z$-module.
\end{lemma}

\begin{proof}
	The second statement follows from the first as $\Z$ is a Dedekind domain.
	The first statement is proven for geometric chains in \cite[Lemma 34]{Lipy14}.
	The proof for geometric cochains, even accounting for our different definition of degeneracy, is the same.
	We give the argument for chains.

	Suppose $\uW$ satisfies $n \uW = 0$ for some $n \in \Z$ with $n \neq 0$.
	Representing $\uW$ by $W$, we can decompose $W$ as in the proofs of \cref{L: Lip L10,L: Lipy12} so that $W = W_1 \sqcup W_2 \sqcup \cdots$ with each $W_i$ being the union of isomorphic-up-to-sign connected manifolds over $M$ with the isomorphism classes of components in $W_i$ and $W_j$ distinct of $i\neq j$.
	Then $n \uW$ is represented by taking $n$ copies of each $W_i$ if $n > 0$ or, if $n < 0$, taking $|n|$ copies of each $-W_i$.
	By \cref{L: co/chains well defined}, $n W \in Q_*(M)$, so each component of $n W_i$ is trivial or has small image.

	Suppose $n W_i$ is trivial for some $i$.
 	Then each component of $n W_i$ is trivial or the total number of components of $n W_i$ is zero when counted with sign.
	But this implies that each component of $W_i$ is trivial or that $W_i$ must also have the total number of components be zero when counted with sign.
	So $W_i$ is trivial.
	It is also clear that if $W_i$ is trivial, then so is $n W_i$, so $W_i$ is trivial if and only if $n W_i$ is trivial.

	Now let $W'$ be the union of those components $W_i$ that are not trivial.
	Then $nW$ is the union of trivial components and $n W'$, so by \cref{L: Lipy12}, we have $n W' \in Q_*(M)$.
	Furthermore, since none of the components of $n W'$ are trivial, $n W'$ must be degenerate.
	In particular, $n W'$ must have small rank, which implies that $W'$ must have small rank.
	Further, since $n W' \in Q_*(M)$, we have $\bd (n W') = n (\bd W') \in Q_*(M)$ by \cref{L: bd defined}.
	So by the same argument as above, $\bd W'$ must be a union of prechains that are trivial or of small rank.
	So $W'$ is degenerate.

	Altogether, we have now shown that $W$ is the union of a trivial prechain and a degenerate prechain, so $W \in Q_*(M)$.
\end{proof}

\subsubsection{Products of manifolds over \texorpdfstring{$M$}{M}}

In this section we define various products of elements of $PC_*^\Gamma(M)$ and $PC^*_\Gamma(M)$, all coming from the external products or fiber products defined above.
When we need transversality, $M$ will be without boundary.
These products will ultimately become our cup, cap, intersection, and exterior products, but we introduce them here as products on $PC_*^\Gamma(M)$ and $PC^*_\Gamma(M)$ and derive some further properties as we will need some of this material in the next section to define creasing.
The first time reader can fairly safely skip this section for now and return to it later as needed.

We begin with the products coming from fiber products.
These are only partially defined, as fiber products require transversality.

\begin{definition}\label{D: PC products}
	Given a manifold without boundary $M$, the fiber products of transverse manifolds over $M$ determine partially-defined products of the following forms:
	\begin{align*}
		PC^*_\Gamma(M) \times PC^*_\Gamma(M)& \to PC^*_\Gamma(M)\\
		PC^*_\Gamma(M) \times PC_*^\Gamma(M)& \to PC_*^\Gamma(M).
	\end{align*}
	If, furthermore, $M$ is oriented, then there is also a partially-defined product
	$$PC_*^\Gamma(M) \times PC_*^\Gamma(M) \to PC_*^\Gamma(M).$$
	In each case, the product is defined when the reference maps $r_V \colon V \to M$ and $r_W \colon W \to M$ are transverse.

		\begin{itemize}
			\item If $V,W\in PC^*_\Gamma(M)$ are transverse, we define their product to be the fiber product $V \times_M W$ with its fiber product co-orientation of \cref{D: pullback coorient}. Note that the fiber product is proper by \cref{L: co-orientable pullback} as the composition of proper maps is proper.

			\item If $V \in PC^*_\Gamma(M)$ and $W \in PC_*^\Gamma(M)$ are transverse, we define their product to be $V \times_M W$ with the orientation induced from the orientation of $W$ and the co-orientation of the pullback $V \times_M W \to W$; see \cref{D: pullback coorient} and the discussion following \cref{D: co-orientations}.
			Note that $V \times_M W$ is compact, as $W$ is compact and the pullback map is proper by \cref{L: co-orientable pullback} and the properness of $r_V$.

			\item If $M$ is oriented and $V,W \in PC_*^\Gamma(M)$, we define their product to be $V \times_M W$ with the fiber product orientation of \cref{S: orientation of fiber products}.
			Note that $V \times_M W$ is compact as a closed subset of the compact set $V \times W$.
	\end{itemize}

	We show below that these products are all well defined when the transversality conditions are met, as transversality is preserved under isomorphisms of manifolds over $M$, and isomorphic manifolds over $M$ have isomorphic fiber products over $M$.

	In all cases we continue to denote the product by $V \times_M W$, allowing context to determine which (co\nobreakdash-)orientations apply.
\end{definition}



In the last case, we really need $M$ to be oriented in general, as we have observed in \cref{R: what products exist} that if $M$ is not orientable the fiber product of orientable manifolds over $M$ may note be orientable.

\begin{comment}
In the cases where $r_V$ and $r_W$ are transverse embeddings, these products are represented by just taking intersections, with the orientations or co-orientations given explicitly in \cref{P: normal pullback,P: cap of immersions,P: orient intersection}.
If $r_V$ and $r_W$ are immersions, these descriptions hold locally.
\end{comment}

\begin{lemma}\label{L: product preserves iso}
	When the transversality conditions are met, the products of \cref{D: PC products} are well defined.
	In particular, if $\phi_V \colon V \to V'$ and $\phi_W \colon W \to W'$ are orientation- or co-orientation-preserving diffeomorphisms of transverse manifolds over $M$, then $\phi_V \times \phi_W$ restricts, as appropriate to the products of \cref{D: PC products}, to a corresponding orientation- or co-orientation-preserving diffeomorphism $V \times_M W \to V' \times_M W'$.

	Similarly, if exactly one of $\phi_V \colon V \to V'$ or $\phi_W \colon W \to W'$ is an orientation- or co-orientation-\textit{reversing} diffeomorphism, then $\phi_V \times \phi_W$ restricts, as appropriate to the products of \cref{D: PC products}, to a corresponding orientation- or co-orientation-\textit{reversing} diffeomorphism $V \times_M W \to V' \times_M W'$.
\end{lemma}
\begin{proof}
	It is clear that the transversality conditions are preserved if representatives $r_V \colon V \to M$ or $r_W \colon W \to M$ are replaced with (oriented or co-oriented) isomorphic manifolds over $M$.
	It is also clear from the universal property that isomorphic manifolds over $M$ lead to diffeomorphic fiber products.
	So our main challenge is to verify the correct behavior of orientations and co-orientations.

	We first consider $V$, $W$, and $M$ all oriented.
	Suppose we have $f \colon V \to M$ isomorphic to $f' \colon V' \to M$ via $\phi_V \colon V \to V'$ and similarly for $W$; see \cref{D: equiv triv and small}.
	The map $\phi_V \times \phi_W$ restricts to give our diffeomorphism from $P = V \times_M W$ to $P' = V' \times_M W'$, so, considering the definition of the fiber product orientation in \cref{S: orientation of fiber products}, we obtain bundle map isomorphisms covering $\phi_V \times \phi_W \colon P \to P'$ of the following exact sequences (or we can think of the bottom sequence as pulled back to $P$ by $\phi_V \times \phi_W$).
	\begin{equation}
	\begin{tikzcd}
		0 \arrow[r] & TP \arrow[r,"D\pi_V \oplus D\pi_W"] \arrow[d] &[1cm] \pi_V^*(TV) \oplus \pi_W^*(TW) \arrow[r,"\pi_V^*(Df)-\pi_W^*(Dg)"] \arrow[d] &[1.7cm] (f\pi_V)^*TM \arrow[r] \arrow[d] & 0 \\
		0 \arrow[r] & TP' \arrow[r,"D\pi_{V'} \oplus D\pi_{W'}"] & \pi_{V'}^*(TV') \oplus \pi_{W'}^*(TW') \arrow[r,"\pi_{V'}^*(Df')-\pi_{W'}^*(Dg')"] & (f'\pi_{V'})^*TM \arrow[r] & 0
	\end{tikzcd}
	\end{equation}
	As the maps $\phi_V$ and $\phi_W$ are orientation preserving, it follows that the two righthand vertical maps are oriented bundle isomorphisms.
	Consequently, if we orient $TP$ and $TP'$ as in \cref{S: orientation of fiber products} and use the isomorphisms to make the splittings of the sequences compatible, then the lefthand map must also be an isomorphism of oriented bundles.
	(Looking at the level of individual tangent space fibers, the reader can also compare with \cite[Sections 9.1.1 and 9.3.1]{RamBas09}.)

	For the other cases, we first observe that being an orientation or co-orientation preserving isomorphism is a local property.
	In particular, a diffeomorphism $V \to V'$ is orientation preserving if and only its derivative is orientation preserving at each $x \in V$, and by \cref{L: co-or preserving/reversing} a similar statement holds for co-orientations.
	Consequently, to show that a diffeomorphism over $M$ is (co-)orientation preserving, it suffices to show that this property holds over each set of an open cover of $M$.

	So let $U$ be any Euclidean subset of $M$, and suppose we give $U$ an arbitrary orientation.
	Considering the case of co-oriented precochains, by \cref{C: co-or preserving is or preserving} the restrictions of $\phi_V$ and $\phi_W$ to $f^{-1}(U)$ and $g^{-1}(U)$ are co-orientation preserving if and only if they are orientation preserving with respect to the induced orientations.
	Note that this statement is independent of the choice of fixed orientation for $U$.
	Since we assume that $\phi_V$ and $\phi_W$ are co-orientation preserving, they are thus orientation preserving over $U$ with respect to the induced orientations.
	It follows from the preceding argument that $\phi_V \times \phi_W$ restricts to an orientation-preserving diffeomorphism of the fiber products over $U$.
	By \cref{P: compare cup and intersection orientations}, there is a fixed sign relation depending only on the dimensions between the fiber product orientations obtained by the above construction and the orientations of the fiber products induced from the orientation of $U$ and the co-oriented fiber product (this is the distinction considered there between $V \times^o_M W$ and $V \times^c_M W$).
	Since $\phi_V \times \phi_W$ provides an orientation-preserving diffeomorphism with respect to the fiber product oriented the first way, it also provides an orientation-preserving diffeomorphism with respect to the fiber product oriented the other way.
	Hence by \cref{C: co-or preserving is or preserving} again, $\phi_V \times \phi_W$ restricts to a co-orientation-preserving diffeomorphism $f^{-1}(U) \times_U g^{-1}(U) \to (f')^{-1}(U) \times_U (g')^{-1}(U)$.
	Since $U$ was arbitrary, $\phi_V \times \phi_W$ gives a co-orientation-preserving diffeomorphism on all of $V \times_M W$.

	Lastly, for the fiber product $PC^*_\Gamma(M) \times PC_*^\Gamma(M) \to PC_*^\Gamma(M)$, we again begin by considering what happens for the preimage of a Euclidean set $U$ in $M$ with an arbitrary orientation $\beta_U$.
	In this case the restriction $g \colon g^{-1}(U) \to U$ is a map between oriented manifolds and so if $W$ has orientation $\beta_W$, this map has an induced co-orientation $(\beta_W, \beta_U)$.
	Now, by definition, $\beta_P$ is the desired orientation of $f^{-1}(U) \times_U g^{-1}(U)$ for this kind of product if and only $(\beta_P,\beta_W)$ is the pullback co-orientation of $f^{-1}(U) \times_U g^{-1}(U) \to g^{-1}(U)$. And this is the case if and only if $(\beta_P,\beta_W)*(\beta_W,\beta_U) = (\beta_P, \beta_U)$ is the co-orientation of the co-oriented fiber product of $f$ and $g$ restricted to $f^{-1}(U)$ and $g^{-1}(U)$, obtained from $g$ being given the induced co-orientation as above.
	Furthermore, as $\phi_W$ is orientation-preserving, its restriction to a diffeomorphism $g^{-1}(U)$ to $(g')^{-1}(U)$ is also co-orientation preserving with respect to the induced co-orientations by \cref{C: co-or preserving is or preserving}.
	But now we have seen that the product $\phi_V \times \phi_W$ of co-orientation-preserving diffeomorphisms gives a co-orientation preserving diffeomorphism on $f^{-1}(U) \times_U g^{-1}(U)$.
	So now again by \cref{C: co-or preserving is or preserving}, $\phi_V \times \phi_W$ is an orientation-preserving diffeomorphism with respect to the orientations induced by the orientation of $U$ and the fiber product co-orientations.
	But we have just recalled that these are exactly the orientations that we want for these products.
\end{proof}

\begin{comment}
\begin{remark}\label{R: product preserves reverse}
	It is easy to modify the preceding arguments to show that if exactly one of $\phi_V \colon V \to V'$ or $\phi_W \colon W \to W'$ is an orientation- or co-orientation-\textit{reversing} diffeomorphism, then $\phi_V \times \phi_W$ restricts to a corresponding orientation- or co-orientation-\textit{reversing} diffeomorphism $V \times_M W \to V' \times_M W'$.
\end{remark}
\end{comment}

Given the preceding lemma, from here on we will usually not explicitly distinguish between maps (co\nobreakdash-)oriented maps $V \to M$ and their isomorphism classes when discussing products.






The next lemma will be critical in \cref{S: products} toward showing that these products extend to well-defined, though only partially-defined, products of geometric chains and cochains.
It will also be needed much sooner to show that the creasing construction is well defined.
This construction is used, in turn, to demonstrate the existence of Mayer--Vietoris sequences.

For the statement, recall that we use $Q(M)$ to stand for $Q_*(M)$ or $Q^*(M)$ as appropriate.

\begin{lemma}\label{L: pullback with Q}
	For any of the products of \cref{D: PC products}, if either $V$ or $W$ is in $Q(M)$ then so is $V \times_M W$.
	In fact, if $V$ or $W$ is trivial then $V \times_M W$ is trivial, if $V$ or $W$ has small rank then $V \times_M W$ has small rank, and if $V$ or $W$ is degenerate then $V \times_M W$ is degenerate.
\end{lemma}

\begin{proof}
	We provide the proof if $V \in Q(M)$; the other case is similar.
	By assumption $V$ is the disjoint union of trivial and degenerate chains or cochains, so it suffices to consider independently the possibilities that $V$ is trivial or degenerate.

	If $\rho$ is a (co\nobreakdash-)orientation reversing diffeomorphism of $V$ over $M$, then $\rho \times_M \id_W$ is a (co\nobreakdash-)orientation reversing diffeomorphism of $V \times_M W$ by \cref{L: product preserves iso}.

	Next assume that $V$ is degenerate, so in particular it has small rank.
	Recall that the tangent bundle of a fiber product is the fiber product of the tangent bundles by \cref{L: tangent of pullbacks}, and so the derivative is the fiber product of derivatives.
	Note that the fiber product of two linear maps, one with a non-trivial kernel, must also have a non-trivial kernel: If $A,B$ are linear maps with a common codomain and $v \in \ker(A)$, then $(v,0)$ is in the kernel of the fiber product of $A$ and $B$.
	So if the differential of $r_V$ has non-trivial kernel everywhere so will the derivative of any fiber product with $r_V$.
	Thus $V \times_M W$ has small rank.

	Now we recall that $\bd(V \times_M W)$ is, up to (co\nobreakdash-)orientations, the disjoint union of $(\bd V) \times_M W$ and $V \times_M (\bd W)$.
	We have just shown that $V \times_M (\bd W)$ must have small rank.
	As $V$ is degenerate, $\bd V$ is a disjoint of trivial and small rank manifolds over $M$, and so by the preceding arguments $(\bd V) \times_M W$ will be a union of trivial and small rank manifolds over $M$.
	Altogether, $V \times_M W$ is degenerate.
\end{proof}

\begin{remark}\label{R: degen2}
	As noted in \cref{R: degen1}, it is this lemma that fails if we attempt to simplify the definition of degeneracy by requiring each connected component of a degenerate prechain or precochain to have small rank and a boundary that is a union of trivial and small rank pre(co)chains.
	In fact, it is possible to construct a $V$ and $W$ such that $V$ is a non-trivial prechain that is degenerate in this stronger sense but such that $V \times_M W$ has multiple components that are each non-trivial and of small rank but such that the boundary of each component is non-trivial and not of small rank.
	So $V \times_M W$ would not be in a version of $Q(M)$ defined using this stronger, but simpler, notion of degeneracy.
	Of course it is in $Q(M)$ with our actual definitions by the preceding lemma.
\end{remark}

We have similar results for the exterior products studied in \cref{S: exterior products}, which are always fully defined:


\begin{lemma}\label{L: ext product preserves iso}
	Suppose $f \colon V \to M$ and $g \colon W \to N$ are maps of manifolds with corners with either $V$ and $W$ both oriented or $f$ and $g$ both co-oriented.
	Then the corresponding exterior product $PC_*^\Gamma(M) \otimes PC_*^\Gamma(N) \to PC_*^\Gamma(M \times N)$ or $PC^*_\Gamma(M) \otimes PC^*_\Gamma(N) \to PC^*_\Gamma(M \times N)$ is well defined.

	In particular, if $\phi_V \colon V \to V'$ and $\phi_W \colon W \to W'$ are orientation- or co-orientation-preserving diffeomorphisms over $M$ and $N$, respectively, then $\phi_V \times \phi_W$ is a corresponding orientation- or co-orientation-preserving diffeomorphism $V \times W \to V' \times W'$.

	Similarly, if exactly one of $\phi_V \colon V \to V'$ or $\phi_W \colon W \to W'$ is an orientation- or co-orientation-\textit{reversing} diffeomorphism, then $\phi_V \times \phi_W$ is a corresponding orientation- or co-orientation-reversing diffeomorphism $V \times W \to V' \times W'$.
\end{lemma}
\begin{proof}
	The oriented case is standard. For the co-oriented case, we can proceed analogously to the proof of \cref{L: product preserves iso}.
	In particular, we can choose Euclidean neighborhoods $A \subset M$ and $B \subset N$ given arbitrary orientations.
	Then, as in that preceding proof, isomorphisms of product co-orientations over $A \times B$ follow from isomorphisms of product orientations over $A \times B$, now using \cref{P: compare exterior orientations} rather than \cref{P: compare cup and intersection orientations}.
\end{proof}


\begin{lemma}\label{L: exterior Q}
	Consider two pre(co)chains $f \colon V \to M$ and $g \colon W \to N$.
	If $V \in Q(M)$ or $W \in Q(N)$ then $f \times g \colon V \times W \to M \times N$ is in $Q(M \times N)$.
\end{lemma}

\begin{proof}
	We provide the proof if both maps are in $PC^*$ and $V \in Q^*(M)$; the other cases are similar.
	By assumption $V$ is the disjoint union of trivial and degenerate chains or cochains, so it suffices to consider independently the possibilities that $V$ is trivial or degenerate.

	If $\rho$ is a co-orientation-reversing diffeomorphism of $V$ over $M$, then $\rho \times \id_W$ is a co-orientation-reversing diffeomorphism of $V \times_M W$ by \cref{L: ext product preserves iso}.
	So if $V$ is trivial so is $V \times W$.

	Next assume that $V$ is degenerate, so in particular it has small rank.
	At any point, the derivative of $f \times g$ is a matrix with $Df$ and $Dg$ on the block diagonals, so $f \times g$ has small rank.
	Now we recall that $\bd(V \times W)$ is, up to co-orientations, the disjoint union of $(\bd V) \times_M W$ and $V \times_M (\bd W)$.
	We have just shown that $V \times_M (\bd W)$ must have small rank.
	As $V$ is degenerate, $\bd V$ is a disjoint union of trivial and small rank manifolds over $M$, and so by the preceding arguments $(\bd V) \times_M W$ will be a union of trivial and small rank manifolds over $M$.
	Altogether, $V \times_M W$ is degenerate.
\end{proof}

\begin{comment}
	The next theorem, which is yet another Leibniz formula, in this case for the pullback of a co-oriented manifold over $M$ by an oriented manifolds over $M$, will be used in Section \ref{S: intersection map} to demonstrate that our intersection map $\mc I$, relating geometric cochains to cubical cochains of a cubulation, is a chain map.
	This map is a critical component in relating geometric cohomology to other cohomology theories and is also central to the main result about cup products in \cite{FMS-flows}.

	\begin{theorem}\label{T: Leibniz cap}
		Let $M$ be a manifold without boundary.
		Suppose $V \in PC^*_\Gamma(M)$ and $W \in PC_*^\Gamma(M)$ with transverse reference maps.
		Consider $V \times_M W$ as representing an element of $PC_*^\Gamma(M)$ as constructed in Definition \ref{D: PC products}.
		Then in $PC_*^\Gamma(M)$, $$\bd(V \times_M W) = \left[(-1)^{\dim(V \times_M W)} (\bd V) \times_M W\right] \bigsqcup V \times_M \bd W.$$
	\end{theorem}
	\begin{proof}
		By Definition \ref{D: PC products} the terms $\bd(V \times_M W)$, $(\bd V) \times_M W$, and $V \times_M \bd W$ are each oriented by considering, as appropriate, the pullback of the co-oriented map $V \to M$ or $\bd V \to M$ to a co-oriented map to $W$ or $\bd W$ and then using the orientation of $W$ or $\bd W$ to determine an orientation of the pullback.
		So we compute and compare these orientations by first considering the pullback co-orientations as defined in Definition \ref{D: pullback coorient}.
		We proceed by analogy to the proof of the Leibniz rule for the pullback of co-oriented maps in Theorem \ref{leibniz}, utilizing the computations already performed there.

		Recall, in brief, from \ref{D: pullback coorient} that to co-oriented the pullback $P = V \times_M W \to W$ we first construct a composition $V\xhookrightarrow{e} M \times \R^N \to M$ and find a Quillen orientation for the normal bundle $\nu V$ of $e(V) \subset M \times \R^N$ as determined by the co-orientation of $V \to M$.
		Then we pull back via $W \times \R^N \to M \times \R^N$ to obtain a normal bundle, also labeled $\nu V$, of $P \subset W \times \R^N$.
		Then we co-orient $P \to W$ locally by $(\beta_P,\beta_W)$ so that $\beta_P \wedge \beta_{\nu V} = \beta_W \wedge \beta_E$, where $\beta_E$ represents the standard orientation of $\R^N$.
		In the case at hand, we can assume $\beta_W$ to represent the global orientation of $W$, and then $\beta_P$ becomes a global orientation for $P$.
		This is the orientation given to $V \times_M W$ in Definition \ref{D: PC products}.

		Let $\nu\bd P$ denote an outward pointing normal vector in the tangent bundle to $P$ at a boundary point of $P$, and let $\beta_{\nu\bd P}$ denote the corresponding orientation.
		Then, by definition, $\bd P$ is oriented at that point by $\beta_{\bd P}$ so that $\beta_{\nu\bd P} \wedge \beta_{\bd P} = \beta_P$.
		In other words, with $\beta_P$, $\beta_W$, $\beta_{\nu V}$, and $\beta_E$ given, $\bd P$ is oriented by $\beta_{\bd P}$ such that $\beta_{\nu\bd P} \wedge \beta_{\bd P} \wedge \beta_{\nu V} = \beta_W \wedge \beta_E$.

		Now, recall that $\bd(V \times_M W) = (\bd V) \times_M W \bigsqcup V \times_M \bd W$ as spaces and consider a point in $(\bd V) \times_M W$.
		By Theorem \ref{leibniz}, at such a point the pullback co-orientation of $(\bd V) \times_M W \to W$ agrees with boundary co-orientation of the pullback $P = V \times_M W \to W$, as described again in the preceding paragraph.
		So continuing to let $(\beta_P,\beta_W)$ denote the pullback co-orientation of $P \to W$ and recalling that the boundary co-orientation utilizes the \textit{inward} normal, the boundary co-orientation of $(\bd V) \times_M W \to W$ is the composite $(\beta_{\bd P}, \beta_{\bd P} \wedge -\beta_{\nu\bd P})*(\beta_P,\beta_W)$ for any $\beta_{\bd P}$.
		But if we choose $\beta_{\bd P}$ to represent the orientation of $\bd P$ found above by orienting $P$ and then taking its boundary orientation, we have $\beta_P = \beta_{\nu\bd P} \wedge \beta_{\bd P} = (-1)^{\dim(\bd P)}\beta_{\bd P} \wedge \beta_{\nu\bd P}$.
		So the boundary co-orientation of $(\bd V) \times_M W \to W$ is the composite
		$$(\beta_{\bd P}, \beta_{\bd P} \wedge -\beta_{\nu\bd P})*((-1)^{\dim(\bd P)}\beta_{\bd P} \wedge \beta_{\nu\bd P},\beta_W) = (-1)^{\dim(\bd P)+1}(\beta_{\bd P},\beta_W).$$
		Thus the resulting orientation of $(\bd V) \times_M W$ is $(-1)^{\dim(P)}$ times the orientation of $\bd P$ obtained by taking the oriented boundary of $V \times_M W$.

		Next we consider a point in $V \times_M \bd W$.
		From the Leibniz rule computation for co-orientations, the co-orientation of the pullback $V \times_M \bd W \to \bd W$ is $(\beta_{\bd P},\beta_{\bd W})$ if and only if $\beta_{\bd P} \wedge \beta_{\nu V} = \beta_{\bd W} \wedge \beta_E$.
		If $\nu\bd P$ is an outward normal then this is equivalent to $\beta_{\nu\bd P} \wedge \beta_{\bd P} \wedge \beta_{\nu V} = \beta_{\nu\bd P} \wedge \beta_{\bd W} \wedge \beta_E = \beta_W \wedge \beta_E$.
		But taking $\beta_P$ and $\beta_{\bd P}$ as found above using our given orientation of $\beta_W$, we have $\beta_{\nu\bd P} \wedge \beta_{\bd P} = \beta_P$, and so the condition is equivalent to $\beta_{P} \wedge \beta_{\nu V} = \beta_W \wedge \beta_E$, which holds by our choice above of $\beta_P$.
		So the orientation of $V \times_M \bd W$ agrees with the orientation of $\bd P$ and we obtain overall $$\bd(V \times_M W) = \left[(-1)^{\dim(P)} (\bd V) \times_M W\right] \bigsqcup V \times_M \bd W.$$

		Finally, we use that $\dim(P) = \dim(V)+\dim(W)-\dim(M)$.
	\end{proof}
\end{comment}

\begin{comment}
	\textbf{Products of immersions.}
	In \cref{S: co-or product immersion} we saw that the fiber product of immersions $V,W \into M$ of co-oriented manifolds over $M$ have particularly simple descriptions.
	Geometrically, the fiber product is just the intersection\footnote{Technically, it is the intersection only \textit{locally} as, for example, $V$ might intersect $W$ transversely in a double point of $W$, but by restricting to small enough neighborhoods of $V$ and $W$ we can treat the resulting $V \times_M W$ locally as two intersection points that just happen to be in the same place.
		The reader unhappy with that kind of thinking can restrict their attention to embeddings or just interpret each statement here as a local statement for suitably small neighborhoods of $V$ and $W$.
	} of $V$ and $W$, and if the co-orientations are given by the Quillen orientations of the normal bundles $\nu V$ and $\nu W$ (see \cref{R: immersion}), then $V \times_M W$ has normal bundle $\nu V \oplus \nu W$ co-oriented with Quillen normal orientation given by $\beta_{\nu V} \wedge \beta_{\nu W}$.

	We note in this brief section the corresponding nice descriptions of second and third products of \cref{D: PC products} when $V$ and $W$ are immersed in $M$.
	In both cases, $P = V \times_M W$ is again simply the intersection of $V$ and $W$ in $M$, so we need only describe the orientation of $P$.

	For the second product, $P$ is oriented using the pullback co-orientation of $V \times_M W \to W$ and the orientation of $W$.
	If $\nu V$ is the normal bundle of $V$ in $M$, then taking $N = 0$ in \cref{D: pullback coorient},
	the pullback co-orientation is $(\beta_P,\beta_W)$ if $\beta_P \wedge \beta_{\nu V} = \beta_W$ up to positive scalar, where we recall our abuse of notation letting $\nu V$ also represent the pullback of $\nu V$ to $W$, where it can then be interpreted as the normal bundle to $P$ in $W$.
	Then by the discussion following \cref{D: co-orientations} this determines an orientation of $P$ by taking $\beta_W$ to agree with our given orientation of $W$.
	In other words, the orientation of the product $P = V \times_M W$ is the orientation $\beta_P$ so that $\beta_P \wedge \beta_{\nu V}$ agrees with the given orientation of $W$ when $\beta_{\nu V}$ is the restriction to $P$ of the Quillen normal orientation of $V$ determined by the co-orientation of $r_V \colon V \to M$ (identifying the restriction of the normal bundle of $V$ to $P$ as the normal bundle to $P$ in $W$).

	For the third product, we have by Joyce's convention that \red{I'm not getting a simple description of this because of Joyce's desription.
		Note that it requires using a specific identification of $TV \oplus TW$ with $TP \oplus TM$.
		I compute that even if we use identify $TM = \nu W \oplus TP \oplus \nu V$, $TV = \nu W \oplus TP$, and $TW = TP \oplus \nu V \oplus TP$ consistently, there's a sign of $(-1)^{m-w}$ from the determinant of Joyce's isomorphism, and then he enforces another factor of $(-1)^{mw}$ by hand.
		Yuck.
		And this is before comparing the actual orientations of things - yuck.
		We'll have to think some more about this.
	}.
\end{comment}

\subsection{Splitting and creasing}\label{S: splitting and creasing}

In this section we discuss the closely related notions of splitting and creasing, both of which are concerned with breaking chains and cochains into smaller pieces.
The idea is somewhat analogous to the role subdivisions play in classical singular homology and cohomology theory, and both will be essential in our discussion of Mayer--Vietoris sequences,

The idea of splitting is to take a manifold with corners $W$ over a manifold without boundary $M$ and split it along a codimension one submanifold with corners into two manifolds with corners, $W^+$ and $W^-$.
This have various uses.
In particular, when $W$ represents a cycle or cocycle, the creasing construction then shows that $W$ is homologous (or cohomologous) to $W^+ + W^-$.


\subsubsection{Splitting}\label{S: splitting}
In this section, we begin by consolidating and extending some results previously encountered in \cref{E: manifold decomposition,S: codim 0 and 1 co-or}, namely the splitting of a manifold with corners over a manifold without boundary $r_W \colon W \to M$ into pieces $W^+$ and $W^-$ with common boundary $W^0$ using a function $\phi \colon M \to \R$.
In this subsection, we use our previous results to establish the facts we will need in the context of prechains and precochains, and in the next subsection we will utilize these spaces to perform creasing.

Our standing assumptions throughout this section will be that $M$ is a manifold without boundary and $\phi \colon M \to \R$ is a smooth map having $0$ as a regular value in the classical sense, i.e.\ for all $x \in \phi^{-1}(0)$ the differential $D_x\phi$ is nonzero.
We also assume an element of $PC_*^\Gamma(M)$ or $PC^*_\Gamma(M)$ represented by $r_W \colon W \to M$ such that $0$ is also a regular value for $\phi r_W$.
By \cref{D: regular value} this means that $\phi r_W$ is transverse to $0$, which is equivalent to assuming $0$ is a regular value in the classical sense for the restriction of $\phi r_W$ to each stratum of $W$.
By \cref{L: transverse to pullback}, $0$ is a regular value for $\phi r_W$ precisely when $r_W$ is transverse to $0 \times_\R M$, which is simply $\phi^{-1}(0)$, as observed in \cref{E: manifold decomposition}.

We let
\begin{align*}
M^0 = \phi^{-1}(0) && M^- = \phi^{-1}((-\infty,0]) &&M^+ = \phi^{-1}([0,\infty)).
\end{align*}
 Analogously,
\begin{align*}
W^0 = (\phi r_W)^{-1}(0) && W^- = (\phi r_W)^{-1}((-\infty,0]) && W^+ = (\phi r_W)^{-1}([0,\infty)).
\end{align*}
We sometimes write $M^\pm$ for statements that could involve either $M^+$ or $M^-$, and similarly for $W$.

\begin{remark}
	For simplicity of notation, we primarily use $0$ as our value in $\R$ at which to perform splittings, though it should be clear that we could split $W$ using any regular value of $\phi$ and $\phi r_W$ to obtain analogous results.
\end{remark}

\begin{comment}
\begin{lemma}\label{L: 0 transverse M0}
	Zero is a regular value for $\phi r_W$ if and only if $r_W$ is transverse to the inclusion of $M^0$ into $M$.
\end{lemma}
\begin{proof}
	As $0$ is a regular value for $\phi$, by classical differential topology $M^0$ is an embedded codimension-one submanifold of $M$, and in a neighborhood of each point of $M^0$, the map $\phi$ behaves up to diffeomorphisms like the standard projection of $\R^m$ to the first coordinate; see Section 1.4 \cite{GuPo74}.
	In particular, at each $z \in M^0$, we have $T_zM^0 = \ker(D_z\phi)$.
	So the linear subspace spanned by a vector $v \in T_zM$ is transverse to $T_zM^0$ if and only if its image under $D_z\phi$ is non-zero.
	It follows that $r_W$ is transverse to $M^0$ if and only if $\phi r_W$ is transverse to $0$.
\end{proof}
\end{comment}

\begin{lemma}\label{L: pm0 as fiber products}
	With our standing assumptions, there are diffeomorphisms
	\begin{align*}
		M^0 & \cong 0 \times_{\R} M &W^0 & \cong 0 \times_{\R} W & W^0& \cong M^0 \times_M W \\
		M^- & \cong (-\infty,0] \times_{\R} M &W^- & \cong (-\infty,0] \times_{\R} W& W^-  &\cong M^- \times_M W\\
		M^+ & \cong [0,\infty) \times_{\R} M & W^+ & \cong [0,\infty) \times_{\R} W& W^+ &\cong M^+ \times_M W.
	\end{align*}
	In particular, these spaces are all manifolds with corners.
	Note that it is possible for some of these spaces to be empty.
\end{lemma}
\begin{proof}
	These diffeomorphisms are discussed in \cref{E: manifold decomposition}.
	The first two columns hold by direct observation, while the rightmost column is a consequence of \cref{P: pullback functoriality}.
\end{proof}

The rightmost column above demonstrates $W^0$, $W^-$, and $W^+$ as fiber products over $M$.
We can use these descriptions to realize these spaces as prechains or precochains.
Note: depending on $\phi$ and $r_W$, some of these spaces may be empty, in which case appropriate versions of the following statements hold vacuously.

\begin{lemma}\label{L: W0 cochain}
	Suppose $W \in PC^*_\Gamma(M)$, and $0$ is a regular value of $\phi \colon M \to \R$ and $\phi r_W \colon W \to \R$.
	Let the inclusions $M^\pm \into M$ have their tautological co-orientations, and give the inclusion $M^0 \into M$ the co-orientation determined by its normal vector field oriented by pulling back the positively-oriented normal vector field over $0$ in $\R$.
	Then $W^0$, $W^-$, and $W^+$, defined as the fiber products of $M^0$, $M^-$, and $M^+$ with $W$ over $M$, are elements of $PC^*_\Gamma(M)$.
	The co-orientations of $W^\pm \to M$ are the restrictions to $W^\pm$ of the co-orientation of $W$.
	Furthermore,
	\begin{align*}
		\bd(W^-) &=  (-W^0) \bigsqcup (\bd W)^- \\
		\bd (W^+) &= W^0 \bigsqcup (\bd W)^+\\
		(\bd W)^0 &= -\bd (W^0).
	\end{align*}
\end{lemma}
\begin{proof}
	As $M^\pm$ and $M^0$ are all closed subsets of $M$, their inclusions are all proper maps, so with the co-orientations assigned above, they all represent elements of $PC^*_\Gamma(M)$.
	By \cref{L: transverse to pullback}, the map $r_W \colon W \to M$ is transverse to $M^0$ and $M^\pm$.
	It follows that $W^0$, $W^-$, and $W^+$, defined as the fiber products of $M^0$, $M^-$, and $M^+$ with $W$, are elements of $PC^*_\Gamma(M)$ by \cref{D: PC products,L: product preserves iso}.
	The co-orientations have been discussed previously in \cref{S: codim 0 and 1 co-or}, and in particular we have the boundary computations from \cref{E: codim 1 pullbacks,C: co-orient W0}.
\end{proof}


\begin{lemma}\label{L: W0 chain}
	Suppose $W \in PC_*^\Gamma(M)$, and $0$ is a regular value of $\phi \colon M \to \R$ and $\phi r_W \colon W \to \R$.
	Let $\R$, $(-\infty, 0]$, and $[0,\infty)$ have their standard orientations, and give $0$ its positive orientation.
	Now orient $W^0$, $W^-$, and $W^+$ by realizing them as oriented fiber products of $0$,  $(-\infty, 0]$, and $[0,\infty)$ with $\phi r_W \colon W \to \R$ over $\R$.
	Then the restrictions of $r_W \colon W \to M$ to $W^0$, $W^-$, and $W^+$ realize elements of $PC_*^\Gamma(M)$.
	When $W^\pm$ are nonempty, their orientations agree with the orientation of $W$.
	Furthermore, as elements of $PC_*^\Gamma(M)$, we have\footnote{Compare the signs with those in \cref{L: W0 cochain}.}
	\begin{align*}
	\bd(W^-) &=  W^0 \bigsqcup (\bd W)^- \\
	\bd(W^+) &= (-W^0) \bigsqcup (\bd W)^+\\
	(\bd W)^0 &= -\bd (W^0).
	\end{align*}
\end{lemma}

\begin{proof}
	Since $W$ is compact, so will be $W^0$, $W^-$, and $W^+$ as closed subsets of $W$.
	So giving each space the orientation from the statement of the lemma, the restrictions of $r_W$ to $W^0$, $W^-$, and $W^+$ are elements of $PC_*^\Gamma(M)$.

	We consider a point $x$ in the interior of $W^+$ with an open neighborhood $N$ also in the interior of $W^+$ and given the orientation restricted from $W$.
	As the orientations of the fiber products are determined locally, the orientation of $W^+$ at $x$ will be consistent with its orientation in the restricted fiber product $[0,\infty) \times_\R N$, which will be the same as its orientation in $(0,\infty) \times_{(0,\infty)} N$.
	But this is the same as the initial orientation of $N$ as a subset of $W$ by \cref{P: oriented fiber product basic properties}.
	The same argument holds for $W^-$.

	We then have by \cref{P: oriented fiber boundary}, our conventions for oriented boundaries, and the standard computations for the boundaries of $(-\infty,0]$ and $[0,\infty)$ that
	\begin{equation*}
			\bd W^- = \bd ((-\infty,0] \times_\R W) = (0 \times_\R W)  \sqcup ((-\infty,0] \times_\R \bd 	W) = W^0 \sqcup (\bd W)^-,
	\end{equation*}
	and
	\begin{equation*}
			\bd W^+ = \bd ([0,\infty) \times_\R W) = (-0 \times_\R W)  \sqcup ([0,\infty) \times_\R \bd W) 	= (-W^0) \sqcup (\bd W)^+.
	\end{equation*}
	We also compute
	\begin{equation*}
			\bd W^0 = \bd (0 \times_\R W) = - 0 \times_\R \bd W = -(\bd W)^0. \qedhere
\end{equation*}
\end{proof}

Lastly, we will occasionally need to consider the preimage of an interval $[p,q] \subset \R$.
If $p < q$ are regular values for $\phi$, then $\phi^{-1}([p,q])$ will be an embedded manifold with boundary in $M$ that we denote $M^{[p,q]}$.
If further $\phi r_W$ is transverse $p$ and $q$ then we can form $W^{[p,q]} = M^{[p,q]} \times_M W$.
We also let $W^p = \phi^{-1}(p) \times_M W$, and similarly for $q$.

\begin{lemma}\label{L: Wpq cochain}
	If $p < q$ are regular values for $\phi \colon M \to \R$, if $W \in PC^*_\Gamma(M)$, and if the inclusion $M^{[p,q]} \into M$ is given its tautological co-orientation, then
	$W^{[p,q]} \in PC^*_\Gamma(M)$ and
	$$\bd W^{[p,q]} = W^p \sqcup -W^q \sqcup (\bd W)^{[p,q]}.$$
	Similarly, if $W \in PC_*^\Gamma(M)$, then so is $W^{[p,q]}$, its orientation agrees with that of $W$, and
	$$\bd W^{[p,q]} = -W^p \sqcup W^q \sqcup (\bd W)^{[p,q]}.$$
\end{lemma}
\begin{proof}
	As $M^{[p,q]}$ is a closed subset of $M$, its inclusion into $M$ is proper, so with the tautological co-orientation,  $M^{[p,q]} \into M$ is an element of $PC^*_\Gamma(M)$.
	We note that we only need to check transversality at $p$ and $q$, as the inclusion of the interior of $M^{[p,q]}$ to $M$ is transverse to any map.

	In the precochain case, as $M^{[p,q]}$ is a closed subset of $M$, its inclusion into $M$ is proper, so with the tautological co-orientation,  $M^{[p,q]} \into M$ is an element of $PC^*_\Gamma(M)$.  Thus the fiber product $W^{[p,q]} = M^{[p,q]} \times_M W$ is an element of $PC^*_\Gamma(M)$ by \cref{D: PC products,L: product preserves iso}, and the boundary formula follows from the Leibniz rule analogously to our computations above for $W^\pm$.

	In the prechain case, we observe that $W^{[p,q]} \cong (\phi r_W)^{-1}([p,q])$ by similar arguments as for $W^\pm$ in \cref{E: manifold decomposition}, so $W^{[p,q]}$ is compact.
	The agreement of orientation and the boundary formula are similar to the arguments of \cref{L: W0 chain}.
\end{proof}




\subsubsection{Creasing}\label{S: creasing}



\begin{comment}
To make this precise, suppose we are given a function $\varphi \colon W \to (-1,1)$ with $0$ a regular value, i.e.\ that $\varphi$ is transverse to the inclusion of $0$ into $(-1,1)$.
We define $W^+$ to be $\varphi^{-1}( [0,1))$, we define $W^-$ to be $\varphi^{-1} ((-1, 0])$, and we define $W^0 = \varphi^{-1}(0)$.
We have already seen this construction in \cref{E: manifold decomposition} and the examples in \cref{S: codim 0 and 1 co-or}, although we here replace the codomain $\R$ with $(-1,1)$ for reasons that will become apparent momentarily.
Nonetheless, the results of those examples carry over in the obvious way.

As in those earlier examples, $W^+$, $W^-$, and $W^0$ are manifolds with corners whose inclusions into $W$ are proper maps.
They are also the fiber products $[0,1)\times_\R W$, $(-1,0] \times_\R W$ and $0 \times_\R W$.
In those previous examples, we also considered the case of having an intermediary manifold $M$ so that $\varphi$ was a composition $W \xr{r_W} M \xr{\phi} \R$ with $0$ a regular value of both $\phi$ and $\phi r_W$.
In this case we also observed in \cref{E: manifold decomposition} that $W^0$ and $W^\pm$ are diffeomorphic to the respective fiber products $M^0 \times_M W$ and $M^{\pm} \times_M W$.
We will need these settings once we have finished with some preliminaries.
\end{comment}

We now discuss the creasing construction of \cite[Section 2.4]{Lipy14}, though we use different orientation conventions and also consider versions involving co-orientations.

Suppose we are given a function $\varphi \colon W \to (-1,1)$ with $0$ a regular value, i.e.\ that $\varphi$ is transverse to the inclusion of $0$ into $(-1,1)$.
Below $\varphi$ will typically be a composition $W \xr{r_W} M \xr{\phi} (-1,1)$.
As in the preceding section, we define $W^+$ to be $\varphi^{-1}( [0,1))$, we define $W^-$ to be $\varphi^{-1} ((-1, 0])$, and we define $W^0 = \varphi^{-1}(0)$.
Note that here our codomain is $(-1,1)$ for reasons that will become apparent momentarily, though this is certainly still consistent with our maps being to $\R$ as above, using $(-1,1) \subset \R$.

By \cite[Lemma~9]{Lipy14}, the idea of creasing is that the topological space $W \times [0,1]$ can be given the structure of a manifold with corners, which we will write $\Cre(W)$, satisfying $\bd \Cre(W) = W \sqcup -W^+ \sqcup -W^- \sqcup - \Cre(\bd W)$.
We call the manifold with corners $\Cre(W)$ the \textbf{creasing homotopy} of $W$.
The creasing homotopy depends on $\varphi$, but we usually leave it tacit in the notation.
We will define $\Cre(W)$ by using pullbacks, which will provide an arguably simpler description of $\Cre(W)$ than found in \cite{Lipy14}.

To define $\Cre(W)$ via pullbacks, we need another map in addition to $\varphi$.
We let $D$ be the semi-open pentagonal region of the plane given by
$$D = \{(x,y) \in \R^2 \mid -1<x<1, 0 \leq y \leq 2-|x|\}.$$
Then $D$ is a manifold with corners with a smooth proper projection map $\pi \colon D \to (-1,1)$ given by $\pi(x,y) = x$.
We see that $\bd D$ has three pieces, say $D_x$, $D_+$, and $D_-$, corresponding respectively to the intersection of $D$ with the $x$-axis, the graph of $y = 2-x$ over $[0,1)$, and the graph of $y = 2+x$ over $(-1, 0]$.
We orient all three pieces by rightward pointing tangent vectors in the plane and give $D$ itself the standard planar orientation so that $\pi$ restricts to oriented diffeomorphisms from $D_x$, $D_+$, and $D_-$ onto their images in $(-1,1) = D_x$.
Then as oriented manifolds with corners, $$\bd D = D_x \sqcup -D_- \sqcup -D_+.$$

To obtain analogous boundary formulas for creasings of cochains, we let the projections of $D_x$, $D_+$, and $D_-$ to $(-1,1)$ be co-oriented by taking the rightward orientations to the rightward orientations, and $\pi \colon D \to (-1,1)$ to be co-oriented by taking the standard planar orientation to the rightward orientation.
In this case, as co-oriented manifolds with corners over $(-1,1)$, we have
$$\bd D = D_x \sqcup -D_- \sqcup -D_+.$$

The projection $\pi \colon D \to (-1,1)$ restricts to submersions from $D_x$, $D_+$, and $D_-$ onto their images, and so a map $\varphi \colon W \to (-1,1)$ from a manifold with corners is transverse to $\pi$ if and only if it is transverse to the map from the point at the tip of the pentagon to $(-1,1)$.
This is equivalent to the requirement that the restriction of $\varphi$ to every stratum of $W$ have $0$ as a regular value, i.e.\ that $0$ is a regular value for $\varphi$ by \cref{D: regular value}.

We can now officially define creasing.

\begin{definition}
	Suppose $W$ is a manifold with corners and $\varphi \colon W \to (-1,1)$ is a smooth map with a regular value at $0$.
	We define the \textbf{creasing homotopy} to be the pullback
	$$\Cre(W) = D\times_{(-1,1)} W \to W.$$
	We note that $\Cre(W)$ does depend on the map $\varphi$, though we typically omit it from the notation as the choice of $\varphi$ will usually be clear from context.
	When necessary for clarity, we may write $\Cre_\varphi(W)$.

	Typically in practice, $W$ will arise with a map $r_W \colon W \to M$ representing an element of $PC_*^\Gamma(M)$ or $PC^*_\Gamma(M)$, and in this case our map $\varphi \colon W \to (-1,1)$ will generally be given as a composition $W \xr{r_W}M \xr{\phi} (-1,1)$ for some smooth $\phi \colon M \to (-1,1)$ with $0$ as a regular value.
	In this case, the condition that $\varphi$ have $0$ as a regular value is equivalent to the condition that $r_W$ be transverse to the submanifold $M^0 = \phi^{-1}(0)$ by \cref{L: transverse to pullback}, and as the creasing is governed by $\phi$, we sometimes write $\Cre_\phi(W)$ when we want to emphasize the dependence on $\phi$.
	When $r_W \colon W \to M$ represents an element of $PC_*^\Gamma(M)$, we treat $D\times_{(-1,1)} W$ as oriented by the convention for fiber products of oriented manifolds from \cref{S: orientation of fiber products}, using our fixed orientation of $D$, and we will show in a moment that if $W$ is compact so is $\Cre(W)$.
	Thus composing the pullback map $\Cre(W) \to W$ with $r_W$, we obtain another element of $PC_*^\Gamma(M)$ given by $D\times_{(-1,1)} W \to W \xr{r_W}M$.
	Similarly, when $r_W \colon W \to M$ represents an element of $PC^*_\Gamma(M)$, we treat $\Cre(W) \to W$ as co-oriented by the pullback conventions of \cref{D: pullback coorient}, using our given co-orientation of $D \to (-1,1)$, and this is a proper map by \cref{L: co-orientable pullback}.
	Thus composing the pullback map with the co-oriented map $r_W$, we obtain another element of $PC^*_\Gamma(M)$ given by $\Cre(W) \to W \xr{r_W} M$.
	In either case, we denote the composition $\Cre(W) \to W \xr{r_W}M$ by $r_{\Cre(W)}$.

	We will regularly abuse notation by allowing $\Cre(W)$ to refer to the space $D\times_{(-1,1)} W$, the pullback $D\times_{(-1,1)} W \to W$, or the element of $PC_*^\Gamma(M)$ or $PC^*_\Gamma(M)$ formed by the preceding constructions.
	It should usually be clear from context which is meant at any point.
\end{definition}


\begin{comment}
	\begin{convention}\label{C: regular value setup}
		In what follows it will typically be useful, and sometimes necessary, in the cases of both chains and cochains to assume that $M$ is a manifold without boundary and $\varphi \colon W \to (-1,1)$ has the form $W \xr{r_W}M \xr{\phi} (-1,1)$ with $\phi \colon M \to (-1,1)$ having $0$ as a regular value so that $M^0$ and $M^\pm$ are fiber products and thus manifolds with corners as in Example \ref{E: manifold decomposition}.
		Also as observed there, in this case $\varphi$ has $0$ as a regular value if and only if $r_W$ is transverse to $M^0$, in which case we also have $W^0 = M^0 \times_M W$ and $W^\pm = M^\pm \times_M W$.
		Unless noted, we will always assume this situation when working with creasing, i.e.\ that we have a fixed $\phi \colon M \to (-1,1)$ with $0$ a regular value and that creasing is defined with respect to the composition $\varphi = \phi r_W$, assuming $r_W$ is transverse to $M^0$.
	\end{convention}

	As usual, we typically abuse notation and, depending on context, may write $\Cre(W)$ to refer to the space $D\times_{(-1,1)} W$, the space together with its map to $W$, or, as we shall see below, refer to

	If we are treating $\varphi \colon W \to (-1,1)$ as oriented, then we interpret $\Cre(W)$ as oriented.
	If we are treating $\varphi \colon W \to (-1,1)$ as co-oriented, then we interpret $\Cre(W)$ as co-oriented.
\end{comment}

To justify the notion of creasing as a type of homotopy, at least topologically, we have the following lemma and corollary.
Note that these results are purely in the topological category, not the smooth category.

\begin{lemma}
	Suppose given a projection $\pi \colon X \times Y \to X$ and a map $g \colon W \to X$ transverse to $\pi$.
	Then the fiber product $(X \times Y) \times_X W$ is homeomorphic to $Y \times W$.
\end{lemma}

\begin{proof}
	We have $(X \times Y) \times_X W = \{(x,y,w) \in X \times Y \times W \mid x = g(w)\}$.
	A homeomorphism $(X \times Y) \times_X W \to Y \times W$ is then given by $(x,y,w) \mapsto (y,w)$ with inverse given by $(y,w) \mapsto (g(w),y,w)$.
\end{proof}

\begin{corollary}
	$\Cre(W)$ is homeomorphic to $I \times W$. In particular, if $W$ is compact then so is $\Cre(W)$.
\end{corollary}

\begin{proof}
	We observe that $D$ is homeomorphic to $(-1,1) \times I$ via a homeomorphism that preserves $\pi$ by taking $(x,y)$ to $\left(x, \frac{y}{2-|x|}\right)$.
	As a homeomorphism over a space $X$ induces a homeomorphism of fiber products, the corollary now follows from the preceding lemma.
\end{proof}


See \cref{F: creasing} for a sketch of a creasing homotopy of the teardrop manifold.

\begin{figure}[h!]
	\documentclass[tikz]{standalone}

\begin{document}
	\begin{tikzpicture}[scale=4]
	\coordinate (a) at (0,1.2);
	\coordinate (b) at (0,0.2);
	\coordinate (c) at (1.2,1);
	\coordinate (d1) at (1.1,.5);
	\coordinate (d2) at (1.4,.5);
	\coordinate (e) at (1.2,0);

	\draw[out=120, in=180] (b) to (a);
	\draw[out=60, in=-10, dotted] (b) to (a);

	\draw (a) -- (c);
	\draw (b) -- (e);

	\draw[out=180, in=130] (c) to (d1);
	\draw[out=-10, in=120] (c) to (d2);
	\draw[out=-100, in=120] (d1) to (e);
	\draw[out=-100, in=40] (d2) to (e);

	\draw (d1) to (d2);

	\node at (0, .7) {$W$};
	\node at (1.2, .75) {$W^+$};
	\node[scale=.7] at (1.23, .55) {$f^{-1}(p)$};
	\node at (1.22, .33) {$W^-$};
	\end{tikzpicture}
\end{document}
	\caption{Creasing homotopy}
	\label{F: creasing}
\end{figure}

\begin{lemma}\label{E: bd crease}
	Let $M$ be a manifold without boundary.
	Suppose given $\phi \colon M \to (-1,1)$ with $0$ a regular value and $r_W \colon W \to M$ transverse to $M^0 = \phi^{-1}(0)$, representing an element of $PC^\Gamma_*(M)$ or $W \in PC_\Gamma^*(M)$.
	Then
\begin{equation*}
	\bd(\Cre_\phi(W)) = (D_x \sqcup -D_- \sqcup -D_+)\times_{(-1,1)} W \bigsqcup -D\times_{(-1,1)}\bd W = W \sqcup -W^- \sqcup -W^+ \sqcup -\Cre_\phi(\bd W),
\end{equation*}
	again interpreting these formulas in $PC^\Gamma_*(M)$ or $PC_\Gamma^*(M)$, respectively, by composing the pullback maps to $W$ with $r_W \colon W \to M$ and the pullback maps to $\bd W$ with $\bd r_W \colon \bd W \to M$.
\end{lemma}

\begin{proof}
The first equality comes from our boundary formulas for $D$ and our Leibniz rules for pullbacks from \cref{P: oriented fiber boundary,leibniz}.
For the second equality we use, for example, that $\pi$ is an orientation preserving diffeomorphism over $(-1,1)$ from $D_+ \cong [0,1)$ onto its image in $(-1,1)$, and so we have an orientation-preserving diffeomorphism over $W$ given by $D_+\times_{(-1,1)} W \cong [0,1)\times_{(-1,1)} W = W^+$, and similarly in the co-oriented case.
The argument is analogous for the $W^-$ and $W$ terms.
\end{proof}

\begin{comment}
	Next let us see how to use creasing in the context of chains and cochains.

	If $W \in PC_*^{\Gamma}(M)$ represented by $r_W \colon W \to M$, then $\Cre(W)$ is also compact and oriented, and we have a map $r_{\Cre(W)} = r_W\pi_W \colon W \to M$ with $\pi_W$ being the projection of $\Cre(W) = D\times_{(-1,1)}W$ to $W$.
	Thus $r_{\Cre(W)} \colon \Cre(W) \to M$ gives us a new element of $PC_*^\Gamma(M)$.
	If $W \in PC^*_\Gamma(M)$, then to get an element $\Cre(W) \in PC^*_\Gamma(M)$ we will need to assume that $\varphi \colon W \to (-1,1)$ is a composition of maps $W \xr{r_W}M \xr{\phi} (-1,1)$.
	We can then form the space $\Cre(W)$ as the pullback by $\phi r_W$ of $\pi \colon D \to (-1,1)$.
	As $\pi$ is proper and co-oriented, the pullback $\pi^* = \pi_W \colon \Cre(W) \to W$ will be proper and co-oriented by Lemma \ref{L: co-orientable pullback} and Definition \ref{D: pullback coorient}.
	We then let $r_{\Cre(W)} = r_W \pi^*$ to obtain an element of $\Cre(W) \in PC^*_\Gamma(M)$.
	By \cite[Propositions 7.4]{Joy12} and Theorem \ref{leibniz}, our boundary formulas for $\Cre(W)$ continue to hold as elements of $PC_*^{\Gamma}(M)$ or $PC^*_{\Gamma}(M)$.
\end{comment}

\begin{comment}
\begin{convention}\label{C: regular value setup}
	In order to crease an element of $PC_*^\Gamma(M)$ or $PC^*_\Gamma(M)$ represented by $r_W \colon W \to M$, we need a smooth map $\varphi \colon W \to (-1,1)$ so that the composite $\varphi \colon W \xr{r_W}M \xr{\phi} (-1,1)$ has $0$ as a regular value.
	In this case $M^0$ and $M^\pm$ are fiber products and thus manifolds with corners as in \cref{E: manifold decomposition}.
	Also as observed there, in this case $\varphi$ has $0$ as a regular value if and only if $r_W$ is transverse to $M^0$, in which case we also have $W^0 = M^0 \times_M W$ and $W^\pm = M^\pm \times_M W$.
	Unless noted, we will always assume this situation when working with creasing, i.e.\ that we have a fixed $\phi \colon M \to (-1,1)$ with $0$ a regular value and that creasing is defined with respect to the composition $\varphi = \phi r_W$ with $r_W$ transverse to $M^0$.
\end{convention}
\end{comment}

To next promote the construction $\Cre(-)$ to an operator on $C_*^{\Gamma}(M)$ or $C^*_\Gamma(M)$, we will need the following corollary of \cref{L: pullback with Q}.

\begin{comment}
	The lemma itself will also be important when we consider products below in Section \ref{S: products}.

	\begin{lemma}\label{L: pullback with Q}
		Let $M$ be a manifold without boundary.
		Suppose $T \in PC_*^\Gamma(M)$ (or $PC^*_\Gamma(M)$), and that $T \in Q(M)$.
		Then $T \times_M W \in Q(M)$ and $W \times_M T \in Q(M)$ for any $W$ in $PC_*^\Gamma(M)$ (or $PC^*_\Gamma(M)$) that is transverse to $T$.
	\end{lemma}
	\begin{proof}
		We consider only $T \times_M W$ as the arguments for $W \times_M T$ are the same.
		By assumption $T$ is the disjoint union of trivial and degenerate chains, so it suffices to consider independently the possibilities that $T$ is trivial or degenerate.

		If $\rho$ is a (co\nobreakdash-)orientation reversing diffeomorphism of $T$ over $M$, then $\rho \times_M \id_W$ is a (co\nobreakdash-)orientation reversing diffeomorphism of $T \times_M W$, by Joyce's construction in the oriented case and by Remark \ref{R: co-or restriction or switch} in the co-oriented case.

		Next assume that $T$ is degenerate, so in particular it has small rank.
		Recall that the tangent bundle of a fiber product is the fiber product of the tangent bundles \cite[Theorem 5.47]{Wed16}, and so the derivative is the fiber product of derivatives.
		Note that the fiber product of two linear maps, one with a non-trivial kernel must also have a non-trivial kernel: If $A,B$ are linear maps with a common codomain and $v \in \ker(A)$, then $(v,0)$ is in the kernel of the fiber product of $A$ and $B$.
		So if the differential of $r_W$ has non-trivial kernel everywhere so will the derivative of any fiber product with $r_W$.
		Thus $T \times_M W$ has small rank.

		Next we recall $\bd(T \times_M W) = (\bd T) \times_M W \bigsqcup \pm T \times_M (\bd W)$.
		We have just shown that $T \times_M (\bd W)$ must have small rank.
		As $T$ is degenerate, $\bd T$ is a disjoint of trivial and small rank manifolds over $M$, and so by the preceding arguments $(\bd T) \times_M W$ will be a union of trivial and small rank manifolds over $M$.
		Altogether, $T \times_M W$ is degenerate.
	\end{proof}
\end{comment}

\begin{corollary}\label{C: creasing Q}
	Let $M$ be a manifold without boundary.
	Suppose given $\phi \colon M \to (-1,1)$ with $0$ a regular value and $r_T \colon T \to M$ transverse to $M^0 = \phi^{-1}(0)$.
	If $T \in Q(M)$, then so are $T^+$, $T^-$, $T^0$, and $\Cre(T)$.
\end{corollary}

\begin{proof}
	As observed in \cref{E: manifold decomposition}, with our assumptions about $\phi$ and $r_T$, the spaces $T^\pm$ and $T^0$ are fiber products over $M$ of $T$ with $M^\pm$ and $M^0$.
	So in this case the claim follows from \cref{L: pullback with Q}.

	For $\Cre(T)$, the pullback projection $\pi \colon \Cre(T) \to T$ is of small rank, as $\dim(\Cre(T))>\dim(T)$, from which it follows that $r_{\Cre(T)} = r_T\pi$ is of small rank.
	We also have $\bd \Cre(T) = T \sqcup - T^+ \sqcup -T^- \sqcup -\Cre(\bd T)$ by \cref{E: bd crease}.
	By assumption and the preceding paragraph, $T, T^\pm \in Q(M)$, and by the preceding sentence $\Cre(\bd T)$ is of small rank.
	Thus all components of $\bd \Cre(T)$ are trivial or of small rank, and so $\Cre(T)$ is degenerate.
\end{proof}

\begin{proposition}
	Let $M$ be a manifold without boundary.
	Suppose given $\phi \colon M \to (-1,1)$ with $0$ a regular value and that all creasing is done with respect to $\phi$.
	If $V$ and $W$ are any two representatives of $\uW \in C_*^\Gamma(M)$ whose reference maps are transverse to $M^0$, then $\Cre_\phi(V)$ and $\Cre_\phi(W)$ represent the same element of $C_*^\Gamma(M)$.
	Thus if the equivalence class $\uW$ contains any representative that is transverse to $M^0$, there is a well-defined element $\underline{\Cre(W)} \in C_*^\Gamma(M)$.
	Similarly for $C^*_\Gamma(M)$.
\end{proposition}

\begin{proof}
	The proofs for chains and cochains are the same, so we provide that with chains.

	If $r_V \colon V \to M$ and $r_W \colon W \to M$ represent the same class in $C_*^\Gamma(M)$ then $V \sqcup -W \in Q_*(M)$, and if $V$ and $W$ are both transverse to $M^0$ then so is $V \sqcup -W$.
	So $\Cre(V\sqcup-W)$ is defined, and by \cref{C: creasing Q}, $\Cre(V \sqcup -W) \in Q_*(M)$.
	Using our assumption that all creasing maps are compositions of the fixed $\phi \colon M \to (-1,1)$ with the reference maps and the properties of fiber product orientations, we have $\Cre(V \sqcup -W) = \Cre(V) \sqcup \Cre(-W) = \Cre(V) \sqcup -\Cre(W)$.
	Thus $\Cre(V)$ and $\Cre(W)$ represent the same element of $C_*^\Gamma(M)$.
\end{proof}

Finally, we come to the punchline of creasing:

\begin{theorem}\label{T: cohomology creasing}
	Let $M$ be a manifold without boundary.
	Suppose $\uW \in H_*^\Gamma(M)$ and that $\uW$ has a representative $r_W \colon W \to M$ that is transverse to $M^0$, defined with respect to some $\phi \colon M \to (-1,1)$ with $0$ a regular value.
	Then $\uW = \underline{W^+} + \underline{W^-} \in H_*^\Gamma(M)$.
	Similarly for $H^*_\Gamma(M)$.
\end{theorem}

\begin{proof}
	Again the proofs for homology and cohomology are the same so we focus on homology.

	We have seen that, with our assumptions, $\uW$ yields a well-defined element $\underline{\Cre(W)}$ represented by $\Cre(W)$.
	Computing in $C_*^\Gamma(M)$ we have
	\begin{align*}
		\bd \underline{\Cre(W)}& = \underline{\bd \Cre(W) }\\
		& = \underline{W \sqcup -W^+ \sqcup -W^- \sqcup -\Cre(\bd W)}\\
		& = \uW -\underline{W^+}-\underline{W^-}.
	\end{align*}
	In the last line we have used that $\underline{\bd W} = 0$ so that $\bd W \in Q_*(M)$; hence $\Cre(\bd W) \in Q_*(M)$ by \cref{C: creasing Q} and so $\underline{\Cre(\bd W)} = 0 \in C_*^\Gamma(M)$.
	The theorem follows.
\end{proof}

\begin{comment}
	Let $f \colon W \to \R$ and $p$ be a regular value of $f$.
	Define $W^+$ to be $f^{-1} [p, \infty)$ and $W^-$ to be $f^{-1} (-\infty, p]$.
	By Theorem~1 of \cite{Lipy14} (and
	also transversality as developed in in Section 6 of \cite{Joy12}), $W^+$ and $W^-$ are manifolds with corners.
	The following is Lemma~9 of \cite{Lipy14}.

	\begin{proposition}\label{P: creasing}
		Let $f \colon W \to \R$ and $p$ be a regular value of $f$.
		There is a manifold-with corners structure on the topological manifold $W \times [0,1]$, called the creasing of $W$ at $p$,
		whose boundary is the disjoint union of $W$ with its orientation
		reversed, $W^+$, $W^-$ and the creasing of ${\bd W}$ at $p$.
	\end{proposition}

	We denote the creasing of $W$ at $p$ by $\Cre(W)$, suppressing $f$ and $p$ from notation.
	See Figure~\ref{F: creasing} for a sketch of a creasing of the teardrop manifold.
\end{comment} %
	% !TEX root = ../foundations.tex

\section{Basic properties of geometric (co)homology and equivalence with singular (co)homology}\label{S: basic properties}

In this section we establish the basic properties of geometric homology and cohomology, eventually showing that they are equivalent to (absolute) singular homology and cohomology on manifolds using a result of Kreck and Singhof.
In the case of homology, we will also construct an explicit isomorphism from smooth singular homology to geometric homology.
It is more challenging to find an explicit isomorphism in the case of cohomology, and, in fact, the bulk of \cref{S: transversality} will be dedicated to such a construction.

\subsection{Functoriality and homotopy properties}\label{S: functoriality}

Given a \textit{continuous} map of smooth manifolds $f \colon M \to N$, we define in this section the induced maps $f_* \colon H_*^\Gamma(M) \to H_*^\Gamma(N)$ and $f^* \colon H^*_\Gamma(N) \to H^*_\Gamma(M)$ and show that they are independent of $f$ up to homotopy.
In subsequent sections, we will sometimes write the homology map simply as $f$ when the context is clear.

We also treat ``wrong way'' maps, though they require extra hypotheses.
In particular, if $f$ is proper and co-oriented it induces a covariant map on cohomology $f_* \colon H^*_\Gamma(M) \to H^{*+n-m}_\Gamma(N)$ independent of $f$ up to proper homotopy,
while if $f$ is proper and $M$ is oriented, we have a contravariant homology map $f^* \colon H_*^\Gamma(N) \to H_{*+m-n}^\Gamma(M)$.

In the covariant cases, $M$ and $N$ may both be manifolds with corners if we assume all maps and homotopies to be smooth (primarily to avoid smoothing arguments for maps of manifolds with corners).
In the contravariant case, we need to use both transversality and smoothing, and so for this case we consider only $M$ and $N$ without boundary.

The construction of $f_*$ for homology is relatively straightforward and can essentially be found in \cite[Section 6]{Lipy14}, though we provide additional details.
The covariant case of cohomology is similar to that for homology.
For the contravariant cases, there is slightly more work, and our argument here for cohomology parallels Kreck's in \cite{Krec10}, which is slightly different than the sketch in Lipyanskiy \cite[Section 6]{Lipy14} in that we choose to perturb $f$ rather than the reference map for the cohomology class in order to obtain transversality for the purpose of performing pullbacks.

As part of our constructions, we will see that if $F \colon M \times I \to N$ is a homotopy (proper or co-oriented as need be) and $r_W \colon W \to M$ represents a geometric cycle or cocycle, then the composition $F \circ (r_W \times \id_I)$ provides a homology or cohomology from $F(-,0)r_W \colon W \to N$ to $F(-,1)r_W \colon W \to N$.
Such homotopies, which we dub ``universal homotopies,'' will be critical in later sections for making geometric chains and cochains transverse to each other.

\subsubsection{Covariant functoriality of geometric homology and cohomology}\label{S: covariant functoriality}

In this section we consider the covariant behavior of geometric chains and cochains under maps and homotopies.
Both $M$ and $N$ may have corners in this section unless noted otherwise.
We begin with smooth maps but will generalize later to continuous maps via smooth approximation, at which time we will assume $M$ and $N$ are without boundary.

If $r_W \colon W \to M$ is in $PC_*^\Gamma(M)$ and $f \colon M \to N$ is a smooth map, then the composition $fr_W \colon W \to N$ is in $PC^\Gamma_*(N)$.
Consistent with our notation of writing $W \in PC_*^\Gamma(M)$, we write the image in $PC_*^\Gamma(N)$ as $f(W)$.
Similarly, if $f$ is proper and co-oriented we obtain a map $PC^*_\Gamma(M) \to PC^{*+n-m}_\Gamma(N)$ that we also write as $W \to f(W)$.
In this case the change of degree is because $W$ represents an element of degree $m-w$ in $PC^*_\Gamma(M)$ and of degree $n-w$ in $PC^{*}_\Gamma(N)$.
Functoriality is clear, and both $\bd(f(W))$ and $f(\bd W)$ are represented by $fr_Wi_{\bd W}$.
To obtain chain maps\footnote{We will not require here the signs that sometimes accompany chain maps of non-zero degree.} $C_*^\Gamma(M) \to C_*^\Gamma(N)$ and $C^*_\Gamma(M) \to C^{*+n-m}_\Gamma(N)$
(that we also write as $f$), it suffices to show that if $W \in Q(M)$ then $f(W) \in Q(N)$.
This is the content of the following lemma.

\begin{lemma}\label{L: Q preservation}
	If $r_W \colon W \to M$ represents an element of $Q_*(M)$ (or $Q^*(M)$) and $f \colon M \to N$ is any (co-oriented proper) smooth map, then $fr_W \colon W \to N$ is in $Q_*(N)$ (or $Q^{*+n-m}(N)$, co-orienting $fr_W$ with the composition co-orientation.).
\end{lemma}

\begin{proof}
	First consider $W \in Q_*(M)$.
	By assumption, $W$ is the disjoint union of a trivial manifold over $M$ given by $r_T \colon T \to M$ and a degenerate manifold over $M$ given by $r_D:D \to M$.
	If $\rho \colon T \to T$ is an orientation-reversing diffeomorphism of $W$ such that $r_T\rho = r_T$ then also $fr_T\rho = fr_T$, so $fr_T \colon T \to N$ is also trivial.
	Furthermore, if $r_D$ has small rank then certainly so does $fr_D$.
	Similarly, $\bd(fr_D)$ is a union of trivial and small rank manifolds over $N$, and so $fr_D:D \to M$ is degenerate.

	The proof for $W \in Q^*(M)$ is the same using that the composition of proper maps is proper and the composition of co-oriented maps is co-oriented.
\end{proof}

\begin{corollary}[Lipyanskiy's Theorem 4]\label{C: homology chain map}
	Given a smooth map $f \colon M \to N$ of manifolds (possibly with corners), there is an induced chain map $C_*^\Gamma(M) \to C_*^\Gamma(N)$ given by $\uW \to \underline{f(W)}$, and this construction gives a covariant functor from the category of smooth manifolds and smooth maps to chain complexes over $\Z$.
\end{corollary}

\begin{corollary}
	Given a smooth proper co-oriented map $f \colon M \to N$ of manifolds (possibly with corners), there is an induced chain map $C^*_\Gamma(M) \to C^{*+n-m}_\Gamma(N)$ given by $\uW \to \underline{f(W)}$, and this construction gives a covariant functor from the category of smooth manifolds and smooth maps to cochain complexes over $\Z$ with chain maps of degree $n-m$.
\end{corollary}

Next we consider behavior with respect to homotopies.
For the remainder of the section, and for simplicity of notation, we leave degree shifts tacit in the notation; for example, for a chain complex $A_*$, we write $x \in A_*$ to mean an element of any degree and simply write $f \colon A_* \to B_*$ even when $f$ is a chain map of non-zero degree.

\begin{convention}\label{homotopy product co-orientation convention}
	If $r_W \colon W \to M$ is co-oriented, then we co-orient $r_W \times \id_I \colon W \times I \to M \times I$ so that if $(\beta_W,\beta_M)$ is the co-orientation at $x \in W$ then $(\beta_W \wedge \beta_I,\beta_M \wedge \beta_I)$ is the co-orientation at $(x,t)$ for any $t \in I$, with $\beta_I$ being the standard orientation of the interval.
\end{convention}

We will also need the following observations for the co-oriented cases.

\begin{lemma}
	Let $r_W \colon W \to M$ be co-oriented, and let $r_W \times \id \colon W \times I \to M \times I$ be co-oriented by \cref{homotopy product co-orientation convention}.
	For $j = 0,1$, let $i_j \colon W = W \times j \into W \times I$ and $k_j \colon M = M \times j \into M \times I$ be the inclusions, co-oriented with the usual boundary co-orientation (\cref{D: boundary co-orientation}).
	Then for $j = 0,1$, the following diagram of co-oriented maps commutes:
	\[
	\begin{tikzcd}[column sep=large]
		W \arrow[r, "r_W"] \arrow[hookrightarrow,d, "i_j"'] & M \arrow[hookrightarrow,d, "k_j"] \\
		W \times I \arrow[r, "r_W \times \id"] & M \times I.
	\end{tikzcd}
	\]
\end{lemma}

\begin{proof}
	Commutativity as maps is clear, so we focus on the co-orientations.

	Let $w \in W$, and suppose the co-orientation of $r_W$ at $w$ is represented by $(\beta_W, \beta_M)$.
	Let $\nu_W$ and $\nu_M$ denote inward pointing normal vector fields to $W \times j$ and $M \times j$ in $W \times I$ and $M \times I$, respectively. If $j=0$, then the co-orientation of $i_0$ at $w$ is $(\beta_W, \beta_W \wedge \beta_{\nu_W}) = (\beta_W, \beta_W \wedge \beta_I)$, where $\beta_I$ is the standard orientation of $I$, while if $j=1$, the co-orientation of $i_1$ is $(\beta_W, \beta_W \wedge \beta_{\nu_W}) = (\beta_W, \beta_W \wedge (-\beta_I))$. The co-orientations of $k_0$ and $k_1$ are analogous.

	So for $j=0$, the co-orientation of the composition down then right is $$(\beta_W, \beta_W \wedge \beta_I) * (\beta_W \wedge \beta_I, \beta_M \wedge \beta_I) = (\beta_W, \beta_M \wedge \beta_I).$$
	The composition right then down is $$(\beta_W, \beta_M) * (\beta_M, \beta_M \wedge \beta_I) = (\beta_W , \beta_M \wedge \beta_I).$$
	So we have commutativity.
	When $j=1$, the computation is the same except that $\beta_I$ is replaced with $-\beta_I$ in both formulas.
\end{proof}

\begin{corollary}\label{C: universal homotopy boundary co-orientation}
	Let $F \colon M \times I \to N$ be a co-oriented homotopy from $f$ to $g$ (see \cref{D: co-oriented homotopy}), and let $r_W \colon W \to M$ be a co-oriented map.
	Then $F \circ (r_W \times \id_I)$ is a co-oriented homotopy from $fr_W$ to $gr_W$.
\end{corollary}

\begin{proof}
	Again, the thing to check is the co-orientations. We continue to utilize the notation of the preceding lemma.

	By \cref{D: co-oriented homotopy}, the composition of the co-oriented boundary inclusion $k_0 \colon M = M \times 0 \into M \times I$ with $F$ is the co-oriented map $-f$, while the composition of the co-oriented boundary inclusion $k_1 \colon M = M \times 1 \into M \times I$ with $F$ is the co-oriented map
	$g$. On the other hand, the ``top and bottom'' components of the boundary of $F \circ (r_W \times \id_I)$ are the co-oriented compositions $F (r_W\times \id) i_0$ and $F (r_W\times \id) i_1$.
	But by the preceding lemma, for $j=0,1$, we have $F (r_W\times \id) i_j = F k_j r_W$.
	So when $j=0$, we have
	$$ F (r_W\times \id) i_0 = F k_0 r_W = -f r_W,$$
	and when $j=1$, we have
	$$ F (r_W\times \id) i_1 = F k_1 r_W = g r_W.$$

	So $F (r_W\times \id)$ is a co-oriented homotopy from $fr_W$ to $gr_W$ as desired.
\end{proof}

The next lemma shows that a homotopy $M \times I \to N$ of the above form cannot ``promote'' a manifold over $M$ that is in $Q(M)$ to one that is not in $Q(N)$.

\begin{lemma}\label{L: dessicated homotopy}
	Let $W \in PC_*^\Gamma(M)$ (or $W \in PC^*_\Gamma(M)$).
	Let $F \colon M \times I \to N$ be a smooth homotopy; if $W \in PC^*_\Gamma(M)$ suppose further that $F$ is proper and co-oriented.
	Then $F \circ (r_W \times \id_I) \colon W \times I \to N$ is in $PC_*^\Gamma(N)$ (or $PC^*_\Gamma(N))$.
	Furthermore,
	if $r_W \colon W \to M$ is trivial, of small rank, or degenerate then so are
	$F \circ (r_W \times id_I) \colon W \times I \to N$ and $F(-,j) \circ r_W \colon W \to N$ for $j=0,1$ (for either co-orientation of $F(-,j)$, which is co-orientable by \cref{L: co-orientable homotopies}).
\end{lemma}

\begin{proof}
	The last statement concerning $F(-,j) \circ r_W \colon W \to N$ holds by \cref{L: Q preservation} and its proof.

	For	$F \circ (r_W \times id_I)$, we first suppose $W \in PC_*^\Gamma(M)$.
	In this case it is clear that $F \circ (r_W \times \id_I) \in PC_*^\Gamma(N)$ using the product orientation on $W \times I$.

	If $\rho \colon W \to W$ is an orientation-reversing diffeomorphism such that $r_W \circ \rho = r_W$, then
	$\rho \times \id_I \colon W \times I \to W \times I$ is an orientation-reversing self-diffeomorphism of $W \times I$ such that $F \circ (r_W \times id_I) \circ (\rho \times \id_I) = F \circ (r_W \times id_I)$.
	So $W \times I \to N$ is trivial.

	\begin{comment}
		% $F \circ (f \times id_I) \circ (\rho \times \id_I) = F \circ (f\rho \times \id_I) = F \circ (f \times \id _I)$.
	\end{comment}

	Next, if the derivative $Dr_W$ of the reference map $r_W$ has non-trivial kernel at a point $x \in W$ then so does the derivative $D(r_W \times \id_I)$ at each $(x,t) \in W \times I$, and thus so
	will $D(F \circ (r_W \times id_I)) = DF \circ D(r_W \times id_I)$.
	So $W \times I \to N$ has small rank if $r_W$ does.

	By definition, if $W \to M$ is degenerate then it has small rank and $\bd W = T \sqcup S$ with $T$ trivial and $S$ small rank.
	We have shown that $W \times I \xr{F\circ(r_W\times\id_I)} N$ then has small rank, so its suffices to consider its boundary,
	which is the union (up to signs) of $\bd W \times I$ and $W \times \bd I$.
	But $\bd W \times I = (T \times I) \sqcup (S \times I)$, which by our previous
	arguments are trivial and of small rank respectively.
	And the components of $W \times \bd I$ can be written for $j = 0, 1$ as $W \times j = W \xr{r_W} M \xr{F(-,j)} N$, and thus are of small rank since $r_W$ is.
	This completes the argument for $PC_*^\Gamma(M)$.

	If $W \in PC^*_\Gamma(M)$, then $F$ is proper and co-oriented by assumption, and we noted above our co-orientation convention for $r_W \times \id_I$.
	To see the latter is proper, if $K$ is compact in $M \times I$, then $K \subset \pi_M(K) \times I$, which is also compact.
	Then $(r_W \times \id_I)^{-1}(K) \subset (r_W \times \id_I)^{-1}(\pi_M(K) \times I) = r_W^{-1}(\pi_M(K)) \times I$, which is compact as $r_W$ is proper.
	As compositions of proper co-oriented maps are proper and co-oriented, $F \circ (r_W \times \id_I) \colon W \times I \to N$ is well defined in $PC^*_\Gamma(N)$.
	The remaining arguments are analogous to the arguments above for $PC_*^\Gamma(M)$.
\end{proof}

\begin{corollary}
	Let $V, W \in PC_i^\Gamma(M)$ or $V, W \in PC^i_\Gamma(M)$.
	Let $F \colon M \times I \to N$ be a smooth homotopy; if $V,W \in PC^i_\Gamma(M)$ suppose further that $F$ is proper and co-oriented.
	If $V,W$ represent the same element of $C_i^\Gamma(M)$ and $j=0,1$, then $F(-,j) \circ r_V: V \to N$ and	$F(-,j) \circ r_W \colon W \to N$ represent the same element of $C_i^\Gamma(N)$ (for either chosen co-orientation of $F(-,j)$) and $F \circ (r_V \times id_I) \colon V \times I \to N$ and $F \circ (r_W \times id_I) \colon W \times I \to N$ represent the same element of $C_{i+1}^\Gamma(M)$.
	The analogous fact holds for $V, W \in PC^i_\Gamma(M)$.
\end{corollary}

\begin{proof}
	We know $V, W \in PC_i^\Gamma(M)$ represent the same element of $C_i^\Gamma(M)$ if and only if $V \sqcup -W \in Q_*(M)$, so the corollary follows from \cref{L: dessicated homotopy}.
	Similarly for cochains.
\end{proof}

\begin{corollary}\label{C: homotopy}
	Suppose $F \colon M \times I \to N$ is a smooth homotopy between maps $f,g \colon M \to N$.

	If $\uW \in C_*^\Gamma(M)$ is a cycle then $\underline{f(W)}$ and $\underline{g(W)}$ are homologous cycles in $N$.

	If $\uW \in C^*_\Gamma(M)$ is a cocycle and $F$ is proper and a co-oriented homotopy from $f$ to $g$ (see \cref{D: co-oriented homotopy}), then $\underline{f(W)}$ and $\underline{g(W)}$ are cohomologous cocycles in $N$.
\end{corollary}

\begin{proof}
	By the preceding corollary we may work with any representative $r_W \colon W \to M$ of $\uW$.
	First suppose $W \in PC_*^\Gamma(M)$.
	Identifying $W \times I$ with $W \times_{pt} I$, by \cref{P: oriented fiber boundary} we have $\bd (W \times I) = \bd W \times I \bigsqcup (-1)^{w} W \times \bd I$.
	As $W$ is a cycle, $\bd W \in Q_*(M)$, and hence so is $F \circ ((\bd r_W) \times \id_I) \colon \bd W \times I \to N$ by the preceding lemma.
	Thus in $C^\Gamma_*(N)$ the boundary of $F \circ (r_W \times \id_I) \colon W \times I \to N$ is represented up to signs by the restriction of
	$F \circ (r_W \times \id_I)$ to $W \times \bd I = W \times (\{1\} \sqcup \{-0\})$, applying our conventions for the boundary of $I$ with its standard orientation.
	As $F \circ (r_W \times \id_I)|_{W \times 1} = gr_W$ and $F \circ (r_W \times \id_I)|_{W \times 0} = fr_W$, we see that up to the overall sign $(-1)^{w}$, the boundary of $F \circ (r_W \times \id_I)$ is represented in $C_*^\Gamma(N)$ by the disjoint union of $gr_W \colon W \to N$ and the negative of $fr_W \colon W \to N$.
	Thus $fr_W \colon W \to N$ and $gr_W \colon W \to N$ represent homologous cycles.

	Next consider the case of $W$ a cocycle. By \cref{C: universal homotopy boundary co-orientation,D: co-oriented homotopy}, $$\bd (F \circ (r_W \times \id_I)) = gr_W \amalg -fr_W \amalg H,$$ where $H$ is the homotopy $F \circ (r_W \times \id_I) \circ i_{\bd W \times \id_I} = F \circ (r_{\bd W} \times \id_I) \colon \bd W \times I \to N$.
	But $\bd W \in Q^*(M)$, and hence so is $H \colon (\bd W) \times I \to N$ by \cref{L: dessicated homotopy}.
	Thus we have
	that $fr_W \colon W \to N$ and $gr_W \colon W \to N$ represent cohomologous cocycles.
\end{proof}

We now come to the culmination of this section.
The first part of the following theorem is Lipyanskiy's Theorem 5 in \cite{Lipy14}.

\begin{theorem}\label{T: homology homotopy functor}
	Given a \textnormal{continuous} map $f \colon M \to N$ of manifolds without boundary\footnote{Possibly we can still let $M$ and $N$ be manifolds with corners and obtain a true statement, but we simplify our assumption here to avoid treating the question of smooth approximations in that setting.}, it induces a map $f_* \colon H_*^\Gamma(M) \to H_*^\Gamma(N)$ that depends only on the homotopy class of $f$.
	If $f$ is proper and co-oriented, it also induces $f_* \colon H^*_\Gamma(M) \to H^*_\Gamma(N)$ of degree $n-m$, which depends only on the proper homotopy class and co-orientation of $f$.
\end{theorem}

\begin{proof}
	First consider the case of homology.
	Let $g$ be any smooth approximation to $f$.
	Then by \cref{C: homology chain map}, $g$ induces a chain map $C_*^\Gamma(M) \to C_*^\Gamma(N)$ and hence a map $H_*^\Gamma(M) \to H_*^\Gamma(N)$.
	We show that this map is independent of the choice of $g$.
	Let $h$ be any other smooth map homotopic to $f$ (and so also homotopic to $g$).
	The continuous homotopy from $g$ to $h$ can be smoothly approximated by a smooth homotopy $H \colon M \times I \to N$ from $g$ to $h$ \cite[Theorem III.2.5]{Kos93}.
	The cycles represented by $g(W)$ and $h(W)$ are homologous by \cref{C: homotopy}.


	The cohomological case is the same by taking proper smooth approximations, which we show can be found in the proof of \cref{T: basic trans}, and co-orienting the homotopies using \cref{L: co-orientable homotopies} so that the approximations $g$ and $h$ are co-oriented homotopic to $f$.
\end{proof}

There is another very useful application of \cref{C: homotopy} that will be needed frequently below.
To explain, we first observe that if we have a map $r_W \colon W \to M$ representing an element of $H_*^\Gamma(M)$, then, in general, this homology class is not preserved under homotopies of $r_W$.
For example, a map representing a cycle might be homotopic to maps that do not represent cycles by pulling apart cancelling boundary components during the homotopy or maps of small rank may be homotopic to maps that do not have small rank.
So, in general, geometric homology and cohomology do not behave well with respect to homotopies of the elements of $PC(M)$; this is the case also with ordinary singular chains modeled on simplices.
This will cause difficulties below, for example when we want to alter a map representing a cycle by a homotopy to make it transverse to some other cycle.
The solution will be to define and use what we call universal homotopies.

\begin{definition}\label{D: universal homotopy}
	Given two maps $f,g \colon W\to M$, we say that there is a \textbf{universal homotopy} from $f$ to $g$ if there is a smooth homotopy $H \colon M\times I\to M$ with $H(-,0)=\id_M$ and such that $g=H(-,1)\circ f$; in other words, if $f$ and $g$ are homotopic by a composition of the form $W \times I \xr{f\times \id} M \times I \xr{H} M$ with $H(-,0)$ the identity. In this case, we call the composition a \textbf{universal homotopy} from $f$ to $g$. We say that the universal homotopy is proper if $f$, $g$, and $H$ are proper maps.
\end{definition}

If one thinks of an ordinary homotopy from $f$ to $g$ as a way of obtaining $g$ by deforming $f$, then we think of a universal homotopy as deforming $f$ by first performing $f$ and then deforming $M$.


\begin{proposition}\label{P: universal homotopy}
	Suppose there is a universal homotopy from $f \colon W\to M$ to $g \colon W\to M$. If $f \colon W\to M$ represents an element of $H_*^\Gamma(M)$, then $g = H(-,1)f \colon W\to M$ represents the same element. Similarly, if $f \colon W\to M$ represents an element of $H^*_\Gamma(M)$ and the universal homotopy is proper, then $g \colon W\to M$ represents the same element.
\end{proposition}

\begin{proof}
	The proposition is immediate from \cref{C: homotopy}, taking $M=N$ there and letting $F$ be the map $H \colon M \times I \to M$ of \cref {D: universal homotopy} that realizes the universal homotopy. In the case of cohomology, we co-orient $H$ using the co-orientation induced by the tautological co-orientation of $H(-,0) = \id \colon M \to M$; see \cref{D: homotopy co-orientation}.
\end{proof}

\subsubsection{Contravariant functoriality of geometric cohomology}\label{S: cohomology pullback}

In this section, we assign to a continuous map $f \colon M \to N$ of manifolds without boundary a map $H^*_\Gamma(N) \to H^*_\Gamma(M)$.
If $f$ is proper and $M$ is oriented, we also have a map $f^* \colon H_*^\Gamma(N) \to H_{*+m-n}^\Gamma(M)$.
As in the preceding section, we first consider a smooth map $f$ and then generalize to the continuous case using smooth approximations.
For simplicity of notation in what follows, we will not always explicitly write the degree shift for the homology map.

First, suppose $W \in PC^*_\Gamma(N)$ is represented by a proper co-oriented map $r_W \colon W \to N$, and let $g \colon M \to N$ be a \textit{smooth} map such that $g$ is transverse to $r_W$.
We emphasize that there is no need for assumptions that $g$ be co-oriented or proper.
We define $g^*(W)$ to be the pullback $r_W \times_N g \colon W \times_N M \to M$, co-oriented by our standard convention from \cref{D: pullback coorient}.
The pullback is proper by \cref{L: co-orientable pullback}.
Also, using our standard convention for notating dimensions, the pullback has dimension $w+m-n$, and we have $m - (w+m-n) = n - w$. In other words, the index of $W$ as a precochain of $N$ is the same as the index of $W \times_N M$ as a precochain of $M$.
So, when defined, $g^*$ takes elements of $PC_\Gamma^i(N)$ to elements of $PC_\Gamma^i(M)$.

Similarly, if $W \in PC_*^\Gamma(N)$ and $g \colon M \to N$ is smooth and transverse to $r_W$ and if $M$ is oriented, the pullback $W \times_N M$ has the pullback orientation of \cref{S: orientation of fiber products} and maps to $M$ by projection.
Furthermore, if $g$ is proper, then $W \times_N M$ is compact, as we show in the following lemma.

\begin{lemma}\label{L: compact pullback}
	Suppose $g \colon M \to N$ is a proper map from a manifold with corners to a manifold without boundary.
	Suppose $W$ is a compact manifold with corners and that $r_W \colon W \to N$ is transverse to $g$.
	Then $W \times_N M$ is compact.
\end{lemma}

\begin{proof}
	As $W$ is compact, so is $r_W(W)$, and as $g$ is proper, $g^{-1}(r_W(W))$ is compact.
	Now we observe that we must have $W \times_N M \subset W \times g^{-1}(r_W(W)) \subset W \times M$.
	So $W \times_N M$ is compact.
\end{proof}

The following lemma is similar to \cref{L: pullback with Q}, as is its proof, though there the focus was on fiber products, not pullbacks.
It will be useful later to allow $M$ to be a manifold with corners here.

\begin{comment}
	\red{GBF: Might want to try to combine those into a single lemma somewhere at some point, but it looks like it might be less messy, if a bit redundant, not to.}
\end{comment}

\begin{lemma}\label{L: pullback map Q}
	Suppose $M$ is a manifold with corners and $N$ is a manifold without boundary.
	Suppose $r_S \colon S \to N$ is trivial (oriented or co-oriented) or has small rank, and let $g \colon M \to N$ be a smooth map transverse to $r_S$.
	Then the pullback $g^*(S) = S \times_N M \to M$ is also trivial or has small rank, respectively.
	Consequently, if $S \in Q(N)$ then $S \times_N M \to M$ is in $Q(M)$.
\end{lemma}

\begin{proof}
	If $\rho$ is a (co\nobreakdash-)orientation reversing diffeomorphism of $S$ over $N$, then the restriction of $\rho \times \id_M$ to $S \times_N M \subset S \times M$ is a (co\nobreakdash-)orientation reversing diffeomorphism of $S \times_N M$ over $M$ by \cref{L: product preserves iso}.

	Next, suppose $r_S \colon S \to N$ has small rank.
	Suppose at the point $x \in S$ we have $v \in \ker(D_x r_S)$.
	By \cref{L: tangent of pullbacks}, if $(x,y) \in S \times M$, the tangent space $T_{(x,y)}(S \times_N M)$ is the pullback $T_xS\times_{T_{(r_S(x),g(y))}N} T_y M$, and we are interested in the map from this space under the derivative of the projection $\pi \colon S \times M \to M$.
	As $v \in \ker (D_xr_S)$, we have $(v,0) \in T_{(x,y)}(S \times_N M)$, and this projects to $0$ in $T_yM$.
	So $(v,0) \in \ker(D_{(x,y)}\pi)$.
	This shows that the pullback has small rank.

	The final statement is then a consequence of the definitions as in the proof of \cref{L: Q preservation}.
\end{proof}

The requirement that $g \colon M \to N$ must be transverse to $W$ means that even with the preceding lemmas we cannot define chain maps $g^* \colon C^*_\Gamma(N) \to C^*_\Gamma(M)$ or $g^* \colon C_*^\Gamma(N) \to C_*^\Gamma(M)$ because for any fixed $g$ there may be geometric cochains none of whose representatives are transverse to $g$.
Nonetheless, given any continuous $f \colon M \to N$ of manifolds without boundary we can define a map in geometric cohomology $f^* \colon H^*_\Gamma(N) \to H^*_\Gamma(M)$, and if $M$ is oriented and $f$ is proper, we can further define a map in geometric homology $f^* \colon H_*^\Gamma(N) \to H_{*+m-n}^\Gamma(M)$.
The constructions are as follows.

\begin{definition}\label{D: cohomology pullback and homology transfer}
	Suppose $f \colon M \to N$ is a continuous (not necessarily smooth, co-oriented, or proper) map of manifolds without boundary and $\uW \in H^*_\Gamma(N)$ is represented by $r_W \colon W \to N$.
	Let $g \colon M \to N$ be any smooth map homotopic to $f$ that is transverse to $r_W$ (which we know exists by \cref{T: basic trans}).
	Then we define $f^*(\uW) \in H^*_\Gamma(M)$ as the cohomology class represented by the pullback of $W$ by $g$, i.e.\ $g^*(W) = W \times_N M \to M$.

	If also $M$ is oriented and $f$ is proper, and if $\uW \in H_*^\Gamma(N)$ is represented by $r_W \colon W \to N$, then let $g$ be a smooth proper map properly homotopic to $f$ and transverse to $r_W$ (also by \cref{T: basic trans}).
	Then we define $f^*(\uW) \in H_*^\Gamma(M)$ as the homology class represented by the pullback of $W$ by $g$, i.e.\ $g^*(W) = W \times_N M \to M$.
\end{definition}

\begin{proposition}\label{P: cohomology pullback}
	Suppose $f \colon M \to N$ is a continuous (not necessarily smooth, co-oriented, or proper) map of manifolds without boundary.
	The map $f^* \colon H^k_\Gamma(N) \to H^k_\Gamma(M)$ is well defined and depends only on the homotopy class of $f$.
	Similarly, if $M$ is oriented and $f$ is proper, then the map $f_* \colon H_k^\Gamma(N) \to H_{k+m-n}^\Gamma(M)$ is well defined and depends only on the proper homotopy class of $f$
\end{proposition}

\begin{proof}
	We treat only the cohomology case, as the homology arguments are analogous taking the appropriate maps and homotopies to be proper, which is consistent with our applications of \cref{T: basic trans,T: homotopy trans} below.

	First, let $r_W \colon W \to N$ represent a cocycle, and suppose $g_0 \colon M \to N$ is smooth, homotopic to $f$, and transverse to $r_W$.
	By \cref{D: cohomology pullback and homology transfer}, $f^*(\uW)$ is represented by the pullback $g^*(W) = W \times_N M \to M$.
	We first observe this represents a cocycle.
	Because  $M$ is without boundary, we have by \cref{leibniz} that $\bd(W\times_N M) = (\bd W) \times_N M$.
	Then as $\bd W \in Q^*(N)$ by assumption, we have $(\bd W) \times_N M \in Q^*(M)$ by \cref{L: pullback map Q}.
	Hence $g^*(W) = W \times_N M \to M$ represents a cocycle.

	Next suppose we have two maps $g_0, g_1 \colon M \to N$ that are smooth, homotopic to $f$, and transverse to $r_W$.
	Let $G'$ be a smooth homotopy from $g_0$ to $g_1$, which exists by \cite[Proposition 9.2.33]{MaDo92} (or, if $f$ is proper and $g_0$ and $g_1$ are smooth proper maps properly homotopic to $f$, there is a smooth proper such $G'$ by \cite[Proposition 9.2.35]{MaDo92}).
	Then by \cref{T: homotopy trans} there exists a smooth homotopy $G \colon M \times I \to N$ from $g_0$ to $g_1$ such that $G$ is transverse to $r_W$.
	In particular, we apply \cref{T: homotopy trans} with the map $f$ of that theorem being our smooth homotopy $G'$, whose boundary consists of $g_0$ and $g_1$; then the resulting map $h(-,1)$ of that theorem will be our desired $G$.

	Now, consider the composition $W \times_N (M \times I) \xr{r_W \times_N G} M \times I \to M$ of the pullback of $r_W$ by $G$ and the projection $M \times I \to M$.
	As $r_W$ is proper and co-oriented, so is the pullback, and the projection $M \times I \to M$ has its standard co-orientation $(\beta_M \wedge \beta_I,\beta_M)$ and is proper.
	So this composition represents a geometric cochain in $M$.
	Furthermore, the boundary of this composition is the disjoint union of $\bd W \times_N (M \times I) \to M \times I \to M$ and, up to sign, $$W \times_N \bd(M \times I) = \left( W \times_N (M \times \{1\})\right) \sqcup \left(- W \times_N (M \times \{0\})\right) \to M \times I \to M$$ as $M$ is a manifold without boundary.
	Since $W$ is assumed to be a cocycle, $\bd W \in Q^*(N)$.
	So $\bd W \times_N (M \times I) \to M \times I$ is in\footnote{Note that even if $M$ is without boundary, $M \times I$ will have boundary (unless $M$ is empty).
	This instance is the main reason we have allowed such constructions as $PC^*(-)$ and $Q^*(-)$ to take manifolds with non-empty boundary as inputs up to this point.
	In this particular case, since $N$ does not have boundary, all of our ``smooth'' maps to $N$ are automatically smooth in the strong sense of \cite{Joy12}, as are then $W \times_N (M \times I) \to M \times I$ and $M \times I = M \times_{pt} I \to M$ as these are both pullbacks in the smooth category.}
	$Q^*(M \times I)$ by \cref{L: pullback map Q}, and then by the proof of \cref{L: Q preservation} the projection to $M$ preserves triviality and small rank.
	Therefore, the composite $\bd W \times_N (M \times I) \to M \times I \to M$ is in $Q^*(M)$.
	The other terms correspond to the pullbacks of $W$ via $g_0$ and $g_1$, so $g_0^*(W)$ and $g_1^*(W)$ represent cohomologous cocycles.

	Finally, as we have noted that $f^*$ is not a fully-defined chain map, we must show directly that $f^*$ as a cohomology map does not depend on the choice of precocycle representing a cohomology class.
	So suppose $r_{W_0} \colon W_0 \to N$ and $r_{W_1} \colon W_1 \to N$ represent the same cohomology class.
	In this case there will be a $Z \in PC^*_\Gamma(N)$ such that $\bd \underline{Z} = \underline{W_0} - \underline{W_1}.$
	As $\bd \underline{Z} = \underline{\bd Z}$ and $\underline{W_0} - \underline{W_1} = \underline{W_0 \sqcup - W_1}$ by \cref{D: chains and cochains,L: co/chains well defined}, the above equality means by \cref{L: co/chains well defined} that $\bd Z \sqcup -(W_0 \sqcup -W_1) \in Q^*(N)$.
	By \cref{T: basic trans} and its proof, we may choose a smooth $g$ homotopic to $f$ that is transverse to all of $Z$ (and so also $\bd Z$), $W_0$, and $W_1$.
	Then by \cref{L: pullback map Q}, the pullback by $g$ given by
	$$\left[\bd Z \sqcup -(W_0 \sqcup -W_1)\right] \times_N M = ((\bd Z) \times_N M) \sqcup (-W_0 \times_N M) \sqcup (W_1 \times_N M)$$
	is in $Q^*(M)$.
	As $M$ has no boundary, $(\bd Z) \times_N M = \bd (Z \times_N M)$.
	It follows that $$\bd \underline{(Z \times_N M)} = \underline{(W_0 \times_N M)} \sqcup \underline{(-W_1 \times_N M)}.$$
	So $W_0 \times_N M$ and $W_1 \times_N M$ represent cohomologous cocycles, and $f^*$ does not depend on the choice of representing precocycle.

	Thus $f^*$ does not depend on our choices and depends on $f$ only through its homotopy class.
\end{proof}

\begin{comment}
\begin{remark}\label{R: homology transfer}
	Analogously, given $f \colon M \to N$, one could also define a contravariant pullback functor of homology groups $f^*: H_*^\Gamma(N) \to H_*^\Gamma(M)$ if $f$ is proper and $M$ and $N$ are both oriented (so that the pullback can be oriented).
	The following lemma shows that the pullback of a compact map by a proper map is compact.
	For the orientation, we observe that given an oriented map $r_V \colon V \to N$ that represents $\uV$ and a map $g \colon M \to N$ that is properly homotopic to $f$ and transverse to $r_V$, the orientations of $V$ and $N$ induce a co-orientation on $r_V$ and so a co-orientation on the pullback $V \times_N M \to M$ by \cref{D: pullback coorient}.
	This in turn induces an orientation on $V \times_N M$ given the orientation of $M$.
\end{remark}
\end{comment}

\subsection{Mayer--Vietoris sequences}

In this sections we consider Mayer--Vietoris sequences for homology and cohomology.
In particular, we present covariant homology and cohomology sequences and a contravariant cohomology sequence.

Throughout our treatment, we will assume that $U$ and $V$ are open subsets of a manifold without boundary $M$ and that $U \cap V \neq \emptyset$.
If $U \cap V = \emptyset$, the Mayer--Vietoris sequences follow from more elementary considerations.
We do allow the possibility that $U \subset V$ or $V \subset U$.

For the covariant homology sequence, the maps $H_*^\Gamma(U) \to H_*^\Gamma(M)$, etc., induced by inclusion will be those defined in \cref{S: covariant functoriality}.
For a covariant cohomology sequence, however, the maps of the preceding section will not work, as the inclusion of an open set into a manifold is not generally proper.
Rather, to obtain a covariant cohomology sequence we will need to use in at least some of the terms a variant of geometric cohomology supported on open subsets, denoted $H^*_\Gamma(M|_U)$.
We will not provide an in-depth study of this variant cohomology, but it will not take us too far afield to introduce it here and treat its Mayer--Vietoris sequence in parallel with the homology sequence.
We justify this added effort by observing that we will have $H^*_\Gamma(M|_M) = H^*_\Gamma(M)$, so this sequence may be useful in studying our usual geometric cohomology.
However, the reader more interested in our standard geometric homology and cohomology can bypass this next section and go right to the following section.


\subsubsection{Cohomology supported on open sets}

In this brief section we provide the definition and some immediate properties for cohomology supported on open sets.

\begin{definition}
	Let $M$ be a manifold without boundary and $U \subset M$ an open subset.
	Define $C_\Gamma^*(M|_U) \subset C_\Gamma^*(M)$ to be the subcomplex consisting of elements of $C_\Gamma^*(M)$ that can be represented by $r_W \colon W \to M$ in $PC^*_\Gamma(M)$ with the image of $W$ contained in $U$.
	Let $H_\Gamma^*(M|_U) = H^*(C_\Gamma^*(M|_U))$.
\end{definition}

It is easy to observe that $C_\Gamma^*(M|_U)$ is a chain complex as it is closed under addition and taking boundaries using that the sum $\underline{W_1}+\underline{W_2}$ of two elements of $C^*_\Gamma(M|_U)$ represented by elements of $PC^*_\Gamma(M)$ with image in $U$ can be represented by the disjoint union $W_1 \sqcup W_2$ with image in $U$, and similarly the boundary of a precochain with image in $U$ has image in $U$.

In fact, we can reformulate this definition as follows.

\begin{lemma}
	Let $PC_\Gamma^*(M|_U) \subset PC_\Gamma^*(M)$ be the subset consisting of $r_W \colon W \to M$ with image in $U$, and let $Q^*(M|_U)$ be the elements of $Q^*(M)$ with image in $U$.
	Then $C_\Gamma^*(M|_U)$ is isomorphic to the set of equivalence classes of $PC_\Gamma^*(M|_U)$ under the relation $V\sim_U W$ if $V \sqcup -W \in Q^*(M|_U)$.
\end{lemma}

\begin{proof}
	We first observe that $\sim_U$ is an equivalence relation on $PC_\Gamma^*(M|_U)$ by the same proof as \cref{L: cancel Q}, assuming all maps have image in $U$.
	Furthermore, if $V,W \in PC_\Gamma^*(M|_U)$ and $V\sim_UW$ then $V\sim W$ in the sense of \cref{L: cancel Q}.
	So letting $\hat C_\Gamma^*(M|_U)$ temporarily denote the
	equivalence classes under $\sim_U$, we have a well-defined map $f \colon \hat C_\Gamma^*(M|_U) \to C_\Gamma^*(M|_U)$.
	By definition, elements of $C_\Gamma^*(M|_U)$ can be represented by elements of $PC_\Gamma^*(M|_U)$, so $f$ is surjective.
	Now suppose $W_1,W_2 \in PC^*_\Gamma(M|_U)$ represent the same element of $C^*_\Gamma(M|_U)$.
	Then $W_1\sim W_2$ in the sense of \cref{L: cancel Q}, i.e.\ $W_1 \sqcup -W_2 \in Q^*(M)$, but $W_1$ and $W_2$ each have support in $U$, so $W_1 \sqcup -W_2 \in Q^*(M|_U)$ and $W_1\sim_U W_2$.
	So $f$ is injective.
\end{proof}

The cohomology groups $H_\Gamma^*(M|_U)$ are functorial in the sense that if $U \subset W \subset M$ are open sets then we have $C_\Gamma^*(M|_U) \subset C_\Gamma^*(M|_W) \subset C_\Gamma^*(M|_M) = C_\Gamma^*(M)$, and these induce maps $H_\Gamma^*(M|_U) \to H_\Gamma^*(M|_W) \to H_\Gamma^*(M)$.


As Mayer--Vietoris sequences are often the engines of induction arguments and as many inductions start with Euclidean balls, we provide the following computation, which is akin to the dimension axiom or the Poincar\'e lemma; see \cref{E: dimension}.

\begin{proposition}\label{P: P lemma}
	Let $M$ be an $m$-dimensional smooth manifold, and $U \subset M$ an open set that is diffeomorphic to an open ball and whose closure in $M$ is compact.
	Then $H^n_\Gamma(M|_U)$ is $\Z$ if $n = m$ and is zero otherwise.
\end{proposition}


\begin{proof}[Proof of \cref{P: P lemma}]
	If $r_W \colon W \to M$ is proper and has image in $U$, then $W$, as the preimage of the closure of $U$, must be compact.
	Also, $U$ is orientable, and so by choosing an orientation of $U$ every co-oriented $r_W \colon W \to U$ determines an orientation of $W$ as explained in \cref{S: co-orientations}.
	Thus $C^*_\Gamma(M|_U) \cong C_{m-*}^\Gamma(U)$, and the result follows from \cref{E: dimension} (the dimension axiom), homotopy functoriality of geometric homology, and that $U$ is homotopy equivalent to a point.
\end{proof}



\subsubsection{Covariant Mayer--Vietoris sequences}

We now turn to our covariant Mayer--Vietoris sequences.


A key tool in proving the existence of these sequences will be creasing, which we introduced in \cref{S: creasing}.
For geometric homology and cohomology, creasing replaces the role that subdivision plays in the classical singular theories.
In some sense creasing is simpler, as creasing only needs to be applied once while subdivision often needs to be iterated.
In order to perform creasing, the following definitions will be useful.

\begin{definition}
	Let $U,V$ be non-empty open subsets of a manifold $M$ such that $U \cap V \neq \emptyset$.
	We will call a surjective\footnote{The condition that $\phi$ be surjective is required if $U\setminus V$ or $V\setminus U$ is empty, i.e.\ if $U \subset V$ or $V \subset U$.} function $\phi \colon U \cup V \to [-1/2,1/2]$ a \textbf{separating function for $\mathbf U$ and $\mathbf V$} if $\phi(U\setminus V) = -1/2$ and $\phi(V\setminus U) = 1/2$.
	Such a function can always be found by an application of Urysohn's Lemma to find a continuous function with this property and then applying the Smooth Approximation Theorem \cite[Theorem III.2.5]{Kos93}.
	We use $\pm 1/2$ as our endpoints to be consistent with the maps $\phi \colon M \to (-1,1)$ used in the definition of creasing.
	Note that we only assume that $U \setminus V \subset \phi^{-1}(-1/2)$ and not that $U \setminus V  = \phi^{-1}(-1/2)$, and similarly for $V \setminus U$.

	Suppose now $\phi$ is a separating function for $U$ and $V$ and that we have a map $r_W \colon W \to U \cup V$.
	We will say that $\phi$ \textbf{separates $\mathbf W$ over $\mathbf U$ and $\mathbf V$} if $0$ is a regular value for the composite $\phi r_W$, meaning that $\phi r_W$ is transverse to the inclusion of $0$ into $[-1/2,1/2]$.
	In particular, $0$ is a regular value for the restriction of $\phi r_W$ to each stratum of $W$.
	Again, such $\phi$ exist, as for an arbitrary separating function the set of regular values is the complement of a set of measure zero in $[-1/2,1/2]$ by Sard's Theorem.
	This is the standard statement of Sard's Theorem when $W$ is a smooth manifold without boundary, but note that for any countable set of maps $W_i \to U \cup V$ from smooth manifolds without boundary we can find a value that is regular for all $\phi r_{W_i}$, as a countable union of sets of measure zero still has measure $0$.
	In particular, we can take the $W_i$ to be the strata of a manifold with corners or even of multiple manifolds with corners.
    Once we have found such a common regular value $p$, we can then replace $\phi$ by its composition with an orientation-preserving diffeomorphism of $[-1/2,1/2]$ that takes $p$ to $0$, for example the linear fractional transformation $x \mapsto \frac{x-p}{-4px+1}$.


	If $\phi$ is separating for $W$ over $U$ and $V$, we may perform creasing of $W$ along $(\phi r_W)^{-1}(0)$.
\end{definition}

We are now ready to demonstrate the existence of covariant Mayer--Vietoris sequences.
In the following statements, we let $i_U \colon U \cap V \to U$ and $i_V \colon U \cap V \to V$ be the inclusion maps, and we write $(i_U, -i_V)$ for the map $C_*^{\Gamma}(U \cap V) \to C_*^{\Gamma}(U) \oplus C_*^{\Gamma}(V)$ that takes $\uW$ to $(i_U(\uW), -i_V(\uW))$.
More generally, we use the notation $(a,b)$ for elements of a direct sum of groups.


\begin{theorem}\label{T: relative MV}
	Let $M$ be a manifold without boundary.
	For any pair of open sets $U$ and $V$ in $M$ there are Mayer--Vietoris exact sequences
	%	\red{Fix tikz arrows?}
	\begin{equation*}
		\begin{tikzcd}[
			column sep = small,
			arrow style = math font,
			cells = {nodes = {text height = 2ex,text depth = 0.75ex}}
			]
			\cdots \arrow[r] & H^{\Gamma}_{k}(U \cap V) \arrow[r, "(i_U{,} -i_V)"] & [25pt] H^{\Gamma}_k(U) \oplus H^{\Gamma}_{k}(V)
			\arrow[r] & H_{k}^{\Gamma}(U \cup V) \arrow[r] & H_{k-1}^{\Gamma}(U \cap V) \arrow[r] & \cdots
		\end{tikzcd}
	\end{equation*}
	and
	\begin{equation*}
		\begin{tikzcd}[
			column sep = small,
			arrow style = math font,
			cells = {nodes = {text height = 2ex,text depth = 0.75ex}}
			]
			\cdots \arrow[r] & H_{\Gamma}^{k}(M|_{U \cap V}) \arrow[r, "(i_U{,} -i_V)"] & [25pt] H_{\Gamma}^k(M|_U) \oplus H_{\Gamma}^{k}(M|_V)
			\arrow[r] & H^{k}_{\Gamma}(M|_{U \cup V}) \arrow[r] & H^{k+1}_{\Gamma}(M|_{U \cap V}) \arrow[r] & \cdots.
		\end{tikzcd}
	\end{equation*}
\end{theorem}

\begin{proof}
	The proof parallels standard proofs of the existence of Mayer--Vietoris sequences for singular homology.
	The proofs for homology and cohomology are analogous, so we give the cohomological case, which is slightly more exotic.

	Let $S^*$ denote the quotient of $C_{\Gamma}^*(M|_U) \oplus C_{\Gamma}^{*}(M|_V)$ by the image of $C_{\Gamma}^*(M|_{U \cap V})$ under the map $(i_U, -i_V)$, with $i_U$ and $i_V$ being the inclusions.
	Then we have a short exact sequence
	\begin{equation}\label{E: homology MV SES}
		\begin{tikzcd}[
			column sep = small,
			arrow style = math font,
			cells = {nodes = {text height = 2ex,text depth = 0.75ex}}
			]
			0 \arrow[r] &
			C_{\Gamma}^{*}(M|_{U \cap V}) \arrow[r, "(i_U{,}-i_V)"] &[25pt]
			C_{\Gamma}^*(M|_U) \oplus C_{\Gamma}^{*}(M|_V) \arrow[r] &
			S^* \arrow[r] &
			0.
		\end{tikzcd}
	\end{equation}
	This short exact sequence generates a long exact cohomology sequence, and the theorem will follow from showing there is a quasi-isomorphism $\psi \colon S^* \to C_{\Gamma}^*(M|_{U \cup V})$.
	Our quasi-isomorphism will be induced by the map $C_{\Gamma}^*(M|_U) \oplus C_{\Gamma}^{*}(M|_V) \to C_{\Gamma}^*(M|_{U \cup V})$ that takes $(\underline{W_1},  \underline{W_2})$ to $\underline{W_1}+\underline{W_2}$, letting context determine whether we think of $\underline{W_1}$ as an element of $C_{\Gamma}^*(M|_U)$ or $C_{\Gamma}^*(M|_{U \cup V})$ and similarly for $\underline{W_2}$.
	This induces a well defined map $\psi \colon S^* \to C_{\Gamma}^*(M|_{U \cup V})$ as it takes elements in the image of $(i_U,-i_V)$ to $0$.
	To establish the quasi-isomorphism, we use creasing.

	First suppose a cocycle $\uW \in C_{\Gamma}^*(M|_{U \cup V})$ represented by $r_W \colon W \to U \cup V$.
	Let $\phi \colon U \cup V \to [-1/2,1/2]$ separate $W$ over $U$ and $V$.
	Then $W^- \in C_{\Gamma}^*(M|_{U})$ and $W^+ \in C_{\Gamma}^*(M|_{V})$, and by \cref{T: cohomology creasing}, we have $\uW = \underline{W^-}+\underline{W^+} \in H^*_\Gamma(M)$.
	But by the creasing construction if $W$ has image in $U \cup V$ then so does $\Cre(W)$, so also $\uW = \underline{W^-}+\underline{W^+} \in H_*^\Gamma(M|_{U \cup V})$.
	We have $\bd W \in Q^*(U \cup V) \subset Q^*(M)$ by assumption that $W$ is a cocycle.
	By the computation in \cref{E: codim 1 pullbacks}, we have $\bd( W^-) = -(W^0) \sqcup (\bd W)^-$, and $\bd (W^+) = W^0 \sqcup (\bd W)^+$.
	AS $\bd W \in Q^*(M)$ with image in $U \cup V$, we have $(\bd W)^\pm \in Q^*(M)$ with respective images in $U$ and $V$ by \cref{C: creasing Q}, so $(\bd W)^- = 0 \in C^*_\Gamma(M|_U)$ and similarly $(\bd W)^+ = 0 \in C^*_\Gamma(M|_V)$.
	Thus $(\underline{W^-}, \underline{W^+})$ is an element of $C_{\Gamma}^*(M|_U) \oplus C_{\Gamma}^{*}(M|_V)$ whose boundary is $(-\underline{W^0},\underline{W^0})$, which is in the image of $(i_U,-i_V)$.
	Therefore, $(\underline{W^-}, \underline{W^+})$ represents an element of
	$H^*(S^*)$ that maps to $\uW \in H^*_\Gamma(M_{U \cup V})$.
	Thus $\psi$ is surjective.

	Next, suppose $\uW$ is a cocycle in $S^*$, represented by $(\underline{W_1},\underline{W_2}) \in C^*_\Gamma(M|_U) \oplus C^*_\Gamma(M|_V)$, that maps to zero in $H_{\Gamma}^*(M|_{U \cup V})$.
	Thus there is
	some $Z \in PC^*_\Gamma(M|_{U \cup V})$ such that $\bd Z \sqcup - W_1 \sqcup -W_2 = T \in Q^*(M|_{U \cup V})$.
	Using that $\underline T = 0$ in $C^*_\Gamma(M|_{U \cup V})$, it will be useful to write this as $\underline{\bd Z} = \bd \underline{Z} = \underline{W_1} + \underline {W_2} \in C^*_\Gamma(M|_{U \cup V})$.

	Now let us choose a function $\phi \colon U \cup V \to [-1/2,1/2]$ that is separating for $Z$, $W_1$, and $W_2$ over $U$ and $V$.
  	We claim that $(\underline{Z^-}+\underline{\Cre(W_1)},\underline{Z^+}+\underline{\Cre(W_2)}) \in C^*_\Gamma(M|_U) \oplus C^*_\Gamma(M|_V)$ represents an element of $S^*$ whose boundary is $\uW$.


	To compute we use that $\bd \underline{Z} = \underline{W_1} + \underline {W_2} \in C^*_\Gamma(M|_{U \cup V})$ implies by \cref{C: creasing Q} that $(\bd Z)^\pm \sqcup (- W_1)^\pm \sqcup (-W_2)^\pm \in Q^*(M|_{U \cup V})$.
	Furthermore, as all the ``minus'' terms are supported in $U$ and all the ``plus'' terms are supported in $V$, we have $\underline{(\bd Z)^- }= \underline{W_1^-} + \underline{W_2^-}$ and $\underline{(\bd Z)^+} = \underline{W_1^+} + \underline{W_2^+}$ in $C^*_\Gamma(M|_U)$ and $C^*_\Gamma(M|_V)$, respectively.

	Now we compute using again \cref{E: codim 1 pullbacks,E: bd crease}:
	\begin{align*}
		\bd (\underline{Z^-}+\underline{\Cre(W_1)},\underline{Z^+}+\underline{\Cre(W_2)})& = (\bd (\underline{Z^-})+\bd \underline{\Cre(W_1)},\bd (\underline{Z^+})+\bd\underline{\Cre(W_2)})\\
		& = (-\underline{Z^0} + \underline{(\bd Z)^-}+\underline{W_1} -\underline{W_1^-} - \underline{W_1^+} -\underline{\Cre(\bd W_1)},\\
		&\qquad\qquad \underline{Z^0} + \underline{(\bd Z)^-} +\underline{W_2} -\underline{W_2^-} -\underline{W_2^+} -\underline{\Cre(\bd W_2)})\\
		& = (-\underline{Z^0} + \underline{W_1^-} + \underline{W_2^-} +\underline{W_1} -\underline{W_1^-} -\underline{W_1^+} -\underline{\Cre(\bd W_1)},\\
		&\qquad\qquad \underline{Z^0}+ \underline{W_1^+}+ \underline{W_2^+} +\underline{W_2} -\underline{W_2^-} -\underline{W_2^+} -\underline{\Cre(\bd W_2)})\\
		& = (-\underline{Z^0} + \underline{W_2^-} + \underline{W_1} -\underline{W_1^+} -\underline{\Cre(\bd W_1)},\\
		&\qquad\qquad \underline{Z^0}+ \underline{W_1^+} + \underline{W_2} -\underline{W_2^-} -\underline{\Cre(\bd W_2))})\\
		& = (\underline{W_1},\underline{W_2})+ (\underline{-Z^0},\underline{Z^0}) + (\underline{W_2^-}-\underline{W_1^+}, \underline{W_1^+} -\underline{W_2^-}) \\
		&\qquad\qquad -(\underline{\Cre(\bd W_1)},\underline{\Cre(\bd W_2)}).
	\end{align*}
	It remains to show that this is equal to $\uW = (\underline{W_1},\underline{W_2})$ in $S^*$.
	The middle two terms are in the image of $(i_U, -i_V)$, so they represent $0$ in $S^*$.
	It remains to show that $(\underline{\Cre(\bd W_1)},\underline{\Cre(\bd W_2))})=0$ in $S^*$.

    This term is obtained by applying the creasing construction to the entries of $(\bd W_1,\bd W_2)$, which by assumption is $0$ in $S^*$.
	In general, suppose $A \in PC^*_{\Gamma}(M|_U)$ and $B \in PC^*_\Gamma(M|_V)$ such that $(\underline{A}, \underline{B}) \in C^*_\Gamma(M|_U) \oplus C^*_\Gamma(M|_V)$ represents zero in $S^*$.
	This means that $(A,B)$ is equivalent in $C^*_\Gamma(M|_U) \oplus C^*_\Gamma(M|_V)$ to an element of the form $(i_U,-i_V)(\underline{C})$ for some $C \in PC^*_{\Gamma}(M|_{U \cap V})$, i.e.\ that $A \sqcup -i_U(C) \in Q^*(M|_U)$ and $B \sqcup i_V(C) \in Q^*(V)$.
	Creasing preserves both support and membership in $Q^*$ by definition and by \cref{C: creasing Q}, so we have
	$$\Cre(A \sqcup -i_U(C)) = \Cre(A) \sqcup -\Cre(i_U(C)) = \Cre(A) \sqcup -i_U(\Cre(C))$$
	and this is an element of $Q^*(M|_U)$.
	The equivalent computations holds for the other term.
	So in $C^*_\Gamma(M|_U)\oplus C^*_\Gamma(M|_U)$, we have
    $$0= (\underline{\Cre(A)} -\underline{i_U(\Cre(C))}, \underline{\Cre(B)} + \underline{i_V(\Cre(C))}) = (\underline{\Cre(A)},\underline{\Cre(B)}) +(i_U(\underline{\Cre(C)}), -i_V(\underline{\Cre(C)})).$$
	Consequently $(\underline{\Cre(A)},\underline{\Cre(B)})$ represents $0$ in $S^*$.

	Altogether, we have show $\psi$ is injective.
\end{proof}

\begin{comment}








	Next, suppose $\uW$ is a cocycle in $S^*$, represented by $(W_1,W_2) \in PC^*_\Gamma(M|_U) \oplus PC^*_\Gamma(M|_V)$, that maps to zero in $H_{\Gamma}^*(M|_{U \cup V})$.
	Thus there is
	some $Z \in PC^*_\Gamma(M|_{U \cup V})$ such that $\bd Z \sqcup - W_1 \sqcup -W_2 = T \in Q^*(M|_{U \cup V})$.
	In $PC^*_\Gamma(M|_{U \cup V})$, we can write
	\[\bd \underline{Z} = \underline{bd Z} = \underline{W_1} + \underline{W_2} + \underline T.\]
	Of course $\underline{T} = 0 \in PC^*_\Gamma(M|_{U \cup V})$, but it will be useful in the following computation
	Now let us choose a separating function and consider $(Z^-+\Cre(W_1),Z^++\Cre(W_2)) \in PC^*_\Gamma(M|_U) \oplus PC^*_\Gamma(M|_V)$; note that as $W_1$ and $W_2$ are in the boundary of $Z$, a separating function for $Z$ can also be used for creasing $W_1$ and $W_2$.
	We compute using again \cref{E: codim 1 pullbacks,E: bd crease}:
	\begin{align*}
		\bd (Z^-+\Cre(W_1),Z^++\Cre(W_2))& = (\bd (Z^-)+\bd \Cre(W_1),\bd (Z^+)+\bd\Cre(W_2))\\
		& = (-Z^0 + W_1^- + W_2^- + T^-+W_1 -W_1^- -W_1^+ -\Cre(\bd W_1),\\
		&\qquad\qquad Z^0+ W_1^++ W_2^+ + T^++W_2 -W_2^- -W_2^+ -\Cre(\bd W_2))\\
		& = (-Z^0 + W_2^- + T^-+W_1 -W_1^+ -\Cre(\bd W_1)),\\
		&\qquad\qquad Z^0+ W_1^+ + T^++W_2 -W_2^- -\Cre(\bd W_2)))\\
		& = (W_1,W_2)+ (-Z^0,Z^0) + (W_2^--W_1^+, W_1^+ -W_2^-) + (T^-,T^+)\\
		&\qquad\qquad -(\Cre(\bd W_1),\Cre(\bd W_2)).
	\end{align*}
	The second and third terms are in $\im(i_U,-i_V)$, while $(T^-,T^+) \in Q^*(M|_U) \oplus Q^*(M|_V)$ by \cref{C: creasing Q}.
	The last term is obtained by applying the creasing construction to the terms of $(\bd W_1,\bd W_2)$, which by assumption is $0$ in $S^*$.
	In other words, $(\bd W_1,\bd W_2)$ can be represented as the union of an element of $Q^*(M|_U) \oplus Q^*(M|_V)$ with an element in the image of $(i_U,-i_V)$.
	But creasing preserves both support and membership in $Q^*$ by definition and by \cref{C: creasing Q}, and if $(A,-B) = (i_U,-i_V)(C) = (i_U(C),-i_V(C))$, then $(\Cre(A),\Cre(-B)) = (i_U(\Cre(C)),-i_V(\Cre(C)))$.
	Altogether, we obtain $$(W_1,W_2) = \bd (Z^-+\Cre(W_1),Z^++\Cre(W_2))$$ in $S^*$.
	So $\psi$ is injective.
\end{comment}


\begin{remark}\label{R: MV boundary}
	The connecting homomorphism in this long exact sequence, as defined through this proof, is given by taking a cocycle in $M|_{U \cup V}$,
	using the creasing construction to write it as a sum of cochains in $C^*_\Gamma(M|_{U})$ and $C^*_\Gamma(M|_{V})$, and taking the boundary of the cochain in $M_U$.
	In short, it takes a cocycle represented by $W \to U \cup V$ and sends it to $-W^0 \to U \cap V$ determined by a function that separates $W$ over $U$ and $V$.

	The description for the homology sequence is identical.
\end{remark}

\begin{comment}
	Since $U$ is diffeomorphic
	to an open ball, there is a homotopy $F$ between the identity map on $U$ and a constant map.
	Consider the composite
	of $r_W \times id \colon W \times I \to U \times I$ with $F$ obtain a manifold over $M$, which we call $P_W$.
	If $W$ is trivial then so is $P_W$ as the
	the product of the involution on $W$ with the identity map defines a suitable involution on $W \times I$.
	If $W$ is degenerate then so is $P_W$,
	as it is small rank and its boundary is the union of $W \times \bd I$, which is small rank, and $\bd W \times I$, which is trivial or small rank
	as $\bd W$ is.

	The construction $W \to P_W$ provides a well-defined chain homotopy between
	the identity and the trivial map.
	Thus we have that $H^*_\Gamma(M|_U) \cong H^*_\Gamma(M|_{pt}) \cong H^{*-m}_\Gamma(pt)$.
	Finally, we observe that $H^*_\Gamma(pt) = H_*^\Gamma(pt)$ by definition (observing that a co-orientation of a map $W \to pt$ is equivalent to an orientation of $W$ \red{[GBF: do we need to explain this further? choose a convention?]}), while $H^\Gamma_*(pt)$ agrees with the singular homology of the point by \cite[Theorem 3]{Lipy14}.
\end{comment}

\begin{comment}
	\blue{DS: Theorem~\ref{T: relative MV} and Proposition~\ref{P: P lemma} at least gives us an isomorphism between geometric and singular
		cohomology for compact manifolds, which is what we need for Flows paper.
		I think we get this in substantially more generality, in particular
		interiors of compact manifolds with boundary, but we need to work that out.}
\end{comment}

\subsubsection{Contravariant Mayer--Vietoris sequence}

\begin{comment}
	\red{[GBF: Note - the work in this section uses the following formulas which need to be confirmed at some point and stated clearly: $\bd W^- = W^0+(\bd W)^+$, $\bd W^+ = - W^0+(\bd W)^-$, $\bd W^0 = -(\bd W)^0$.
		The should all follow from the Liebniz rule for pullbacks using $W^- = \varphi^{-1}([0,p]) \times_M W$, $W^+ = \varphi^{-1}([p,1]) \times_M W$, and $W^0 = \varphi^{-1}(p) \times_M W$.
		PLEASE DO NOT ERASE THIS NOTE UNTIL THESE FORMULAS HAVE BEEN ESTABLISHED AND WRITTEN INTO THE PAPER SOMEWHERE]}
\end{comment}

In this section we show that geometric cohomology possesses a contravariant Mayer--Vietoris sequence on manifolds.
This does not seem possible by deriving a long exact sequence from a short exact sequence.
For example, in general the restriction map $C^*_\Gamma(M) \to C^*_\Gamma(U)$ will not be surjective.
Consequently, this sequence will take more work than the covariant ones.

Instead, we proceed with a direct analysis at each term of the exact sequence as in Kreck's proof of the Mayer--Vietoris sequence for his cohomology theory using stratifolds in \cite{Krec10}.
In fact, our argument for exactness parallel's Kreck's fairly closely.
However, our proof that the connecting map is well defined is more complicated than the analogous proof in Kreck (ignoring the extra complications in \cite{Krec10} arising from considerations of collars that we do not need).
This is because Kreck is able to define his version of separating functions directly on his representing objects as $\varphi \colon W \to [0,1]$, which then makes it possible to compare two different $\varphi$s using a cylinder $W \times I$ with different versions of $\varphi$ on each end.
Here, however, we must always use splitting maps that factor through $M$ or risk that even if $W$ is trivial $W^{\pm}$ may not be.
This complicates the proof of \cref{P: connecting} below.

\begin{definition}
	Suppose $U,V \subset M$ are non-empty open subsets and $\uW \in H^k_\Gamma(U \cap V)$ represented by $r_W \colon W \to U \cap V$.
	Of course we can also consider $r_W$ to have image in $U \cup V$.
	Let $\phi \colon M \to [-1/2,1/2]$ separate $W$ over $U$ and $V$.
	Define $\delta(\uW) \in H^{k+1}_\Gamma(U \cup V)$ to be represented by $-W^0 = -\phi^{-1}(0)\times_{U \cup V}W$.
	We recall here that $\phi^{-1}(0)$ can be considered a precochain in $U \cap V$ or $U \cup V$ by the constructions in \cref{S: codim 0 and 1 co-or}.
\end{definition}

The choice to use $-W^0$ rather than $W^0$ in the definition is explained by \cref{R: MV boundary}.
An alternative convention that uses $W^0$ rather than $-W^0$ would be to use $(-i_U,i_V)$ rather than $(i_U,-i_V)$ in our covariant Mayer--Vietoris sequences and the map $\uW \to (-\uW|_U, \uW|_V)$ rather than $\uW \to (\uW|_U, -\uW|_V)$ in our contravariant Mayer--Vietoris sequence below.

Our next goal is to show that $\delta$ is well defined, i.e.\ that it does not depend on the choice of representing precochain or the choice of $\phi$.
We begin with the following lemma:

\begin{lemma}\label{L: different point}
	Suppose $U,V \subset M$ are open subsets of $M$ with $U \cap V$ nonempty.
	Suppose $W \in PC^*_\Gamma(U \cap V)$ represents an element of $H^*_\Gamma(U \cap V)$ and that $\phi \colon U \cup V \to [-1/2,1/2]$ separates $U$ and $V$.
	Let $p, q\in (-1/2,1/2)$ be any points such that $\phi r_W$ is transverse to the inclusions of $p$ and $q$ into $(-1/2,1/2)$, and let $W^p$ and $W^{q}$ be the fiber products $\phi^{-1}(p) \times_M W$ and $\phi^{-1}(q) \times_M W$ considered as elements of  $PC^*_\Gamma(U \cap V)$.
	Then $W^p$ and $W^q$ represent the same element of $H^*_\Gamma(U \cup V)$.
\end{lemma}

\begin{proof}
	We first observe that $\phi^{-1}(p) \to U \cup V$ is a closed inclusion and so proper.
	Thus $r_{W^p} \colon W^p \to U \cup V$ is proper by \cref{L: co-orientable pullback} as the composition of $r_W^* \colon \phi^{-1}(p)\times_{U \cup V}W \to \phi^{-1}(p)$ and $\phi^{-1}(p) \into U \cup V$.
	So $W^p$ represents an element of $PC^*_\Gamma(U \cap V)$, as does $W^q$ for the same reason, noting that the constructions of $W^p$ and $W^q$ are completely analogous.

	Next, we assume without loss of generality that $p<q$.
	Then $\phi^{-1}([p,q])$ will be transverse to $r_W \colon W \to M$, and we have $\phi^{-1}([p,q]) \subset U \cap V$.
	As $\phi^{-1}([p,q])$ is closed in $U \cup V$, its inclusion is proper, and the inclusion can be tautologically co-oriented as a codimension $0$ embedding (see \cref{D: tautological co-orientation}).
	Thus $$\bd(\varphi^{-1}([p,q])\times_{U \cup V} W = (\varphi^{-1}(q)\times_{U \cup V} W)-(\varphi^{-1}(p)\times_{U \cup V} W)+ \varphi^{-1}([p,q])\times_{U \cup V} \bd W.$$
	As $W$ is a cocycle, the last term is in $Q^*(U \cup V)$ (with support in $U \cap V$) by \cref{L: pullback with Q}, so $\varphi^{-1}(p)\times_{U \cup V} W$ and $\varphi^{-1}(q)\times_{U \cup V} W$ are cohomologous.
\end{proof}

\begin{lemma}\label{L: different W}
	Suppose $U,V \subset M$ are open subsets of $M$ with $U \cap V$ nonempty.
	Suppose $W_1, W_2 \in PC^*_\Gamma(U \cap V)$ represent the same element of $H^*_\Gamma(U \cap V)$.
	Let $\phi \colon U \cup V \to [-1/2,1/2]$ separate $U$ and $V$ and suppose $q \in (-1/2,1/2)$ is a regular value for both $\phi r_{W_1}$ and $\phi r_{W_2}$.
	Then $W_1^q$ and $W_2^q$ represent the same element of $H^*_\Gamma(U \cup V)$.
\end{lemma}

\begin{proof}
	By assumption, there is a $Z \in PC^*_{\Gamma}(U \cap V)$ with $\bd Z \sqcup -W_1 \sqcup W_2 \in Q^*(U \cap V)$.
	For our fixed $\phi$, there is by Sard's Theorem a $p \in (-1/2,1/2)$ such that $p \into (-1/2,1/2)$ is transverse to $\phi r_Z$, $\phi r_{W_1}$, and $\phi r_{W_2}$.
	By the preceding lemma, it suffices to show that $W_1^p$ and $W_2^p$ represent the same element of $H^*_\Gamma(U \cup V)$.
	As fiber products with elements of $Q^*(U \cap V)$ are in $Q^*(U \cap V)$ by \cref{L: pullback with Q},
	we have that
	$$\phi^{-1}(p)\times_{U \cap V}(\bd Z \sqcup -W_1 \sqcup W_2) = (\bd Z)^p \sqcup -W_1^p \sqcup W_2^p \in Q^*(U \cap V).$$
	Furthermore, by an obvious modification to \cref{C: co-orient W0}, we have $\bd (Z^p) = -(\bd Z)^p$.
	So  $$-\bd (Z^p) \sqcup -W_1^p \sqcup W_2^p in Q^*(U \cap V),$$
	which implies that $W_1^p$ and $W_2^p$ represent the same cohomology class in $U \cap V$.

	But by the same argument presented at the start of the proof of \cref{L: different point}, the spaces $Z^p$, $W_1^p$, and $W_2^p$ all map properly in $U \cup V$, and so $W_1^p$ and $W_2^p$ also represent the same cohomology class in $U \cup V$.
	We note that they do indeed represent cocycles as $\bd (W_1^p) = - (\bd W_1)^p = -\phi^{-1}(p) \times_{U \cup V} (\bd W_1)$, and since $\bd W_1 \in Q^*(U \cap V)$ by assumption, and so is a union of trivial and degenerate precochains, the same is true of $\phi^{-1}(p) \times_{U \cup V} (\bd W_1)$ by \cref{L: pullback with Q}.
\end{proof}

\begin{proposition}\label{P: connecting}
	The map $H^k_\Gamma(U \cap V) \xr{\delta} H^{k+1}_\Gamma(U \cup V)$ is well defined.
\end{proposition}

\begin{proof}
	Given the preceding lemmas, it suffices to show that if $W \in PC^*_\Gamma(U \cap V)$ represents a cocyle and we have two separating functions  $\phi_1,\phi_2:U \cup V \to [-1/2,1/2]$ and a $p \in (-1/2,1/2)$ that is a regular value for both $\phi_1 r_W$ and $\phi_2 r_W$, then $\phi_1^{-1}(p)\times_{U \cup W} W$ and $\phi_1^{-1}(p)\times_{U \cup W} W$ represent the same element of $H^*_\Gamma(U \cup W)$.
	Indeed, suppose $W_1$ and $W_2$ represent the same element of $H^*_\Gamma(U \cup V)$ and that we choose functions $\phi_1,\phi_2$ that separate that are respectively separating for $W_1$ and $W_2$ over $U$ and $V$, as in the construction of $\delta$.
	Again by Sard's Theorem, we can find a point $p_3 \in (-1/2,1/2)$ such that $p_3$ is a regular value for all three of $\phi_1 r_{W_1}$, $\phi_2 r_{W_1}$, and $\phi_2 r_{W_2}$.
	Then by \cref{L: different point}, $\phi_1^{-1}(0) \times_{U \cup V} W_1$ and $\phi_2^{-1}(0) \times_{U \cup V} W_2$ represent the same elements of $H^*_\Gamma(U \cup V)$ respectively as $\phi_1^{-1}(p_3) \times_{U \cup V} W_1$ and $\phi_2^{-1}(p_3) \times_{U \cup V} W_2$.
	But by \cref{L: different W}, $\phi_2^{-1}(p_3) \times_{U \cup V} W_2$ represents the same cohomology class as $\phi_2^{-1}(p_3) \times_{U \cup V} W_1$.
	So it suffices to show that $\phi_1^{-1}(p_3) \times_{U \cup V} W_1$ represents the same cohomology class as $\phi_2^{-1}(p_3) \times_{U \cup V} W_1$.

 	So let us suppose a single precochain $W$ and two separating functions $\phi_1,\phi_2$. The preceding argument shows it suffices to show for some common regular value $p$ that $\phi_1^{-1}(p) \times_{U \cup V} W$  and $\phi_2^{-1}(p) \times_{U \cup V} W$ represent the same element of $H^*_\Gamma(U \cap V)$.
	Actually it will be simpler to show instead that for regular values $p_1 < p_2$ are $\phi_1,\phi_2$, respectively, that  $\phi_1^{-1}(p_1) \times_{U \cup V} W$  and $\phi_2^{-1}(p_2) \times_{U \cup V} W$ represent the same element of $H^*_\Gamma(U \cap V)$.
	This suffices again by \cref{L: different point}.
	Again there is no trouble finding such $p_1$ and $p_2$ by Sard's Theorem.

	Now, suppose there exists $q$ with $-1<q<p_2$ such that $\phi_1^{-1}(p_1) \subset \phi_2^{-1}([-1,q])$.
	By postcomposing $\phi_1$ with a diffeomorphism of $[-1,1]$, we may suppose that $p_1<q$.
	Let us choose $u_1,u_2$ such that $-1<u_1<p_1<u_1<q$.
	Using the Urysohn lemma we can find a continuous $\hat \Phi$ on $U \cup V$ such that:

	\begin{enumerate}
		\item $\hat \Phi$ is equal to $\phi_1$ on $\phi_1^{-1}[u_1,u_2]$,
		\item $\hat\Phi$ is equal to $\phi_2$ on $\phi_2^{-1}([q,1])$,

		\item $\hat\Phi$ takes $U\setminus V$ to $-1$,

		\item $\hat\Phi^{-1}(p_1) = \phi_1^{-1}(p_1)$,

		\item $\hat\Phi^{-1}(p_2) = \phi_2^{-1}(p_2)$.
	\end{enumerate}
	Furthermore, by a sufficiently small homotopy $\hat\Phi$ can be approximated by a smooth function $\Phi$ that preserves the last two properties and is still separating.
	Thus by using $\Phi$ and applying the previous case for a single separating function but two different separating points, we see that splitting at $\phi_1^{-1}(p_1)$ and $\phi_2^{-1}(p_2)$ produce the same image of $\delta$.
	While this argument is written for $\phi_1^{-1}(p_1)$ ``below some $q<p_2$,'' clearly an analogous argument holds for $\phi_1^{-1}(p_1)$ ``above some $q>p_2$,'' or with the roles of $\phi_1$ and $\phi_2$ reversed.

	Lastly, we have to consider the case where there is not a $q$ as in the preceding paragraph that allows us to separate $\phi_1^{-1}(p_1)$ from $\phi_2^{-1}(p_2)$.
	In this case we will modify $\phi_2$ in an appropriate way.
	Let use choose $r$ with $p_2 <r<1$.
	Define $\hat \Phi$ as follows:

	\begin{enumerate}
		\item On $\phi_2^{-1}([-1,r])$, take $\hat \Phi = \phi_2$,

		\item On $\phi_2^{-1}([r,1])$ use the Urysohn Lemma to construct a continuous function $\phi_2^{-1}([r,1]) \to [r,1]$ that takes $\phi_2^{-1}(r) \cup [\phi_2^{-1}([r,1]) \cap \phi_1^{-1}(p_1)]$ to $r$ and $V\setminus U$ to $1$; let $\hat \Phi$ be this constructed function on $\phi_2^{-1}([r,1]$.
	\end{enumerate}

	Now, we can modify $\hat \Phi$ by an $\epsilon$-homotopy fixing $\phi_2^{-1}([-1,r])$ and $\Phi^{-1}(1)$ to a smooth $\Phi$ with $\Phi^{-1}(p_2) = \phi_2^{-1}(p_2)$ and such that there is a $q$, $r<q<1$, with $\phi_1^{-1}(p_1) \subset \Phi^{-1}([-1,q])$.
	We can now choose a $p_2'$ with $q<p_2'<1$ that separates $W$ via $\Phi$.
	From our previous cases, we know $\Phi^{-1}(p_2')\times_{U \cup V}W$ is cohomologous to $\Phi^{-1}(p_2)\times_{U \cup V}W = \phi_2^{-1}(p_2)\times_{U \cup V}W$, but $\Phi^{-1}(p_2')$ and $\phi_1^{-1}(p_1)$ are now related as in the preceding case.

	Altogether we have shown that $\delta$ is independent of choices.
\end{proof}

\begin{lemma}\label{L: natural connection}
	The connecting map $\delta$ is natural, i.e.\ for a continuous map of triples $f \colon (M;U',V') \to (N;U,V)$ the following diagram commutes for all $k$:
	\[
	\begin{tikzcd}
		H^k_\Gamma(U \cap V) \arrow[r, "\delta"] \arrow[d, "f^*"] & H^{k+1}_\Gamma(U \cup V) \arrow[d, "f^*"] \\
		H^k_\Gamma(U' \cap V') \arrow[r, "\delta"] & H^{k+1}_\Gamma(U' \cup V')
	\end{tikzcd}
	\]
\end{lemma}

\begin{proof}
	Consider $\uW \in H^k_\Gamma(U \cap V)$ represented by $r_W \colon W \to U \cap V$.
	Let $\phi \colon U \cup V \to [-1,1]$ be a separating function for $U,V$.
	We may assume that $0 \in [-1,1]$ is a separating point for $W$, by postcomposing $\phi$ with a diffeomorphism of $[-1,1]$ if necessary.

	Next we note that $\phi f$ separates $U'$ and $V'$ but it may not be smooth.
	This can be remedied by a small homotopy of $f$ to $g$.
	This homotopy may not preserve the property that $f(U') \subset U$ and $f(V') \subset V$ but we may choose the homotopy small enough that $(\phi g)^{-1}([-1/3,1/3]) \subset (\phi f)^{-1}(-1/2,1/2)$, which will be sufficient to ensure that $(\phi g)^{-1}([-1/3,1/3]) \subset U' \cap V'$.
	We may further assume that the homotopy has been chosen so that $g$ is transverse to $r_{W^0} \colon W^0 \to N$.

	We claim that the transversality of $g$ with $r_{W^0}$ implies that $0$ is a regular value for the composition $\phi gr_W^*$, where $r_W^*$ is the pullback of $W$ to $M$ via $g$, i.e.\ $r_W^*: W \times_N M \to M$.
	To see this, recall that the tangent bundle of the pullback is the pullback of the tangent bundles \cite[Theorem 5.47]{Wed16}, so at a point $(x,y)$ of $(\phi gr^*_{W})^{-1}(0)$ with $x \in W$, $y \in M$, the tangent space of $W \times_N M$ is $T_x W \times_{T_{( r_W(x),g(y))}} T_yM$.
	As $0$ is a regular value for $\phi r_W$, there must be a vector $\xi \in T_xW$ with $D_x(\phi r_W)(\xi)\neq 0$.
	Consider $D_xr_W(\xi)$, which must also be nonzero.
	As $g$ is assumed transverse to $W^0$, there must be $\alpha \in T_xW^0$ and $\beta \in T_yM$ such that $(D_xr_W)(\xi) = (D_xr_W)(\alpha) + (D_yg) (\beta)$.
	Rewriting, $D_yg (\beta) = D_xr_W(\xi-\alpha)$.
	As $\alpha \in T_xW^0$, we have $$D_{r_W(x)}\phi(D_yg (\beta)) = D_{r_W(x)}\phi \circ D_xr_W(\xi-\alpha) = D_x(\phi r_W)(\xi)-D_x(\phi r_W)(\alpha) = D_x(\phi r_W)(\xi)\neq 0.$$
	So, recalling that $Dr_W^*$ is simply the projection to $TM$, the pair $( \xi-\alpha, \beta)$ is a non-zero vector in $T_xW\times_{T_{( r_W(x), g(y))}}T_y M$ that maps by $D(\phi gr_W^*)$ to a non-zero vector in $T_{0}[-1,1]$.
	As $(x,y)$ was an arbitrary point of $(\phi gr^*_{W})^{-1}(0) \subset W \times_N M$, this shows that $0$ is a regular value for $\phi gr^*_W$.

	Now we have from the definitions that $f^*\delta(W)$ is represented by $(-W^0) \times_N M = -(N^0 \times_N W) \times_N M = -(\phi^{-1}(0) \times_N W) \times_N M$, where $(\phi^{-1}(0) \times_N W)$ is a fiber product over $M$ and the whole expression is the pullback of this fiber product to $N$ via $g \colon M \to N$.
	On the other hand, $\delta f^*(W) = -(W \times_N M)^0 = -M^0 \times_M (W \times_N M) = -(\phi g)^{-1}(0) \times_M (W \times_N M)$.
	Here $W \times_N M$ is the pullback to $M$ and then we take the fiber product with $(\phi g)^{-1}(0)$.
	In both cases these correspond to pairs of points $(x,y) \in W \times M$ with $r_W(x) = g(y)$ and $\phi(r_W(x)) = \phi(g(y)) = 0$.
	In other words, as spaces these are both precisely the limit of the following diagram together with its map to $M$:
	\[
	\begin{tikzcd}
		& M \arrow[d,"g"] & \\
		W \arrow[r, "r_W"] & N \arrow[r, "\phi"] & {[-1,1]} & \arrow[l,hook'] 0
	\end{tikzcd}
	\]
	Thus $f^*\delta(\uW)$ and $\delta f^*(\uW)$ are represented by the same map, and we only need to check co-orientations.

	We return to the definitions of the pullback and fiber product co-orientations.
	It suffices to compare $(W \times_N M)^0$ and $W^0 \times_N M$ at an arbitrary point.
	We first consider the co-orientation in the form $(W \times_N M)^0$.
	For the pullback $W \times_N M \to M$, choose $e \colon W \into N \times \R^K$.
	At our chosen point, fix $\beta_N$ and choose $\beta_W$ so that $\omega_{r_W} = (\beta_W,\beta_N)$.
	Let $\nu$ be the Quillen-oriented normal bundle of $W$ in $N \times \R^K$ so that $\beta_W \wedge \beta_\nu = \beta_N \wedge \beta_E$, where $\beta_E$ is the standard orientation of $\R^K$.
	Then by definition the pullback map from $P = W \times_N M \subset M \times \R^K$ to $M$ is co-oriented by $(\beta_P,\beta_M)$ if we choose $\beta_P$ and $\beta_M$ such that $\beta_P \wedge \beta_\nu = \beta_M \times \beta_E$, recalling that we let $\beta_\nu$ also denote the pulled back normal bundle of $P$ in $M \times \R^K$.
	We suppose we have chosen such $\beta_P$ and $\beta_M$.
	Next, let $M^0 \subset M$ have normal co-orientation $\beta_\phi$ determined by pulling back the standard orientation from $(-1,1)$, and similarly let $\beta_\phi$ denote the pullback normal co-orientation to $(W \times_N M)^0$.
	Then by \cref{P: codim 1 co-orient}, the co-orientation of the pullback $(W \times_N M)^0 = M^0 \times_M (W \times_N M) \to W \times_N M$ is $(\beta_Q,\beta_Q \wedge \beta_\phi)$ for any $\beta_Q$.
	The fiber product $(W \times_N M)^0 \into M$ is then co-oriented by the composition $(\beta_Q,\beta_Q \wedge \beta_\phi)*(\beta_P,\beta_M)$.
	So if we choose $\beta_Q$ so that $\beta_Q \wedge \beta_\phi = \beta_P$ (or equivalently $\beta_Q$ and $\beta_M$ so that $\beta_Q \wedge \beta_\phi \wedge \beta_\nu = \beta_M \wedge \beta_E$), the co-orientation is $(\beta_Q,\beta_M)$.
	\greg{G. note to self: Double check that \cref{P: codim 1 co-orient} still works here given recent changes to its statement.}

	On the other hand, consider $(N^0 \times_N W) \times_N M = W^0 \times_N M \to M$.
	One again we fix $\beta_N$ and $\beta_W$ so that $\omega_{r_W} = (\beta_W,\beta_N)$.
	Again by \cref{P: codim 1 co-orient}, the co-orientation of the pullback $N^0 \times_N W \to W$ is $(\beta_{W^0},\beta_{W^0} \wedge \beta_\phi)$ for any $\beta_{W^0}$, continuing to let $\beta_\phi$ denote any normal co-orientation pulled back via $\phi$.
	If we choose $\beta_{W^0}$ so that $\beta_{W^0} \wedge \beta_\phi = \beta_W$ then we have $r_{W^0} \colon W^0 \to N$ co-oriented by $(\beta_{W^0},\beta_N)$.
	As $W^0 \subset W$, we can embed $W^0$ in $N \times \R^K$ via the composition $W^0 \into W \xhookrightarrow{e}N \times \R^K$, using the same $e$ and $K$ as above.
	As $\beta_W \wedge \beta_\nu = \beta_N \wedge \beta_E$ and $\beta_W = \beta_{W^0} \wedge \beta_\phi$, we have $\beta_{W^0} \wedge \beta_\phi \wedge \beta_\nu = \beta_N \wedge \beta_E$ so that $\beta_\phi \wedge \beta_\nu$ is the Quillen orientation for the normal bundle of $W^0$ in $N \times \R^K$.
	Using this to pull back $W^0 \to N$ to $W^0 \times_N M \to M$, by definition the pullback co-orientation is $(\beta_Q,\beta_M)$ when $\beta_Q$ and $\beta_M$ are chosen so that $\beta_Q \wedge \beta_\phi \wedge \beta_\nu = \beta_M \wedge \beta_E$.
	But this is exactly the same co-orientation we arrived at in the preceding paragraph.
	Thus the co-orientations of the two constructions agree.

	We conclude that the diagram of the lemma commutes.
\end{proof}

\begin{notation}
	If $f \colon U \into M$ is the inclusion of an open subset and $\uW \in C^*_\Gamma(M)$, we may also write $f^*(\uW)$ as $\uW|_U$.
	As such an inclusion $U$ is necessarily transverse to any other map to $M$, we also obtain a well-defined map $PC^*_\Gamma(M) \to PC^*_\Gamma(U)$ that we also write $W \to W|_U$.
	By \cref{P: codim 0 pullback}, $W|_U$ is simply the restriction of $r_W \colon W \to M$ to $r_W^{-1}(U)$.
\end{notation}

\begin{theorem}\label{T: absolute MV}
	Let $U,V \subset M$ be open subsets.
	There is a long exact Mayer--Vietoris sequence
	$$\cdots \to H^k_\Gamma(U \cup V) \xr{i} H^k_\Gamma(U) \oplus H^k_\Gamma(V) \xr{j} H^k_\Gamma(U \cap V) \xr{\delta} H^{k+1}_\Gamma(U \cup V)\to\cdots$$
	with $i(\uW) = (\uW|_U, -\uW|_V)$, $j(\uW_1,\uW_2) = \uW_1|_{U \cap V}+\uW_2|_{U \cap V}$, and $\delta$ as above.
\end{theorem}

\begin{proof}
	Notationally, we always assume $\uW$ is represented by $W$, etc.
	In the following we typically assume an appropriate separating function $\phi \colon U \cup V \to [-1,1]$ such that $\phi(U\setminus V) = -1$ and $\phi(V\setminus U) = 1$ and choose an appropriate separating point without further comment.
	We observe that in this case $(U \cap V)^+ = \phi^{-1}([0,1])\cap(U \cap V)$ is a closed subspace of both $U \cap V$ and of $U$, as $\phi|_U^{-1}([0,1]) = \phi^{-1}([0,1]) \cap (U \cap V)$ given that $\phi(U\setminus V) = -1$.
	So the inclusions $(U \cap V)^+ \into U \cap V$ and $(U \cap V)^+ \into U$ are both proper maps.
	Thus if $W \in PC^*_\Gamma(U \cap V)$, in which case in particular $r_W \colon W \to U \cap V$ is proper, both the fiber products $(U \cap V)^+\times_{U \cap V} W \to U \cap V$ and $(U \cap V)^+\times_{U} W \to U$ will be proper maps.
	In fact, they both have the same domain, which is just $W^+$, and we will write $W^+$ for the corresponding elements of $PC^*_\Gamma(U \cap V)$ or $PC^*_\Gamma(U)$, determining which via context.
	Similarly, we obtain $W^-$ in $PC^*_\Gamma(U \cap V)$ or $PC^*_\Gamma(V)$.
	See \cref{F: MV1}.
	On the other hand, $(U \cap V)^- = \phi^{-1}([-1,0])\cap(U \cap V)$ is not generally closed in $U$ and so we do not obtain a $W^-$ in $PC^*_\Gamma(U)$.

	\begin{figure}[h]
		\input{auxy/mayer_vietoris1.tex}
		\caption{For a proper map $r_W \colon W \to U \cap V$, the restriction to $W^+$ is proper into $U$, while the restriction to $W^-$ is proper into $V$.}
		\label{F: MV1}
	\end{figure}

	Similarly, continuing with our assumptions about $\phi \colon U \cup V \to [-1,1]$, we have $\phi^{-1}([-1,0]) = \phi|_U^{-1}([-1,0])$ so that $\phi^{-1}([-1,0])$ is a closed subset of $U$ and of $U \cup V$.
	Therefore, the inclusions $\phi^{-1}([-1,0]) \into U$ and $\phi^{-1}([-1,0]) \into U \cup V$ are proper so that if $W \in PC^*_\Gamma(U)$, then $W^-$ is well-defined in both $PC^*_\Gamma(U)$ and $PC^*_\Gamma(U \cup V)$.
	See \cref{F: MV2}.
	Analogously, if $W \in PC^*_\Gamma(V)$ then we obtain $W^+$ in both $PC^*_\Gamma(V)$ and $PC^*_\Gamma(V^+)$.

	\begin{figure}[h]
		\documentclass[tikz]{standalone}
\usetikzlibrary{decorations.pathmorphing}
\tikzset{snake it/.style={decorate, decoration=snake}}

\begin{document}
	\begin{tikzpicture}
	\draw[gray, dashed] (0,0) circle (2);
	\node at (0,1){$U$};
	\draw[gray, dashed] (3,0) circle (2);
	\node at (3,1){$V$};
	\draw[gray, snake it] (-2,0) -- (2,0);
	\draw[black, fill=white] (-2,0) circle (1pt);
	\draw[black, fill=white] (2,0) circle (1pt);
	\draw[thick] (1.5, 1.32) -- (1.5, -1.32);
	\node[scale=.6] at (1.5, -1.8){$\phi^{-1}(0)$};
	\node[scale=.6] at (.7, -.3){$W^+$};
	\node[scale=1] at (-1, .3){$W$};
	\end{tikzpicture}
\end{document}
		\caption{For a proper map $r_W \colon W \to U$, the restriction to $W^-$ is proper into $U \cup V$.}
		\label{F: MV2}
	\end{figure}

	Finally, we also note that $\phi^{-1}(0)$ is closed in $U \cap V$, $U$, $V$, and $U \cup V$, so for any $W$ in $PC_\Gamma^*(-)$ for any of these spaces, $W^0$ is also in $PC^*_\Gamma(-)$ for all these spaces.

	These observations will be used freely in the remainder of the proof.

	\textbf{Exactness at $H^k_\Gamma(U \cup V)$.}
	Let $\uW \in H^{k-1}(U \cap V)$.
	Then $i\delta(\uW)$ is represented by $(-W^0, W^0)$.
	By the above discussion, we have $W^+ \in PC^*_\Gamma(U)$, and by applying the same discussion to $\bd (W^+)$ we also have in $U$ that $\bd (W^+) = W^0+(\bd W)^+$ via \cref{E: codim 1 pullbacks}.
	But $(\bd W)^+ \in Q^*(U)$ by \cref{C: creasing Q} as $W$ is a cocycle.
	Thus $W^0$ represents $0 \in H^k_\Gamma(U)$.
	Similarly, $W^0$ represents $0 \in H^k_\Gamma(V)$ using $W^-$.
	So $i\delta = 0$.

	Next suppose $\uW \in H^k_\Gamma(U \cup V)$ with $i(\uW) = 0$.
	Representing $\uW$ by $W$, this means that $W|_U$ and $W|_V$ each bound, in $U$ and $V$ respectively.
	This means there exist $A \in PC^*_\Gamma(U)$ and $S \in Q^*(U)$ with $\bd A = W|_U+S$ and $B \in PC^*_\Gamma(V)$ and $T \in Q^*(V)$ with $\bd B = W|_V+T$.
	We choose a common separating point for $A$ and $B$ and consider $A^-$ and $B^+$, which are both well defined in $U \cup V$ by the discussion above.
	We compute
	\begin{align*}
		\bd(A^-+B^+)& = -A^0 + (\bd A)^- +B^0+ (\bd B)^+\\
		& = -A^0 + (W|_U)^-+S^- +B^0+ (W|_V)^++T^+.
	\end{align*}
	But $(W|_U)^- +(W|_V)^+ = W^-+W^+$, which through the creasing construction is cohomologous to $W$.
	Further, $S^-+T^+ \in Q^*(U \cup V)$, so we see that $W$ is cohomologous to $A^0-B^0$.
	But $A^0-B^0 = \delta(B|_{U \cap V}-A|_{U \cap V})$.

	\textbf{Exactness at $H^k_\Gamma(U) \oplus H^k_\Gamma(V)$.}
	It is immediate that the composition $ji = 0$.

	Now suppose $(\uW_1,\uW_2) \in \ker j \subset H^k_\Gamma(U) \oplus H^k_\Gamma(V)$.
	Using representatives $W_1,W_2$, this means that there is a $Z \in C^k_\Gamma(U \cap V)$ with $\bd Z = W_1|_{U \cap V}+W_2|_{U \cap V}+T$, with $T \in Q^*(U \cap V)$.
	Choose a separating point for all $Z$ and hence automatically $W_1$, $W_2$, and $T$.
	We claim that $\gamma = W_1^- - Z^0 -W_2^+$ represents an element of $H^k_\Gamma(U \cup V)$ whose image under $i$ is $(\uW_1,\uW_2)$.
	We compute
	\begin{align*}\bd \gamma& = \bd (W_1^- - Z^0 -W_2^+)\\
		& = -W_1^0 +(\bd W_1)^- + (\bd Z)^0 - W_2^0 - (\bd W_2)^+\\
		& = -W_1^0 +(\bd W_1)^- + W_1^0+W_2^0+T^0 - W_2^0 - (\bd W_2)^+\\
		& = (\bd W_1)^- +T^0 - (\bd W_2)^+.
	\end{align*}
	As $W_1$ and $W_2$ are cycles, $(\bd W_1)^-$, $(\bd W_2)^+$, and $T^0$ are in $Q^*(U \cup V)$.
	So this boundary is $0$ in $C^*_\Gamma(U \cup V)$, and $\gamma$ represents an element of $H^k_\Gamma(U \cup V)$.

	Next we show that $\gamma|_U$ is cohomologous to $W_1$ in $U$.
	In fact, in $U$ we have
	$$\bd (Z^+) = Z^0+(\bd Z)^+ = Z^0+ W_1^++ W_2^+|_{U}+T^+.$$ So
	$\bd (Z^+) +\gamma|_U = W_1^-+W_1^+ +T^+$, and we see that $\gamma|_U$ is cohomologous in $U$ to $W_1^-+W_1^+$, which is cohomologous to $W_1$ in $U$ via creasing.
	Similarly, in $V$ we have
	$$\bd (Z^-) = -Z^0+(\bd Z)^- = -Z^0+W_1^-|_{V}+ W_2^-+T^-.$$
	So
	$\bd(Z^-) - \gamma|_V = W_2^-+W_2^+ +T^-$, and $-\gamma|_V$ is cohomologous to $W_2^-+W_2^+$, which is cohomologous to $W_2$ in $V$.

	So $(\uW_1,\uW_2) = i(\underline{\gamma})$.

	\textbf{Exactness at $H^k_\Gamma(U \cap V)$.}
	Consider a cocycle $W$ in $C^k_\Gamma(U)$.
	Then $\delta j(\uW)$ is represented by $-W^0$.
	By the discussion above, $W^- \in PC^*_\Gamma(U \cup V)$, and $$\bd (W^-) = -W^0+ (\bd W)^-.$$ As $W$ is a cocycle, $\bd W \in Q^*(U)$, and so $(\bd W)^- \in Q^*(U \cup V)$ via \cref{C: creasing Q}.
	So $\delta j(\uW) = 0 \in H^*_\Gamma(U \cup V)$, and similarly for elements of $H^k_\Gamma(V)$.

	Now suppose $\uW \in H^k_\Gamma(U \cap V)$ and $\delta(\uW) = 0$.
	Representing $\uW$ by $W$ and choosing a separating function and separating point, this means there is a $Z$ in $U \cup V$ such that $\bd Z = -W^0+T$ with $T \in Q^*(U \cup V)$.
	Let $A = Z|_U+ W^+ \in PC^*_\Gamma(U)$ and $B = -Z|_V +W^- \in PC^*_\Gamma(V)$.
	Then
	\begin{align*}
		\bd A& = \phantom{-}\bd Z|_U+ \bd( W^+) = -W^0+T|_U +W^0+(\bd W)^{+} = \phantom{-}T|_U+(\bd W)^{+}\\
		\bd B& = -\bd Z|_V+ \bd (W^-) = \phantom{-}W^0-T|_V -W^0+(\bd W)^{-} = -T|_V+(\bd W)^{-}.
	\end{align*}
	As $\bd W \in Q^*(U \cap V)$ and $T \in Q^*(U \cup V)$, their pullbacks and restrictions are also in the appropriate $Q^*$s, so $(A,B)$ represents an element of $H^k_\Gamma(U) \oplus H^k_\Gamma(V)$.

	We then have
	\begin{align*}
		j(A,B)& = A|_{U \cap V}+B|_{U \cap B}\\
		& = Z|_{U \cap V}+ W^+ - Z|_{U \cap V} +W^-\\
		& = W^++W^-,
	\end{align*}
	which represents $\uW \in H^k(U \cap V)$ via creasing.
\end{proof}

\begin{comment}

	\subsection{The suspension map and the cohomology of Euclidean space}\label{S: suspension}

	\red{I'm leaving it for now, but I'm not sure we need this section anymore (which is a shame as I enjoyed writing it).}

	\begin{proposition}
		$H_\Gamma^*(\R^n) \cong H^*(\R^n)$.
	\end{proposition}

	The proof of the proposition will proceed by an induction over the dimension $n$.
	For $n = 0$, the domains of all cochains must be compact and oriented, so $H^*_\Gamma(\R^0) = H_*^\Gamma(\R^0)$, where the latter represents Lipyanskiy's geometric homology theory, which is isomorphic to singular homology by \cite[Section 10]{Lipy14} (or in this case by direct computation).
	For $n>0$, our general strategy will be derived from the treatment of non-compactly supported piecewise-linear intersection homology in \cite[Section II.2]{BoHab}.
	In particular, we have the following two lemmas modifying \cite[Lemmas 2.2 and 2.3]{BoHab}.

	\begin{lemma}
		Let $M$ be a manifold, and suppose $\uW \in C^{i}_\Gamma(\R \times M)$ is a cocycle with support in $\R_+ \times M$.
		Then $\uW$ is the coboundary of a cochain in $C^{i-1}_\Gamma(\R \times M)$ supported in $\R_+ \times M$.
	\end{lemma}
	\begin{proof}
		Let $p_1 \colon \R \times M \to \R$ and $p_2 \colon \R \times M \to M$ be the projections, and let $r_W \colon W \to \R_+ \times M$ be the reference map.
		Note that $\R_+ \times W$ is a manifold with corners by \cite[Theorem 6.4]{Joy12}.
		So we can define a new map $r_{\R_+ \times W} \colon \R_+ \times W \to \R \times M$ by $r_{\R_+ \times W}(t,x) = (t+p_1r_W(x),p_2r_W(x))$.
		As $r_W$ is proper, so is $r_{\R_+ \times W}$, and its boundary (up to signs) is the sum of $W$ with $\pm r_{\R_+ \times \bd W} \colon \R_+ \times \bd W \to \R \times M$, letting this latter map be constructed analogously to $r_{\R_+ \times W}$.
		As $\bd W$ represents $0$ in $C^*_\Gamma(\R \times W)$, it must be a sum of trivial and degenerate objects.
		But it is easy to see that our construction takes trivial and degenerate objects to trivial and degenerate objects, respectively, so that $\R_+ \times \bd W$ is a sum of trivial and degenerate objects.
		Thus $W$ cobounds as required.
	\end{proof}

	In this section we give an explicit construction of the cohomology suspension isomorphism

	\begin{proposition}[Suspension isomorphism]
		Define the suspension map\footnote{Note that the suspension map $S$ has degree $0$ as the degree of an element of $W$ is determined by the codimension $\dim(M)-\dim(W)$, which is preserved under suspension.
		} $S:PC^*_\Gamma(M) \to PC^{*}_\Gamma(\R \times M)$ so that if $W \in PC^*_\Gamma(M)$ is represented by $r_W \colon W \to M$ then $S(W)$ is represented by $r_{\R \times W} \colon \R \times W \to \R \times M$ defined by $r_{\R \times W}(t,x) = (t,r_W(x))$.
		Then $S$ induces isomorphisms $S \colon H^*_\Gamma(M) \to H^{*}_\Gamma(M)$.
	\end{proposition}
	\begin{proof}
		We first observe that $S$ descends to a well-defined map $C^*_\Gamma(M) \to C^{*}_\Gamma(\R \times M)$, as it is easy to see that it preserves the properties of being trivial or degenerate.
		It is also clearly a chain map.

		Next, we note that $S: C^*_\Gamma(M) \to C^{*}_\Gamma(M)$ is injective: In order for $\R \times W$ to be trivial, each $\{t\} \times W$ would have to be trivial so that $W$ would be trivial, and similarly in order for $\R \times W$ to be degenerate, $W$ would have to be degenerate.
		Thus it suffices to show that the quotient of $C^{*}_\Gamma(M)$ by the image of $C^*_\Gamma(M)$ is acyclic.
		To do so, we must show that if $\uW \in C^{*}_\Gamma(M)$ and $\bd W = S(V)+T$ for some $V \in PC^*_\Gamma(M)$ and $T \in Q^*(M)$ then $W = \bd Z+S(V')+T'$ for some $V' \in PC^*_\Gamma(M)$ and $T' \in Q^*(M)$.

		Consider the map $p \colon \R \times M \to \R$.
		The composite $pr_W$ must have a regular value by Sard's theorem; without loss of generality, we assume this is at $0$.
		As in the creasing construction, let $W_0 = (pr_W)^{-1}(0)$, $W^+ = (pr_W)^{-1}([0,\infty))$, and $W^- = (pr_W)^{-1}((-\infty,0])$.
		We can also think of these as the pullbacks $p^{-1}(0) \times_M W$, etc.
		Similarly, we consider $S(W_0)$ and note that it has an analogous creasing decomposition with $(S(W_0))_0 = W_0$, $S(W_0)^+ = \R_+ \times W_0$, and $S(W_0)^- = \R_- \times W_0$.
		We will also need to consider $\bd (W_0)$.
		From the properties of pullbacks \cite[Proposition 7.4]{Joy12} (from which we get the sign in the following formula using that $W_0 = p^{-1}(0) \times_M W$), we have
		$$\bd(W_0) = -(\bd W)_0 = -S(V+T)_0 = -S(V)_0-T_0 = -V-T_0.$$

		%p^{-1}((-infty,0]) \times_M W

		Next, we consider $W^--(\R_- \times W_0)$.
		Its boundary, again using the sign conventions from Joyce, is
		\begin{align*}
			\bd (W^--(\R_- \times W_0))& = \bd(W^-)-\bd (\R_- \times W_0) \\
			& = [W_0+(S(V)+T)^- ]-[W_0-\R_- \times \bd W_0]\\
			& = (\R_- \times V)+T^-+(\R_- \times \bd W_0)\\
			& = (\R_- \times V)+T^-+(\R_- \times (-V-T_0))\\
			& = (\R_- \times V)+T^--(\R_- \times V)-(\R_- \times T_0)\\
			& = T^--(\R_- \times T_0).
		\end{align*}
		As $T$ is an element of $Q^*(\R \times M)$ and $T^-$ and $T_0$ are pullbacks of $T$ with other precochains, they are also in $Q^*(\R \times M)$ by REF.
		It follows that $\R_- \times T_0$ is also in $Q^*(\R \times M)$, so
		$W^--(\R_- \times W_0)$ is a cocycle in $C_\Gamma^*(\R \times M)$ that is supported in $\R_- \times M$.
		By the preceding lemma, it cobounds, i.e.\ there is a $Z_1$ with $\bd Z_1 = W^--(\R_- \times W_0)+A_1$, with $A_1 \in Q^*(\R \times M)$.

		Analogously, there are $Z_2$ and $A_2$ such that $\bd Z_2 = W^+-(\R_+ \times W_0)+A_2$.

		We now consider the cochain $\Cre(W)-\Cre(S(W_0))-Z_1-Z_2$.
		Its boundary is
		\begin{align*}
			\bd(\Cre(W)&-\Cre(S(W_0))-Z_1-Z_2)\\
			& = W^++W^--W-\Cre(\bd W)-[(\R_+ \times W_0)+(\R_- \times W_0)-S(W_0)-\Cre(\bd S(W_0))] \\
			&\phantom{ = } -W^-+(\R_- \times W_0)-A_1-W^++(\R_+ \times W_0)-A_2\\
			& = -W+S(W_0)-\Cre(\bd W)+\Cre(\bd S(W_0))-A_1-A_2
		\end{align*}

		Now recall that $\bd W = S(V)+T$, while $\bd S(W_0) = \bd(\R \times W_0) = -\R \times \bd W_0 = -S(\bd W_0) = -S(-V-T_0) = S(V+T_0)$.
		So $\Cre(\bd W)-\Cre(\bd S(W_0)) = \Cre(T)-\Cre(S(T_0))$.
		We already know $A_1,A_2 \in Q^*(\R \times M)$.
		Furthermore, $\Cre(T), \Cre(S(T_0)) \in Q^*(\R \times M)$ as we have noted that the suspension of an element of $Q^*(M)$ is in $Q^*(\R \times M)$ and the creasing of an element of $Q^*(\R \times M)$ is in $Q^*(\R \times M)$ by the proof of \cite[Lemma 18]{Lipy14}.

		So, we have established a cohomology between $W$ and $S(W_0)$ as desired.
	\end{proof}
\end{comment}

\subsection{Geometric homology and cohomology are singular homology and cohomology}\label{S: homology is homology}

In this section we apply a theorem of Kreck and Singhof to show that geometric homology and cohomology are isomorphic to singular homology and cohomology on smooth manifolds.

\begin{theorem}\label{T: geometric is singular}
	On the category of smooth manifolds (without boundary) and continuous maps, geometric homology and cohomology are respectively isomorphic to singular homology and cohomology with integer coefficients, i.e.\ $H_*^\Gamma \cong H_*(\cdot ;\Z)$ and $H^*_\Gamma \cong H^*(\cdot ;\Z)$ as functors.
\end{theorem}

\begin{proof}
	This is a consequence of \cite[Theorem 10]{Krec10b} once we verify that $H_*^\Gamma$ and $H^*_\Gamma$ are respectively an ordinary homology theory and an ordinary cohomology theory on the category of smooth manifolds as defined in \cite{Krec10b}.
	This requires the following axioms:

	\begin{enumerate}
		\item\label{I: homotopy functor} $H_*^\Gamma$ is a covariant functor on the category of smooth manifolds (without boundary) and continuous maps between them, and $H^*_\Gamma$ is a contravariant homotopy functor on the same category.

		\item For each triple $(M;U,V)$ with $M$ a smooth manifold and $U,V$ open subsets such that $U \cup V = M$ there are exact (homological or cohomological) Mayer--Vietoris sequences with natural connecting maps $\delta$.

		\item\label{I: neg dim} For all $M$, $H_k^\Gamma(M) = H^k_\Gamma(M) = 0$ for $k<0$.

		\item The Dimension Axiom: $H_k^\Gamma(pt) = H^k_\Gamma(pt) = 0$ for $k\neq 0$ and $H^\Gamma_0(pt) \cong H_\Gamma^0(pt) \cong \Z$.

		\item $H_*^\Gamma$ and $H^*_\Gamma$ are additive: for a manifold $M$ of dimension $0$, each $H_k^\Gamma(M)$ is canonically isomorphic to $\oplus_{x \in M} H_k^\Gamma(x)$ and each $H^k_\Gamma(M)$ is canonically isomorphic to $\prod_{x \in M} H^k_\Gamma(x)$.
	\end{enumerate}

	Axiom \ref{I: homotopy functor} holds from the definitions and \cref{P: homology homotopy functor,P: cohomology pullback}.
	We have Mayer--Vietoris sequences by \cref{T: relative MV,T: absolute MV}.
	The connecting map for the cohomology sequence is natural by \cref{L: natural connection}.
	The connecting map for the homology sequence is natural just as in the standard argument for singular homology: given a map of triples $(M;U',V') \to (N;U,V)$ there is a map of Mayer--Vietoris sequences induced by a map of short exactly sequences of chain complexes of the form of diagram \eqref{E: homology MV SES} (replacing supported cochains complexes with chain complexes), itself induced by functoriality from the maps of chain complexes $C_*^{\Gamma}(U') \to C_*^{\Gamma}(U)$ and similarly for $V$ and $U \cap V$.
	This map of Mayer--Vietoris sequences in particular shows that the connecting map is natural.

	Axiom \ref{I: neg dim} holds trivially for homology as there are no chains of degree $<0$.
	It also holds for cohomology because for $k<0$ any representing cocycle must have small rank and boundary in $Q^*(M)$.
	Thus any such cocycle must be $0 \in C^k_\Gamma(M)$.

	The Dimension Axiom has been proven in \cref{E: dimension}.

	The Additivity Axiom is apparent.
\end{proof}

\begin{example}
	Let us consider $H^*_\Gamma(\R P^2)$.
	We know from the standard computations of $H^*(\R P^2)$ and the above theorem that
	\begin{equation*}
		H_\Gamma^i(\R P^2) =
		\begin{cases}
			\Z_2,&i = 2,\\
			\Z,&i = 0,\\
			0,&\text{otherwise.}
		\end{cases}
	\end{equation*}
	From the viewpoint of geometric cohomology, $H^0(\R P^2)$ is generated by the identity map $\R P^2 \to \R P^2$, which is co-oriented at each point by $(\beta,\beta)$ for any local orientation $\beta$.

	In degree $1$, cochains are represented by co-oriented maps from (unions of) closed intervals or circles.
	For a map from the circle to be co-oriented it cannot represent the non-trivial element $\alpha \in \pi_1(\R P^2)$, and so it must be contractible.
	Any map from a disk must be co-orientable, as the disk is contractible (this can be considered an extension of \cref{L: co-orientable homotopies}).
	Thus by smooth approximation, any smooth co-oriented map from the circle to $\R P^2$ is the boundary of a smooth co-oriented map from the disk, and so represents $0$ in cohomology.
	The same is true for any ``circle of intervals'' that does not represent $\alpha \in \pi_2(\R P^2)$; via homotopy and creasing, such a ``circle'' is the boundary of a polygon.
	On the other hand, consider a map $g \colon \interval \to \R P^2$ that does represent $\alpha$.
	As $\interval$ is contractible, such a map can always be co-oriented.
	We can let $e$ represent the standard unit vector of $\interval$ and suppose that $g$ is co-oriented so that at $0 \in \interval$ the co-orientation is represented by $(e,\beta)$ for some local orientation $\beta$ of $\R P^2$ at $f(0)$.
	Traversing the path, the representation for the co-orientation at $1$ is then $(e,-\beta)$.
	Recalling our boundary conventions, the boundary of $f$ therefore consists of the maps $f|_0:0 \to \R P^2$ co-oriented by $(1,e)*(e,\beta) = (e,\beta)$ and $f|_1:1 \to \R P^2$ co-oriented by $(1,-e)*(e,-\beta) = (e,\beta)$.
	Thus with these co-orientations, $f|_0:0 \to \R P^2$ and $f|_1:1 \to \R P^2$ represent isomorphic manifolds over $\R P^2$.
	As a co-oriented map from any single point to $\R P^2$ is a cocycle, we see by choosing the image to be the basepoint for $\pi_1(\R P^2)$ that twice any such map is a boundary.
	We leave it to the reader to verify that any two such points generate the same cohomology class, and so we have verified that $H^1_\Gamma(\R P^2) = 0$ and $H_\Gamma^2(\R P^2) \cong \Z_2$.
\end{example}

\subsubsection{A more direct comparison of singular and geometric homology}

While it was convenient to cite simultaneously the homology and cohomology versions of the Kreck-Singhof theorem, we can provide a second proof in the homology case that says a bit more.
In particular, as simplices are manifolds with corners and as the standard model simplex comes equipped with an orientation, if we let $S^{sm}_*(M)$ denote the chain complex of smooth singular chains, there is an obvious inclusion map $S^{sm}_*(M) \into PC_*^\Gamma(M)$.
It is not hard to check consistency of the boundary orientations so that this induces a chain map $S^{sm}_*(M) \to C_*^\Gamma(M)$.
We will see that this is a quasi-isomorphism.
Combined with the well-known fact that the inclusion $S^{sm}_*(M) \into S_*(M)$ is a chain homotopy equivalence if $S_*(M)$ is the full singular chain complex on $M$ \cite[Theorem 18.7]{Lee13}, this provides a concrete chain of isomorphisms $H_*(M) \cong H_*^\Gamma(M)$.

It will also be useful below to have the analogous result for cubical singular homology in addition to the more common simplicial singular homology.
Details of cubical singular homology can be found, for example, in (see, e.g., \cite{Mas91} or \cite[Section 8.3]{HW60}).
Rather than maps of simplices $\Delta^k \to M$, the cubical singular chain complex $SK_*(M)$ is generated by maps $\interval^k \to M$ with $\interval$ being the standard interval $\interval = [0,1]$.
The boundary formula is defined so that if $\sigma: \interval^k \to M$ is a singular cube, then
\begin{equation}\label{E: cube bd}
	\bd \sigma = \sum_{i = 1}^k (-1)^i(\sigma \delta_i^0-\sigma \delta^1_i),
\end{equation}
where for $\epsilon\in\{0,1\}$, the map $\delta_i^\epsilon \colon \interval^{k-1} \to \interval^k$ is defined by
$$\delta_i^\epsilon(x_1,\ldots,x_k) = (x_1,\ldots,\epsilon,\ldots, x_k)$$
with $\epsilon$ in the $i$th slot.
The homology of the chain complex $SK_*(M)$ is not isomorphic to singular homology as it does not satisfy the dimension axiom, so one instead forms the normalized complex $NK_*(M)$ by quotienting out the subcomplex of degenerate singular cubes, generated by singular cubes $\sigma \colon \interval^k \to M$ such that $\sigma$ does not depend on at least one of the variables.
In other words, the degenerate singular cubes are those maps $\sigma \colon \interval^k \to M$ that factor through one of the standard projections $\interval^k \to \interval^{k-1}$.
It then holds that $NK_*(M)$ is chain homotopy equivalent to $S_*(M)$.
In fact this holds for $M$ any space and not just a manifold \cite[Theorem 8.4.7]{HW60}.
Of course we will need the smooth version $NK^{sm}_*(M)$ generated by smooth singular cubes and modulo degenerate smooth singular cubes.
We defer to an appendix to this section the proof that $NK^{sm}_*(M) \into NK_*(M)$ is a chain homotopy equivalence.

As the cubes $\interval^k$ are compact manifolds with corners equipped with their standard orientations,
we have inclusion $SK^{sm}_i(M) \into PC_i^\Gamma(M)$ for all $i$.

\begin{lemma}
	The inclusion $SK^{sm}_*(M) \into PC_*^\Gamma(M)$ determines a chain map $NK^{sm}_*(M) \to C_*^\Gamma(M)$.
\end{lemma}

\begin{proof}
	Any degenerate singular cube $\sigma \colon \interval^k \to M$ is also degenerate in the sense of \cref{D: equiv triv and small}.
	In fact, it will have small rank as it filters through a projection.
	Furthermore, if that projection collapses the $i$th coordinate then each face $\sigma \delta_j^\epsilon$ for $j\neq i$ will also be a degenerate small cube and so have small rank, while the term $\pm (\sigma \delta_i^0-\sigma \delta_i^1)$ will be trivial with the trivializing map $\rho$ being the interchange of the two faces.
	Thus, the degenerate smooth singular cubes are elements of $Q_*(M)$, and our map is well defined in each degree.

	We check compatibility of boundary orientations.
	Consider the $n-1$ face $F$ of $\interval^n$ given by $x_i = j$ with $j\in\{0,1\}$.
	Then we have an outward pointing vector given by $(-1)^{j+1}e_i$, where $e_i$ is the vector in the positive $i$th direction.
	In general if we let $\beta_k$ denote the positive orientation corresponding to the $k$th coordinate, then the boundary orientation for $\beta_F$ is the one such that
	$(-1)^{j+1}\beta_i \wedge \beta_F$ is the orientation $\beta_1 \wedge\cdots\wedge \beta_n$ of $\interval^n$.
	Thus the boundary orientation is $(-1)^{j+1+i-1}\beta_1 \wedge \cdots \wedge \hat{\beta}_i \wedge \cdots\beta_n$.
	On the other hand, $\beta_1 \wedge \cdots \wedge \hat{\beta}_i \wedge \cdots\beta_n$ is precisely the standard orientation of $F$ when considering $\interval^n$ as a cubical complex.
	So the boundary orientation of $F$ is $(-1)^{i+j}$ times its orientation in the cubical complex.
	But this corresponds precisely to the formula for the boundary in $K^{sm}_*(M)$ coming from equation \eqref{E: cube bd}.
\end{proof}

\begin{comment}
	As the cubes $\interval^k$ are compact manifolds with corners equipped with their standard orientations,
	we have $SK^{sm}_*(M) \into PC_*^\Gamma$.
	Furthermore, any degenerate singular cube $\sigma \colon \interval^k \to M$ is also degenerate in the sense of Definition \ref{D: equiv triv and small}.
	In fact, it will have small rank as it filters through a projection.
	Furthermore, if that projection collapses the $i$th coordinate then each face $\sigma \delta_j^\epsilon$ for $j\neq i$ will also be a degenerate small cube and so have small rank, while the term $\pm (\sigma \delta_i^0-\sigma \delta_i^1)$ will be trivial with the trivializing map $\rho$ being the interchange of the two faces.
	Thus, the degenerate smooth singular cubes are elements of $Q_*(M)$.
	As in the simplicial case one can check compatibility of boundary orientations so that we have a chain map $NK^{sm}_*(M) \to C_*^\Gamma(M)$.
	\red{NEED TO PROVE THIS IS A CHAIN MAP}
\end{comment}

\begin{theorem}\label{T: hom iso map}
	The maps $H_*(S^{sm}_*(M)) \to H_*^\Gamma(M)$ and $H_*(NK^{sm}_*(M)) \to H_*^\Gamma(M)$ obtained by treating smooth singular simplices and smooth singular cubes as elements of $C_*^\Gamma(M)$ are isomorphisms.
\end{theorem}

\begin{proof}
	Both $H_*(S^{sm}_*(M))$ and $H_*(NK^{sm}_*(M))$ are isomorphic to the standard singular homology groups, so in the following we simply write $H_*$ for either of these theories and provide a uniform proof.

	We apply Theorem \cite[5.1.1]{Frie20}, which is based on standard Mayer--Vietoris techniques.
	In particular, on the category consisting of the open sets of $M$, the maps $\Phi: H_*(-) \to H_*^\Gamma(-)$ provide a natural transformation of functors.
	We need to check the following three properties:

	1.
	On $\emptyset$ or $U \subset M$ with $U$ homeomorphic to $\R^m$, the map $\Phi: H_*(U) \to H_*^\Gamma(U)$ is an isomorphism.
	As both $H_*$ and $H_*^\Gamma(-)$ are homotopy functors, we know from the respective Dimension Axioms (see \cref{E: dimension}) that in this case $H_k(U) = H_k^\Gamma(U) = 0$ for $k\neq 0$, while for $k = 0$ we have the commutative diagram
	\[
	\begin{tikzcd}
		H_0(pt) \arrow[r] \arrow[d, "\Phi"] & H_0(U) \arrow[d, "\Phi"] \\
		H_0^\Gamma(pt) \arrow[r] & H^\Gamma_0(U).
	\end{tikzcd}
	\]
	The horizontal maps are isomorphisms because these are homotopy functors, and the left hand vertical map is an isomorphism because $\Phi$ takes a generator of $H_0(pt) \cong \Z$ to a generator of $H_0^\Gamma(pt) \cong \Z$; see again \cref{E: dimension}.
	So the right hand map is also an isomorphism.

	2.
	$\Phi$ induces a commutative diagram of long exact Mayer--Vietoris sequences.
	This follows from basic homological algebra given the commutativity of the following diagram and its analogue for singular cubical chains
	\[
	\begin{tikzcd}[column sep=large]
		S_*(U \cap V) \arrow[r, "{(i_U, -i_V)}", hook] \arrow[d, "\Phi"] & S_*(U) \oplus S_*(V) \arrow[d, "\Phi \oplus \Phi"] \\
		C_*^\Gamma(U \cap V) \arrow[r, "{(i_U, -i_V)}", hook] & C_*^\Gamma(U) \oplus C_*^\Gamma(V).
	\end{tikzcd}
	\]

	3.
	If $\{U_\alpha\}$ is an increasing collection of open submanifolds of $M$ such that $\Phi \colon H_*(U_\alpha) \to H_*^\Gamma(U_\alpha)$ is an isomorphism for all $\alpha$, then $\Phi \colon H_*(\cup_\alpha U_\alpha) \to H_*^\Gamma(\cup_\alpha U_\alpha)$ is an isomorphism.
	This argument is standard given that both singular (simplicial or cubical) chains and geometric chains are represented by compact spaces: If $W$ represents a cycle in $C_*^\Gamma(\cup_\alpha U_\alpha)$, then $W \to \cup_\alpha U_\alpha$ factors through some particular $U_\beta$, so, as $H_*(U_\beta) \xr{\Phi}H_*^\Gamma(U_\beta)$ is an isomorphism, $\uW$ is in the image of $H_*(U_\beta) \xr{\Phi}H_*^\Gamma(U_\beta) \to H_*^\Gamma(\cup_\alpha U_\alpha)$.
	But then $\uW$ is in the image of composition $H_*(U_\beta) \to H_*(\cup_\alpha U_\alpha) \to H_*^\Gamma(\cup_\alpha U_\alpha)$, so $\Phi \colon H_*(\cup_\alpha U_\alpha) \to H_*^\Gamma(\cup_\alpha U_\alpha)$ is surjective.
	Similarly, if $\Phi \colon H_*(\cup_\alpha U_\alpha) \to H_*^\Gamma(\cup_\alpha U_\alpha)$ maps a class represented by a singular cycle $\xi$ to $0$, then $\xi$ bounds as a geometric cycle, say $\bd W = \xi+T$ for some $T \in Q_*(U)$.
	But by compactness, there is some $\beta$ so that $W$, $T$, and $\xi$ all have image in $U_\beta$.
	So $\xi$ represents a class in $H_*(U_\beta)$ that maps to $0$ in $H_*^\Gamma(U_\beta)$.
	As $\Phi$ is assumed an isomorphism on $U_\beta$, it must be that $\xi$ represents $0$ in $H_*(U_\beta)$, and so it also represents $0$ in $H_*(\cup_\alpha U_\alpha)$.

	It now follows from Theorem \cite[5.1.1]{Frie20} that $\Phi: H_*(M) \to H_*^\Gamma(M)$ is an isomorphism.
\end{proof}

\cref{T: hom iso map} is claimed without proof in \cite[Section 10]{Lipy14}.
Lipyanskiy states ``The fact that the natural maps induce isomorphisms follow from the standard Mayer--Vietoris arguments.'' However, these arguments are not given and, in fact, no Mayer--Vietoris sequence is proven to exist in \cite{Lipy14}, though the main required tool, creasing, is provided.

Unfortunately, providing a direct comparison for cohomology theories is not so straight forward as there is no obvious map between $C^*_\Gamma(M)$ and $S^*(M) = \Hom(S_*(M),\Z)$.
It will take some work in the following sections to develop a geometric connection between these cohomology theories.

\subsection{Appendix: smooth singular cubes}

\begin{proposition}\label{P: singular smooth cubes}
	The inclusion $\psi: NK^{sm}_*(M) \into NK_*(M)$ is a chain homotopy equivalence.
\end{proposition}

\begin{proof}
	The proof is analogous to the simplicial case as given in detail in \cite[Theorem 18.7]{Lee13}, though we need to take care with degenerate cubes, which is not an issue in the simplicial case.
	To account for this, we sketch the proof in \cite{Lee13} but provide some detailed modifications.

	We first observe that the map $SK^{sm}_*(M) \to NK_*(M)$ takes a smooth singular cubical chain to $0$ only if all of its cubes (with non-zero coefficient) are degenerate, and so we do have an injection $NK^{sm}_*(M) \into NK_*(M)$.
	We will define cube-wise a chain homotopy inverse $s \colon NK_*(M) \to NK^{sm}_*(M)$ by starting with a map $\td s \colon SK_*(M) \to SK^{sm}_*(M)$ and passing to quotients.

	Recall from \cref{S: homology is homology} that we write $\delta_i^\epsilon$ for the face inclusions of the standard cubes.
	If $\sigma \colon \interval^k \to M$ is a singular cube, we define homotopies $H_\sigma \colon \interval^k \times \interval = \interval^{k+1} \to M$ so that the following properties hold:
	\begin{enumerate}
		\item\label{I: smooth} $H_\sigma$ is a homotopy from $\sigma$ to a smooth map $\td \sigma \colon \interval^k \to M$.

		\item\label{I: faces} $H_{\sigma \delta_i^\epsilon} = H_\sigma \circ (\delta_i^\epsilon \times \id_\interval)$ so that the construction is compatible along faces.
		More explicitly, $H_{\sigma \delta_i^\epsilon}(x,t) = H_\sigma(\delta_i^\epsilon(x),t)$.

		\item If $\sigma$ is smooth then $H_{\sigma}(x,t) = \sigma(x)$, i.e.\ the homotopy is constant.

		\item\label{I: degen} If $\sigma$ is independent of the coordinate $x_i$ then so is $H_\sigma$.
	\end{enumerate}

	The last condition, which we have added for cubes, ensures that if $\sigma$ is degenerate so will be $H_\sigma$ and $\td \sigma$.

	The construction is by induction on dimension.
	If $\sigma$ is a $0$-cube, then we define $H_\sigma(x,t) = \sigma(x)$, the constant homotopy.
	This satisfies the conditions.
	We then assume $H_\sigma$ defined with these properties for all cubes of dimension $<k$ and extend the definition to $k$-cubes.
	If $\sigma$ is already smooth, then the constant homotopy $H_\sigma(x,t) = \sigma(x)$ satisfies the conditions, noting that if $\sigma$ is smooth then so is each $\sigma \circ \delta_i^\epsilon$.
	If $\sigma$ is not smooth, we consider separately the two cases when $\sigma$ is degenerate or nondegenerate.

	First suppose $\sigma$ is not degenerate.
	By the induction hypothesis and Condition \eqref{I: faces}, $H_\sigma$ is determined on $(\interval^k \times 0) \cup (\bd \interval^k \times \interval)$.
	One can check as in the proof of \cite[Lemma 18.8]{Lee13} that Condition \eqref{I: faces} guarantees that the faces glue to form a continuous map.
	As $\bd \interval^k \into \interval^k$ is a cofibration, there is a retraction $\interval^k \times \interval \to (\interval^k \times 0) \cup (\bd \interval^k \times \interval)$, and the composition determines a homotopy $F \colon \interval^k \times \interval \to M$ such that $F(-,1)$ is smooth on each $k-1$ face of $\bd \interval^k$.
	In fact, this implies that $F(-,1)$ is smooth on all of $\bd \interval^k$ by a minor modification of \cite[Lemma 18.9]{Lee13}.
	So by the Whitney Approximation Theorem \cite[Theorem 6.26]{Lee13}, there is a homotopy rel $\bd \interval^k$ from $F(-,1)$ to a smooth map $\td \sigma \colon \interval^k \to M$; we denote this homotopy $G$.
	Finally, let $u \colon \interval^k \to (0,1]$ be a continuous function that takes $\bd\interval^k$ to $1$ and the interior of the cube to $(0,1)$.
	Then we can define
	\begin{equation*}
		H_\sigma(x,t) =
		\begin{cases}
			F\left(x,\frac{t}{u(x)}\right),&x \in \interval^k, 0 \leq t \leq u(x),\\
			G\left(x,\frac{t-u(x)}{1-u(x)}\right),&x \in \text{Int}(\interval^k), u(x) \leq t \leq 1.\\
		\end{cases}
	\end{equation*}
	One can check as in the proof of \cite[Lemma 18.8]{Lee13} that this is a continuous homotopy that satisfies the first two conditions above, as required.

	Next suppose $\sigma$ is degenerate, i.e.\ there is some coordinate $x_i$ so that $\sigma$ does not depend on $x_i$.
	Let $\pi_i \colon \interval^k\to\interval^{k-1}$ be given by $\pi_i(x_1,\ldots, x_k) = (x_1,\ldots, \hat x_i, \ldots, x_k)$ with the $x_i$ term omitted.
	In this case we let $H_\sigma(x,t) \defeq H_{\sigma \delta_i^0}(\pi_i(x),t) = H_{\sigma \delta_i^1}(\pi_i(x),t)$.
	We claim that if there are multiple coordinates of which $\sigma$ is independent then this definition is independent of the choice of such coordinate.
	This is clear for $1$-cubes for which there is only one possible coordinate.
	Suppose then the claim proven in dimensions $<k$ and that $\sigma \colon \interval^k \to M$ is independent of $x_i$ and $x_j$ with $j<i$.
	Since $\sigma$ is independent of $x_j$, so is $\sigma \circ \delta_i^0$, so inductively $H_{\sigma \delta_i^0}(\pi_i(x),t) = H_{\sigma \delta_i^0\delta_j^0}(\pi_j\pi_i(x),t)$.
	Similarly, using that the $i$th coordinate of the cube is the $i-1$-st coordinate of the $j$th faces, we have $H_{\sigma \delta_j^0}(\pi_j(x),t) = H_{\sigma \delta_j^0\delta_{i-1}^0}(\pi_{i-1}\pi_j(x),t)$.
	But $\delta_i^0\delta_j^0$ and $\delta_j^0\delta_{i-1}^0$ determine the same $k-2$ face of $\interval^k$, and $\pi_j\pi_i(x) = \pi_{i-1}\pi_j(x)$.
	So both constructions give the same $H_\sigma$.

	In this case, Conditions \eqref{I: smooth} and \eqref{I: degen} hold by construction and by induction.
	We must verify Condition \eqref{I: faces}.
	If $\sigma$ is independent of $x_i$, the condition is clear by construction for the faces $\sigma\delta_i^0$ and $\sigma\delta_i^1$.
	For $j\neq i$, first suppose $i<j$.
	As $\sigma$ is independent of $x_i$, so is $\sigma\delta^j_\epsilon$, so
	\begin{align*}
		H_{\sigma\delta_j^\epsilon}(x,t)& = H_{\sigma\delta_j^\epsilon \delta_i^0}(\pi_i(x),t)\\
		& = H_{\sigma \delta_i^0\delta^{j-1}_\epsilon}(\pi_i(x)),t)\\
		& = H_{\sigma \delta_i^0}(\delta_{j-1}^\epsilon\pi_i(x)),t)\\
		& = H_{\sigma \delta_i^0}(\pi_i(\delta_j^\epsilon(x)),t)\\
		& = H_\sigma(\delta_j^\epsilon(x),t).
	\end{align*}
	Here the first equality uses our definition of $H_{\sigma\delta_j^\epsilon}$ as $\sigma\delta_j^\epsilon$ is independent of $x_i$.
	The second equality is an identity for cubical face inclusions.
	The third is Condition \eqref{I: faces} for $H_{\sigma \delta_i^0}$, which holds by induction hypothesis.
	The fourth equality is another cubical identity, and the last is the definition of $H_\sigma$.

	Similarly, if $j<i$, then $\sigma\delta_j^\epsilon$ is independent of its $i-1$-st coordinate, and we compute analogously:

	\begin{align*}
		H_{\sigma\delta_j^\epsilon}(x,t)& = H_{\sigma\delta_j^\epsilon \delta_{i-1}^0}(\pi_{i-1}(x),t)\\
		& = H_{\sigma \delta_{i}^0\delta_{j}^\epsilon}(\pi_{i-1}(x)),t)\\
		& = H_{\sigma \delta_{i}^0}(\delta_{j}^\epsilon\pi_{i-1}(x)),t)\\
		& = H_{\sigma \delta_{i}^0}(\pi_{i}(\delta_j^\epsilon(x)),t)\\
		& = H_\sigma(\delta_j^\epsilon(x),t).
	\end{align*}

	This completes our construction of the homotopies $H_\sigma$.
	We can now define $\td s \colon SK_*(M) \to SK^{sm}_*(M)$ by $\td s(\sigma) = H_\sigma(-,1)$.
	Then if $\td \psi: SK_*^{sm}(M) \to SK_*(M)$ is the inclusion, we have by definition that $\td s \td \psi = \id$.
	We show that $\td\psi\td s$ is chain homotopic to the identity\footnote{Here, finally, is a step that is easier in the cubical setting as we do not need to subdivide prisms into simplices.}.
	Indeed, if $\sigma$ is a singular $k$-cube then treating $H_\sigma$ as a singular $k+1$ cube we have

	\begin{align*}
		\bd H_\sigma& = \sum_{i = 1}^{k+1} (-1)^i\left(H_\sigma \delta_i^0-H_\sigma \delta^1_i\right)\\
		& = \left(\sum_{i = 1}^{k} (-1)^i\left(H_\sigma (\delta_i^0 \times \id_\interval)-H_\sigma (\delta^1_i \times \id_\interval)\right)\right) +(-1)^{k+1}(H_\sigma(-,0)-H_\sigma(-,1))\\
		& = \left(\sum_{i = 1}^{k} (-1)^i\left(H_{\sigma \delta_i^0}-H_{\sigma\delta^1_i}\right)\right) +(-1)^{k+1}(\sigma(-)-\td \psi\td s(\sigma)).
	\end{align*}
	So if we define $\td J(\sigma) = (-1)^{k+1}H_\sigma$, we obtain
	\begin{align*}
		(-1)^{k+1}\bd \td J(\sigma)& = \left(\sum_{i = 1}^{k} (-1)^i\left( (-1)^k\td J(\sigma \delta_i^0)-(-1)^k\td J(\sigma\delta^1_i)\right)\right) +(-1)^{k+1}\left(\sigma(-)-\td \psi\td s(\sigma)\right)\\
		& = (-1)^k\td J(\bd \sigma)+(-1)^{k+1}(\sigma(-)-\td \psi\td s(\sigma)),
	\end{align*}
	so $$\bd \td J(\sigma)+\td J(\bd \sigma) = \sigma(-,0)-\td \psi\td s(\sigma),$$
	which shows that $\td \psi\td s$ is chain homotopic to the identity.

	Finally, we note that, by definition and construction, $\td \psi$, $\td s$, and $\td J$ all take degenerate simplices to degenerate simplices so that these descend to chain maps $\psi \colon NK_*^{sm}(M) \to NK_*(M)$ and $s \colon NK_*(M) \to NK_*^{sm}(M)$ with $s\psi = \id$ and a chain homotopy $J \colon NK_*(M) \to NK_{*+1}(M)$.
\end{proof} %
	% !TEX root = ../foundations.tex

\section{Interaction with cubical structures}\label{S: transversality}

In this section we bring in some auxiliary structures that will help us further develop geometric cohomology and its connections to singular cohomology.
In particular, we equip our manifolds with smooth cubulations.
Many of our results would apply just as well with the more familiar smooth triangulations, but we find cubulations to be more convenient.
In particular, in \cite{FMS-flows} we have considered geometric cochains in the presence of cubulations, demonstrating how to obtain a fully-defined cochain-level cup product via intersection using certain flows developed in terms of the cubulation.
Cup products in geometric cohomology will be discussed in the following section.

Smooth cubulations are analogous to smooth triangulations in that they involve a homeomorphism $M \cong |X|$ between a manifold $M$ and the geometric realization $|X|$ of a cubical complex $X$ such that the restriction to each cubical face is a smooth embedding.
What is slightly different, aside from substituting cubes (i.e.\ copies of $\interval^k$) for simplices, is that cubical complexes are required to have a bit more structure than simplicial complexes, which can always be constructed just by gluing together simplices along faces.
The issue is that any simplicial complex can be given a total ordering of its vertices, and this ordering provides a canonical identification between any simplicial face and the standard model simplex of the same dimension.
By contrast, the natural combinatorial structure on the vertices of the standard cube is not a total ordering but rather a partial ordering.
In particular, if we take the standard cube to be $\interval^k = [0,1]^k \subset \R^k$, then we have $v \leq w$ for two vertices if each coordinate of $v$ is less than or equal to the corresponding coordinate of $w$.
There turn out to be spaces obtained from naively gluing cubes that do not support compatible partial orderings of this type.
So rather when we speak of cubical complexes we will restrict ourselves to complexes that do admit such combinatorial data.
Consequently, each cubical $k$-face comes equipped with an identification with the standard $k$-cube, and hence also a standard orientation.
As we will note below, smooth cubulations of this form exist for any smooth manifold.
In the remainder of this work, ``cubulation'' will always mean a smooth cubulation.

Also analogously to simplicial complexes, cubical complexes possess algebraic cubical chain and cochain complexes and so cubical homology and cohomology that coincides with singular homology and cohomology\footnote{We will show below that cubical homology coincide with singular cubical homology, which coincide with simplicial singular homology by \cite{EM53}.
As all of the involved chain complexes are free, the corresponding cohomologies are also isomorphic by basic homological algebra \cite[Theorem 45.5]{Mun84}.\label{FN: cubical and singular}}.
Our primary goal in this section is to see that there are direct geometrically-defined isomorphisms between cubical (co)homology nd geometric (co)homology.

For this, we first provide some background on cubical complexes and cubical homology and cohomology in \cref{S: cubes,S: cubical cochains}.
Then in \cref{S: cubical and geometric homology} we show that the obvious map that takes a face of a cube complex to its corresponding geometric chain induces an isomorphism from cubical homology to geometric homology.
Next, in \cref{S: transverse cochains}, we consider those geometric cochains that are transverse to a given cubulation and show that their cohomology agrees with the geometric cohomology obtained without that constraint.

The motivation for our interest in cochains that are transverse to the cubulation is that they allow us to define an \textit{intersection map} $\mc I$ from these transverse geometric cochains to the cubical cochains.
If $F$ is a face of the cubulation, $F^*$ its dual cochain, and $W$ is a geometric cochain of complementary dimension to $F$, then the coefficient of $F^*$ in $\mc I(W)$ is simply the geometric intersection number of $W$ with $F$.
This intersection map is defined in \cref{S: intersection map}, which also contains our proof that the intersection map induces a cohomology isomorphism when $H^*(M)$ is finitely generated in each degree.
To implement this proof, we include in \cref{S: dual cubes} a discussion of what we call ``central subdivisions'' of cubical complexes, which are analogous to barycentric subdivisions of simplicial complexes and allow us to construct the cubical dual cells to faces of the cubulation.

\subsection{Cubical complexes and cubulations}\label{S: cubes}

We begin by recalling some notation from \cite{FMS-flows}.
In the context of cubical complexes we write the unit interval as $\interval = [0,1]$ and define the \textbf{standard $n$-cube} to be
\begin{equation*}
	\interval^n = \big\{ (x_1, \dots, x_n) \in \R^n\ |\ 0 \leq x_i \leq 1 \big\}.
\end{equation*}
Denote $\{1, \dots, n\}$ by $\overline{n}$.

Given a partition $F = (F_0, F_{01}, F_1)$ of $\overline n$, it determines a \textbf{face} of $\interval^n$ given by
\begin{equation*}
	\{(x_1, \dots, x_n) \in \interval^n\ |\ \forall \varepsilon \in \{0, 1\},\ i \in F_\varepsilon \Rightarrow x_i = \varepsilon\}.
\end{equation*}
We abuse notation and write $F$ for both the partition and its associated face.
We refer to coordinates $x_i$ with $i \in F_{01}$ as \textbf{free} and to the others as \textbf{bound}.
The \textbf{dimension} of $F$ is its number of free coordinates, and as usual the faces of dimension $0$ and $1$ are called vertices and edges, respectively.
The set of vertices of $\interval^n$ is denoted by $\vertices(\interval^n)$.

For $\varepsilon \in \{0, 1\}$ and $i \in \overline{n}$, we define maps $\delta_i^\varepsilon \colon \interval^{n-1} \to \interval^{n}$ by
\begin{align*}
	\delta_i^\varepsilon(x_1, \dots, x_{n-1}) & = (x_1, \dots, x_{i-1}, \varepsilon, x_i, \dots, x_{n-1}).
\end{align*}
Any composition of these is referred to as a \textbf{face inclusion map}.

We also have projection maps $\pi_i \colon \interval^n \to \interval^{n-1}$ such that
\[\pi_i(x_1, \ldots, x_n) = (x_1, \ldots, \hat x_i, \ldots, x_n),\]
with $\hat x_i$ as usual denoting the omission of the $x_i$ term.
Analogous to the face and degeneracy identities for simplicial sets, these operators satisfy the following relations \cite[Section 4]{GrMa03} (it is also easy and illuminating to work these out on one's own):

\[
	\begin{array}{rlc}
		\delta_j^\eta \delta_i^\varepsilon &= \delta_{i+1}^\varepsilon \delta_j^\eta, &j\leq i,\\
		\pi_i \pi_j &= \pi_j \pi_{i+1}, & j \leq i,\\
		\pi_j \delta^\varepsilon_i &=
			\begin{cases}
				\delta_{i-1}^\varepsilon \pi_j, \\
				\id,\\
				\delta_i^\varepsilon \pi_{j-1},
			\end{cases}
		&\begin{array}{lll}j<i, \\ j=i, \\ j>i. \end{array}
	\end{array}
\]

For $v \in \vertices(\interval^n)$ all coordinates are bound -- that is, $v_{01} = \emptyset$.
Thus
$v$ is determined by the partition of $\overline n$ into $v_0$ and $v_1$, so
we have a bijection from the set of vertices of $\interval^n$ to the power set $\mathcal P(\overline n)$ of $\overline n$, sending $v$ to $v_1$.
The inclusion relation in the power set induces a poset structure on $\vertices(\interval^n)$ given explicitly by
\begin{equation*}
	v = (\epsilon_1, \dots, \epsilon_n) \leq w = (\eta_1, \dots, \eta_n) \iff \forall i,\ \epsilon_i \leq \eta_i.
\end{equation*}
We will freely use the identification of these two posets.
Face embedding maps induce order-preserving maps at the level of vertices.

An \textbf{interval subposet} of $\mathcal P(\overline n)$ is one of the form $[v, w] = \{u \in \mathcal P(\overline n)\ |\ v \leq u \leq w\}$ for a pair of vertices $v \leq w$.
There is a canonical bijection between faces of $\interval^n$ and such subposets, associating to $[v, w]$ the face $F$ defined by $F_\varepsilon = \{i \in \overline{n}\ |\ v_i = w_i = \varepsilon\}$ for $\varepsilon \in \{0, 1\}$.

The posets $\{\mathcal P(\overline n)\}_{n \geq 1}$ play the role for cubical complexes that finite totally ordered sets play for simplicial complexes.
Recall for comparison that one definition of an abstract ordered simplicial complex is as a pair $(V, X)$, where $V$ is a poset and $X$ is a collection of subsets of $V$, each with an induced total order, such that all singletons are in $X$ and subsets of sets in $X$ are also in $X$.
We have the following cubical analogue.

\begin{definition}\label{D:cubical}
	A \textbf{cubical complex} $X$ is a collection $\{ \sigma \}$ of finite non-empty subsets of a poset
	$\vertices(X)$, together with, for each $\sigma \in X$, an order-preserving bijection $\iota_\sigma \colon \sigma \to \mathcal P(\overline n)$ for some $n$, such that:
	\begin{enumerate}
		\item For all $v \in \vertices(X)$, $\{v\} \in X$,
		\item For all $\sigma \in X$ and all $[u,w] \subset \mathcal P(\overline n)$ the set $\rho = \iota_\sigma^{-1}([u,w]) \in X$ and the following commutes
		\begin{equation*}
			\begin{tikzcd} [row sep = tiny, column sep = small]
				\sigma \arrow[rr, "\iota_\sigma"] && \mathcal P(\overline n) \\
				& [-5pt] {[}u,w{]} \arrow[ur, hook] & \\
				\rho \arrow[uu, hook] \arrow[rr, "\iota_\rho"'] && \mathcal P(\overline m).
				\arrow[ul, "\cong"'] \arrow[uu, dashed]
			\end{tikzcd}
		\end{equation*}
	\end{enumerate}
	We refer to an element $\sigma \in X$ as a \textbf{cube} or \textbf{face} of $X$, refer to $\iota_\sigma \colon \sigma \to \mathcal P(\overline{n})$ as its \textbf{characteristic map},
	and refer to $n$ as its \textbf{dimension}.
	If $\rho \subseteq \sigma \in X$, we say that $\rho$ is a \textbf{face} of $\sigma$ in $X$.
	We identify elements in $\vertices(X)$ with the singleton subsets in $X$, referring to them as vertices.
\end{definition}

In analogy with the usual terminology in the simplicial setting, one could call these ``ordered cubical complexes," but we only work with these and have seen little use elsewhere for the unordered version.
Our definition sits between cubical sets \cite{jardine2002cubical} and cellular subsets of the cubical lattice of $\R^\infty$ \cite{kaczynski2006computational}, analogously to the way that abstract ordered simplicial complexes sit between simplicial sets and simplicial complexes.
The following geometric realization construction makes our definition and the cubical lattice definition essentially equivalent.
Just as is the case for simplicial complexes, faces in cubical complexes are completely determined by their vertices.

Consider the category defined by the inclusion poset of a cubical complex $X$ and the subcategory ${\tt Cube}$ of the category ${\tt Top}$ of topological spaces whose objects are the $n$-cubes, identified with $\interval^n$, and whose morphisms are face inclusions.
The characteristic maps of $X$ define a functor from its poset category to $\mathtt{Cube}$, and we define its \textbf{geometric realization} as the colimit of this functor.
A \textbf{cubical structure} or \textbf{cubulation} on a space $S$ is a homeomorphism $h \colon |X| \to S$ from the geometric realization of a cubical complex.
We abuse notation and write $h \circ \iota_{|\sigma|}$ simply as $\iota_\sigma$ for any $\sigma \in X$ when a cubical structure $h \colon |X| \to S$ is understood.

A smooth cubulation is one for which characteristic maps are smooth maps of manifolds with corners.
Smooth cubulations exist for any smooth manifold, as in the following construction of \cite{ShSh92}.
Start with a smooth triangulation (see for example \cite[Theorem 10.6]{MUNK66} for the existence of such).
Consider the cell complex that is dual to its barycentric subdivision.
Intersecting those dual cells with each simplex in the triangulation provides a subdivision of the simplex into cells that are linearly isomorphic to cubes.
Moreover, starting with an ordered triangulation -- obtained for example by taking a barycentric subdivision -- such a cubical decomposition embeds cellularly into the cubical lattice of $\R^\infty$, and thus it is the geometric realization of a cubical complex.

\begin{figure}
	\newcommand*{\xMin}{0}%
\newcommand*{\xMax}{4}%
\newcommand*{\yMin}{0}%
\newcommand*{\yMax}{4}%

\begin{subfigure}{.4\textwidth}
	\centering
	\begin{tikzpicture}[scale=.8]
		\draw[-{Latex[length=2mm]}] (-.5,\yMin)--(-.5,\yMax);
		\draw[-{Latex[length=2mm]}] (-.5,\yMin)--(-.5,\yMax-.5);
		\draw[-{Latex[length=2mm]}] (4.5,\yMin)--(4.5,\yMax);
		\draw[-{Latex[length=2mm]}] (4.5,\yMin)--(4.5,\yMax-.5);

		\draw[-{Latex[length=2mm]}] (\xMin, -.5)--(\xMax, -.5);
		\draw[-{Latex[length=2mm]}] (\xMin, 4.5)--(\xMax, 4.5);

		\draw [very thin,gray] (\xMin, \yMin) -- (\xMin, \yMax) -- (\xMax, \yMax) -- (\xMax, \yMin) -- (\xMin, \yMin);

		\draw [very thin,gray] (0.5*\xMax, \yMin) -- (0.5*\xMax, \yMax);

		\draw [very thin,gray] (\xMin, 0.5*\yMax) -- (\xMax, 0.5*\yMax);
	\end{tikzpicture}
	\caption{\textbf{Not} a cubulation of the torus}
\end{subfigure}\qquad
\begin{subfigure}{.4\textwidth}
	\centering
	\begin{tikzpicture}[scale=.8]
		\draw[-{Latex[length=2mm]}] (-.5,\yMin)--(-.5,\yMax);
		\draw[-{Latex[length=2mm]}] (-.5,\yMin)--(-.5,\yMax-.5);
		\draw[-{Latex[length=2mm]}] (4.5,\yMin)--(4.5,\yMax);
		\draw[-{Latex[length=2mm]}] (4.5,\yMin)--(4.5,\yMax-.5);

		\draw[-{Latex[length=2mm]}] (\xMin, -.5)--(\xMax, -.5);
		\draw[-{Latex[length=2mm]}] (\xMin, 4.5)--(\xMax, 4.5);

		\foreach \i in {\xMin,...,\xMax} {
			\draw [very thin,gray] (\i,\yMin) -- (\i,\yMax);
		}
		\foreach \i in {\yMin,...,\yMax} {
			\draw [very thin,gray] (\xMin,\i) -- (\xMax,\i);
		}
	\end{tikzpicture}
	\caption{A cubulation of the torus}
\end{subfigure}
	\caption{The first cellular decomposition of a torus pictured above does not represent the geometric realization of a cubical complex, as each square has the same set of vertices.
	On the right, each square has been coherently identified with the standard square.
	Therefore, (B) depicts a cubical structure on the torus.}
	\label{F: cubical structure}
\end{figure}

\subsection{Cubical chains and cochains}\label{S: cubical cochains}

We can also define an ``algebraic realization" for a cubical complex in analogy to its geometric realization.
Let $K_*(\interval^1)$ be the usual cellular chain complex of the interval with integral coefficients.
Explicitly, $K_0(\interval^1)$ is generated by the vertices $[\underline{0}]$ and $[\underline{1}]$, and $K_1(\interval^1)$ is generated by the unique 1-dimensional face, denoted $[\underline{0},\underline{1}]$ in the interval subposet notation.
The boundary map is $\bd [\underline{0},\underline{1}] = [\underline{1}]-[\underline{0}]$.

Let $K_*(\interval^n) = K_*(\interval^1)^{ \otimes n}$, with differential defined by the graded Leibniz rule.
Given a face inclusion $\delta_i^{\varepsilon} \colon \interval^n \to \interval^{n+1}$ the natural chain map $K_*(\delta_i^{\varepsilon}) \colon K_*(\interval^1)^{ \otimes n} \to K_*(\interval^1)^{ \otimes n+1}$ is defined on basis elements by
\begin{equation*}
	x_1 \otimes \cdots \otimes x_n \mapsto
	x_1 \otimes \cdots \otimes [\underline{\varepsilon}] \otimes \cdots \otimes x_n.
\end{equation*}
Regarding a cubical complex $X$ as a functor to $\mathtt{Cube}$, we can compose it with the chain functor above to obtain a functor to chain complexes.
The complex of \textbf{cubical chains} of $X$, denoted $K_*(X)$, is defined to be the colimit of this composition.
As one would expect, in each degree it is a free abelian group generated by the cubes of that dimension, and its boundary homomorphism sends the
generator associated to a cube to a sum of generators associated to its codimension-one faces with appropriate signs.
By abuse, we use the same notation and terminology for an element in $X$, its geometric realization in $|X|$,
and the corresponding basis element in $K_*(X)$.
Most commonly we will write $F$ and refer simply to a ``face of $X$.''

We note that for each $\interval^n$ we have the ordered set $\{\e_1, \dots, \e_n\}$ where $\e_i = \frac{\bd\ }{\bd x_i}$.
For any face $F$ of $\interval^n$, the ordered subset $\beta_F = \{\e_i\ |\ i \in F_{01}\}$ defines the \textbf{canonical orientation} of $F$.
In forming the cubical complex $X$, these orientations are preserved, and so each face of $X$ carries an orientation.
These orientations are compatible with our standard generators of $K_*(X)$ in the sense that if we identify $[0,1]^{ \otimes k}$ with $\interval^k$ with its standard orientation then, in the boundary formula, $k-1$ faces appear sign $1$ or $-1$ according to whether or not their standard orientations agree with the boundary orientation of $\interval^k$ as a manifold with corners.

The \textbf{cubical cochain complex} of $X$ (with $\Z$ coefficients) is the chain complex $K^*(X) = \Hom_\Z(K_*(X), \Z)$.
If $F$ is a face of $X$, and correspondingly a generator of $K_*(X)$, then we will write $F^*$ for the dual, i.e.\ the generator of $K^*(X)$ such that $F^*(F) = 1$ and $F^*(E) = 0$ for all other faces $E\neq F$ of $X$.
We will use the convention as in \cite{Mun84} that
$$(dF^*)(\xi) = F^*(\bd \xi).$$

\subsection{Comparing cubical and geometric homology}\label{S: cubical and geometric homology}

Suppose $h \colon |X| \to M$ is a smooth cubulation.
As the cubes of $X$ are compact oriented manifolds with corners, the composition of the inclusion of a cube into $X$ with the map $h$ gives an element of $PC_*^\Gamma(M)$ and hence an element of $C_*^\Gamma(M)$.
Furthermore, the boundary formula for cubes in $K_*(X)$ agrees with the geometric boundary formula, so the cubes of $X$ generate a subcomplex $K^X_*(M) \subset C^\Gamma_*(M)$ that is canonically isomorphic to $K_*(X)$.
As expected this gives the standard homology:

\begin{theorem}\label{T: cubical homology iso}
	The map $\mc J \colon K_*(X) \cong K^X_*(M) \to C^\Gamma_*(M)$ induces an isomorphism of homology groups $H_*(K_*(X)) \to H_*^\Gamma(M)$.
\end{theorem}

\begin{proof}
	The proof is analogous to the proof of \cite[Proposition V.8.3]{Dol72}, which provides an isomorphism between simplicial and singular homology.

	Let $NK_*(M)$ be the normalized singular cubical chain complex of $M$ as in recalled in \cref{S: homology is homology}, and let $NK^{sm}_*(M)$ be the subcomplex generated by smooth cubes.
	By \cref{P: singular smooth cubes,T: hom iso map}, there are quasi-isomorphisms $NK_*(M) \xr{\psi} NK^{sm}_*(M) \to C^\Gamma_*(M)$, the latter induced by observing that smooth singular cubes are elements of $PC^\Gamma_*(M)$ and that degenerate cubes are elements of $Q_*(M)$.

	Next we observe that there is a map $\eta: K^X_*(M) \to NK^{sm}_*(M)$ that takes each cube into its embedding to $M$ (recall that we assume the cubulation is smooth) and that the composition $\phi\eta$ is the map $\mc J \colon K^X_*(M) \to C^\Gamma_*(M)$ of the theorem statement.
	So it suffices to show that $\eta$ is an isomorphism.

	For this, we have the diagram
	\[
	\begin{tikzcd}
		H_*(K^X_*(M)) \arrow[r, "\eta_*"] \arrow[d, "\cong"] & H_*(NK^{sm}_*(M)) \arrow[r, "\psi_*", "\cong"'] & H_*(NK_*(M)) \\
		H_*(CW_*(M)) \arrow[rru, "\Theta", "\cong"', out=0, in=200] & &
	\end{tikzcd}
	\]
	in which $CW_*(M)$ is the CW chain complex of $M$ corresponding to the CW complex structure given by the cubulation and $\Theta$ is the standard isomorphism between CW homology and singular homology as in Dold \cite[Proposition V.1.9]{Dol72}.
	The isomorphism in Dold is developed using simplicial singular homology, but as simplicial singular and cubical singular homology are isomorphic, the argument goes through identically using singular cubes.
	The map on the left is an isomorphism at the chain level as there is an evident isomorphism in this case between the cubical chain complex and the CW chain complex that takes an embedding of a $k$-cube to the corresponding generator of $CW_k(M) = H_k(X^k, X^{k-1})$ (where we assume the expression on the right is singular cubical homology).
	As in Dold, the map $\Theta$ takes a class in $H_k(CW_*(M))$ represented to by a $k$-cycle $z$ in $CW_k(M)$ to the class in $H_k(NK_*(M))$ represented by a singular (cubical) cycle in $NK_k(M)$ that represents the same class as $z$ in $H_k(X^k,X^{k-1})$.
	But all cycles in the image of the vertical map of the diagram are already represented by singular cycles in $NK^{sm}_k(M)$, so the diagram commutes, and it follows that $\eta_*$ is an isomorphism.

	The isomorphism of the theorem is now obtained by composing the maps $K_*^X(M) \to NK^{sm}_*(M) \to C_*^\Gamma(M)$.
\end{proof}

As a corollary of the proof, we have the following useful result concerning the cohomology groups of the complexes
\begin{align*}
	NK^*(M))& \defeq \Hom(NK_*(M),\Z)\\
	NK_{sm}^*(M)& \defeq \Hom(NK^{sm}_*(M),\Z)\\
	K^*_X(M)& \defeq \Hom(K^X_*(M),\Z).
\end{align*}

\begin{corollary}
	The following maps on cohomology induced by restrictions are isomorphisms: $$H^*(NK^*(M)) \to H^*(NK_{sm}^*(M)) \to H^*(K^*_X(M)).$$
\end{corollary}

\begin{proof}
	This follows from basic homological algebra \cite[Theorem 45.5]{Mun84} as $NK_*(M)$, $NK^{sm}_*(M)$, and $K^X_*(M)$ are all free chain complexes, observing that even though $NK_*(M)$ are $NK^{sm}_*(M)$ are defined by taking the quotients of the groups of all singular cubes, $SK_*(M)$ and $SK^{sm}_*(M)$, by the subgroups of degenerate cubes, the degenerate cubes correspond to generators of $SK_*(M)$ and $SK^{sm}_*(M)$, and so each $NK_i(M)$ and $NK^{sm}_i(M)$ is freely generated by the nondegenerate, respectively nondegenerate and smooth, singular $i$-cubes.
\end{proof}

\subsection{Cubically transverse geometric cohomology}\label{S: transverse cochains}

In this section we consider the cochains on $M$ represented by maps $W \to M$ that are transverse to a given cubulation of $M$.

\begin{definition}
	Let $M$ be equipped with a smooth cubulation $|X| \to M$.
	We say that $r_W \colon W \to M$ is \textbf{transverse} to $X$ if $r_W \colon W \to M$ is transverse to each characteristic map of the cubulation.
	In particular, this implies by \cref{L: simple trans} that each induced $\bd^kW \to M$ is naively transverse to each face of the cubulation.
	If $r_W \colon W \to M$ is transverse to $X$ then the same is true for any $r_V \colon V \to M$ isomorphic to $r_W \colon W \to M$ , and so we can define $PC^*_{\Gamma \pf X}(M)$ to be the subset of $PC^*_{\Gamma}(M)$ consisting of those precochains with reference maps transverse to $X$.

	We let $Q^*_{\Gamma \pf X}(M) = PC_{\Gamma \pf X}^*(M) \cap Q^*(M)$ and note that the equivalence relation of \cref{L: cancel Q} descends to an equivalence relation on $PC_{\Gamma \pf X}^*(M)$ such that $V\sim W$ if and only if $V \sqcup -W \in Q^*_{\Gamma \pf X}(M)$.
	The \textbf{geometric cochains of $M$ transverse to $X$}, denoted $C_{\Gamma \pf X}^*(M)$, are the equivalence classes in $PC_{\Gamma \pf X}^*(M)$.
	The set $C_{\Gamma \pf X}^*(M)$ is a chain complexes under the operation $\sqcup$ and with boundary map $\bd$.
	The \textbf{geometric cohomology transverse to $X$} is $H_{\Gamma \pf X}^*(M) \defeq H^*(C_{\Gamma \pf X}^*(M))$.

	When the specific cubulation $X$ is understood, we sometimes simplify the notation to $PC_{\Gamma\pf}^*(M)$, $C_{\Gamma\pf}^*(M)$, and $H_{\Gamma\pf}^*(M)$.
\end{definition}

The proof of \cref{L: co/chains well defined} continues to hold for transverse cochains, and so $r_W \colon W \to M$ represents $0$ in $C^*_{\Gamma \pf X}(M)$ if and only if it is in $Q^*_{\Gamma \pf X}(M)$.
Therefore, the evident map $C^*_{\Gamma \pf X}(M) \to C^*_\Gamma(M)$, which takes the element of $C^*_{\Gamma \pf X}(M)$ represented by $r_W \colon W \to M$ to the element $C^*_\Gamma(M)$ represented by the same map, is a monomorphism of chain complexes, for such an $r_W$ is transverse to $X$ by definition and if it is also in $Q^*(M)$ then it is in $Q^*_{\Gamma \pf X}(M)$.
Thus we will think of $C^*_{\Gamma \pf X}(M)$ as a subcomplex of $C^*_\Gamma(M)$.
A key technical result, which will take the remainder of this section to prove,
is that this inclusion induces a cohomology isomorphism.
In other words, the cochains that are transverse to $X$ are sufficient to compute the cohomology of $M$.

\begin{theorem}\label{T: transverse complex}
	The inclusion $C^*_{\Gamma \pf X}(M) \into C^*_\Gamma(M)$ is a quasi-isomorphism.
\end{theorem}

To show that the inclusion $C^*_{\Gamma \pf X}(M) \into C^*_\Gamma(M)$ is a quasi-isomorphism, it will be necessary to consider the following scenario.
Suppose we have a map $r_V \colon V \to M$ with $V$ a manifold with corners and $M$ a manifold with a cubulation.
Let $\bd V = W$, and suppose $W$ is already transverse to the cubulation.
We will construct a homotopy $h \colon V \times I \to M$ such that $g(-,0) = r_V$, $g(-,1)$ is transverse to the cubulation, and the restriction of $h$ to $W \times I$ is transverse to the cubulation. However, as noted at the end of \cref{S: covariant functoriality}, such a homotopy might not preserve cohomology classes, as we will need below. So we must instead use the universal homotopies of \cref{D: universal homotopy,P: universal homotopy}.


The technique for constructing such homotopies will be modeled on a variety of results in \cite{GuPo74}.
We use the Transversality Theorem and Transversality Homotopy Theorem of \cite[Section 2.3]{GuPo74} as stated.
However, for the Stability Theorem of \cite[Section 1.6]{GuPo74} we will provide details of the proof because the proof is only sketched in \cite{GuPo74} and we will need the result to be generalized in several ways.
Also, the Stability Theorem is not completely correct as stated in \cite[Section 1.6]{GuPo74}; the requirement that the submanifolds of the target be closed sets is omitted\footnote{As stated in \cite{GuPo74}, the claim is that if $f \colon X \to Y$ is transverse to any submanifold $Z$ of $Y$ then this property is stable under small homotopies of $f$; more specifically that if $f_t:X \times I \to Y$ is a homotopy with $f_0$ transverse to $Z$ then there is an $\epsilon>0$ such that $f_t$ is transverse to $Z$ for all $t\in[0,\epsilon)$.
Here is a counterexample:

In the plane $\R^2$, let $Z = \{(x,y)|y = x^2, x\neq 0\}$ and consider maps $g_t: \R \to \R^2$ with
$g_t(x) = (x,t^2+2t(x-t))$.
For each fixed $t$, the image is the line given by $y-t^2 = 2t(x-t)$, which has slope $2t$ and passes through the point $(t,t^2)$.
So the map $g_0$ embeds $\R$ as the x-axis, and as the image does not intersect $Z$, the map $g_0$ is transverse to $Z$.
But for all $t\neq 0$, $g_t$ takes $\R$ to a line that is tangent to $Z$, and so $g_t$ is not transverse to $Z$ for $t\neq 0$, violating the Stability Theorem as stated on page 35 of \cite{GuPo74}.

The error in the proof comes from considering only what happens in neighborhoods of points $x$ such that $f(x) \in Z$ but not points $x$ with $f(x)\notin Z$.
As we can see, the claim breaks down when $f(x)\notin Z$ but every neighborhood of $(x,0)$ in $X \times I$ has a point with image in $Z$.
However, this can be avoided if $Z$ is a closed set in $Y$.}.
The needed versions of these results is established in the following proposition:

\begin{proposition}\label{P: ball stability}
	Suppose $r_V \colon V \to M$ is a proper map from a manifold with corners to a cubulated manifold without boundary.
	Then there is a proper universal homotopy $h \colon V \times I \to M$ such that:
	\begin{enumerate}
		\item $h(-,0) = r_V$,
		\item $h(-,1) \colon V \to M$ is transverse to the cubulation,
		\item if $i_W \colon W \to V$ is the inclusion of a union of boundary components of $V$ with $r_W = r_Vi_W \colon W \to M$ transverse to the cubulation then $h \circ (i_{W} \times \id) \colon W \times I \to M$ is transverse to the cubulation.
	\end{enumerate}
\end{proposition}

Before proving the proposition, which is somewhat technical, we use it to prove \cref{T: transverse complex}, which states that $H^*(C^*_{\Gamma \pf X}(M)) \to H^*(C_\Gamma^*(M))$ is an isomorphism.

\begin{proof}[Proof of \cref{T: transverse complex}]
	The idea of the argument that $H^*(C^*_{\Gamma \pf X}(M)) \to H^*(C_\Gamma^*(M))$ is a surjection is contained already in the proof of \cite[Lemma 15]{Lipy14}, which involves constructing a homotopy to move a cycle into transverse position.
	We elaborate upon that argument.

	Suppose $\uV \in C_\Gamma^*(M)$ is a cocycle represented by $r_V \colon V \to M$.
	By \cref{P: ball stability}, there is a proper universal homotopy $h \colon V \times I \to M$ from $r_V$ to $r_{V'} \colon V' \to M$ with $V'$ transverse to the cubulation.
	By \cref{C: homotopy}, $r_V$ and $r_{V'}$ represent the same cohomology class in $H^*_{\Gamma}(M)$, but the class represented by $r_{V'}$ is in the image of $H^*(C^*_{\Gamma \pf X}(M))$.

	For injectivity, suppose $W \in PC^*_{\Gamma \pf X}(M)$ is transverse to the cubulation and represents zero in $H^*(C_\Gamma^*(M))$.
	Then by definition there is a $V \in PC^*_\Gamma(M)$ with $\bd V = W+T$ for some
	$T \in Q^*(M)$.
	By \cref{P: ball stability} there is a proper universal homotopy $h \colon V \times I \to M$ such that $h(-,1)$ and $h \circ (i_{W} \times \id)$ are both transverse to the cubulation.
	Let $V', W',T' \in PC^*_\Gamma(M)$ be $W$, $V$, and $T$ but with reference maps given respectively by $h(-,1)$, $h(-,1)i_W$, and $h(-,1)i_T$, where $i_W \colon W \to V$ and $i_T \colon T \to V$ are the boundary inclusion maps restricted to the components of $W$ and $T$, respectively.
	As $h \circ (i_{W} \times \id)$ is transverse to the cubulation, $W$ and $W'$ represent the same element of $H^*_{\Gamma \pf X}(M)$ by arguments analogous to the proof of \cref{C: homotopy}.
	But we also have $\bd V' = W'+T'$ with $V'$ in $C^*_{\Gamma \pf X}(M)$ and, by \cref{L: Q preservation}, $T' \in Q^*(M)$.
	So $W'$ represents $0 \in H^*(C^*_{\Gamma \pf X}(M))$.
\end{proof}

It remains to prove \cref{P: ball stability}, which will requires the following technical lemma that is also useful below in the proof of \cref{T: intersection qi}.

\begin{lemma}\label{L: minimizer}
	Let $M$ be a manifold without boundary, and let $\mc U = \{U_j\}$ be a locally finite open covering such that each $\bar U_j$ is compact.
	Suppose given $\varepsilon_j>0$ for each $j$.
	Then there exists a smooth function $\phi \colon M \to \R$ such that $0<\phi(x)<\varepsilon_j$ if $x \in \bar U_j$.
\end{lemma}

\begin{proof}
	Let $\eta_j = \min\{\varepsilon_k \mid \bar U_j \cap \bar U_k\neq \emptyset\}$.
	By the local finiteness and compactness conditions, $\{k \mid \bar U_j \cap \bar U_k\neq \emptyset\}$ is a finite sets and so the $\eta_j$ are well defined.
	Let $\{\psi_j\}$ be a partition of unity subordinate to $\mc U$ and let $\phi_1 = \sum \eta_j\psi_j$.
	For any $x \in M$, this sum is positive.
	If $x \in \bar U_j$ then $\phi_1(x) = \sum_{\{k \mid \bar U_j \cap \bar U_k\neq \emptyset\}} \eta_k\psi_k$.
	But for any such $k$, we have $\eta_k \leq \varepsilon_j$.
	Thus $\phi_1(x) \leq \varepsilon_j$.
	Now take $\phi = \frac{1}{2}\phi_1$.
\end{proof}

We can now prove \cref{P: ball stability}.
In the following $D^N$ is the open unit ball in $\R^N$ and, more generally, $D^N_r$ the open ball of radius $r$.

\begin{proof}[Proof of \cref{P: ball stability}]
	We begin with the case that $V$ is compact, and then we will show how to use the arguments of the compact case to obtain the general case.
	We first construct $F \colon M \times D^N \to M$, for some $N$, such that

	\begin{enumerate}
		\item $F(-,0) = \id \colon M \to M$,
		\item for almost all $s \in D^N$ the composition $V \xr{r_V} M \xr{F(-,s)}M$ is transverse to the cubulation,
		\item there is a ball neighborhood $D_r^N$ of $0$ in $D^N$ such that for all $s \in D_r^N$ the composition $W \xr{i_{W}} V \xr{r_V} M \xr{F(-,s)}M$ is transverse to the cubulation.
	\end{enumerate}

	This will suffice in the compact case as then we can let $s_0$ be any point in $D_r^N$ such that the composition $V \xr{r_V} M \xr{F(-,s_0)}M$ is transverse to the cubulation and define $h(-,t) = F(-,ts_0)r_V$, i.e.\ $h(x,t) = F(r_V(x),ts_0)$.
	The first property of the proposition holds since $F(-,0) = \id$.
	The second property holds by our choice of $s_0$.
	The last property then holds as $ts_0 \in D_r^N$ for all $t \in I$; thus each $h(-,ts_0)i_W$ is transverse to the cubulation, which then implies that $h \circ (i_{W} \times \id)$ is transverse to it as well.
	Furthermore, for $V$ compact any map and homotopy are proper, and this homotopy is universal as it can be decomposed into $r_V \colon V \times I \to M \times I$ and the homotopy $M \times I \to M$ taking $(z,t)$ to $F(z,ts_0)$.

	The construction of $F$ is a small variation of the construction in the Transversality Homotopy Theorem of \cite[Section 2.3]{GuPo74}:
	Let $M_\epsilon$ be an $\epsilon$-neighborhood of $M$ in some $\R^N$ in the sense of the $\epsilon$-Neighborhood Theorem of \cite[Section 2.3]{GuPo74}; in particular,
	$M_\epsilon$ is an $\epsilon$-neighborhood of a proper embedding of $M$ into $\R^N$ that possesses a submersion $\pi \colon M_\epsilon \to M$.
	If $M$ is not compact, then $\epsilon$ is a smooth positive function of $M$ and $M_\epsilon = \{z \in \R^N \mid |z-y|<\epsilon(y) \text{ for some }y \in M\}$.
	Let $f \colon M \times D^N \to M_\epsilon$ be given by $f(y, s) = y + \epsilon(y) s$; as $\epsilon(y)>0$, this is clearly a submersion (onto its image) at all points.
	We let $F \colon M \times D^N \to M$ be the composition $M \times D^N \xr{f}M_\epsilon \xr{\pi}M$.
	Furthermore, the map $H \colon V \times D^N \to M$ given by the composition $$V \times D^N \xr{r_V \times \id} M \times D^N \xr{F} M$$ as well as all the restrictions $H|_{S^k(V)}$
	are submersions.
	In particular, each $H|_{S^k(V)}$ is transverse to any submanifold of $M$, so it follows by the Transversality Theorem of \cite[Section 2.3]{GuPo74} that for any fixed submanifold $Z$ of $M$, each $H|_{S^k(V)}(-,s)$ is transverse to $Z$ for almost all $s \in D^N$.
	In particular, we may take $Z$ to be the interior of any cube $E$ (of any dimension) of the cubulation.
	However, there are countably many cubes in the cubulation of $M$ and finitely many manifolds $S^k(V)$.
	As the countable union of measure zero sets has measure zero, for almost all $s \in D^N$ we have all $H|_{S^k(V)}(-,s) = F(-,s)r_V|_{S^k(V)}$ transverse to all cubical faces.

	It is clear that $F(-,0) = \id_M$, so next we show that if $W$ is a union of boundary components of $V$ with $r_W = r_Vi_W \colon W \to M$ transverse to the cubulation then $F(-,s)r_Vi_W = H(-,s)i_W$ is transverse to the cubulation for all $s$ in some neighborhood $U$ of $0$ in $D^N$.
	It is here that we need to generalize the Stability Theorem of \cite[Section 1.6]{GuPo74}.
	As the Stability Theorem is not necessarily true when the manifolds involved are not closed submanifolds, compact, or controlled in some other way, it is more convenient here to work with the closed cubical faces of the cubulation and with $\bd^kV$ rather than $S^k(V)$.
	We recall that by \cref{L: simple trans}, to prove that two maps of manifolds with corners are transverse it is sufficient to show that their compositions with all pairs of boundary inclusions are naively transverse.

	Let $H_k$ denote the composition $H_k \colon \bd^kV \times D^N \xr{i_{\bd^kV} \times \id}V \times D^N \xr{r_V \times \id} M \times D^N \xr{F} M$.
	As $W$ consists of boundary components of $V$, we must consider the $H_k$, $k\geq 1$.
	We provide the details for $H_1$, the other cases being similar.
	For simplicity of notation, we also assume for the remainder of the proof that $W = \bd V$; in case $W$ is a union of only some of the components of $\bd V$, we can restrict $H_k$, $k>0$, in the following to just those components of $W$.

	Let $E$ be a (closed) face of the cubulation and let $x \in W$.
	As $r_W = H_1(-,0)$ is transverse to the cubulation, for any $x \in W$, either $r_W(x)\notin E$ or $r_W$ is transverse to $E$ at $r_W(x)$.
	In the former case, as $E$ is closed, there is an open neighborhood $A_x$ of $(x,0) \in W \times D^N$ such that $H_1(A_x) \cap E = \emptyset$.
	Now suppose that $r_W(x) \in E$ and is transverse there.
	By appealing to charts, we can suppose without loss of generality (as least locally in a neighborhood of $r_W(x)$) that $M = \R^m$ with $m = \dim(M)$ and $r_W(x) = 0$ and that $E = \{(y_1,\ldots,y_m) \mid y_i\geq 0\text{ for } i \leq \dim E\text{ and } y_i = 0 \text{ for } i>\dim(E)\}$.
	The transversality assumption means that the composition of $D_xr_W \colon T_xW \to T_{r_W(x)}M$ with the projection to the last $m-\dim(E)$ coordinates is a linear surjection.
	As this is an open condition on the Jacobian matrix of $r_W$ at $x$, it follows again that there is an open neighborhood $A_x$ of $(x,0) \in W \times D^N$ such that for each $(x',s)$ in the neighborhood $H_1(-,s)$ is transverse to $E$ at $x'$ (it is possible that $H_1(x',s)$ no longer intersects $E$, but this is fine).
	Taking the union of the $A_x$ over all $x \in W$ gives a neighborhood $B_E$ of $W \times 0$ in $W \times D^N$, and by the Tube Lemma, as $W$ is compact there is a neighborhood of $W \times 0$ of the form $W \times U_E \subset B_E$.
	For each $s \in U_E$, we have $H_1(-,s)$ transverse to $E$.
	Now let $D^N_{1/2}$ be the open ball of radius $1/2$ and $\bar D^N_{1/2}$ its closure.
	As $W \times \bar D^N_{1/2}$ is compact, its image under $H_1$ can intersect only a finite number of faces of the cubulation of $M$; call this collection $\mc E$.
	Then let $U_1$ be the finite intersection $U_1 = D^N_{1/2} \cap \bigcap_{E\in\mc E} U_E$.
	Then $W \times U_1 \subset W \times D^N$ is a neighborhood of $W \times 0$ on which $H_1(-,s)$ is transverse to every cubical face that its image intersects.
	Let $U_k$ be defined similarly for each $k\geq 1$.
	As $W$ has finite depth, $U = \cap U_k$ is a neighborhood of $0$ in $D^N$.
	Let $r>0$ be such that $D^N_r \subset U$.
	Then for every $s \in D^N_r$ we have $H_k(-,s) \colon \bd^kV \to M$ transverse to all $E$ for all $k$ as required.

	This completes the proof of the proposition for $V$ compact.

	Next suppose that $V$ is no longer necessarily compact.
	We show how to apply and extend the preceding arguments.
	We will define a new homotopy $\hat h \colon V \times I \to M$ with the desired three properties of the proposition.

	To begin we will construct $F \colon M \times I \to M$ exactly as above, as its definition did not depend on the compactness of $V$.
	For $V$ not compact, the first two properties listed above for $F$ will continue to hold, but the third relied on compactness and so will not hold any long in general.
	However, let
	$K \subset W$ be compact and $E$ be a closed cube.
	Taking the union of the $A_x$ over all $x \in K$ and intersecting with $K \times D^N$ gives an open neighborhood $B_E$ of $K \times 0$ in $K \times D^N$, such that
	$H_1(-,s)$ is transverse to $E$ at all $x \in K$.
	Furthermore, as $K \times \bar D^N_{1/2}$ is compact, its image under $H_1$ can intersect only a finite number of faces of the cubulation of $M$, so again by applying the tube lemma and then intersecting tubular neighborhoods, we find an open ball $D_{r,K}^N \subset D^N$ centered at $0$ such that for all $x \in K$ and $s \in D_{r,K}^N$ we have $H_1(-,s)$ transverse to the cubulation at $x$.
	Furthermore, as the inclusions $\bd^kV \into W$ for $k\geq 1$ are all proper, we can similarly find $D_{r,K}^N$ so that for all $s \in D_{r,K}^N$ and all $k\geq 1$, we have $H_k(-,s) \colon \bd^kV \to M$ transverse to the cubulation for any $x \in \bd^{k}V$ whose image in $W$ is in $K$.

	Let $\{\mc U_j\}$ be a locally finite covering of $M$ such that each $\bar{\mc U_j}$ is compact.
	As $r_V$ and $r_W = r_V \circ i_{W}$ are proper, each $r^{-1}_W(\bar {\mc U_j})$ is compact in $W$.
	Proceeding as just above with $r_W^{-1}(\bar U_j)$ in place of $W$, we can find for each $j$ an $\varepsilon_{j,1} \leq 1$ so that for every $s \in D^N_{\varepsilon_{j,1}}$ we have $H_1(-,s)$ transverse to all cubical faces of $M$ at every $x \in r^{-1}_W(\bar {\mc U_j})$.
	Analogously, we have $\varepsilon_{j,k}$ for all $k\geq 1$ using $(r_Vi_{\bd^kV})^{-1}(\bar {\mc U_j})$.
	Let $\varepsilon_j = \min\{\varepsilon_{j,k} \mid k\geq 1\}$.
	These minima exist as $V$ has finite depth.

	Now, using \cref{L: minimizer}, we choose a smooth function $\phi \colon M \to \R$ such that for all $x \in M$ we have $0<\phi(x)<\epsilon_j$ if $x \in \bar{\mc U_j}$.
	Let $M\times_\phi D^N = \{(y,s) \in M \times D^N \mid |s|<\phi(y)\}$.
	By our construction, $H_k(-,s) \colon \bd^{k-1}W \to M$ is transverse to the cubulation at each $x$ such that $(x,s)\in(r_Wi_{\bd^{k-1}W} \times \id)^{-1}(M\times_\phi D^N) = \{(x,s) \in \bd^{k-1}W \times I \mid |s|<\phi(r_Wi_{\bd^kW}(x))\}$.
	Unfortunately, however, while $M\times_\phi D^N$ is a neighborhood of $M \times 0$ in $M \times D^N$, there is not necessarily a $U \in D^M$ so that $M \times U \subset M\times_\phi D^M$.
	Thus,
	we cannot construct $h$ from $F$ as above using a fixed $s_0$ as there may be no single $s_0\neq 0$ so that $W \times s_0 \subset M\times_\phi D^N$.

	To account for this, we modify our functions above as follows: Let $\hat f \colon M \times D^N \to M_\epsilon$ be given by $\hat f(y, s) = y +\phi(y) \epsilon(y) s$; as $\phi(y)\epsilon(y)>0$, this is again a submersion onto its image at all points.
	Let $\hat F \colon M \times D^N \to M$ be the composition $M \times D^N \xr{\hat f}M_\epsilon \xr{\pi}M$, and let $\hat H_k$ be the composition $\bd^kV \times D^N \xr{i_{\bd^kV} \times \id}V \times D^N \xr{r_V \times \id} M \times D^N \xr{\hat F} M$ for $k\geq 0$.
	Once again by the Transversality Theorem of \cite[Section 2.3]{GuPo74}, for almost all $s \in D^N$ we have $\hat H_k(-,s)$ transverse to all cubical faces for all $k\geq 0$.
	Letting $s_0$ be any such point we define $\hat h \colon V \times I \to M$ to be $\hat h(x,t) = \hat H(x,ts_0)$, and we claim that this $\hat h$ satisfies the conditions of the proposition.

	The map $\hat h$ is proper, and the first two conditions of the proposition follow immediately from the construction.
	It remains to verify that $\hat h (i_W \times \id) \colon W \times I \to M$ is transverse to the cubulation, and similarly for the higher boundaries.
	Note: we are not claiming an analogue of the third condition above holds for $\hat F$, nor do we need to.
	As we already know from the second condition of the proposition that $\hat h(-,1)$ is transverse to the cubulation and from the hypotheses that $\hat h(-,0)$ is transverse to the cubulation on $W$, it suffices to demonstrated transversality to the cubulation on the restriction of $\hat h$ to $W \times (0,1)$.

	We begin by observing that for $(x,t) \in W \times I$ we can write $\hat h(x,t) \circ (i_W \times \id_I) \colon W \times I \to M$ explicitly as
	$$\hat h(x,t) = \pi(r_W(x)+\phi(r_W(x))\epsilon(r_W(x))ts_0).$$
	So, alternatively, we can observe that $\hat h(x,t) \circ (i_W \times \id_I)$ is the composition
	\begin{equation}\label{E: alt hat h}
		W \times I \xr{\Phi} W \times I \xhookrightarrow{\Psi} W \times D^N \xr{r_W \times \id} M \times D^N \xr{F} M,
	\end{equation}
	with $\Phi(x,t) = (x,\phi(r_W(x))t)$, $\Psi(x,t) = (x,ts_0)$, and noting that on the right we do mean our original $F$ and not $\hat F$.

	The first map $\Phi$ is a diffeomorphism onto its image, which is a neighborhood of $W \times 0$ in $W \times I$, and the map $\Psi$ embeds this linearly into $W \times D^N$.
	The composition of the last two maps is just our earlier map $H_1$.
	By construction, the map $r_W \times \id$ now takes the image of $\Psi\Phi$ into $M\times_\phi D^N$, and so at each point $(z,s)$ in the image of $\Psi\Phi$ if we fix $s$ and consider $H_1(-,s)$ we get by construction a map on $W$ that is transverse to the cubulation.
	But as $\Phi$ is a diffeomorphism onto its image and $\Psi$ is an embedding that is the identity with respect to $W$, we see $\Psi\Phi$ takes a neighborhood of any $(x,t) \in W \times (0,1)$ to a neighborhood of its image in $W \times \R s_0$, where $\R s_0$ is the line in $\R^N$ spanned by $s_0$.
	In particular, the derivative of $\Psi\Phi$ maps the tangent space to $W \times (0,1)$ at $(x,t)$ onto $ T_xW \times \R s_0 \subset T_{\Psi\Phi(x,t)}(W \times D^N)$.
	In particular, this image contains $T_xW \times 0$, and by construction $DH_1$ takes this tangent space to a tangent subspace in $M$ at $\hat h(x,t)$ that is transverse to the tangent space there of any face of the cubulation containing $\hat h(x,t)$.
	The same holds for $k>1$ replacing $W$ in with $\bd^{k-1}W$ in \eqref{E: alt hat h} and $r_W$ with $r_{\bd^{k-1}W}$.
	So we see that $\hat h$ satisfies all the requirements of the proposition.
\end{proof}

\subsection{The intersection map and the isomorphism between cubical and geometric cohomology}\label{S: intersection map}

To define the intersection map, we introduce an augmentation map as in singular homology theory.
For this we first need a quick lemma.

\begin{lemma}\label{L: Q0}
	If $W \in Q_0(M)$, then $W$ has the same number of positively and negatively oriented points.
\end{lemma}

\begin{proof}
	As elements of $PC_0^\Gamma(M)$ cannot be degenerate, if $W \in Q_0$ then $W$ must be trivial, and so there is an orientation-reversing diffeomorphism $\rho$ of $W$ such that $r_W\rho = r_W$.
	But a compact $0$-manifold has an orientation-reversing diffeomorphism if and only there are the same number of points with each orientation.
\end{proof}

\begin{definition}\label{D: aug}
	We define the \textbf{augmentation map} $\aug \colon PC^\Gamma_0(M) \to \Z$ as follows: If $W \in PC^\Gamma_0(M)$ then $W$ is the disjoint union of a finite number of points, each with orientation denoted $1$ or $-1$.
	We let $\aug(W)$ be the sum of the orientations of the points in $W$, interpreting $1$ and $-1$ as integers.
	By \cref{L: Q0}, an element of $PC^\Gamma_0(M)$ can be in $Q_0(M)$ only if this sum is $0$, so the augmentation descends to a homomorphism $\aug \colon C^\Gamma_0(M) \to \Z$.
	Furthermore, if $W \in PC_1^\Gamma(M)$ then, as usual, $\aug(\bd W) = 0$, so $\aug$ further descends to a homomorphism $\aug \colon H_0^\Gamma(M) \to \Z$.
\end{definition}

Later, we will construct in general a partially-defined intersection map $C^*_\Gamma(M) \otimes C_*^\Gamma(M) \to C_*^\Gamma(M)$.
In general, this is delicate as geometric chains and cochains do not have fixed representatives.
However, at the moment we do not need this full generality to define the intersection map we will need to compare geometric cohomology and cubical cohomology.
This is reflected in the following more limited definition:

\begin{definition}\label{D: intersection number}
	Suppose $M$ is an $m$-manifold without boundary and $W \in PC_\Gamma^i(M)$ and $N \in PC_{i}^\Gamma(M)$ are transverse.
	We define the \textit{intersection number} $I_M(W,N)$ (or simply $I(W,N)$ if $M$ is clear from context) by $$I_M(W,N) = \aug(W \times_M N),$$ with $W \times_M N$ as defined in \cref{D: PC products}.
\end{definition}

We observe that this definition makes sense as $W$ and $N$ are transverse with complementary dimensions and $W \times_M N$ is an element of $PC_0^\Gamma(M)$.
In fact, in this case in order for transversality to hold the maps $r_W \colon W \to M$ and $r_N \colon N \to M$ must have full rank at each $x \in W$ and $y \in N$ such that $r_W(x) = r_N(y)$.
As having full rank is an open condition, the maps will also have full rank on neighborhoods of these points.
In particular, by the Implicit Function Theorem, they must be immersions on neighborhoods of these points.
So, locally, the orientation of $N$ determines an orientation of $T_yN$, which we can consider to be a subspace of $T_{y}M$, slightly abusing notation to identify $y$ and $r_N(y)$ via the local immersion.
Furthermore, in a neighborhood of $x$ the co-orientation of $r_W$ determines an orientation of the normal bundle of the local immersion of $W$, and we can take the fiber of the normal bundle at $r_W(x)$ to be $T_yN$.

\begin{lemma}\label{L: intersection number}
	The intersection number $I_M(W,N)$ is equal to signed count of intersection points of $W$ and $N$, counting an intersection point with $+1$ if the normal co-orientation of $W$ agrees with the orientation of $N$ and $-1$ otherwise.
\end{lemma}

\begin{proof}
	This follows directly from \cref{C: complementary cap}.
	\begin{comment}
		We first recall the construction of the pullback orientation $W \times_M N \to N$.
		As $W$ and $N$ are immersed near their geometric intersections, we can restrict to these immersed regions of $W$ and $N$ and so take the dimension of the Euclidean factor to be $0$ in Definition \ref{D: pullback coorient}.
		So then $\nu W$ is the oriented normal bundle of $W$ determined by the co-orientation in the immersed region, and we pull this back to be a normal bundle of $W \times_M N$ in $N$.
		In this simplified situation, Definition \ref{D: pullback coorient} tells us that the co-orientation of the pullback is the normal co-orientation corresponding to this pullback bundle, which is just the restriction of the normal bundle to the intersection point.
		So the co-orientation at each intersection point can be written as $(1,1 \wedge \beta_{\nu W}) = (1, \beta_{\nu W})$.
		So now by the discussion following Definition \ref{D: co-orientations}, we orient each point of the pullback by $1$ if $\beta_{\nu W}$ agrees with the orientation of $N$ and $-1$ otherwise.
		The lemma follows.
	\end{comment}
\end{proof}

\begin{lemma}\label{L: Q-trivial intersection}
	Suppose $W \in PC_\Gamma^i(M)$ and $N \in PC_{i}^\Gamma(M)$ are transverse and that $W \in Q^i(M)$.
	Then $I(W,N) = 0$.
\end{lemma}

\begin{proof}
	By \cref{L: pullback with Q}, we know $W \times_M N \in Q_0(M)$, so $I(W,N) = \aug(W \times_M N) = 0$ by definition and \cref{L: Q0}.
\end{proof}

\begin{definition}\label{D: intersection homomorphism}
	Given the manifold without boundary $M$ cubulated by $X$, we define the \textbf{intersection map} $\mc I \colon C^*_{\Gamma \pf X}(M) \to K^*(X)$ by $$\mc I(\uW) = \sum_F I_M(W,F)F^*,$$ where the sum is taken over faces $F$ of the cubulation $X$ such that $\dim(F)+\dim(W) = \dim(M)$ and the $W$ on the right hand side is any element of $PC^*_{\Gamma \pf X}(M)$ representing $\uW$.

	In particular, for a face $F$ of dimension $\dim(M)-\dim(W)$, we have $$\mc I(\uW)(F) = I_M(W,F) = \aug(W \times_M F).$$
\end{definition}

\begin{proposition}
	The intersection map $\mc I$ is a well-defined chain map.
\end{proposition}

\begin{proof}
	If $W, W' \in PC^*_{\Gamma \pf X}(M)$ are two representatives of $\uW$ then $W \sqcup -W' \in Q^*(M)$ and it is transverse to $X$.
	So for any face $F$ we have $\aug(W \times_M F)-\aug(W' \times_M F) = \aug((W \sqcup -W') \times_M F) = 0$ by \cref{L: Q-trivial intersection}.
	So $\mc I(\uW)$ does not depend on the choice of $W$.

	To see that $\mc I$ is a chain map, let $W$ be transverse to the cubulation and representing $\uW$.
	Then we compute for any face $f$ of $X$ that
	\begin{align*}
		\mc I(\bd\uW)(f)& = \aug((\bd W) \times_M f)\\
		& = \aug(W \times_M \bd f)\\
		& = \mc I(\uW)(\bd f).
	\end{align*}
	For the second equality, we use \cref{P: Leibniz cap} together with the facts that the augmentations are both trivial unless $\dim(W \times_M f) = 1$ and that $\aug(\bd (W \times_M F)) = 0$.
\end{proof}

\begin{example}\label{E: coho 0 generator}
	Let $M$ be any connected manifold without boundary and given a cubulation $X$, let $\uW \in C^*_{\Gamma \pf X}(M)$ be represented by the tautologically co-oriented identity map $M \to M$, and let $v$ be a (positively-oriented) $0$-dimensional vertex of $X$.
	By \cref{P: cap with 1}, we have $M \times_M v = v$, and so
	\[\mc I(\uW)(v) = \aug(M \times_M v) = \aug(v) = 1.\]
	In other words, $\mc I(\uW)$ takes the value $1$ on each vertex of $X$, so it is the standard generator of $H^0(K^*_X(M)) \cong \Z$.
	It will therefore follow from \cref{T: intersection qi}, below, that the tautologically co-oriented identity map $M \to M$ generates $H^0_\Gamma(M) \cong \Z$, as promised in \cref{E: first examples}.
\end{example}

Our goal now is to show that the intersection map $\mc I$ induces an isomorphism $H^i_{\Gamma \pf X}(M) \to H^i(K^*(M))$ whenever $H^i_\Gamma(M)$ and $H^i(K^*(M))$ are finitely generated.
Recall that we already know these groups are abstractly isomorphic by \cref{T: transverse complex,T: geometric is singular} and the footnote on page \pageref{FN: cubical and singular}.
We begin in the next section by using the cubical structure to start building an inverse map.

\subsubsection{Dualization in cubes}\label{S: dual cubes}

Analogous to barycentric subdivisions of simplices, we will need to consider standard subdivisions of cubes.
For this we let $\jinterval$ denote the interval $\interval = [0,1]$ thought of as the (non-disjoint) union $[0,1/2] \cup [1/2,1]$.
We can then write $\jinterval^n = \left([0,1/2] \cup [1/2,1]\right)^n$, with the idea being that we consider $\interval^n$ as the union of $2^n$ \textit{subcubes} of side length $1/2$, each of which is the product within $\interval^n$ of $n$ factors, each factor is equal to either $[0,1/2]$ or $[1/2,1]$.
We refer to $\jinterval^n$ with this structure as the \textbf{central subdivision} of $\interval^n$.

Analogously to the case with $\interval^n$, each cube $S$ of $\jinterval^n$ possesses faces (some of which it shares with other $n$-cubes) consisting of the subsets of $S$ in which some variables have been bound to the values $0$, $1$, or $1/2$.
In general we refer to such faces as \textbf{faces of $\jinterval^n$}.
If no variable of such a face is bound to $0$ or $1$, we say that we have an \textbf{internal face of $\jinterval^n$}, otherwise we call it an \textbf{external} face.
External faces are all subsets of $\bd \interval^n$; internal faces are not subsets of $\bd \interval^n$.

To each face $F$ of $\interval^n$, we refer to the point $\hat F$ at which all its free variables are equal to $1/2$ as the \textbf{center} of $F$.
Each vertex is its own center.
To each face $F$ of $\interval^n$ we define its \textbf{dual face in $\interval^n$}, or simply its \textbf{dual}, to be the face $F^\vee$ of $\jinterval^n$ determined as follows:
\begin{itemize}
	\item If $i \in F_{01}$ (i.e.\ the coordinate $x_i$ is free in $F$), then $x_i = 1/2$ in $F^\vee$.

	\item If $i \in F_0$ (i.e.\ the coordinate $x_i$ is bound to $0$ in $F$), then $x_i$ is free in $[0,1/2]$ in $F^\vee$.

	\item If $i \in F_1$ (i.e.\ the coordinate $x_i$ is bound to $1$ in $F$), then $x_i$ is free in $[1/2,1]$ in $F^\vee$.
\end{itemize}

It is clear that $F$ and $F^\vee$ have complementary dimensions and that they intersect naively transversely in the center of $F$.

\begin{lemma}
	The set function $F \to F^\vee$ is a bijection between the faces of $\interval^n$ and the internal faces of $\jinterval^n$.
\end{lemma}

\begin{proof}
	Injectivity is clear as two different faces of $\interval^n$ will have different partitions of $\bar n$.

	Next consider an internal face of $\jinterval^n$.
	By definition this is a set in which some set of variables $A$ has been bound to $1/2$ while two other sets of variables, $B$ and $C$, are free on $[0,1/2]$ or $[1/2,1]$, respectively.
	But this is $F^\vee$ for the face $F$ with partition $(B,A,C)$.
	So our function is surjective.
\end{proof}

As each face $F$ of $\interval^n$ carries a natural orientation $\beta_F$ determined by the order of its free variables (or the orientation $1$ for vertices), this provides $F^\vee$ with the corresponding normal co-orientation.
In other words, $F^\vee$ is co-oriented at all points by the co-orientation $(\beta_{F^\vee}, \beta_{F^\vee} \wedge \beta_F)$, where $\beta_{F^\vee}$ is an arbitrary orientation of $F^\vee$.
Assigning $F^\vee$ this co-orientation, we interpret its embedding in $\interval^n$ as representing a cochain in $\interval^n$ of index $\dim(F)$.
We define a map $\Psi \colon K^*(\interval^n) \to PC_\Gamma^*(\interval)$ by $\Psi(F^*) = F^\vee$.

We introduce one more piece of notation for the following lemma.
If $f$ is a face in $\jinterval^n$ co-oriented and considered as an element of $PC^*_\Gamma(\interval^n)$, we let $\bd_{\text{int}}f$ denote the union of the internal faces of $\bd f$ and $\bd_{\text{ext}}f$ denote the union of the external faces of $f$.
If $f$ is co-oriented, then we interpret the terms of $\bd_{\text{int}}f$ and $\bd_{\text{out}}f$ as co-oriented with the boundary co-orientations.
We extend $\bd_{\text{int}}$ to a linear operator in the obvious way.

\begin{lemma}\label{L: dualizing bijection}
	For any face $F$ of $\interval^n$, we have $$\Psi(d F^*) = \bd_{\text{int}}\Psi(F^*) = \bd_{\text{int}}F^\vee.$$
\end{lemma}

\begin{proof}
	We first show that there is a bijection between the interior faces of $\Psi(d F^*)$ and the set of $f^\vee$ such that the corresponding $f^*$ have non-zero coefficients in $dF^*$.
	Then we will return to carefully consider the co-orientations.

	So let $F$ be a face of $\interval^n$.
	Recall that $F$ is determined by the partition $(F_0,F_{01}, F_1)$ of $\overline{n}$ corresponding respectively to variables set to $0$, free variables of $F$, and variables set to $1$.
	By definition, $(dF^*)(f) = F^*(\bd f)$ and so the only faces participating in $dF^*$ are those that have $F$ as a boundary, in other words those faces whose free variables are those in $F_{01}$ plus one more variable from $F_0$ or $F_1$.

	Now let us consider again $F^\vee$, which we recall is obtained by setting all variables in $F_{01}$ to $1/2$ and letting the variables in $F_0$ and $F_1$ become free variables on $[0,1/2]$ or $[1/2,1]$ respectively.
	Each boundary face of $F^\vee$ is then obtained by either setting one of the variables in $F_0$ to $0$ or $1/2$ or one of the variables in $F_1$ to $1/2$ or $1$.
	Furthermore, the internal faces are those where the variable in $F_0$ or $F_1$ has been set to $1/2$.
	So, in summary, an internal face of $F^\vee$ has all of the variables in $F_{01}$ as well as exactly one other variable set to $1/2$ and the rest remain free over the appropriate domains, namely $[0,1/2]$ for those in $F_0$ and $[1/2,1]$ for those in $F_1$.
	But, from the definition of dualization $f \to f^\vee$, this exactly describes the duals $f^\vee$ of the faces participating in $dF^*$, which is sufficient due to the bijection of \cref{L: dualizing bijection}.

	It remains now to consider the signs.
	We recall that the boundary formula for cubes has the form
	$$\bd f = \sum_{i = 1}^k (-1)^i(f \delta_i^0-f \delta^1_i),$$ where the $\delta$s denote the embeddings of the faces.
	In $K_*(\interval^n)$, we can shorten this notation to $$\bd f = \sum_{i = 1}^k (-1)^i(f_i^0-f^1_i),$$ letting $f_i^j$, $j \in \{0,1\}$, denote the $i$th ``front or back face'' according to $j = 1$ or $j = 0$.
	In particular, we note that there are two factors, both $i$ and $j$, affecting sign.

	So now let us again fix a face $F$ of $\interval^n$ and let $f$ be a face of dimension $\dim(F)+1$ that includes $F$ in its boundary as $F = f_i^j$.
	Representing $f$ as $f = (f_0,f_{01},f_1)$, we obtain $F$ by setting the $i$th variable of $f_{01}$ to $j$.
	From the coboundary formula, we have
	\begin{equation*}
		(dF^*)(f) = F^*(\bd f) = (f_i^j)^*\left(\sum_{i = 1}^k (-1)^i(f_i^0-f^1_i)\right)
		= (f_i^j)^*((-1)^{i+j}f_i^j) = (-1)^{i+j}.
	\end{equation*}
	So
	$f^*$ occurs in $dF^*$ with coefficient $(-1)^{i+j}$.
	Thus we must show that the coefficient of $f^\vee$ in $\bd_{\text{int}} (F^\vee)$ is $(-1)^{i+j}$.

	Now consider $F^\vee$.
	We can write its normal co-orientation as $(\beta_{F_0 \cup F_1},\beta_{F_0 \cup F_1} \wedge \beta_{F_{01}})$.
	Let $k \in F_0 \cup F_1$ be the unique index that is free in $f$ but bound in $F$ and so also free in $F^\vee$.
	Then $f^\vee$ is the boundary component of $F^\vee$ with $x_k$ set to $1/2$.
	In this case the boundary co-orientation of $f^\vee$ in $F^\vee$ is $(\beta_{F_0 \cup F_1 \setminus \{k\}},\beta_{F_0 \cup F_1 \setminus \{k\}} \wedge (-1)^{j+1} \beta_{e_k})$, where $e_k$ is the unit vector in the direction of increasing $x_k$.
	The sign $(-1)^{j+1}$ is because if $j = 1$ then in $F^\vee$ the variable $x_k$ is free on $[1/2,1]$ so the inward normal vector at $1/2$ points in the direction of $e_k$, and the opposite if $j = 0$.
	So the boundary co-orientation for $f^\vee$ in $\interval^n$ as a piece of $\bd F^\vee$ is the composition $$(\beta_{F_0 \cup F_1 \setminus \{k\}},\beta_{F_0 \cup F_1 \setminus \{k\}} \wedge (-1)^{j+1} \beta_{e_k})*(\beta_{F_0 \cup F_1},\beta_{F_0 \cup F_1} \wedge \beta_{F_{01}}).$$

	Meanwhile, the natural co-orientation for $f^\vee$ is $$(\beta_{f_0 \cup f_1},\beta_{f_0 \cup f_1} \wedge \beta_{f_{01}}).$$ But now we observe that $f_0 \cup f_1 = F_0 \cup F_1 \setminus \{k\}$.
	So
	\begin{align*}
		(\beta_{F_0 \cup F_1 \setminus \{k\}},\beta_{F_0 \cup F_1 \setminus \{k\}} \wedge (-1)^{j+1} \beta_{e_k})&*(\beta_{F_0 \cup F_1},\beta_{F_0 \cup F_1} \wedge \beta_{F_{01}})\\
		& = (-1)^{j+1}(\beta_{f_0 \cup f_1},\beta_{f_0 \cup f_1} \wedge \beta_{e_k})*(\beta_{f_0 \cup f_1} \wedge \beta_{e_k},\beta_{f_0 \cup f_1} \wedge \beta_{e_k} \wedge \beta_{F_{01}})\\
		& = (-1)^{j+1}(\beta_{f_0 \cup f_1},\beta_{f_0 \cup f_1} \wedge \beta_{e_k} \wedge \beta_{F_{01}})\\
		& = (-1)^{j+i}(\beta_{f_0 \cup f_1},\beta_{f_0 \cup f_1} \wedge \beta_{f_{01}})\\
	\end{align*}
	In the second express after the first equal sign we have used that the expression for the co-orientation of $F^\vee$ is independent of the choice of local orientation for $F^\vee$.
	So we replace $\beta_{F_0 \cup F_1}$ with $\beta_{f_0 \cup f_1} \wedge \beta_{e_k}$.
	But we know that $k$ is the $i$th variable of $f_{01}$, so
	$\beta_{e_k} \wedge \beta_{F_{01}} = (-1)^{i-1}\beta_{f_{01}}$, which we use in the last equality.
\end{proof}

\begin{lemma}\label{L: ext faces}
	Suppose for any $n-1$ face $E$ of $\interval^n$ with $F<E$ we let $F_E^\vee$ denote the dual of $F$ in $E$ (ignoring co-orientation).
	Then the exterior faces $\bd_{\text{ext}}F^\vee$ of $F^\vee$ correspond exactly to the $F_E^\vee$ as $E$ ranges over the $n-1$ faces of $\interval^n$ containing $F$.
	Furthermore, if $F^\vee$ is given the normal co-orientation induced by $F$ as above, then the co-orientation of $F^\vee_E \to F^\vee \to \interval^n$ as a piece of the boundary of $F^\vee$ is given by $(\beta_{F_E^\vee},\beta_{F_E^\vee} \wedge \beta_v \wedge \beta_F)$, where $\beta_{F_E^\vee}$ is any arbitrary orientation for $F_E^\vee$, $v$ is an inward pointing normal at $E$, and $\beta_F$ is the orientation of $F$.
\end{lemma}

\begin{proof}
	The $n-1$ face $E$ contains $F$ if and only if there is an $i \in F_0$ such that $i \in E_0$ or an $i \in F_1$ such that $i \in E_1$.
	Now from the definitions, a face $f$ of $F^\vee$ is an external face if and only if there is an $i \in F_0$ such that $x_i$ is bound to $0$ in $f$ or an $i \in F_1$ such that $x_i$ is bound to $1$ in $f$.
	Thus any $f$ in $\bd_{ext}F^\vee$ must be contained in an $E$ that contains $F$.
	Conversely, if $F<E$ and $i \in F_0$ with $x_i = 0$ defining $E$, then there is an external face of $F^\vee$ for which $x_i = 0$, $x_k = 1/2$ for all $k \in F_{01}$, and all other variables are free.
	Similarly, if $F<E$ and $i \in F_1$ with $x_i = 1$ defining $E$, then there is an external face of $F^\vee$ for which $x_i = 1$, $x_k = 1/2$ for all $k \in F_{01}$, and all other variables are free.
	So we have a bijection between external faces of $\bd F^\vee$ and the $n-1$ faces $E$ containing $F$.
	Furthermore, after we have set $x_i$ to $0$ or $1$ as appropriate, the behavior in the remaining variables shows that such a face has precisely the form of $F^\vee_E$.

	It remains to consider the co-orientations.
	Suppose $f$ is an external face of $F^\vee$ in the $n-1$ face $E$, i.e.\ $f = F^\vee_E$.
	The co-orientation of $F^\vee$ in $\interval^n$ and the co-orientation of $f$ in $E$ are both the normal co-orientations determined by the orientation of $F$.
	As $f$ is part of $\bd F^\vee$, its orientation in $\interval^n$ is therefore the composite $$(\beta_f,\beta_f \wedge \beta_v)*(\beta_f \wedge \beta_v,\beta_f \wedge \beta_v \wedge \beta_F) = (\beta_f,\beta_f \wedge \beta_v \wedge \beta_F),$$ where $\beta_f$ is any arbitrary orientation for $f$, $v$ is the inward pointing normal of $f$ in $F^\vee$, which in this case is also an outward pointing normal of $E$ in $\interval^n$, and we use $\beta_f \wedge \beta_v$ as a convenient orientation for $F^\vee$.
\end{proof}

\subsubsection{Dualization in cubical complexes}

Now suppose that $X$ is any cubical complex.
We obtain the \textbf{centrally subdivided cubical complex $sd(X)$} by replacing each cube $\interval^n$ with $\jinterval^n$.
Now suppose that $M$ is an $n$-manifold without boundary cubulated by the cubical complex $X$ via $\phi \colon |X| \to M$.
For a given face $F$ of an $n$-cube $B$ of $X$, we can extend the above definitions, and abuse notation, by letting $F$ refer also to the image of $F$ in $M$ and letting $F_B^\vee$ also denote the composition $F_B^\vee \into \interval^n \into |X| \xr{\phi} M$, where $F_B^\vee$ is the dual of $F$ in $B$.
Similarly, we consider this version of $F_B^\vee$ in $M$ to be co-oriented by the normal co-orientation obtained from the orientation of $F$.
Moreover, in the cubulated $M$ we extend our earlier definition and write $F^\vee = \sqcup_{F<B} F^\vee_B$, where the union is taken over all $n$-cubes $B$ having $F$ as a face.
We similarly extend $\Psi$ so that $\Psi(F^*) = F^\vee = \sum_{F<B} F_{B}^\vee$.

\begin{lemma}\label{L: dual chain map}
	$\Psi$ gives a chain map $K_X^*(M) \to C_\Gamma^*(M)$.
\end{lemma}

\begin{proof}
	We will show that for any face $F$ of the cubulation $X$ of $M$ we have $\Psi(\bd F^*) = \bd\Psi(F^*)$.

	By \cref{L: dualizing bijection} and the definition of the extended $\Psi$, we know that $\Psi(dF^*) = \sum_{F<B} \bd_{\text{int}}F^\vee_B$, so we must only show that $\sum_{F<B} \bd (F^\vee_B) = \sum_{F<B} \bd_{\text{int}}(F^\vee_B)$, in other words that all of the external faces of the terms of $\Psi(dF^*)$ cancel in the sum $\sum_{F<B} \bd (F^\vee_B)$.

	For this, let us fix an $n$-cube $B$ containing $F$, and let $E$ be an $n-1$ face of $B$ containing $F$.
	As $M$ is an $n$-manifold, there is exactly one other $n$-cube, say $B_1$, in the cubulation that shares the face $E$ with $B$.
	Furthermore, by \cref{L: ext faces}, $E$ contains external faces of $\bd(F^\vee_B)$ and $\bd(F^\vee_{B_1})$ and they both correspond geometrically to $F^\vee_E$.
	Also by the lemma, the co-orientation as a boundary piece of $F^\vee_B$ is $(\beta_{F_E^\vee},\beta_{F_E^\vee} \wedge \beta_{v_B} \wedge \beta_F)$, where $v_B$ is an inward pointing vector of $B$ at $E$, and the co-orientation as a boundary piece of $F^\vee_{B_1}$ is $(\beta_{F_E^\vee},\beta_{F_E^\vee} \wedge \beta_{v_{B_1}} \wedge \beta_F)$, where $v_{B_1}$ is an inward pointing vector of $B_1$ at $E$.
	Since a normal vector to $E$ that is inward pointing with respect to $B_1$ is outward pointing with respect to $B$, and vice versa, these are opposite co-orientations, so these two boundary pieces cancel in $\bd \Psi(F^*)$.
\end{proof}

Morally speaking, $\Psi$ will be our inverse to the intersection map $\mc I$, but unfortunately the elements of $PC^*_\Gamma(M)$ that represent the image of $\Psi$ are not transverse to the cubulation.
So we need another step.

\subsubsection{Pushing the dual cubulation}

To remedy the problem with the image of $\Psi$ consisting of elements of $PC^*_\Gamma(M)$ that are not transverse to the cubulation, we construct a homotopy of $M$ to itself that pushes these cochains into transverse position with respect to the cubulation, which we regard as fixed.
Moreover, we want to do so in a way such that the intersection number of the pushed $F^\vee$ with $F$ will be $1$, as expected.
Constructing such a homotopy is the purpose of the following technical lemma.
As we will see in the proof, the actual construction is a bit fiddly in order to ensure all the needed properties.
The payoff is in \cref{T: intersection qi}, which can be found in the next subsection.

\begin{lemma}
	Suppose $M$ is a manifold without boundary with a cubulation $X$.
	Then there is a smooth map $h \colon M \to M$ such that for each face $F$ of $X$ the following hold:
	\begin{enumerate}
		\item $h|_{F^\vee} \colon F^\vee \to M$ is transverse to $X$
		\item the only $\dim(F)$-face of $X$ that intersects $h(F^\vee)$ is $F$
		\item $I_M(h(F^\vee),F) = 1$.
	\end{enumerate}
\end{lemma}

\begin{proof}
	We will construct our homotopy in multiple steps.
	The basic idea is first to construct small isotopies near the centers $\hat F$ of the faces $F$ that push the corresponding $F^\vee$ into transverse position with $F$ and so that the intersection number of the shifted $F^\vee$ with $F$ is $1$, giving the third condition.
	These will also be designed to ensure enough transversality in neighborhoods of the $\hat F$.
	Then we perform a more global shift to push the rest of $sd(X)$ into transverse position with $X$ to achieve the first condition, while leaving fixed the isotopies already constructed near the $\hat F$.
	We also ensure the global shift is small enough to provide the second condition and also not disrupt the third.

	Throughout we may assume that $M$ is embedded properly in some $\R^N$ with an $\epsilon$-neighborhood $M^\epsilon$ (with $\epsilon$ a function of $x \in N$) and a submersion $\pi \colon M^\epsilon \to M$; see \cite[Section 2.3]{GuPo74}.
	We can then consider $M$ as a metric space with the subspace metric.

	To create a template and explain the basic idea, we first consider the standard $m = \dim(M)$-cube $\interval^m = [0,1]^m \subset \R^m$.
	Let $e_i$ denote the unit tangent vector of $\R^m$ pointing in the positive $i$th direction.
	Let $f = (f_0,f_{01},f_1)$ be a face of $\interval^m$, let $\hat f$ be its center, and define the vector $v_f$ by $v = \sum_{i \in f_1} e_i-\sum_{i \in f_0} e_i$ (if $F_0 = F_1 = \emptyset$, let $v = 0$).
	At points of $f$, the vector $v_f$ points outward from the cube.
	The vector $v_f$ is also tangent to $f^\vee$, and for small $\delta>0$ the points $\hat f-\delta v_f$ lie in the interior of the face $f^\vee$.
	So, if we were to translate $f^\vee$ by $\delta v_f$ for a sufficiently small positive $\delta$, we would obtain a translate of $f^\vee$ within the $m-\dim(f)$ plane containing it.
	As this plane is orthogonal to $f$ and intersects it at the single point $\hat f$, the only intersection of $f$ with the translated $f^\vee+\delta v_f$ is at $\hat f$ with a point in the image of the interior of $f^\vee$.
	As the translation preserve the tangent plane of $f^\vee$, this is a transverse intersection, and the translation preserves the normal co-orientation so that the intersection number of $f^\vee+\delta v_f$ with $f$ is $1$.
	Furthermore, these properties are preserved under a small perturbation of $v_f$.
	But now we note that by the Transversality Theorem of \cite{GuPo74} that for almost all vectors $u$ in $\R^m$, translation by $u$ takes all open faces of $\jinterval^m$ into transverse position with respect to all open faces of $\interval^m$.
	In particular, there is such a vector $z_f$ arbitrarily close to $\delta v_f$.

	Next, suppose we are given a small ball $B^f_r$ centered at $\hat f$ in $\R^m$ with $r<1/4$ so that $B^f_r$ does not intersect any faces of $\interval^m$ that do not intersect $f$.
	Let $r_1$ be such that $0<r_1<r$, and let $B^f_{r_1}$ similarly be the ball of radius $r_1$ centered at $\hat f$.
	We choose a smooth function $\eta \colon \R^m \to [0,1]$ such that $\eta(x) = 1$ on $B^f_{r_1}$ and $\eta(x) = 0$ outside of $B^f_r$.
	Now choose $\delta\ll r_1$ and let $Z$ be the vector field on $\R^m$ given by $Z(x) = \eta(x)z_f$, with $|z_f-\delta v_f|\ll \delta$.
	Let $\Phi \colon \R^m \to \R^m$ be the map obtained by flowing by $Z$ from $t = 0$ to $t = 1$.
	Outside of $B^F_{r}$, the map $\Phi$ is the identity.
	If we take $\delta$ and $|z_f-\delta v_f|$ sufficiently small with respect to $r_1$ then there is a closed ball neighborhood $W_f$ of $\hat f$ in $B^f_{r_1}$ containing both $\hat f$ and $\hat f-z_f$
	on which the map $\Phi$ acts identically to the translation of the preceding paragraph.
	Later we will also need further closed ball neighborhood $W^f_1$ and $W^f_2$ of $\hat f$ with $W^f_1 \subset int(W^f)$, $W^f_2 \subset int(W^f_1)$, and such that $\hat f, \hat f-z_f \in W^f_2$.
	This is again possible by taking $\delta$ and the small perturbation $z_f$ of $\delta v_f$ sufficiently small.
	We also observe that the distance between the set $\Phi(f^\vee-(f^\vee \cap W^2_f))$ and the union of $\dim(F)$-faces of $\interval^m$ is positive.

	We now translate this procedure to the cubulation of $M$.
	Suppose $F$ is a face of the cubulation $X$, and choose an $m$-cube $E$ of the cubulation containing $F$ as a face.
	By assumption there is a smooth diffeomorphism $\phi$ from $\interval^m$ to $E$ that is compatible with the cubical structure.
	Let $f$ be the face of $\interval^m$ that maps to $F$.
	Then $\phi$ takes $\hat f$ to the center $\hat F$ and $f^\vee_{\interval^m}$ to one of the components of $F^\vee$.
	By the definition of smooth maps, there is a neighborhood $U$ of $\hat f$ in $\R^m$ on which there is defined a smooth map $\phi_U:U \to M$ and such that $\phi_U$ agrees with $\phi$ on $U \cap \interval^m$.
	As $\phi$ is an embedding at $\hat f$, by the Inverse Function Theorem we may choose $U$ such that $\phi_U$ is a diffeomorphism from $U$ onto its image, and by making $U$ smaller if necessary, we can also assume $U$ to be an open Euclidean ball $B^F_r$ centered at $\hat f$ whose image in $M$ intersections only faces of $X$ that contain $F$.
	Identifying this ball in $\R^m$ with its image in $M$ via $\phi$, we can perform a flow $\Phi$ as in the preceding paragraph to push $F^\vee_E$ into transverse position with $F$ (note: we keep $F$ fixed under the flow) and with intersection number $1$.
	Furthermore, this procedure pushes the other components $F^\vee_C$ for $C\neq E$ away from $F$ (slightly), and so $I(\Phi(F^\vee),F) = 1$ and also the distance between the set $\Phi(F^\vee-(F^\vee \cap W^2_F))$ and the union of $\dim(F)$-faces of $\interval^M$ is positive.
	By choosing again our $z_f$ if necessary (among almost all vectors arbitrarily close to $\delta v_f$, we can ensure that the restriction of $\Phi$ to the intersection of any face of $sd(X)$ with $W^2_F$ is transverse to the cubulation.

	Next, we can apply this procedure at all faces $F$ simultaneously by choosing a sufficiently small $B^F_r$ neighborhood of each $\hat F$ so that they are all disjoint; we may let each $r$, $r_1$, and $\delta$ depend on $F$ though we do not include this in the notation.
	We then generalize $\Phi$ by allowing a corresponding flow on all balls simultaneously.
	This provides a smooth map $\Phi \colon M \to M$ that satisfies the second two conditions of the theorem, as we can also choose the balls small enough that the ball around $\hat F$ does not intersect any other $\dim(F)$ face of $X$.
	We define $W = \cup_F W_F$, $W^1 = \cup_F W^F_1$, and $W^2 = \cup_F W^F_2$.
	By construction, the restriction of $\Phi$ to the intersection of $W$ with any face of $sd(X)$ is transverse to all faces of $X$.
	We must further modify $\Phi$ to ensure transversality in general.

	As in the proof of \cref{T: transverse complex}, we next follow the construction in \cite[Section 2.3]{GuPo74}.
	We have $M$ embedded in some Euclidean space $\R^N$ with an $\epsilon$-neighborhood $M^\epsilon$ and a submersion $\pi \colon M^\epsilon \to M$.
	As we are happy with the map $\Phi$ as constructed so far on $W$, we let $\rho \colon M \to [0,1)$ be a smooth function that is $0$ on $W^1$ and $>0$ on $M-W_1$.
	We will fine tune $\rho$ a bit more soon.
	Let $S$ be the unit ball in $\R^N$.
	We now consider the map $H \colon M \times S \to M$ defined by $H(x,s) = \pi(\Phi(x)+\rho(x)\epsilon s)$.
	At all points $(x,s)$ such that $\rho(x)>0$, i.e.\ on $M-W_1$, this is a submersion (and so transverse to all faces of $X$), and for all $(x,s)$ such that $\rho(x) = 0$, i.e.\ on $W_1$, we have $H(x,s) = \Phi(x)$.

	Now let $G$ be the interior of any face of the cubical subdivision $sd(X)$.
	At any point $x \in G \cap W_1$, we already have that for any fixed $s_0 \in S$ the map $H|_G(-,s_0) = \Phi|_G(-)$ is transverse at $x$ to any face of $X$.
	Furthermore, by the Transversality Theorem of \cite{GuPo74}, for almost every $s_0 \in S$ and for any face $F$ of $X$, the map $H|_G(-,s_0)$ is transverse to the interior of $F$ at all points on the submanifold $G-G \cap W_1$ of $G$.
	But there are a finite number of faces of $X$, so for almost every $s_0 \in S$ and for every face $F$ of $X$, the map $H|_G(-,s_0)$ is transverse to the interior of every face of $F$, which implies it is transverse to every face of $F$.
	But there are also only a finite number of faces of $sd(X)$, and so for almost every $s_0 \in S$, $H|_G(-,s_0)$ is transverse to $F$ for every $G$ and every $F$.
	In particular, each $H|_{F^\vee}$ determines an element of $PC^*_{\Gamma \pf X}(M)$ (that $H|_{F^\vee}$ is the output of an ambient isotopy in a neighborhood of $\hat F$ allows us to co-orient the components of $H$ via orientations of their tubular neighborhoods, which are also pushed around isotopically).

	To complete the proof we must do one last thing: we must fine tune $\rho$ to ensure that in forming $H(-,s_0)$ to ensure all of the required transversality we have not pushed any $F^\vee$ so far as to create new intersections with faces of $X$ of complementary dimensions beyond the single intersection of $H(F^\vee,s_0)$ with $F$ inside $W$.
	For this we do the following.

	Recall as noted above that, by our construction, if $F$ is any face of $X$ then the distance between $\Phi(F^\vee-(F^\vee \cap W^2))$ and the union of $\dim(F)$-faces of $X$ is positive.
	Now suppose $x \in M-W_2$ and consider a compact neighborhood $\bar U_x$ of $x$ in $M-W_2$.
	By the above, if $\Phi(\bar U_x \cap F^\vee)$, then there is a positive distance $\varepsilon_{x,F}$ between $\bar U_x \cap F^\vee\neq \emptyset$ and the $\dim(F)$-skeleton of the cubulation.
	As $\bar U_x$ is compact, it intersects only a finite number of $F^\vee$ as $F$ ranges over all faces of $X$, and we let $$\varepsilon_x = \min\{\varepsilon_{x,F} \mid \bar U_x \cap F^\vee\neq \emptyset\}.$$
	So, by construction, any map $g \colon M \to M$ such that $d(z,g(z))<\varepsilon_x$ for all $z \in \bar U_x$ satisfies the property that if $z \in F^\vee \cap \bar U_x$ then $g(z)$ is not contained in a $\dim(F)$ face of $X$.

	Now, suppose we have constructed such a $\varepsilon_x$ for all $x$ in $M-W_2$.
	Then the interiors $U_x$ of the $\bar U_x$ cover $M-W_2$, and we can take a locally finite refinement $\mc U$.
	By \cref{L: minimizer}, we can find a smooth function $\rho_1 \colon M-W_2 \to \R$ so that $0<\rho_1(z)<\varepsilon_x$ if $z \in \bar U_x$ for $U_x \in \mc U$.
	Let $\rho_2 = \rho\rho_1 \colon M \to \R$.
	This is smooth and well defined on all of $M$ as $\rho(x) = 0$ for all $x \in W_1$ and $W_2 \subset int(W_1)$.
	We also see that $0<\rho_2(x)<\rho_1(x)$ for all $x \in M-W_2$.
	So now if we replace $H$ with $H_2 \colon M \times S \to M$ defined by $H_2(x,s) = \pi(\Phi(x)+\rho_2(x)\epsilon s)$, then for any $s_0$ and for any $x \in (M-W_2) \cap F^\vee$, we have that $H_2(x,s)$ does not intersect any face of dimension $\dim(F)$ as desired.
	We can now set $h(-) = H_2(-,s_0)$ for almost all $s_0$ to achieve all three required conditions, noting that the conclusions of the preceding paragraphs did not depend on the choice of $\rho$.
\end{proof}

\subsubsection{The intersection map is an isomorphism for finitely-generated cohomology groups}

\begin{theorem}\label{T: intersection qi}
	If $M$ is a manifold without boundary cubulated by $X$, the intersection map $\mc I \colon H^i_{\Gamma \pf X}(M) \to H^i(K_X^*(M))$ is an isomorphism whenever $H^i(M)$ is finitely generated.
	It is a surjection even if $H^i(M)$ is not finitely generated.
\end{theorem}

\begin{proof}
	Above we constructed a chain map $\Psi \colon K_X^*(M) \to C_\Gamma^*(M)$ by taking $F^*$ to the geometric cochain represented by the inclusion of $F^\vee$ into $M$, but the image did not lie in $C^*_{\Gamma \pf X}(M)$.
	We now modify that construction to define $\psi:K_X^*(M) \to C_{\Gamma \pf X}^*(M)$ by taking $F^*$ to the element of $C^*_{\Gamma \pf X}(M)$ represented by the composition of the inclusion of $F^\vee$ into $M$ with $h$.
	To see that this is a chain map, we observed that $\Psi$ commutes with boundaries in the proof of \cref{L: dual chain map}, and we know that $h$ commutes with boundaries by the discussion leading up to \cref{T: homology homotopy functor}, co-orienting $h$ using that it is homotopic to the identity.
	Furthermore, by construction $\psi(F^*)$ represents a map that is transverse to the cubulation for each generator $F^*$ of $K_X^*(M)$.
	So $\psi$ determined a chain map to $C_{\Gamma \pf X}^*(M)$ as desired.

	Furthermore, we see by the preceding lemma and the definition that $\mc I\psi$ is the identity map.
	In particular, then, $\mc I$ induces a cohomology surjection $H^*_{\Gamma \pf X}(M)\onto H^*(K_X^*(M))$.

	Now, we know from \cref{T: geometric is singular} and the isomorphism between cubical and singular cubical cohomologies that these groups are both isomorphic to $H^*(M)$, and a surjective map of isomorphic finitely-generated abelian groups is an isomorphism (as $\Z$-is Noetherian).
\end{proof}

We conjecture that $\mc I$ is an isomorphism in the general case but have not been able to prove this.

\begin{remark}\label{R: intersection map extension}
	Putting together the isomorphism $\mc I \colon H^i_{\Gamma \pf X}(M) \to H^i(K_X^*(M))$ with the inverse of the isomorphism $H^*_{\Gamma \pf X}(M) \to H^*_\Gamma(M)$, it is sometimes useful to abuse notation and speak of the intersection-induced isomorphism
	$H^*_\Gamma(M) \to H^i(K_X^*(M))$.
	Of course this map is given by taking a cohomology class representative that is transverse to the cubulation and applying the intersection map $\mc I$ to find a cubical cocycle representing the target cohomology class.
\end{remark} %
	% !TEX root = ../foundations.tex

\section{Products of geometric chains and cochains}\label{S: products}

In this section we consider products of geometric chains and cochains, first simply as chains and cochains and then as pairings on homology and cohomology.
These pairings are all built from the fiber products and exterior products of maps as defined in \cref{S: orientations and co-orientations}.
However, while the exterior products were fully defined, the fiber products required transversality of $f \colon V \to M$ and $g \colon W \to M$ in order for $V \times_M W$ to be a well-defined manifold with corners possessing an oriented or co-oriented map to $M$.
Consequently, the fiber products do not induce fully-defined chain- and cochain-level products such as a pairing $C^*_\Gamma(M) \otimes C^*_\Gamma(M) \to C^*_\Gamma(M)$.
At best we can hope for a partially-defined (co)chain-level pairing, though even this is not clear once we take into account that a geometric chain or cochain is not represented simply by a single isomorphism class of a map $V \to M$ but is rather an equivalence class of such mappings up to triviality and degeneracy.

In \cref{S: chain products}, we address this issue and show that there is a natural notion of transversality among chains and cochains, despite the ambiguity in the representative prechains and precochains.
We use this to provide partially-defined cup, cap, and intersection pairings among geometric chains and cochains.
We consider it important to have such pairings, even when only partially defined, as cochain algebras contain much information that is lost on passage to cohomology.
For example, the singular cochains of a space are what carry the $E_\infty$-algebra structure, while passage to cohomology often contains just shadows of this structure, such as the Steenrod squares.
We also provide fully-defined chain and cochain exterior products, though these give us less trouble.
In \cref{S: (co)chain properties}, we collect the various properties of these partially-defined products, mostly based on properties we have established for fiber products of maps in earlier sections.

In \cref{S: homology products}, we then turn to the resulting products in geometric homology and cohomology, which we show are fully defined, providing cup, cap, intersection, and exterior products.
We also utilize this machinery along with our cubulations to construct a universal coefficient theorem when the cohomology is finitely generated.
In \cref{S: usual cup}, we show that the cup product in geometric cohomology is isomorphic to the singular cohomology cup product.

\begin{comment}
	in this section endow geometric cochains with a commutative product given by intersection of immersed submanifolds, or pullback more generally.
	This product is partially defined, as it must be if it is to be commutative and induce the cup product in cohomology.
	The interplay between intersection and cup product dates back to the beginning of the subject, giving the latter product its name.
	But, to our knowledge, we are the first to construct a cochain level comparison between these approaches.
	Somewhat surprisingly, the construction ends up being delicate since our cochains are themselves equivalence classes in a
	highly non-trivial way.

	\red{D: Should say that Joyce gives a cochain level product for his theory but to our knowledge this is the first for the Lipyanksiy theory.}

	BCOMMENT
	While Joyce carefully develops transversality for manifolds with corners in Section 6 of \cite{Joy12}, we only requires transversality for maps into a manifold without corners or boundary.
	In this case the definition is equivalent to transversality when restricted to each $S^k$.
	If $V$ and $W$ are manifolds over $M$,
	we use the notation $V \pf W$ to denote
	transversality of maps $r_V$ and $r_W$.

	When $W$ and $V$ are embedded, pullback is intersection, and the normal bundle of the intersection is canonically (once an ordering of $W$
	and $V$ is fixed) the direct sum of the normal bundles.

	The following is essentially the content of Section~5 of \cite{Lipy14}.

	\begin{proposition}\label{P: product}
		The pullback product $\uplus_M$ extends linearly to a well-defined, though only partially-defined, product on $C_\Gamma^*(M)$, which in turn passes to a fully-defined product on $H_\Gamma^*(M)$.
	\end{proposition}

	\begin{proof}
		To show that $\uplus_M$ is well-defined, we check that if $f = 0$ in $C_\Gamma^*(M)$, so is $f \uplus_M g$ for any $g \colon V \to M$.
		If $f$ is trivial, the co-orientation-reversing involution on $W$ defines one on $W \times V$ which restricts to the pullback.
		Recall that the tangent bundle of the pullback is the pullback of tangent bundles, and the derivative is the pullback of derivatives, so, if the differential of $f$ has non-trivial kernel everywhere so will the derivative of any pullback with $f$.

		That the pullback product passes to cohomology follows from the Leibniz rule for pullbacks, which in turn follows from that for the product $W \times V$.
		We defer the proof that this product on cohomology is fully defined until after the proof of Theorem~\ref{T: transverse complex} below.
	\end{proof}
	ECOMMENT

	Recall that if $r_W \colon W \to M$ is an immersion, then a co-orientation is equivalent to an orientation of the normal bundle of $W$.
	Explicitly in coordinates of a tangent space, if $n_1 \wedge \cdots \wedge n_w$ is an orientation of the normal bundle, the
	corresponding co-orientation sends $b_1 \wedge \cdots \wedge b_{m-w}$ to $b_1 \wedge \cdots \wedge b_{m-w} \wedge n_1 \wedge \cdots \wedge n_w$.

	\begin{proposition}
		If $r_V$ and $r_W$ are immersions which are co-oriented through orientations of $\nu_{V \subset M}$ and $\nu_{W \subset M}$
		then the co-orientation of the pullback agrees with the co-orientation through the identification of the normal bundle of the pullback with
		$\nu_{V \subset M} \oplus \nu_{W \subset M}$
	\end{proposition}

	\begin{proof}
		For once, we find dimensions better to track than codimensions.
		At a point in the pullback there are neighborhoods in $V$, $W$ and $M$
		so that the manifolds themselves, and thus their tangent spaces, are arranged as follows.
		The neighborhood of $M$ is diffeomorphic to $\R^m$,
		spanned by $e_1, \cdots, e_m$; under this identification, the neighborhood of $V$ is the subspace given by the
		span of $e_1, \cdots, e_v$ and that of $W$ is the span of $e_{m-w+1}, \cdots, e_m$.
		In these coordinates the pullback is just $P = V \cap W$, which is the span of $e_{m-w+1}, \cdots, e_v$.
		Our preferred orientations of spans of standard basis
		vectors - which is what $V$, $W$ their intersection and all normal bundles are - are the class in which those basis vectors occur in order.

		The normal co-orientation of $V$ sends the preferred orientation $e_1 \wedge \cdots \wedge e_v$ to
		$e_1 \wedge \cdots \wedge e_v \wedge e_{v+1} \wedge \cdots \wedge e_m$, which is the preferred orientation of $M$.
		The normal co-orientation of $W$ sends $e_{m-w+1} \wedge \cdots \wedge e_m$ to $e_{m-w+1} \wedge \cdots \wedge e_m \wedge e_1 \wedge
		\cdots \wedge e_w$, which is $(-1)^{w(m-w)}$ times the preferred orientation of $M$.
		The normal co-orientation of $P = V \cap W$ sends $e_{m-w+1} \wedge \cdots \wedge e_{v}$ to
		$$ e_{m-w+1} \wedge \cdots \wedge e_{v} \wedge e_{v+1} \wedge \cdots \wedge e_m \wedge e_1 \wedge \cdots \wedge e_{m-w},$$
		thus sending the preferred orientation of $P$ to $(-1)^{v(m-v)}$ times the preferred orientation of $M$.

		We now must wade through our definition of co-orientation of the pullback and in particular Equation~\ref{co-or stuff}.
		Since we must consider $P \subset V \times W$, we
		alter notation to distinguish basis vectors in $V$ by calling them $e_i^V$ and similarly for $W$.
		Then the tangent vectors for $P$ in $V \times W$
		are naturally ``diagonal'', with a basis $e_{m-w+1}^V \oplus e_{m-w+1}^W, \cdots, e_v^V \oplus e_v^W$.
		For $M \times M$, we set
		the basis $e_1, \cdots, e_m, e_1', \cdots, e_m'$.

		We choose
		a convenient set of isomorphisms for Equation~\ref{co-or stuff}, in particular identifications of normal bundles as well as
		``breaking up'' tensor products to minimize additional signs.
		Namely, identify $ \Or TP \otimes \Or(\nu_{P \subset V \times W})$ with $ \Or T \times V \cong \Or T V \otimes \Or TW $ by sending
		$e_{m-w+1}^V \oplus e_{m-w+1}^W, \cdots, e_v^V \oplus e_v^W$ to
		$$ (e_1^V \wedge \cdots \wedge e_v^V) \wedge (e_{m-w+1}^V \oplus e_{m-w+1}^W \wedge \cdots \wedge e_v^V \oplus e_v^W) \wedge
		(e_{v+1}^W \wedge \cdots \wedge e_m^W).$$
		Here we use additional parenthesis so that one can see that an orientations using normal vectors is appended with parts coming
		both before and after the orientation
		in question.
		With this choice, which must followed below as well, the standard orientation of $P$ is sent to the standard orientation of $V \times W$.

		Applying the normal co-orientations of $V$ and $W$ as calculated above to the standard orientation of $V \times W$ will yield
		$(-1)^{w(m-w)}$ times the standard orientation of $M \times M$.
		Consulting the diagram of Equation~\ref{co-or stuff} we ask
		which orientation of $M$ corresponds with this under the identification of $\Or TM \otimes \Or( \nu_{P \subset V \times W})$
		with $ \Or T M \otimes \Or TM$ which is compatible with our identification
		$ \Or TP \otimes \Or(\nu_{P \subset V \times W})$ with $ \Or T \times V \cong \Or T V \otimes \Or TW $.
		That identification sends $e_1 \oplus e_1' \wedge \cdots \wedge e_m \oplus e_m'$ to
		$$ (e_1 \wedge \cdots \wedge e_v) \wedge (e_1 \oplus e_1' \wedge \cdots \wedge e_m \oplus e_m') \wedge (e_{v+1}' \wedge \cdots \wedge e_m').$$
		This expression differs from the standard orientation of $M \times M$ by a sign of $v(m-v)$, as we must move
		$e_{v+1} \oplus e_{v+1}' \wedge \cdots \wedge e_m \oplus e_m'$ past $(e_{1} \oplus e_{1}' \wedge \cdots \wedge e_v \oplus e_v')$ in order
		to obtain something equivalent to the standard orientation.

		\red{D: Help - in looking this over I think we have an extra $(-1)^{w(m-w)}$ coming from the application of the co-orientation of $W$.
			So I wrote the argument
			with that, but now we do not have agreement in the end...}

	\end{proof}

\end{comment}



\subsection{Chain- and cochain-level products and transversality}\label{S: chain products}

In this section, we develop chain- and cochain-level products, as well as study some other aspects of chain- and cochain-level transversality.
We begin in \cref{S: simple products} with the case of ``simple products,'' in which two chains or cochains can be represented by transverse prechains or precochains, in which case we can take the fiber product.
Building on this case, we then define a more general notion of transversality for chains and cochains that allows for some amount of bilinear behavior.
In \cref{S: product pullbacks}, we consider pullbacks of cochains, and in \cref{S: Kronecker} we obtain a Kronecker pairing for geometric chains and cochains.
Finally, in \cref{S: exterior chain products}, we define exterior products of chains and cochains.


\subsubsection{Transversality and products}\label{S: simple products}
In this section we define a notion of transversality of geometric chains and cochains, which allows us to define (co)chain-level cup, cap, and intersection products.

We begin in \cref{D: cochain trans} with the naive case in which our (co)chains possess transverse representing pre(co)chains.
We call such (co)chains \textbf{simply transverse} (see \cref{D: cochain trans}), and this is sufficient to define cup, cap, and intersection products via fiber products.
However, as we will discuss below, this definition is insufficient to obtain a product that behaves bilinearly.
To obtain this property, we introduce the more general concept of compound transversality in \cref{D: multicup}.
This allows for bilinear behavior of products of simply transverse (co)chains, although compound transversality itself does not seem to provide a satisfactory bilinear product.
The key problem has to do with demonstrating the existence of products that are independent of choices of representing pre(co)chains.
We will discuss the difficulties further after establishing some definitions and results.

\begin{comment}
	For example, one might consider the case where $\uV$ and $\uW$ can be represented by disjoint unions of pre(co)chains $V = \bigsqcup V_i$ and $W = \bigsqcup W_j$ such that the pairs $(V_i,W_j)$ are not necessarily transverse but such that for each such pair there are alternative representatives, say $(V'_{ij},W'_{ji})$, depending on both $i$ and $j$, with $\underline{V_i} = \underline{V'_{ij}}$, $\underline{W_j} = \underline{W'_{ji}}$, and $V'_{ij}$ transverse to $W'_{ji}$.
	Then one would like to have products of $\uV$ and $\uW$ using these representatives.
\end{comment}


We begin with simple transversality.

\begin{definition}\label{D: cochain trans}
	Let $M$ be a manifold without boundary.
	We say that $\uV, \uW \in C^*_{\Gamma}(M)$ are \textbf{simply transverse} as geometric cochains if there exist representatives $V,W \in PC^*_\Gamma(M)$ such that $V$ and $W$ are transverse as manifolds with corners mapping to $M$.
	We call the data of such a pair $(V,W)$ a \textbf{simple transverse representation} for the pair $(\uV,\uW)$.

	We define simple transversality similarly if $\uV \in C^*_{\Gamma}(M)$ and $\uW \in C_*^{\Gamma}(M)$ or if $M$ is oriented and $\uV, \uW \in C_*^{\Gamma}(M)$.
\end{definition}

\begin{definition}\label{D: cochain products}
	Let $M$ be a manifold without boundary.
	For $\uV, \uW \in C^*_{\Gamma}(M)$ simply transverse, we define the \textbf{cup product} $\uV \uplus \uW \in C^*_\Gamma(M)$ to be the geometric cochain represented by the fiber product $V \times_M W$ for some simple transverse representation $(V,W)$ of $(\uV,\uW)$.

	Analogously, if $\uV \in C^*_{\Gamma}(M)$ and $\uW \in C_*^{\Gamma}(M)$ are simply transverse, we define the \textbf{cap product} $\uV \nplus \uW \in C_*^\Gamma(M)$ by $V \times_M W$ for some simple transverse representation $(V,W)$.

	If $M$ is oriented and $\uV,\uW \in C_*^\Gamma(M)$ are simply transverse, we define the \textbf{intersection product} $\uV \bullet \uW \in C_*^\Gamma(M)$ by $V \times_M W$ for some simple transverse representation $(V,W)$.

	In each context, the given product $V \times_M W$ is as defined in \cref{D: PC products}, as $V$ and $W$ are transverse by assumption.
\end{definition}

\begin{comment}
	It will follow from \cref{T: multicup}, below, that these products are well defined, independent of the choices of representing chains and cochains.
	This is not immediately clear, as a geometric (co)chain $\uV$ has in general an infinite number of representatives in $PC(M)$ that may or may not be transverse to any other given element of $PC(M)$; see \cref{E: bad transversality}.
	Rather than stop to prove well-definedness in the simple transversality case, we proceed toward our more general definition of transversality.
\end{comment}

The first main result of this section is that these products are well defined as operations on simply transverse geometric chains or cochains, independent of the prechain or precochain representatives chosen.
This is not immediately clear, as a geometric (co)chain $\uV$ has in general an infinite number of representatives in $PC(M)$ that may or may not be transverse to any other given element of $PC(M)$; see \cref{E: bad transversality}.

\begin{theorem}\label{T: cochain product}
	Given simply transverse $\uV$ and $\uW$, the cup, cap, or intersection products of \cref{D: cochain products} are well defined, independent of choice of simple transverse representation.
\end{theorem}


Rather than prove this theorem here, we first provide some further discussion and development.
\cref{T: cochain product} will then follow directly as a special case of \cref{T: multicup}, below.

\begin{comment}

	\begin{proof}
		For simplicity, we just give the proof for $\uplus$; the other arguments are identical.
		We also write $V+W$ and $V-W$ rather than $V \sqcup W$ and $V\sqcup-W$ to make the following easier to read.

		Let $\uV,\uW$ be simply transverse geometric cochains with simple transverse representation $(V, W)$ as in \cref{D: cochain trans}.
		Similarly, let $(V',W')$ be another simple transverse representation.
		By assumption, $V-V', W-W' \in Q^*(M)$.
		For each of these precochains, we have their essential decompositions $V = V_E \sqcup V_{TI} \sqcup V_{NI}$, etc.

		We must show that $V \times_M W$ and $V' \times_M W'$ represent the same element of $C^*_\Gamma(M)$, i.e.\ that
		$$[(V_E+V_{TI}+V_{NI}) \times_M (W_E+W_{TI}+W_{NI})] - [(V'_E+V'_{TI}+V'_{NI}) \times_M (W'_E+W'_{TI}+W'_{NI})]$$
		is in $Q^*(M)$.
		Writing out, we have
		\begin{align*}(V_E+V_{TI}+V_{NI}) \times_M (W_E+W_{TI}+W_{NI}) = &V_E \times_M W_E+V_E \times_M W_{NI}+V_{NI} \times_M W_E+V_{NI} \times_M W_{NI}\\
			&+V_{TI} \times_M (W_E+W_{TI}+W_{NI})+(V_E+V_{NI}) \times_M W_{TI}.
		\end{align*}
		As $V_{TI},W_{TI} \in Q^*(M)$, the terms on the second line are all in $Q^*(M)$ by \cref{L: pullback with Q}.
		A similar decomposition holds for the primed versions, so we must show that
		\begin{equation}\label{E: intersect}
			(V_E \times_M W_E+V_E \times_M W_{NI}+V_{NI} \times_M W_E+V_{NI} \times_M W_{NI})-(V'_E \times_M W'_E+V'_E \times_M W'_{NI}+V'_{NI} \times_M W'_E+V'_{NI} \times_M W'_{NI})
		\end{equation}
		is in $Q^*(M)$.

		From \cref{C: essential trans} we know that if $V_E$ is transverse to $W$, and in particular $W_E$, then so is $V'_E$, and so we can form $V'_E \times_M W$.
		Thus, taking the union of $V_E \times_M W_E -V'_E \times_M W'_E$ with the trivial precochain $-V'_E \times_M W_E+V'_E \times_M W_E$, we have
		\begin{equation*}
			V_E \times_M W_E-V'_E \times_M W_E+V'_E \times_M W_E-V'_E \times_M W'_E = (V_E-V'_E) \times_M W_E+V' \times_M (W_E-W'_E).
		\end{equation*}
		By \cref{L: essential}, $V_E-V'_E$ and $W_E-W'_E$ are trivial, and the transverse intersection of any cochain with a trivial cochain is in $Q^*(M)$ by \cref{L: pullback with Q}.
		Thus $V_E \times_M W_E -V'_E \times_M W'_E \in Q^*(M)$ by \cref{L: Lipy12} (which says that if we take the disjoint union of a precochain with an element of $Q^*(M)$ and obtain an element of $Q^*(M)$ then our original precochain is in $Q^*(M)$).

		The remaining terms of \eqref{E: intersect} each involve an element of small rank and so are of small rank by \cref{L: pullback with Q}.
		So to see that what's left is an element of $Q^*(M)$, it suffices to show that the boundary of the remaining terms consists of cochains that are trivial or of small rank.
		By \cref{leibniz}, this boundary is
		\begin{multline}\label{E: boundary}
			(\bd V_E) \times_M W_{NI}+(-1)^{m-v}V_E \times_M \bd W_{NI}+
			(\bd V_{NI}) \times_M W_E+(-1)^{m-v}V_{NI} \times_M \bd W_E\\
			+(\bd V_{NI}) \times_M W_{NI}+(-1)^{m-v}V_{NI} \times_M \bd W_{NI}
			-(\bd V'_E) \times_M W'_{NI}-(-1)^{m-v}V'_E \times_M \bd W'_{NI}\\
			-(\bd V'_{NI}) \times_M W'_E-(-1)^{m-v}V'_{NI} \times_M \bd W'_E- (\bd V'_{NI}) \times_M W'_{NI}-(-1)^{m-v}V'_{NI} \times_M \bd W'_{NI}.
		\end{multline}
		Every term except for the pairs $(-1)^{m-v}V_E \times_M \bd W_{NI}-(-1)^{m-v}V'_E \times_M \bd W'_{NI}$ and $(\bd V_{NI}) \times_M W_E-(\bd V'_{NI}) \times_M W'_E$ involves a fiber product with an element of small rank and so is of small rank by \cref{L: pullback with Q}.

		For $V_E \times_M \bd W_{NI}-V'_E \times_M \bd W'_{NI}$, we recall that the data of two maps being transverse includes the assumption of transversality on boundaries.
		So as in our argument employed above to show $V_E \times_M W_E -V'_E \times_M W'_E$ is trivial, we know that $V_E$ and $V'_E$ must each be transverse to both $\bd W_{NI}$ and $\bd W'_{NI}$ by \cref{C: essential trans}.
		So by an analogous computation we arrive at
		\begin{equation*}
			V_E \times_M \bd W_{NI}-V'_E \times_M \bd W'_{NI} + V'_E \times_M \bd W_{NI} - V'_E \times_M \bd W_{NI} = (V_E- V'_E) \times_M \bd W_{NI}+ V'_E \times_M (\bd W_{NI}-\bd W'_{NI}).
		\end{equation*}
		We then have that $V_E - V'_E$ is trivial by \cref{L: essential}, and $\bd W_{NI}-\bd W'_{NI}$ is in $Q^*(M)$ by \cref{L: same NI}.
		Thus these fiber products are in $Q^*(M)$ by \cref{L: pullback with Q}, and so $V_E \times_M \bd W_{NI}-V'_E \times_M \bd W'_{NI} \in Q^*(M)$ by \cref{L: Lipy12}.
		By an analogous argument $(\bd V_{NI}) \times_M W_E-(\bd V'_{NI}) \times_M W'_E \in Q^*(M)$.
		In particular, these can be written as disjoint unions of components that are trivial or of small rank.

		Therefore, the boundary \eqref{E: boundary} can be completely decomposed into cochains that are trivial or have small rank,
		and this completes our proof that $V \times_M W-V' \times_M W' \in Q^*(M)$.
	\end{proof}

\end{comment}

\begin{comment}
	\subsubsection{Chain and cochains transversality and products}\label{S: trans products}
\end{comment}

To motivate our next step, suppose now a geometric cochain $\uV$ is simply transverse to two other geometric cochains of the same degree, $\underline{W_1}$ and $\underline{W_2}$.
This means we can form $\uV\uplus \underline{W_1}+\uV\uplus \underline{W_2}$, and we would like for this to equal $\uV\uplus (\underline{W_1} + \underline{W_2})$.
The problem is that it is not apparent from the definitions whether or not $\uV$ is simply transverse to $\uW_1+\uW_2$, as the simple transversality of the pairs $(\uV,\underline{W_1})$ and $(\uV,\underline{W_2})$ might be realized by representatives $V_1,V_2, W_1,W_2 \in PC^*_\Gamma(M)$ with $\underline{V_1} = \underline{V_2}$ so that $V_1$ and $W_1$ are transverse as spaces mapping to $M$ and $V_2$ and $W_2$ are transverse as spaces mapping to $M$, but neither $V_1$ nor $V_2$ is transverse to $W_1 \sqcup W_2$.
It is also not apparent how to find a $V_3$ representing $\uV$ that is transverse to both $W_1$ and $W_2$.
The simplest solution would then seem to be to just define
$\uV\uplus (\underline{W_1}+\underline{W_2})$ to be represented by $(V_1 \times_M W_1) \sqcup (V_2 \times_M W_2)$, so long as this is well defined.
The next definition builds on this idea.

\begin{comment}
	To do so, however, we must make sure that such a construction is independent of the choices involved.
	This is what we turn to now.
	We first show in \cref{P: multicup} that the products can be made linear in one variable as just described, and then we use that to provide a more general multilinearlity in \cref{T: multicup}.


	\begin{proposition}\label{P: multicup}
		Suppose $\sum_i \underline{W_i} = \sum_j \underline{W'_j} \in C_\Gamma^a(M)$ and that all $\underline{W_i}$ and $\underline{W_j'}$ are simply transverse to $\uV \in C_\Gamma^b(M)$.
		Then $$\sum_i \uV \uplus \underline{W_i} = \sum_j \uV\uplus \underline{W'_j}.$$
		Analogous statements hold with the sum in the first factor and for the cap and intersection products.
	\end{proposition}

	\begin{proof}
		We provide the proof for the cup product, the other arguments being analogous.

		We suppose each $\underline{W_i}$ represented by $W_i \in PC^*_\Gamma(M)$, and similarly for each $\underline{W'_j}$.
		Let $V_i$ and $V_j'$ all be representatives of $\uV$ with $V_i$ transverse to $W_i$ and $V'_j$ transverse to $W'_j$.
		We must show that $\bigsqcup_i V_i \times_M W_i$ and $\bigsqcup_j V'_j \times_M W'_j$ represent the same cochain, i.e.\ that $$\left(\bigsqcup_i V_i \times_M W_i\right) \sqcup \left(-\bigsqcup_j V'_j \times_M W'_j\right) \in Q^*(M).$$
		As $\bigsqcup_i W_i$ and $\bigsqcup_j W'_j$ represent the same geometric cochain, we have $\left(\bigsqcup_i W_i\right) \sqcup - \left(\bigsqcup_j W'_j \right) \in Q^*(M)$, so it suffices to prove the following:
		If $\bigsqcup_k W_k \in Q^*(M)$, $V_k$ is transverse to $W_k$ for all $k$, and all $V_k$ represent the same geometric cochain, then $$\bigsqcup_k V_k \times_M W_k \in Q^*(M).$$

		For each $W_k$, we consider its essential decomposition $$W_k = W_{k,E} \sqcup W_{k,TI} \sqcup W_{k,NI}.$$
		As each $W_{k,TI} \in Q^*(M)$, we have each $V_k \times_M W_{k,TI} \in Q^*(M)$ by \cref{L: pullback with Q}.

		We next consider $\bigsqcup_k W_{k,E}$, which is trivial by \cref{L: Q essential}.
		By \cref{L: trivial structure}, each connected component, say $\mc W$, appearing in one of the $W_{k,E}$ either has a co-orientation-reversing automorphism or appears zero times in all of $\bigsqcup_k W_{k,E}$ when counting with sign.
		If $\mc W$ has a co-orientation-reversing automorphism, then $\mc W$ is trivial and $V_k \times \mc W$ is trivial for any $V_k$ transverse to $\mc W$ by \cref{L: pullback with Q}.
		Otherwise, for each occurrence of $\mc W$ in some $W_{k,E}$, there is an occurrence of $-\mc W$ in some $W_{\ell,E}$.
		By \cref{T: cochain product}, $V_k \times_M \mc W$ and $V_\ell \times_m \mc W$ represent the same cochain, so
		$$(V_k \times_M \mc W) \sqcup -(V_\ell \times_M \mc W) = (V_k \times_M \mc W)\sqcup(V_\ell \times_M -\mc W) \in Q^*(M).$$
		Continuing this way with pairs of oppositely-co-oriented components of $\bigsqcup_k W_{k,E}$, and noting as in the proof of \cref{L: trivial structure} that each $\mc W$ occurs only a finite number of times, we see that $\bigsqcup_k V_k \times_M W_{k,E} \in Q^*(M)$.

		It remains to show that $\bigsqcup_k V_k \times_M W_{k,NI} \in Q^*(M)$.
		By \cref{L: Q essential}, we have $\bigsqcup_k W_{k,NI} \in Q^*(M)$.
		By definition, we can write $\bigsqcup_k W_{k,NI} = W_{tr} \sqcup W_{d}$ as the disjoint union of a trivial precochain and a degenerate precochain.
		By the same procedure as just above, if $\mc W$ is a connected component of $W_{tr}$, then either it has a co-orientation-reversing automorphism or it appears zero times in all $W_{tr}$ counting with sign.
		So, again as above, either either $V_k \times_M \mc W \in Q^*(M)$ or we can have pairs $V_k$ and $V_\ell$ with $(V_k \times_M \mc W) \sqcup (V_\ell \times_M -\mc W) \in Q^*(M)$.
		So it remains to consider the terms involving $W_d$.

		Let $W_d = \bigsqcup_k Y_k$, where $Y_k$ consists of those components of $W_d$ contributed by $W_{k,NI}$.
		We consider $\bigsqcup_k V_k \times_M Y_k$.
		Each $Y_k$ has small rank, hence so does each $V_k \times_M Y_k$ by \cref{L: pullback with Q}.
		Therefore, $\bigsqcup_k (V_k \times_M Y_k)$ has small rank, and it suffices to show that $\bd (\bigsqcup_k V_k \times_M Y_k)$ is the union of a trivial precochain and one of small rank.
		By \cref{leibniz},
		$$\bd \left(\bigsqcup_k V_k \times_M Y_k\right) = \bigsqcup_k ((\bd V_k) \times_M Y_k ) \sqcup (-1)^{m-v} (V_k \times_M \bd Y_k).$$
		We note that these terms are defined as the transversality of $V_k$ and $Y_k$ includes transversality with the boundaries.
		As the $Y_k$ have small rank, each $(\bd V_k) \times_M Y_k$ has small rank by \cref{L: pullback with Q}.
		As $W_d = \bigsqcup_k Y_k$ is degenerate by definition, $\bd (\bigsqcup_k Y_k) = \bigsqcup_k \bd Y_k$ can be written as $\bigsqcup_k\bd Y_k = A_{tr} \sqcup A_{sm}$, with $A_{tr}$ trivial and $A_{sm}$ of small rank.
		But then if $\mc A$ is a component of $\bd Y_k$ in $A_{sm}$, we have $V_k \times \mc A$ of small rank, and for the connected components of $A_{tr}$ we can once again recognize that either $\mc A$ has a co-orientation-reversing automorphism or appears zero times in all of $A_{tr}$ counting with co-orientation.
		So again repeating our earlier argument either $V_k \times_M \mc A$ is trivial or we can find pairs of components $\mc A$ in $\bd Y_k$ and $-\mc A$ in $\bd Y_\ell$ (possibly with $k=\ell$) such that $(V_k \times_M \mc A) \sqcup (V_\ell \times_M -\mc A) \in Q^*(M)$.
		So in particular this expression is a union of a trivial precochain and one of small rank by definition of $Q^*(M)$.
		Continuing in this way through all connected components, all of $\bigsqcup_k(V_k \times_M \bd Y_k)$ can be partitioned into trivial precochains and precochains of small rank.
	\end{proof}

	We now generalize yet again to fully multilinear versions of the cup, cap, and intersection products.
	Again, this requires enough transversality for all fiber products to be defined, but now we allow cochain representatives in both the first and second factors to vary.
\end{comment}


\begin{definition}\label{D: multicup}
	Let $M$ be a manifold without boundary.
	Let $\uV, \uW \in C_\Gamma^*(M)$, and suppose $\uV$ and $\uW$ can be written as finite sums $\uV = \sum_i \underline{V_i} \in C_\Gamma^*(M)$ and $\uW = \sum_j \underline{W_j} \in C_\Gamma^*(M)$ such that each pair $(\underline{V_i},\underline{W_j})$ is simply transverse.
	Then we say that $\uV$ and $\uW$ are \textbf{compound transverse} and define the cup product $\uV\uplus\uW$ as $$\uV\uplus\uW = \sum_{i,j} \underline{V_i}\uplus \underline{W_j},$$
	where the cup products on the right are those of \cref{D: cochain products}.

	We extend the definition of the cap and intersection products analogously.
	In particular, for each simply transverse pair $(\underline{V_i},\underline{W_j})$ as above, there is a simple transverse representation $(V_{ij}, W_{ji})$, and the product $\uV\uplus \uW$, $\uV\nplus \uW$, or $\uV\bullet\uW$ is represented by $\sum_{ij}V_{ij} \times_M W_{ji}$.
\end{definition}


\begin{comment}
	\begin{theorem}
		Let $M$ be a manifold without boundary.
		Suppose $\underline{V_1}, \underline{V_2} \in C_\Gamma^*(M)$ are both transverse to $\uW \in C_\Gamma^*(M)$.
		Then $\underline{V_1 + V_2}$ is transverse to $\uW$ and $$\underline{V_1 + V_2} \uplus \uW = \underline{V_1} \uplus \uW + \underline{V_2} \uplus \uW.$$
		The equivalent statements hold for the cap and intersection products and with the sum in the second variable.
	\end{theorem}
	\begin{proof}
		By definition, we can write

	\end{proof}
\end{comment}

We now demonstrate these products are well defined, noting that \cref{T: cochain product} occurs as the special case in which the sums for $\uV$ and $\uW$ in \cref{D: multicup} each have only one term.
The proof utilizes the essential decompositions of prechains and precochains developed in \cref{S: essential decomp}.

\begin{theorem}\label{T: multicup}
	The products of compound transverse chains and cochains of \cref{D: multicup} are well defined.
	In particular, they do not depend on the decompositions of $\uV$ and $\uW$ into sums of geometric chains or cochains.
\end{theorem}

\begin{proof}
	We provide the argument for the cup product, the other proofs being analogous.
	Suppose $\uV = \sum_i \underline{V_i} = \sum_k \underline{V'_k}$ and $\uW = \sum_j \underline{W_j} = \sum_\ell \underline{W'_\ell}$.
	Suppose the pairs $(\underline{V_i},\underline{W_j})$ and $(\underline{V'_k},\underline{W'_\ell})$ are simply transverse.
	We must show that $\sum_{i,j} \underline{V_i}\uplus \underline{W_j} = \sum_{k,\ell} \underline{V'_k}\uplus \underline{W'_\ell}$.
	The assumptions mean that for each pair $(\underline{V_i},\underline{W_j})$, there are simple transverse representatives we can choose and call $(V_{ij}, W_{ji})$, and similarly for the primed versions.
	Then we must show that $$\left(\bigsqcup_{i,j} V_{ij} \times_M W_{ji}\right) \sqcup \left(-\bigsqcup_{k,\ell} V'_{k\ell} \times_M W'_{\ell k}\right) \in Q^*(M).$$

	For each $V_{ij}$, we have its essential decomposition $$V_{ij} = V_{ij,E} \sqcup V_{ij,TI} \sqcup V_{ij,NI},$$
	and by \cref{T: minimal rep}, the cochain $\underline{V_{ij}}$ is also represented by a precochain of the form $Z_{ij} \sqcup V_{ij,NI}$, where $Z_{ij}$ is the minimal essential precochain of $\underline{V_{ij}}$.
	As $Z_{ij} \sqcup V_{ij,NI}$ is obtained from $V_{ij}$ by removing some components, we may assume that all $V_{ij}$ in fact have this form without disturbing our transversality assumptions, and similarly for the $W_{ji}$, $V'_{k\ell}$, and $W'_{\ell k}$.
	Furthermore, as $V_{ia}$ and $V_{ib}$ represent the same $\underline{V_i}$, they will have the same minimal essential cochain, which we can therefore write simply as $V_{i,E}$ rather than $Z_{ij}$, and again similarly for the other precochains.

	\begin{comment}
	Next, let us choose for each $j$ and $\ell$ particular representatives $W_j$ and $W'_\ell$ for $\underline{W_j}$ and $\underline{W'_\ell}$.
	By the same argument as just above, the essential component of this $W_j$ is the same as that of the other $W_{ji,E}$ up to trivial terms, so, again abusing notation but with no negative impact, we can take the essential component of $W_j$ to also be $W_{j,E}$, and the notation is consistent.
	Similarly for the $W'_\ell$.
	\end{comment}
	\begin{comment} Again we can decompose each $W_j$ into $W_{j,E} \sqcup W_{j,TI} \sqcup W_{j,NI}$, and by Corollary \ref{C: Q essential}, we can throw out some trivial components (which contribute trivial fiber products) and assume that these are the same $W_{j,E}$ as above.
		Similarly for the $W'_\ell$.
		As each $W_{j,TI}$ and $W'_{\ell,TI}$ is in $Q^*(M)$, it follow from Lemma \ref{L: Lipy12} that
		$$\left(\bigsqcup_j \left(W_{j,E} \sqcup W_{j,NI}\right)\right) \sqcup \left(-\bigsqcup_\ell \left(W'_{\ell,E} \sqcup W'_{\ell,NI}\right)\right) \in Q^*(M).$$
	\end{comment}

	We have $\left(\bigsqcup_j W_{j}\right) \sqcup -\left(\bigsqcup_\ell W'_{\ell}\right) \in Q^*(M)$ by assumption, so by \cref{L: Q essential}, $\left(\bigsqcup_j W_{j,E}\right) \sqcup -\left(\bigsqcup_\ell W'_{\ell,E}\right)$ must be trivial.
	Therefore, by \cref{L: trivial structure}, each connected component, say $\mc W$, appearing in one of the $W_{j,E}$ or $W'_{\ell, E}$ either has a co-orientation-reversing automorphism or appears zero times in all of $\left(\bigsqcup_j W_{j,E}\right) \sqcup -\left(\bigsqcup_\ell W'_{\ell,E}\right)$ when counting with sign.
	If $\mc W$ has a co-orientation-reversing automorphism, then $\mc W$ is trivial and so cannot appear in $W_{j,E}$ or $W'_{\ell,E}$.
	So, for each occurrence of $\mc W$ in $\left(\bigsqcup_j W_{j,E}\right) \sqcup -\left(\bigsqcup_\ell W'_{\ell,E}\right)$, there is an occurrence of $-\mc W$.
	Suppose $\mc W \subset W_{a,E}$ and $-\mc W \subset W_{b,E}$.
	Then, in particular, $\mc W \subset W_{ai}$ and $-\mc W \subset W_{bi}$ for all $i$, and so $\mc W$ is simply transverse to each $V_{ia}$ and $V_{ib}$.
	So $\bigsqcup_i V_{ia}$ and $\bigsqcup_i V_{ib}$ both represent $\uV \in C^*_\Gamma(M)$ and are transverse to $\mc W$.
	Thus
	$$\left(\bigsqcup_i V_{ia} \times_M \mc W\right) \sqcup \left(\bigsqcup_i V_{ib} \times_M -\mc W\right) = \left(\bigsqcup_i V_{ia} \times_M \mc W\right) \sqcup \left(-\bigsqcup_i V_{ib} \times_M \mc W\right) \in Q^*(M)$$
	by \cref{L: pullback with Q}, and similarly if one or both occurrences of $\pm \mc W$ are components of one of the $W'_{\ell, E}$, in which case the corresponding $V_{ia}$ or $V_{ib}$ is replaced with $V'_{k\ell}$ or $V'_{k\ell}$ .
	Continuing in this way, all of $$\left(\bigsqcup_{i,j} V_{ij} \times W_{j,E}\right) \sqcup \left(-\bigsqcup_{k,\ell} V'_{k\ell} \times_M W'_{\ell,E}\right)$$
	is in $Q^*(M)$.

	So it remains to show that
	\begin{equation}\label{E: multicup NI}
		\left(\bigsqcup_{i,j} V_{ij} \times W_{ji,NI} \right) \sqcup \left(- \bigsqcup_{k,\ell} V'_{k\ell} \times_M W'_{\ell k,NI}\right)
	\end{equation}
	is in $Q^*(M)$.
	As each $W_{ji,NI}$ and $W'_{\ell k,NI}$ has small rank, each component of \eqref{E: multicup NI} is of small rank by \cref{L: pullback with Q}.
	So it suffices to show that the boundary of \eqref{E: multicup NI} is a union of trivial and small rank precochains.
	The boundary terms of the form $(\bd V_{ij}) \times W_{ji,NI}$ and $(\bd V'_{k\ell}) \times_M W'_{\ell k,NI}$ all have small rank, again by \cref{L: pullback with Q}.
	So we consider
	$$\left(\bigsqcup_{i,j} V_{ij} \times \bd W_{ji,NI} \right) \sqcup \left(-\bigsqcup_{k,\ell} V'_{k\ell} \times_M \bd W'_{\ell k,NI}\right)$$
	(we can ignore the sign from the boundary formula, as all terms are multiplied by the same sign $(-1)^{m-v}$ in taking the boundary).

	We now consider the essential decompositions of the $\bd W_{ji,NI}$ and $\bd W'_{\ell k,NI}$.
	By \cref{L: pullback with Q}, any fiber product involving a TI component will be trivial and any fiber product involving an NI component will have small rank.
	So we must consider the terms $V_{ij} \times \left(\bd W_{ji,NI}\right)_E$ and $V'_{ij} \times \left(\bd W'_{ji,NI}\right)_E$.
	By \cref{L: same NI}, since $W_{ji}$ and $W_{ja}$ represent the same cochain for any $i,a$, we have that $\bd W_{ji,NI}$ and $\bd W_{ja,NI}$ represent the same cochain (and similar for the $W'$).
	So by \cref{C: Q essential}, these have the same minimal essential part and $\left(\bd W_{j,NI}\right)_E$ is the minimal essential part together with something trivial.
	As any fiber product with something trivial is trivial, we can concentrate on the minimal essential part; we now abuse notation and let $\left(\bd W_{j,NI}\right)_E$ stand just for the minimal essential part.
	So if we can show that
	$$\left(\bigsqcup_j \bd W_{j,NI}\right) \sqcup -\left(\bigsqcup_\ell \bd W'_{\ell,NI}\right)$$
	is in $Q^*(M)$ then we can proceed by the same argument we used above for $\left(\bigsqcup_j W_{j,E}\right) \sqcup -\left(\bigsqcup_\ell W'_{\ell,E}\right)$ to show that $\left(\bigsqcup_{i,j} V_{ij} \times W_{j,E}\right) \sqcup \left(-\bigsqcup_{k,\ell} V'_{k\ell} \times_M W'_{\ell,E}\right) \in Q^*(M)$.

	But, again, we know that
	$\left(\bigsqcup_j W_{j}\right) \sqcup \left(-\bigsqcup_\ell W'_{\ell}\right) \in Q^*(M)$, so by \cref{L: Lipy12}
	$$\left(\bigsqcup_j \left(W_{j,E} \sqcup W_{j,NI}\right)\right) \sqcup \left(-\bigsqcup_\ell \left(W'_{\ell,E} \sqcup W'_{\ell,NI}\right)\right) \in Q^*(M),$$
	as the $W_{j,TI}$ and $W'_{\ell, TI}$ are in $Q^*(M)$.
	Thus by \cref{L: Q essential}, $$\left(\bigsqcup_j W_{j,NI}\right) \sqcup \left(-\bigsqcup_\ell W'_{\ell,NI}\right) \in Q^*(M).$$
	So by \cref{L: bd defined}, the boundary
	$$\left(\bigsqcup_j \bd W_{j,NI}\right) \sqcup \left(-\bigsqcup_\ell \bd W'_{\ell,NI}\right)$$ is in $Q^*(M)$, as required.
\end{proof}



\begin{comment}
	\begin{corollary}
		Let $M$ be a manifold without boundary.
		Suppose $\underline{V_1}, \underline{V_2} \in C_\Gamma^*(M)$ are both simply transverse to $\uW \in C_\Gamma^*(M)$.
		Then $\underline{V_1 + V_2}$ is compound transverse to $\uW$ and $$\underline{V_1 + V_2} \uplus \uW = \underline{V_1} \uplus \uW + \underline{V_2} \uplus \uW.$$
		The equivalent statements hold for the cap and intersection products and with the sum in the second variable.
	\end{corollary}
	\begin{proof}
	This follows immediately from the definitions and \cref{D: cochain products,D: multicup,T: multicup}, taking in the theorem statement $\uV = \underline{V_1} + \underline{V_2}$ and letting $\uW$ simply be itself without splitting it into a sum.
	\end{proof}
\end{comment}

\cref{T: multicup} can be read to imply bilinear behavior of \textit{simply} transverse chains and cochains, although this is somewhat definitional: \cref{D: multicup} \textit{defines} $\uV \uplus (\underline{W_1} + \underline{W_2})$ as  $(\uV \uplus \underline{W_1}) + (\uV \uplus \underline{W_2})$, where the cup products on the right are those of \cref{D: cochain products}, and then \cref{T: multicup} tells us this is well defined.

The conundrum is that compound transverse chains and cochains again do not necessarily behave bilinearly, for essentially the same reasons that simple transversality does not provide bilinear products without extending to a broader definition of transversality:
The definition of compound transversality assumes fixed decompositions $\uV = \sum_i \underline{V_i}$ and $\uW = \sum_j \underline{W_j}$.
But now suppose we have $\uV \in C_\Gamma^*(M)$ that is \textit{compound} transverse to both $\underline{W_1}, \underline{W_2} \in C_\Gamma^*(M)$.
Again we would like $\uV \uplus (\underline{W_1} + \underline{W_2})  = (\uV \uplus \underline{W_1}) + (\uV + \underline{W_2})$, but $\uV \uplus \underline{W_1}$ and $\uV \uplus \underline{W_2}$ might be computed by the formula above using two different ways of writing $\uV$ as a sum, and \cref{T: multicup} does not cover this situation.

Conjecturally, we could mirror the preceding program by introducing an appropriate even broader notion of transversality, showing it is well defined, and then declaring it to provide a notion of bilinearity for products of compound transverse chains and cochains.
Even more conjecturally, this process can be repeated to all levels of ``$n$-transversality'' (with $n=1$ being simple transversality, $n=2$ being compound transversality, etc.) to provide well defined ``level $n$ products'' that definitionally provide bilinearity at the preceding the level.
But we will not pursue this project here.

\subsubsection{Exterior products}\label{S: exterior chain products}

As a complement to \cref{S: simple products}, we observe in this section that the exterior products defined in \cref{S: exterior products} give rise to well-defined products for geometric chains and cochains.
Unlike the cup, cap, and intersection products, these are fully defined, as the exterior products do not require any transversality assumptions.

\begin{definition}\label{D: exterior chain}
	Suppose $\uV \in C_*^{\Gamma}(M)$ and $\uW \in C_*^{\Gamma}(N)$ are represented by $V \in PC_*^{\Gamma}(M)$ and $W \in PC_*^{\Gamma}(N)$.
	Then we define the \textbf{exterior chain product (or chain cross product)} $$\times \colon C_*^{\Gamma}(M) \times C_*^{\Gamma}(N) \to C_*^{\Gamma}(M \times N)$$ by $\uV \times \uW = \underline{V \times W}$, giving the product of oriented manifolds the standard product orientation, as in \cref{S: oriented product}.

	Similarly, suppose $\uV \in C^*_{\Gamma}(M)$ and $\uW \in C^*_{\Gamma}(N)$ are represented by $V \in PC^*_{\Gamma}(M)$ and $W \in PC^*_{\Gamma}(N)$.
	Then we define the \textbf{exterior cochain product (or cochain cross product)}
	$$\times \colon C^*_{\Gamma}(M) \times C^*_{\Gamma}(N) \to C^*_{\Gamma}(M \times N)$$ by $\uV \times \uW = \underline{V \times W}$, using the co-orientation of a product of co-oriented maps as defined in \cref{D: co-oriented exterior}.
\end{definition}

As is standard for singular homology and cohomology, we use the symbol $\times$ for both products, allowing context to determine which product is meant.

\begin{proposition}
	The exterior chain and cochain products are well defined.
\end{proposition}

\begin{proof}
	We note that product of proper maps is proper by \cite[Proposition I.10.1.4]{Bou98}.

	It remains to show that if $V'$ and $W'$ are alternative representatives of $V$ and $W$ then $(V \times W) \sqcup -(V' \times W') \in Q(M \times N)$.
	We will show that $(V \times W) \sqcup -(V' \times W) \in Q(M \times N)$, then the general case follows from an equivalent argument with $W$.
	But we need only observe that $(V \times W) \sqcup -(V' \times W) = (V \sqcup -V') \times W$ and then apply
	\cref{L: exterior Q}.
\end{proof}


\subsubsection{Pullbacks of cochains}\label{S: product pullbacks}
\greg{Should this section go here?}
We now return to the main program of \cref{S: chain products} --- the behavior of geometric chains and cochains, and particularly the need for transversality assumptions.

In \cref{S: cohomology pullback}, we showed that a continuous map $f \colon M \to N$ of manifolds without boundary yields a well-defined cohomology map $f^* \colon H^*_\Gamma(N) \to H^*_\Gamma(M)$.
In this section we consider $f^*$ as a partially-defined map of cochain complexes $C^*_\Gamma(N) \to C^*_\Gamma(M)$.

\begin{definition}\label{D: transverse to map}
	Let $f \colon M \to N$ be a smooth map of manifolds without boundary, and let $\uV \in C^*_\Gamma(N)$.
	We will say that $\uV$ is \textbf{transverse to $f$} if $\uV$ has a representative $r_V \colon V \to N$ such that $r_V$ is transverse to $f$.
	In this case we define the pullback $f^*(\uV)$ to be $\underline{f^*(V)} \in C^*_\Gamma(M)$.

	We will write the set of cochains transverse to $h$ as $C^*_{\Gamma \pf f}(N)$.
\end{definition}

We notice that the transversality situation here is simpler than the more general ones in the preceding section, as $f$ is a fixed map.

\begin{proposition}
	Given a smooth map of manifolds without boundary $f \colon M \to N$, the set $C^*_{\Gamma \pf f}(N)$ is a subcomplex of $C^*_{\Gamma}(N)$, and the map $f^*:C^*_{\Gamma \pf f}(N) \to C^*_{\Gamma}(M)$ is a well-defined chain map.
\end{proposition}

\begin{proof}
	To show that $f^*$ is well defined on $C^*_{\Gamma \pf f}(N)$ we must show that it does not depend on the choice of representative $V$.
	Suppose $V$ and $V'$ both represent $\uV$ and are transverse to $f$.
	Then $V \sqcup -V'$ is transverse to $f$ and an element of $Q^*(N)$.
	So by \cref{L: pullback map Q}, $f^*(V \sqcup -V')$, which is by definition $(V \sqcup -V') \times_N M = (V \times_N M) \sqcup (-V' \times_N M)$ mapping to $M$, is an element of $Q^*(M)$.
	So $f^*(V)$ and $f^*(V')$ represent the same element of $C^*_{\Gamma}(M)$.
	Thus $f^*$ is well defined.

	If $\uV,\uW \in C^*_\Gamma(N)$ are represented by $r_V \colon V \to N$ and $r_W \colon W \to N$ that are transverse to $f$, then $\uV+\uW$ can be represented by $V \sqcup W$, which will also be transverse to $f$.
	So $C^*_{\Gamma \pf f}(N)$ is closed under addition.
	If $r_V \colon V \to N$ is transverse to $f$ then so is $-r_V$, i.e.\ $r_V$ with the opposite co-orientation, so $C^*_{\Gamma \pf f}(N)$ is closed under negation.
	The empty map $\emptyset \to N$ is always transverse to $f$ (since there are no points at which to check the tangent space condition), and so $0 \in C^*_{\Gamma \pf f}(N)$.
	Finally, if $\uV$ is represented by $r_V \colon V \to N$ transverse to $f$, then by definition $\bd V \to N$ is transverse to $f$, so $\bd \uV \in C^*_{\Gamma \pf f}(N)$.
	Therefore, $C^*_{\Gamma \pf f}(N)$, is a subcomplex of $C^*_{\Gamma}(N)$.

	To see that $f^*$ is a homomorphism, let $V,W \to N$ represent elements of $C^*_{\Gamma}(N)$ that are transverse to $f$.
	Then
	$$f^*(\uV+\uW) = \underline{f^*(V \sqcup W)} = \underline{f^*(V) \sqcup f^*(W)} = \underline{f^*(V)}+\underline{f^*(W)},$$
	using the definitions and obvious properties of the pullback.
	Furthermore, as $\bd M = \emptyset$, $f^*$ is a chain map by \cref{leibniz}, .
\end{proof}

\begin{remark}
	While $C^*_{\Gamma \pf f}(N)$ is a subcomplex, it is not closed under taking cup products, even when they are well defined.
	As an example, let $f \colon M \to N$ be the inclusion of the $x$-axis into the plane $\R^2$.
	Let $V$ be represented by an embedding of $\R$ into $\R^2$ as the line $y = x$, and let $W$ similarly correspond to $y = -x$, with any co-orientations.
	Then $\uV \uplus \uW$ is represented by the embedding of the origin into $\R^2$, but this is not transverse to $f$, even though both $V$ and $W$ are transverse to $f$.
\end{remark}

\subsubsection{Kronecker pairing}\label{S: Kronecker}
\greg{Should this section go here?}
Using similar arguments to those in \cref{S: product pullbacks}, we consider a partially-defined Kronecker-type evaluation $C^*_\Gamma(M) \to \Hom(C_*^\Gamma(M),\Z)$.

The partially-defined cap product yields a partially-defined pairing
$$C^i_\Gamma(M) \times C_i^\Gamma(M) \xr{\nplus} C_0^\Gamma(M) \xr{\aug}\Z,$$
where $\aug \colon C_0^\Gamma(M) \to \Z$ is the augmentation map of \cref{D: aug}.
We consider here the extent to which this pairing corresponds to a function $C^i_\Gamma(M) \to \Hom(C_i^\Gamma(M),\Z)$.
This situation is closely related to the preceding discussion of pullbacks.

\begin{definition}\label{D: transverse to cohain}
	Let $\uV \in C^i_\Gamma(M)$ be a geometric cochain.
	We write $C_i^{\Gamma \pf \uV}(M)$ for the subgroup of $C_i^\Gamma(M)$ generated by geometric $i$-chains simply transverse to $\uV$.
\end{definition}

\begin{proposition}
	Given a geometric cochain $\uV \in C^i_\Gamma(M)$, the map $\aug(\uV\nplus -):C_i^{\Gamma \pf \uV}(M) \to \Z$ is a well-defined homomorphism.
\end{proposition}

\begin{proof}
	We first observe that $\uV\nplus -$ is defined on all elements of $C_i^{\Gamma \pf \uV}(M)$.
	If $\uW \in C_i^\Gamma(M)$ can be written as a sum $\uW = \sum \underline{W_i}$ with each $\underline{W_i}$ simply transverse to $\uV$, then $\uV\nplus \uW$ is well defined as $\sum \uV\nplus \underline{W_i}$ by \cref{T: multicup}.
	The element $0 \in C_i^\Gamma(M)$, as represented by the empty map, is simply transverse to $\uV$ with $\uV\nplus 0 = 0$, and if $\uW$ is transverse to $\uV$ then so is $-\uW$.
	\cref{D: multicup,T: multicup} imply that $\uV \nplus -$ is a homomorphism.
	We know that $\aug$ is a homomorphism, so the proposition follows.
\end{proof}

So given $\uV \in C^i_\Gamma(M)$, we obtain an element of $\Hom\left(C_i^{\Gamma \pf \uV}(M),\Z\right)$, but of course we will not in general obtain an element of $\Hom\left(C_i^{\Gamma}(M), \Z \right)$ due to transversality requirements.

\begin{comment}
	\begin{proof}
		If $\uV,\uW \in C^*_\Gamma(N)$ are represented by $f \colon V \to N$ and $g \colon W \to N$ that are transverse to $h$, then $\uV+\uW$ is represented by $V \sqcup W$, which will also be transverse to $h$.
		So $C^*_{\Gamma \pf h}(N)$ is closed under addition.
		If $f \colon V \to N$ is transverse to $h$ then so is $-f$, i.e.\ $f$ with the opposite co-orientation, so $C^*_{\Gamma \pf h}(N)$ is closed under taking negatives.
		The empty map $\emptyset \to N$ is always transverse to $h$ (since there are no points at which to check the tangent space condition), and so $0 \in C^*_{\Gamma \pf h}(N)$.
		Finally, if $\uV$ is represented by $f \colon V \to N$ transverse to $h$, then by definition $\bd V \to N$ is transverse to $h$, so $\bd \uV \in C^*_{\Gamma \pf h}(N)$.
		Therefore, $C^*_{\Gamma \pf h}(N)$, is a subcomplex of $C^*_{\Gamma}(N)$.

		To how that $h^*$ is well defined on $C^*_{\Gamma \pf h}(N)$ we must show that it does not depend on the choice of representative $V$.
		Suppose $V$ and $V'$ both represent $\uV$ and are transverse to $h$.
		Then $V \sqcup -V'$ is transverse to $h$ and an element of $Q^*(N)$.
		So by \cref{L: pullback map Q}, $h^*(V \sqcup -V')$, which is by definition $(V \sqcup -V') \times_N M = (V \times_N M) \sqcup (-V' \times_N M)$ mapping to $M$, is an element of $Q^*(M)$.
		So $h^*(V)$ and $h^*(V')$ represent the same element of $C^*_{\Gamma}(M)$.
		Thus $h^*$ is well defined.

		To see that $h^*$ is a homomorphism, let $V,W \to N$ represent elements of $C^*_{\Gamma}(N)$ that are transverse to $h$.
		Then
		$$h^*(\uV+\uW) = \underline{h^*(V \sqcup W)} = \underline{h^*(V) \sqcup h^*(W)} = \underline{h^*(V)}+\underline{h^*(W)},$$
		using the definitions, obvious properties of the pullback and \cref{L: co/chains well defined}.
		Furthermore, $h^*$ is a chain map by \cref{leibniz}, as $\bd M = \emptyset$.
	\end{proof}
\end{comment}


\subsection{Properties of the chain and cochain products}\label{S: (co)chain properties}

Now that we have defined cup, cap, intersection, and exterior products of geometric chains and cochains and shown that these products are well defined, at least when the necessary transversality and orientation conditions hold, they immediately inherit many of the properties demonstrated in \cref{S: orientations and co-orientations}.
We provide below some tables listing these properties and the locations of the previous results that support them.
The references are typically to results that involve only transversality of a pair of prechains or precochains, but in the chain and cochain setting they generalize to the more general products of \cref{D: multicup} by applying them to each summand.

For example, suppose $\uV, \uW \in C^*_\Gamma(M)$ are compound transverse.
This means we can write $\uV = \sum_i \underline{V_i} \in C_\Gamma^*(M)$, $\uW = \sum_j \underline{W_j} \in C_\Gamma^*(M)$ with each pair $(\underline{V_i},\underline{W_j})$ simply transverse.
And this means that there are representatives $V_{ij},W_{ji} \in PC_\Gamma^*(M)$ such that for all $i$ and $j$, we have $\underline{V_{ij}} = \underline{V_i}$, $\underline{W_{ji}} = \underline{W_j}$, and $V_{ij}$ transverse to $W_{ji}$.
We then have $\uV\uplus \uW$ represented by
$\sum_{i,j} V_{ij} \times_M W_{ji}$.
By \cref{P: graded comm}, we have
$$\sum_{i,j} V_{ij} \times_M W_{ji} = \sum_{i,j} (-1)^{(m-v)(m-w)}W_{ji} \times_M V_{ij} = (-1)^{(m-v)(m-w)}\sum_{i,j} W_{ji} \times_M V_{ij},$$
and this last expression represents $(-1)^{(m-v)(m-w)} \uW\uplus \uV$.
So we obtain the cup product commutativity formula $$\uV\uplus\uW = (-1)^{(m-v)(m-w)}\uW\uplus\uV$$ for compound transverse cochains.

The more complicated exceptions to this inheritance of properties from the pre(co)chain properties concern associativity and naturality, which we will address below in a separate section.

In the tables that follow, we assume to hold all transversality required for each expression to be defined.
For intersection products, we assume that the underlying manifold is oriented.
Unless stated otherwise, our default notations for cup, cap, and intersection products will have manifolds with corners $V$ and $W$ mapping to a manifold without boundary $M$.
Our default notations for chain and cochain cross products will assume $V \to M$ and $W \to N$.
We explain the further assumptions and notations prior to each table of formulas

\subsubsection{Boundary formulas}

For our first table, with formulas involving boundaries, we also invoke the well-definedness of boundaries of geometric chains and cochains, see \cref{L: co/chains well defined}.
The cup, cap, and intersection products require (simple or compound) transversality of $\uV$ and $\uW$; the exterior products have no transversality requirements.
In the first line, we use that $\uV \times \uW = \uV \times_M \uW$, when $M$ is a point.

\begin{center}
	\begin{tabular}{|l|c|l|}
		\hline
		Chain cross product &$\bd(\uV \times \uW) = (\bd \uV) \times \uW+ (-1)^{v}\uV \times \bd \uW$&\cref{P: oriented fiber boundary}\\
		\hline
		Cochain cross product&$\bd(\uV \times \uW) = (\bd \uV) \times \uW+ (-1)^{m-v}\uV \times \bd \uW$&\cref{P: boundary of exterior product}\\
		\hline
		Cup product&$\bd (\uV \uplus \uW) = (\bd \uV) \uplus \uW+ (-1)^{m-v} \uV \uplus \bd \uW$&\cref{leibniz}\\
		\hline
		Cap product&$\bd(\uV\nplus \uW) = (-1)^{v+w-m} (\bd \uV)\nplus \uW + \uV\nplus\bd \uW$&\cref{P: Leibniz cap}\\
		\hline
		Intersection product &$\bd (\uV \bullet \uW) = (\bd \uV) \bullet \uW + (-1)^{m-v}\uV \bullet \bd \uW$&\cref{P: oriented fiber boundary}\\
		\hline
	\end{tabular}
\end{center}

\subsubsection{Commutativity formulas}

For the commutativity properties listed below, $\tau$ is the transposition map $\tau \colon N \times M \to M \times N$.
The cup and intersection products require transversality of $\uV$ and $\uW$; the exterior products have no transversality requirements.
In the first line, we again use that $\uV \times \uW = \uV \times_M \uW$, when $M$ is a point.

\begin{center}
	\begin{tabular}{|l|c|l|}
		\hline
		Chain cross product&$\tau(\uV \times \uW) = (-1)^{vw}\uW \times \uV$&\cref{P: commute oriented fiber}\\
		\hline
		Cochain cross product&$\tau^*(\uV \times \uW) = (-1)^{(m-v)(n-w)}\uW \times \uV$&\cref{P: exterior commutativity}\\
		\hline
		Cup product&$\uV\uplus \uW = (-1)^{(m-v)(m-w)} \uW\uplus \uV$&\cref{P: graded comm}\\
		\hline
		Intersection product&$\uV\bullet \uW = (-1)^{(m-v)(m-w)}\uW\bullet \uV$&\cref{P: commute oriented fiber}\\
		\hline
	\end{tabular}
\end{center}

\subsubsection{Unital properties}\label{S: unital properties}

For the following unital properties, we write $pt$ to refer to the point with its positive orientation.
We will write $\underline{pt}$ for the geometric chain given by $\id_{pt}:pt \to pt$ or for the geometric cochain given by the canonically co-oriented identity map $\id_{pt}:pt \to pt$.
Similarly, $\uM$ represents the geometric chain or cochain determined by $\id_M \colon M \to M$, canonically co-oriented in the cochain case.
Technically, $M$ must be compact for $\id_M$ to represent a chain, but the corresponding formulas hold more broadly at the referenced locations and so these identities could be taken as statements involving a broader class of geometric chains.
Note that, as a cochain, $\uM \in C^0_\Gamma(M)$, and \cref{P: projection pullbacks} shows that these behave like the singular cochain $1$.
We also let $\pi_1 \colon M \times N \to M$ and $\pi_2 \colon N \times M \to M$ denote the projections.
In the first formula for the cap product with $\uM$, $M$ is assumed oriented, and the first $\uV$ in the formula is represented by $V \to M$ as a cochain while the second $\uV$ in the formula is $V \to M$ as a chain with the induced orientation on $V$; see \cref{P: cap with identity M}.
In the second cap product formula, both instances of $\uV$ are as chains.
As $\id_M$ is transverse to all other maps, the following hold for all $\uV$.

\begin{center}
	\begin{tabular}{|l|c|l|}
		\hline
		Identity for chain cross product&$\uV \times \underline{pt} = \underline{pt} \times \uV = \uV$& Straightforward\\
		\hline
		Identity for cochain cross product&$\uV \times \underline{pt} = \underline{pt} \times \uV = \uV$& \cref{P: co-oriented exterior unit}\\
		\hline
		Cochain cross product with $1$&\begin{tabular}{c}$\pi_1^*\uV = \uV \times \underline{N}$\\$\pi_2^*\uV = \underline{N} \times \uV$ \end{tabular} &\cref{P: projection pullbacks}\\
		\hline
		Cup product with $1$&$\uV\uplus\uM = \uM\uplus \uV = \uV$&\cref{C: cup with identity}\\
		\hline
		Cap product with $\underline M$&$\uV\nplus \uM = \uV$ &\cref{P: cap with identity M}\\
		\hline
		Cap with product with 1&$\uM\nplus \uV = \uV$&\cref{P: cap with 1}\\
		\hline
		Intersection product with $\uM$ &$\uM\bullet \uV = \uV\bullet \uM = \uV$&\cref{P: oriented fiber product basic properties}\\
		\hline
	\end{tabular}
\end{center}

\subsubsection{Mixed properties}\label{S: mixed formulas}

The next grouping concerns properties that involve multiple kinds of products.
We recall that $\diag \colon M \to M \times M$ is the diagonal map.
For these properties we assume maps $V,W \to M$ and $X,Y \to N$.
We also have projections $\pi_M \colon M \times N \to M$ and $\pi_N \colon M \times N \to N$.
As in \cref{S: unital properties}, $\uM$ represents the geometric chain or cochain determined by $\id_M \colon M \to M$.
Then the last formula follows from \cref{P: compare cup and intersection orientations} by recalling from \cref{P: cap with identity M} that, when $M$ is oriented, the fiber product with $\id_M \colon M \to M$ takes a precochain to the corresponding chain with the induced orientation.
Again, $M$ must be compact for $\id_M$ to represent a chain, but the corresponding formulas hold more broadly at the referenced locations and so these identities could be taken as statements involving a broader class of geometric chains.

The first and last properties require that $\uV$ and $\uW$ be transverse.
The second holds for all $\uV, \underline{X}$.
The next three require that $\uV$ be transverse to $\uW$ and that $\underline{X}$ be transverse to $\underline{Y}$.



\begin{center}
	\begin{tabular}{|l|c|l|}
		\hline
		Cup from cross& $\uV\uplus \uW = \diag^*(\uV \times \uW)$&\ref{P: cross to cup}	\\
		\hline
		Cross from cup&$\uV \times \underline{X} = \pi_M^*(\uV)\uplus\pi_N^*(\underline{X})$& \ref{C: cross is cup}	\\
		\hline
		Cup of crosses&$(\uV \times \underline{X})\uplus (\uW \times \underline{Y}) = (-1)^{(m-w)(n-x)} (\uV\uplus \uW) \times (\underline{X}\uplus \underline{Y})$ &	\ref{C: criss cross}\\
		\hline
		Cap of crosses &$(\uV \times \underline{X})\nplus (\uW \times \underline{Y}) = (-1)^{(x+y-n)(m-v)} (\uV \nplus \uW) \times (\underline{X}\nplus \underline{Y})$ & \ref{P: cap cross}\\
		\hline
		Intersect. of crosses &$(\uV \times \underline{X})\bullet (\uW \times \underline{Y}) = (-1)^{(m-w)(n-x)}(\uV\bullet \uW) \times (\underline{X}\bullet \underline{Y})$&\ref{P: oriented interchange}\\
		\hline
		Cup and intersect. &$(\uV\uplus \uW)\nplus \uM = (-1)^{(m-v)(m-w)}(\uV\nplus \uM)\bullet(\uW\nplus \uM)$&\ref{P: compare cup and intersection orientations}\\
		\hline
	\end{tabular}
\end{center}

\subsubsection{Immersion formulas}

While geometric chains and cochains do not have unique representatives by prechains and precochains, we recall that if the two terms can be represented by transverse embeddings, then the cup, cap, and intersection products are represented by their intersection and we have simple formulas for the orientations or co-orientations; see \cref{P: normal pullback,P: cap of immersions,P: orient intersection}, respectively.
For cap and intersection products, the special cases where the (co)chains have complementary dimensions are further specified in \cref{C: complementary cap,C: orient complementary intersection}.

\subsubsection{Naturality and associativity formulas}

Formulas for associativity and naturality of geometric chain and cochain products are more delicate than our preceding formulas because they require sufficient transversality of more than two objects.
This would require some careful assumptions even just for maps of manifolds, but the ambiguity in representation of geometric chains and cochains makes the situation even more complicated.
To start, there is the question of which transversality we mean, as we have defined both simple and compound transversality for chains and cochains in \cref{S: simple products}.
If we limited ourselves to simple transversality, then we would be able to invoke results like \cref{C: fiber assoc} fairly directly, though even here there are a number of conditions that must be met.

If we want to work with compound transversality, the situation quickly becomes much more complicated.
For example, if $\uV$ and $\uW$ are compound transverse, then by definition we can write finite sums $\uV = \sum_i \underline{V_i}$ and $\uW = \sum_j \underline{W_j}$ and then find transverse representatives $V_{ij}$ and $W_{ji}$ of $\underline{V_i}$ and $\underline{W_j}$.
Then $\uV\uplus \uW$ is represented by $\sum_{ij}V_{ij} \times_M W_{ji}.$
Now suppose $Z$ is compound transverse to $\uV\uplus \uW$.
Then there must be similar decompositions of $Z$ and $\uV\uplus \uW$ into simply transverse pairs, but it is not clear that this condition can necessarily be written in terms of the $V_{ij}$ and $W_{ji}$ so that we can then take advantage of \cref{C: fiber assoc}.
So, rather than attempt to pursue the most general case, we will impose some extra restrictions in what follows so that we can utilize \cref{C: fiber assoc} and its analogues for the intersection product and for the cap product with a cup product.
These assumptions can be simplified if we only wish to consider simple transversality.

Similar concerns arise for our naturality formulas, as, for example, pulling back a cup product by a map $h$ requires that $h$ be transverse to the cup product, so we again have an interaction of three maps, leading to similar issues.

\textbf{Naturality.}
As noted above, naturality of cup and cap products requires some extra care to ensure not just that chains and cochains are appropriately transverse but that there are also the appropriate transversalities with respect to the maps we pull back by.
This requires a good number of further assumptions; see \cref{R: multiproducts,P: 3 out of 4 trans}.

So suppose first $h \colon N \to M$ is a map of manifolds without boundaries and that $\uV,\uW \in C^*_{\Gamma}(M)$.
For naturality of cup products we will assume not just that $\uV$ and $\uW$ are compound transverse, but we also require decompositions into finite sums $\uV = \sum_i \underline{V_i}$, $\uW = \sum_j \underline{W_j}$ such that each pair $(\underline{V_i},\underline{W_j})$ has representatives $V_{ij}$ and $W_{ji}$ such that:
\begin{itemize}
	\item $V_{ij}$ and $W_{ji}$ are transverse and
	\item $V_{ij}$, $W_{ji}$, and $V_{ij} \times_M W_{ji}$ are all transverse to $h$.
\end{itemize}
If we assume that $\uV$ and $\uW$ are simply transverse, then we need only representatives $V$ and $W$ such that $V$ and $W$ are transverse and, $V$, $W$, and $V \times_M W$ is transverse to $h$.

For naturality of cap products, we suppose $h \colon N \to M$, $\uV \in C^*_\Gamma(M)$, and $\uW \in C_*^\Gamma(N)$.
Then, via \cref{P: natural cap}, it is sufficient to assume decompositions $\uV = \sum_i \underline{V_i}$ and $\uW = \sum_j \underline{W_j}$ with representative pairs $(V_{ij}, W_{ji})$ such that each $V_{ij}$ is transverse to $h$ and each $W_{ji}$ is transverse to the pullback $V_{ij} \times_M N \to N$.
In the simple version we just need $V$ and $W$ with $V$ transverse $h$ and $W$ transverse to $V \times_M N \to N$.

The exterior products are simpler.
The naturality of the chain cross product requires no assumption, while the naturality of the cochain cross product requires only that $V$ and $W$ have representatives that are respectively transverse to $h$ and $k$.

With these assumptions, we have the following formulas.

\begin{center}
	\begin{tabular}{|l|c|l|}
		\hline
		Chain cross product&$(h \times k)(\uV \times \uW) = h(\uV) \times k(\uW)$ &Straightforward\\
		\hline
		Cochain cross product&$(h \times k)^*(\uV \times \uW) = h^*(\uV) \times k^*(\uW)$ &\cref{P: natural exterior}\\
		\hline
		Cup product &$h^*(\uV\uplus \uW) = h^*(\uV) \uplus h^*(\uW)$&\cref{C: fiber natural pullback}\\
		\hline
		Cap product &$\uV \nplus h(\uW) = h(h^*(\uV)\nplus \uW)$&\cref{P: natural cap}\\
		\hline
	\end{tabular}
\end{center}

\medskip\noindent\textbf{Associativity.}

For the associativity formulas we add a manifold with corners $X$ either mapping to $M$ for the cup, cap, and intersection products or to a third target manifold $Q$ for the cross products.
Once again, there are no special requirements for the exterior products in the first two formulas below.
For the other associativity formulas, while this might not encompass the most general possibility, in order to ensure associativity in a compound transversality setting, we assume decompositions into finite sums $\uV = \sum_i \underline{V_i}$, $\uW = \sum_j \underline{W_j}$, and $\underline{X} = \sum_k \underline{X_k}$ such that for each triple $(\underline{V_i},\underline{W_j}, \underline{X_k})$ there are representatives $V_{i,jk}$, $W_{j,ik}$, and $X_{k,ij}$ such that the following pairs are transverse: $(V_{i,jk}, W_{j,ik})$, $(W_{j,ik},X_{k,ij})$, $(V_{i,jk} \times_M W_{j,ik},X_{k,ij})$, and $(V_{i,jk}, W_{j,ik} \times_M X_{k,ij})$.
These assumptions will allow us to invoke \cref{C: fiber assoc,P: OC mixed associativity,P: oriented fiber mixed associativity}.
For simple transversality versions, we need only assume $V, W, X$ such that $(V, W)$, $(W,X)$, $(V \times_M W,X)$, and $(V, W \times_M X)$ are transverse pairs (and by \cref{P: 3 out of 4 trans}, the condition on the last pair is redundant).

We leave the reader to formulate associativity for products of larger collections of maps.

\begin{center}
	\begin{tabular}{|l|c|l|}
		\hline
		Chain cross product& $(\uV \times \uW) \times \uX = \uV \times (\uW \times \uX)$&Straightforward\\
		\hline
		Cochain cross product& $(\uV \times \uW) \times \uX = \uV \times (\uW \times \uX)$&\cref{P: exterior associativity}\\
		\hline
		Cup product &$(\uV\uplus \uW)\uplus\uX = \uV\uplus(\uW\uplus X)$&\cref{C: fiber assoc} \\
		\hline
		Cup/cap & $(\uV \uplus \uW)\nplus \uX = \uV\nplus(\uW\nplus\uX)$& \cref{P: OC mixed associativity}\\
		\hline
		Intersection product &
		$(\uV\bullet\uW)\bullet\uX = \uV\bullet(\uW\bullet \uX)$&\cref{P: oriented fiber mixed associativity} and following\\
		\hline
	\end{tabular}
\end{center}

We note that with our assumptions, these triple products exhibit a form of linearity in each variable akin to that discussed in \cref{S: simple products}.

\subsection{Homology and cohomology products}\label{S: homology products}

In this section we observe that the partially-defined cup, cap, and intersection products of geometric chains and cochains give rise to fully-defined products of geometric homology and cohomology classes.
Similarly, we obtain external homology and cohomology products, although this is more evident as external products are already fully defined for geometric chains and cochains.

\begin{theorem}\label{T: (co)homology products}
	Let $M$ and $N$ be manifolds without boundary.
	The chain cross product, cochain cross product, cup product, cap product, and, if $M$ is oriented, intersection product induce fully-defined bilinear maps
	\begin{align*}
		\times \colon & H^\Gamma_*(M) \otimes H^\Gamma_*(N) \to H^\Gamma_*(M \times N)\\
		\times \colon & H_\Gamma^*(M) \otimes H_\Gamma^*(N) \to H_\Gamma^*(M \times N)\\
		\uplus \colon & H_\Gamma^*(M) \otimes H_\Gamma^*(M) \to H_\Gamma^*(M)\\
		\nplus \colon & H_\Gamma^*(M) \otimes H^\Gamma_*(M) \to H^\Gamma_*(M)\\
		\bullet \colon & H^\Gamma_*(M) \otimes H^\Gamma_*(M) \to H^\Gamma_*(M).\\
	\end{align*}
\end{theorem}

We will prove \cref{T: (co)homology products} below.
The basic idea, in the case of the last three products, will be to show that we can represent pairs of homology or cohomology classes by simply transverse representatives, which is accomplished by \cref{T: transverse reps}, below, and then form the usual fiber products.
We therefore have the following immediate consequence.

\begin{theorem}
	The homology cross product, cohomology cross product, cup product, cap product, and, if $M$ is oriented, intersection product satisfy the properties enumerated in \cref{S: (co)chain properties}, except for the boundary formulas.
\end{theorem}

\begin{remark}
	For the naturality and associativity properties which require additional assumptions about transversality of representatives, those assumptions can all be met in the (co)homology setting.
	For example, for naturality of the cup product with respect to a map $h \colon N \to M$, once transverse representatives $V$ and $W$ of $\uV$ and $\uW$ have been found, we can use \cref{T: basic trans,R: countable trans} to replace $h$ with a homotopic map transverse to $V$, $W$, and $V \times_M W$, and by \cref{P: cohomology pullback}, we can use this transverse map to pull back our cohomology classes.
	For the cap product, we can again use \cref{T: basic trans} to assume $h$ transverse to $V$ and then \cref{T: transverse reps}, below, to choose a $W$ in the desired homology class transverse to the pullback $V \times_M N \to N$.
	Similarly, for associativity, we use \cref{T: transverse reps,R: countable trans2} to choose representatives $V$, $W$, and $X$ first so that $W$ is transverse to $V$ and then so that $X$ is transverse to $V \times_M W$ and $W$.
\end{remark}

To prove \cref{T: (co)homology products}, we will use the following important theorem.

\begin{theorem}\label{T: transverse reps}
Let $r_W \colon W \to M$ be a proper map from a manifold with corners to a manifold without boundary, and let $\uV \in H_*^\Gamma(M)$ or $\uV \in H^*_\Gamma(M)$.
Then there is an $r_V \colon V \to M$ representing $\uV$ such that $V$ is transverse to $W$.

Furthermore, if $V$ is transverse $W$ and there is a pre(co)chain $Z$ such that $\bd Z \sqcup -V \in Q(M)$, i.e.\ $\uV = 0$ as a homology or cohomology class, then such a $Z$ can be chosen so that it is transverse to $W$.
\end{theorem}

We will prove \cref{T: transverse reps} below in \cref{S: transverse maps}.
For now we use it to prove \cref{T: (co)homology products}.



\begin{comment}
Suppose V is a cycle transverse to N that bounds. Then there is a Z transverse to N that realizes it:

Choose Y so that \bd Y \sqcup  -V \in Q. Y may not be transverse.

Do a universal homotopy on Y \sqcup V so that Y becomes transverse and the trace of V is transverse. Let Y’ and V’ be the results and let W be the trace of the V homotopy. Let Z = Y’ \sqcup -W, which is transverse. Also \bd Y’ \sqcup -V’ is in Q.

Then \bd Z \sqcup -V is
(\bd Y’ \sqcup - V’ \sqcup V \sqcup -V)
Which is in Q.

\end{comment}



\begin{proof}[Proof of \cref{T: (co)homology products}]
	For the exterior products, by \cref{S: exterior chain products} we already have fully-defined maps
	\begin{align*}C^\Gamma_*(M) \times C^\Gamma_*(N)& \to C^\Gamma_*(M \times N)\\ C_\Gamma^*(M) \times C_\Gamma^*(N)& \to C_\Gamma^*(M \times N).
	\end{align*}
	These are easily seen to be bilinear and $\Z$-balanced (i.e.\ they satisfy $(a\uV) \times \uW = a(\uV \times \uW) = \uV \times a\uW$ for any $a \in \Z$).
	Moreover, these are chain maps: for the chain cross product this follows from the standard boundary formula for oriented products and our boundary conventions, \cref{Con: oriented boundary}, and for the cochain cross product this follows from \cref{P: boundary of exterior product}, recalling our indexing convention for cochains.
	The existence of the homology and cohomology cross products now follows from standard homological algebra.



	\begin{comment}
	 show that if we are given homology or cohomology classes (depending on the particular product), then they can be represented by chains or cochains $\uV$ and $\uW$ that are simply transverse and that the product does not depend on such a choice.
	The general idea of the proof is relatively standard and analogous to the proof of \cref{T: transverse complex}.
	We provide the details here modulo a technical lemma that we will prove below.
	\end{comment}

	To define the other products, we utilize \cref{T: transverse reps}.
	Given (co)homology classes represented by $r_V \colon V \to M$ and $r_W \colon W \to M$, we can find by \cref{T: transverse reps} a map $r_V' \colon V' \to M$ that represents the same (co)homology class as $V$ and is transverse to $W$.
	We then represent the product by the (oriented or co-oriented) fiber product of $r_V'$ and $r_W$.

	To show that this gives a well-defined (co)homology class, we can suppose that $r_V'' \colon V'' \to M$ is another map transverse to $r_W$ representing the same (co)homology class as $r_V \colon V \to M$.
	Suppose $r_Z:Z \to M$ with $\bd Z \sqcup V' \sqcup -V'' \in Q(M)$, provides the (co)homology.
	We must show that $V' \times_M W$ and $V'' \times_M W$ are (co)homologous.

	Now by \cref{T: transverse reps}, we can assume that $Z$ is also transverse to $W$.
	\begin{comment}
		 To do so, we claim we can find a proper universal homotopy $H \colon Z \times I \to M$ from $r_Z$ to an $r_Z':Z' \to M$ with $Z = Z'$ such that the restrictions of $H$ to $V' \times I$ and $V'' \times I$ are transverse to $r_W$ and $r_Z'$ is also transverse to $r_W$.
		We will prove this is possible in \cref{P: perturb transverse to map}.
		Then $H|_{V' \times I}$ and $H|_{V'' \times I}$ give (co)homologies from the (co)chains represented by $V'$ and $V''$ to (co)chains that are (co)homologous to each other via $r_Z'$.
		Altogether, this gives a (co)homology from $V'$ to $V''$ via maps that are transverse to $r_W$.
		Call the (co)chain representing this last (co)homology $\mc V$ so that $(\bd \mc V) \sqcup V' \sqcup -V'' \in Q(M)$ and $\mc V$ is transverse to $W$.
	\end{comment}
	So we can form $Z \times_M W$, and we have $$\bd (Z \times_M W) = \pm \left( (\bd Z) \times_M W \right) \sqcup \pm \left( Z \times_M \bd W \right)$$ via the appropriate boundary formulas, with the precise signs depending on which kind of product we are considering (see \cref{P: oriented fiber boundary,leibniz,P: Leibniz cap}).
	As $W$ represents a (co)cycle, $\bd W \in Q(M)$ by \cref{R: cycles and boundaries}, so $Z \times_M \bd W \in Q(M)$ by \cref{L: pullback with Q}.
	Meanwhile, we have
	$$((\bd Z) \times_M W) \sqcup (V' \times_M W) \sqcup -(V'' \times_M W) = \left((\bd Z) \sqcup V' \sqcup -V''\right) \times_M W,$$
	which is again in $Q(M)$ by \cref{L: pullback with Q}.
	Altogether then,
	\begin{multline*}
	\bd (Z \times_M W)  \sqcup (V' \times_M W) \sqcup -(V'' \times_M W)\\
	  = \pm \left( (\bd Z) \times_M W \right) \sqcup \pm \left( Z \times_M \bd W \right) \sqcup (V' \times_M W) \sqcup -(V'' \times_M W)\\
	 = \left[\pm Z \times_M \bd W\right] \sqcup \left[ \pm \left((\bd Z) \times_M W \right) \sqcup (V' \times_M W) \sqcup -(V'' \times_M W)\right].
	\end{multline*}
	We need to have $+(\bd Z) \times_M W$ in the bottom line of this computation, but we can obtain that by replacing $Z \times_M W$ with $-Z \times_M W$ if necessary.
	Then the bottom line will be an element of $Q(M)$.

	So $V' \times_M W$ and $V'' \times_M W$ are (co)homologous.
	\qedhere
	\begin{comment}
		To finish the proof, we need an analogue of \cref{P: ball stability} that allows us to construct the homotopy $H$.
		This is the content of \cref{P: perturb transverse to map} below.
	\end{comment}
\end{proof}

\begin{comment}
NEED TO PULL THIS OUT

The arguments of the proof of \cref{T: (co)homology products} also demonstrate the following important theorem:



\greg{Once this section is edited and fixed up, we need to pull out a theorem that says that given $W$, (1) any cocycle can be represented by a pre(co)chain $V$ transverse to $W$ and (2) given such a cocycle transverse to $W$ that bounds, we can cobound it with something transverse to $W$. For sketch of proof, see comment after this note marker
}
\end{comment}


\subsubsection{Representing (co)homology classes by transverse maps}\label{S: transverse maps}.

In this section, we prove \cref{T: transverse reps}.
The main tool will be \cref{P: perturb transverse to map}, stated below.
This proposition is analogous to \cref{P: ball stability} with the difference being that instead of making a map transverse to the faces of a cubulation we must make a map transverse to another map.
We will explain how to modify the proof of \cref{P: ball stability} to accomplish this.
We change notation slightly from that of \cref{T: transverse reps} to make it more consistent with \cref{P: ball stability}, which we hope will ease comparison of the two results for the reader.

\begin{proposition}\label{P: perturb transverse to map}
	Suppose $r_V \colon V \to M$ and $r_X \colon X \to M$ are proper maps from manifolds with corners to a manifold without boundary.
	Then there is a proper homotopy $H \colon M \times I \to M$ such that $H(-,0) = \id$ and $H(-,1)r_V \colon V \to M$ is transverse to $r_X$.

	Furthermore, given another proper map $r_W \colon W \to M$ that is transverse to $r_X$, we can choose the homotopy $H$ above so that also the resulting universal proper homotopy of $W$ given by $W \times I \xr{r_W \times \id_I} M \times I \xr{H} M$ is transverse to $r_X$.
\end{proposition}

Proving \cref{P: perturb transverse to map} will involve \cref{L: all transversality is wrt embeddings}, which, in our current notation, says that two maps $r_V \colon V \to M$ and $r_X \colon X \to M$ are transverse if and only when we replace $r_X$ with an embedding $e \colon X \into M \times \R^n$ that projects to $r_X$, then $e$ is transverse to $r_V \times \id_{\R^n}$.
So this lemma allows us to replace transversality of arbitrary maps with transversality in which one map is an embedding.
The following lemma says, roughly speaking, that when we construct such an embedding $e$, then $e(X)$ does not run off to infinity in the $\R^n$ factors over compact subsets of $M$.

\begin{comment}
	\red{GOING TO MOVE ELSEWHERE - HEREHERE}

	\begin{lemma}\label{L: all transversality is wrt embeddings}
		Let $f \colon V \to M$ and $g \colon W \to M$ be smooth maps from manifolds with corners to a manifold without boundary.
		Let $e \colon W \to M \times \R^n$ be an embedding such that $\pi e = g$, where $\pi$ is the projection $M \times \R^n \to M$.
		Then $f$ and $g$ are transverse if and only if $e$ is transverse to $f \times \id_{\R^n} \colon V \times \R^n \to M \times \R^n$.
	\end{lemma}

	\begin{proof}
		It suffices to assume that $V$ and $W$ are without boundary.
		Otherwise we can apply the following argument to each pair of strata of $V$ and $W$.

		Suppose that $f$ and $g$ are transverse, i.e.\ that if $f(v) = g(w)$ then $Df(T_vV)+Dg(T_wW) = T_{f(v)}M$.
		For each $w \in W$, we can write $e(w) = (g(w),e_\R(w)) \in M \times \R^n$.
		Now suppose $w \in W$ and $(v,z) \in V \times \R^n$ such that $e(w) = (f \times \id_{\R^n})(v,z)$.
		Then we have $(g(w),e_\R(w)) = (f(v),z)$.
		The image of the derivative of $f \times \id_{\R^n}$ at such a point will span $Df(T_vV) \oplus T_z(\R^{n}) = Df(T_vV) \oplus \R^{n}$, while the derivative of $e$ will take $a \in T_w(W)$ to $Dg(a)+ De_{\R}(a)$.
		But the image of $D(f \times \id_{\R^n})$ already includes $0 \oplus \R^{n}$, so
		subtracting off the second summand, $D(f \times \id_{\R^{n}})(T_{(v,z)}(V \times \R^n))+De(T_wW)$ contains $Dg(a)$.
		It follows that $D(f \times \id_{\R^{n}})(T_{(v,z)}(V \times \R^n))+De(T_wW)$ contains $Df(T_vV) \oplus 0$, $Dg(T_wW) \oplus 0$, and $0 \oplus \R^n$.
		Since $f$ and $g$ are transverse and $D(f \times \id_{\R^{n}})(T_{(v,z)}(V \times \R^n))+De(T_wW)$ is a vector space, it therefore contains all of $T_{f(v)}M \oplus \R^n = T_{e(w)}(M \oplus \R^n)$.
		So $f \times \id_{\R^n}$ and $e$ are transverse.

		Next suppose $f \times \id_{\R^n}$ and $e$ are transverse and that $f(v) = g(w) \in M$.
		Suppose $e(w) = (g(w),z)$.
		Then $e(w) = (f \times \id_{\R^n})(v,z)$.
		So, by definition and assumption,
		\begin{equation}\label{E: Quillen transverse}
			D(f \times \id_{\R^{n}})(T_{(v,z)}(V \times \R^n))+De(T_wW) = T_{e(w)}(M \times \R^n) = T_{f(v)}M \oplus \R^n.
		\end{equation}
		As $\pi$ is a submersion, the image of this tangent space under $D\pi$ is all of $T_{f(v)}M$.
		But $(D\pi)(De) = D(\pi e) = Dg$, so $(D\pi \circ De)(T_wW) = Dg(T_wW)$.
		Furthermore, letting $\pi_V \colon V \times \R^n \to V$ be the projection, we have $(D\pi)(D(f \times \id_{\R^{n}})) = D(\pi(f \times \id_{\R^{n}})) = D(f\pi_V) = (Df)(D\pi_V)$, so, as $D\pi_V \colon T_{(v,z)}(V \times \R^n) \to T_vV$ is surjective, we have $(D\pi)(D(f \times \id_{\R^{n}}))(T_{(v,z)}(V \times \R^n)) = Df(T_vV)$.
		So applying $D\pi$ to equation \eqref{E: Quillen transverse}, we get $Df(T_vV)+Dg(T_wW) = T_{f(v)}M$, and $f$ is transverse to $g$.
	\end{proof}

\end{comment}



\begin{lemma}\label{L: compact preimage}
	Let $r_X \colon X \to M$ be a proper map from a manifold with corners to a manifold without boundary, let $\pi_M \colon M \times \R^n \to M$ be the projection, and let $e \colon X \to M \times \R^n$ be an embedding such that $\pi_Me = r_X$.
	Then if $L \subset M$ is compact, there exists a close ball $\bar B^n_L \subset \R^n$ such that $e(r_X^{-1}(L)) \subset L \times \bar B^n_L$.
\end{lemma}

\begin{proof}
	As $L$ is compact and $g$ is proper, $r_X^{-1}(L)$ is compact.
	So $e(r_X^{-1}(L))$ is compact, as is its image under the projection $\pi_{\R^n} \colon M \times \R^n \to \R^n$.
	Let $\bar B^n_L \subset \R^n$ be a closed ball containing this projection.
	Then $\pi_Me(r_X^{-1}(L)) = r_X(r_X^{-1}(L)) \subset L$ and $\pi_{\R^n}e(r_X^{-1}(L)) \subset \bar B^n_L$.
	So $e(r_X^{-1}(L)) \subset L \times \bar B^n_L$.
\end{proof}

We can now use \cref{L: all transversality is wrt embeddings,L: compact preimage} to augment the proof of \cref{P: ball stability} to a proof of \cref{P: perturb transverse to map}.

\begin{proof}[Proof of \cref{P: perturb transverse to map}]
	Suppose $e \colon X \to M \times \R^n$ is an embedding such that $\pi e = r_X$, with $\pi \colon M \times \R^n \to M$ the projection.
	Such an embedding always exists by \cref{C: embed V}.
	By \cref{L: all transversality is wrt embeddings}, it suffices to show that there is a proper  homotopy $H \colon M \times I \to M$ such that
	\begin{enumerate}
		\item $H(-,0) = \id$,
		\item $(H(-,1)r_V) \times \id_{\R^n} \colon V \times \R^n \to M \times \R^n$ is transverse to $e \colon X \to M \times \R^n$, and
		\item $(H (W \times \id_I)) \times \id_{\R^n}: W \times I \times \R^n \to M \times \R^n$ is transverse to $e$.
	\end{enumerate}
	To do so, we will run through the proof of \cref{P: ball stability} again, adapting it to this altered situation and referring back to that proof for some of the details.

	\begin{comment}
		\red{Note to Dev and Anibal: I know this is a bit redundant and \cref{P: ball stability} is arguably a special case with $n = 0$ (though there are a few other differences), but the proof of \cref{P: ball stability} is already very complicated so I didn't want to just do this more general version earlier.}
	\end{comment}

	As in the proof of \cref{P: ball stability}, we begin with the case where $W$ is compact.
	We will construct $F \colon M \times D^N \to M$, with $D^N$ the unit ball centered at $0$ in $\R^N$ for some $N$, such that

	\begin{enumerate}
		\item $F(-,0) = \id \colon M \to M$,
		\item for almost all $s \in D^N$ the composition $V \times \R^n \xr{r_V \times \id_{\R^n}} M \times \R^n \xr{F(-,s) \times \id_{\R^n}} M \times \R^n$ is transverse to $e \colon X \to M \times \R^n$,

		\item there is a ball neighborhood $D_r^N$ of $0$ in $D^N$ such that for all $s \in D_r^N$ the composition $W \times \R^n \xr{r_W \times \id_{\R^n}} M \times \R^n \xr{F(-,s) \times \id_{\R^n}}M \times \R^n$ is transverse to $e$.
	\end{enumerate}

	Given such an $F$, we let $s_0$ be any point in $D_r^N$ such that $V \times \R^n \xr{r_V \times \id_{\R^n}} M \times \R^n \xr{F(-,s_0) \times \id_{\R^n}} M \times \R^n$ is transverse to $e \colon X \to M \times \R^n$.
	Then let $H(-,t) = F(-,ts_0)$, i.e.\ $H(y,t) = F(y,ts_0)$.
	Then $H(-,0) = F(-,0) = \id$, and $H(-,1)r_V$ will be transverse to $r_X$ by our choice of $s_0$ and \cref{L: all transversality is wrt embeddings}.
	Finally, $ts_0 \in D_r^N$ for all $t \in I$, each $F(-,ts_0)r_W \times \id_{\R^n}$ is transverse to $e$, so $F(-,ts_0)r_W$ is transverse to $r_X$.
	Each $F(-,ts_0)r_W$ is the restriction of $H \circ (r_W \times \id) \colon W \times I \to M$ to a fixed $W \times t$, so this implies $H \circ (r_W \times \id)$ is transverse to $r_X$.
	This does not provide the properness of $H$, which we will discuss below.

	We now claim that we can construct $F$ almost exactly as in \cref{P: ball stability}.
	Recall that we let $M_\epsilon$ be an $\epsilon$-neighborhood of a proper embedding of $M$ into some $\R^N$ in the sense of the $\epsilon$-Neighborhood Theorem of \cite[Section 2.3]{GuPo74}, with $\epsilon$ a smooth positive function of $M$ and $M_\epsilon = \{z \in \R^N \mid |z-y|<\epsilon(y) \text{ for some }y \in M\}$.
	We may assume $\epsilon$ is a bounded function.
	Let $\pi_\epsilon: M_\epsilon \to M$ be the submersion.
	We define $f: M \times D^N \to M_\epsilon$ by $f(y, s) = y + \eta(y) s$, where $\eta \colon M \to \R$ is a smooth function such that $0 < \eta(y) < \epsilon(y)$ for all $y \in M$.
	As $\eta > 0$, this is clearly a submersion (onto its image) at all points.
	Then we let $F \colon M \times D^N \to M$ be the submersion $M \times D^N \xr{f}M_\epsilon \xr{\pi} M$ and let $\ms F \colon V \times D^N \to M$ be the composition $F \circ (r_V \times \id_{D^N})$.
	It is clear that $F(-,0) = \id_M$,
	Furthermore, the map $\ms F$, as well as each $\ms F|_{S^k(V)}$, is a submersion onto its image.
	Consequently the maps
	$$S^k(V) \times D^N \times \R^n \xr{\ms F \times \id_{\R^n}} M \times \R^n$$
	are also submersions onto their images.
	Thus they are all transverse to all the strata of $X$, embedded by $e$ into $M \times \R^n$.
	It follows by the Transversality Theorem of \cite[Section 2.3]{GuPo74} that for any stratum $S^j(X)$ of $X$, each $\ms F|_{S^k(V)}(-,s) \times \id_{\R^n}$ is transverse to $S^k(X)$ for almost all $s \in D^N$; note that $D^N$ remains our parameter space for invoking the Transversality Theorem, though we no longer write it as the last factor.
	As $X$ and $V$ each have finitely many strata and as the finite union of measure zero sets has measure zero, for almost all $s \in D^N$ we have all $\ms F|_{S^k(V)}(-,s) \times \id_{\R^n}$ transverse to all $S^j(X) \subset M \times \R^n$.
	So $\ms F(-,s) \times \id_{R^n}$ is transverse to $e$ for almost all $s \in D^N$.

	We also observe as in the proof of \cref{P: ball stability} that $H$ can be assumed proper, by replacing $\eta$ with a function with smaller values if necessary.

 	So it remains to show that if $W$ is compact (as we are currently assuming) and $r_W$ is transverse to $r_X$ (or, equivalently, $r_W \times \id_{\R^n}: W \times \id_{R^n} \to M \times \R^n$ transverse to $e$) then $(F(-,s) \circ r_W) \times \id_{\R^n} \colon W \times \R^n \to M \times \R^n$ is transverse to $e$ for all $s$ in some neighborhood $D_r^N$ of $0$ in $D^N$.
	Similarly to \cref{P: ball stability}, it is more convenient here to work with boundaries than with strata, recalling that, by \cref{L: simple trans}, to prove that two maps of manifolds with corners are transverse it is sufficient to show that their compositions with all pairs of boundary inclusions are naively transverse (see \cref{D: naive transversality}).
	Let $\Upsilon_k$ denote the composition $\Upsilon_k \colon \bd^k W \times D^N \xr{i_{\bd^k W} \times \id_{D^N}} W \times D^N \xr{r_W \times \id_{D^N}} M \times D^N \xr{F} M$.
	We must show that there is a $D_r^N$ such that for each $s\in D_r^N$ the maps $\Upsilon_k(-,s) \times \id_{\R^n}$ and $e i_{\bd^jX}$ are naively transverse for all $j,k$.
	We will sometimes provide the details only for $\Upsilon_0$, with $\bd^0 W$ being $W$ itself, when the other cases are analogous.

	We must start with two observations that were not needed in the proof of \cref{P: ball stability}.

	First, as we are currently assuming that $W$ is compact, $L_k = \Upsilon_k(\bd^k W \times \bar D^N_{1/2}) \subset M$ is compact, and so by \cref{L: compact preimage} we have $e(r_X^{-1}(L_k)) \subset M \times \bar B^n_k$ for some closed ball $\bar B^n_k \in \R^n$.
	In particular, this implies that for $|s| \leq 1/2$ only points in the compact set $\bd^k W \times \bar B^n_k$ can be taken by $\Upsilon_k(-,s) \times \id_{\R^n}$ to points of $e(X)$ in $M \times \R^n$.

	Second, we also have to be more careful here about the map $e \colon X \to M \times \R^n$, as its behavior can be more complicated than the embedding of a closed cube of a cubulation, in which each boundary component also embeds.
	Let $X_j = ei_{\bd^jX}(\bd^jX)$, the image of $\bd^jX$ in $M \times \R^n$.
	As the maps $i_{\bd^jX}$ are not necessarily embeddings, it will not generally be the case that $X_j \cong \bd^jX$.
	However, by \cite[Lemma 2.8]{Joy12}, the $i_{\bd^jX}$ are proper maps, so $X_j$ is a closed subset of $M \times \R^n$ (recall that proper maps are closed --- see \cite[Proposition I.10.1.1]{Bou98}).

	Now, suppose\footnote{In this argument we will use the symbol $w$ to refer to points in $W$ or $\bd^k W$, not the dimension of $W$.} $(w,z) \in W \times \bar B^n_0$, and let us fix $X_j$ as above.
	As $r_W$ is transverse to $r_X$ by assumption, $\Upsilon_0(-,0) \times \id_{\R^n} = r_W \times \id_{\R^n}$ is transverse to $e \colon X \to M \times \R^n$.
	So either $(r_W(w),z)\notin X_j$ or $r_W \times \id_{\R^n}$ is naively transverse to $ei_{\bd^jZ}$ at $(w,z)$ (recall \cref{D: naive transversality}).
	In the former case, as $X_j$ is closed, there is an open neighborhood $A_{(w,z)}$ of $(w,0,z) \in W \times D^N \times \R^n$ such that $(\Upsilon_0 \times \id_{\R^n})(A_{(w,z)}) \cap X_j = \emptyset$.
	On the other hand, suppose that $(r_W(w),z) \in X_j$ and $\Upsilon_0(-,0) \times \id_{\R^n}$ is naively transverse to $ei_{\bd^jX}$ at $(w,z)$.
	As $e$ is an embedding, the preimage of $(r_W(w),z)$ in $\bd^jX$ is the preimage of a point of $X$ under the boundary map $i_{\bd^jX}$, which is a finite set of points.
	Let $a \in \bd^jX$ be one point of this preimage.
	As the boundary maps are immersions, there is a neighborhood $C_a$ of $a$ in $\bd^jX$ on which $ei_{\bd^jX}$ restricts to an embedding from an $x-j$ dimensional manifold with corners into $M \times \R^n$.
	In fact, by choosing a chart that takes $0 \subset \R^{x-j}$ to $a$ and using the definition of a smooth map of manifolds with corners, $ei_{\bd^jX}$ (composed with the chart map) extends to a smooth immersion of a neighborhood of $0 \in \R^{x-j}$ into $M \times \R^n$.
	By further appealing to charts and local diffeomorphisms, we can identify a neighborhood of $(r_W(w),z)$ in $M \times \R^n$ with $\R^{m+n}$ and the image of the extension of $ei_{\bd^jX}$ with $\R^{x-j} \times 0 \subset \R^{m+n}$ (cf.\ the Local Immersion Theorem in \cite{GuPo74}).
	Making these identifications, the transversality assumption means that the composition of $D(\Upsilon_0(-,0) \times \id_{\R^n}) \colon T_wW \times T_z\R^n
	\to T_{(r_W(w),z)}(M \times \R^n)$ with the projection to the last $m+n-(x-j)$ coordinates is a linear surjection.
	As this is an open condition on the Jacobian matrix of $\Upsilon_0(-,0) \times \id_{\R^n}$ at $(w,z)$, it follows again that there is an open neighborhood $A_{(w,t),a}$ of $(w,0,z)$ in $W \times D_{1/2}^N \times \R^n$ such that for each $(w',s,z')$ in the neighborhood $\Upsilon_0(-,s) \times \id_{\R^n}$ is transverse to the restriction of $ei_{\bd^jX}$ to a neighborhood of $a$ (cf.\ the Stability Theorem in \cite{GuPo74}).
	As there are a finite number of possible choices for the point $a$ and the transversality assumptions must hold for all of them, by taking $A_{(w,z)} = \cap_a A_{(w,z),a}$ with the finite intersection running over all points of $\bd^jX$ that map to $(r_W(w),z)$, we obtain a neighborhood $A_{(w,z)}$ of $(w,0,z)$ in $W \times D_{1/2}^N \times \R^n$ such that for each $(w',s,z')$ in the neighborhood, $\Upsilon_0(-,s) \times \id_{\R^n} \colon W \times \R^n \to M \times \R^n$ is transverse to $ei_{\bd^jX}$ at $(w',z')$.

	Now, taking the union of the $A_{(w,z)}$ over all $(w,z) \in W \times \bar B^n_0$ gives a neighborhood $G_j$ of $W \times 0 \times \bar B^n_0$ in $W \times D_{1/2}^N \times \bar B^n_0$, and by the Tube Lemma, as $W \times \bar B^n_0$ is compact, there is a neighborhood of $W \times 0 \times \bar B^n_0$ of the form $W \times U_j \times \bar B^n_0$ in $G_j$.
	For each $s \in U_j$, we have $\Upsilon_0(-,s) \times \id_{\R^n} \colon W \times \bar B^n_0 \to M \times \R^n$ naively transverse to $ei_{\bd^jX}$.
	Furthermore, by the choice of $\bar B^n_0$, the map $\Upsilon_0(-,s) \times \id_{\R^n}$ takes no point of $W \times \R^n$ that is in the complement of $W \times \bar B^n_0$ to the image of $X$,
	so in fact $\Upsilon_0(-,s) \times \id_{\R^n} \colon W \times \R^n \to M \times \R^n$ is naively transverse to $ei_{\bd^jX}$.
	Repeating the argument for all of the finite $j$ such that $\bd^jX \neq \emptyset$ and taking $U^0 = \cap_j U_j$, we obtain a neighborhood of $0$ in $D_{1/2}^N$ on which $\Upsilon_0(-,s) \times \id_{\R^n} \colon W \times \R^n \to M \times \R^n$ is naively transverse to all $ei_{\bd^jX}$.
	By equivalent arguments, we can find open sets $U^k$ such that $\Upsilon_k(-,s) \times \id_{\R^n} \colon \bd^k W \times \R^n \to M \times \R^n$ is naively transverse to all $ei_{\bd^jX}$ for all $s \in U^k$ and all $k\geq 0$.
	Finally, taking $r$ sufficiently small so that $D_r^N \subset \cap_k U^k$, we obtain the desired $D_r^N$.

	This completes the proof of the proposition for $W$ compact.

	Next suppose that $W$ is no longer necessarily compact.

	We will continue to utilize $F \colon M \times I \to M$ as defined above, which did not rely on $W$ being compact.
	For $W$ not compact, the first two properties listed above for $F$ will continue to hold, but the third relied on compactness and so need not hold any long in general.
	However, the only places where we required compactness above were in defining $L_k$ and hence $\bar B^n_k$ via \cref{L: compact preimage}, and then in applying the Tube Lemma.
	So now let $K \subset W$ be compact.
	Then each $i_{\bd^k W}^{-1}(K)$ is also compact as the $i_{\bd^k W}$ are proper.
	Proceeding exactly as in the argument above, using $L_k = \Upsilon_k(i_{\bd^k W}^{-1}(K) \times \bar D^N_{1/2})$, we can find an open neighborhood
	$D_{r_K}^N$ of $0$ in $D^N$ such that for all $s \in D_{r_K}^N$ and any $k\geq 0$ we have that $\Upsilon_k(-,s) \times \id_{\R^n} \colon W \times \R^n \to M \times \R^n$ is naively transverse to all $ei_{\bd^j X}$ at all points of $i_{\bd^k W}^{-1}(K) \times \R^n$.\footnote{We note that this condition still concerns transversality of $\Upsilon_k(-,s) \times \id_{\R^n}$ as a map with domain $\bd^k W \times \R^n$, not $i_{\bd^k W}^{-1}(K) \times \R^n$, which may not be a manifold with corners. But we only consider this condition at points of $i_{\bd^k W}^{-1}(K) \times \R^n$.}

	\begin{comment}
		Let $L^K_k = \Upsilon_k(i_{\bd^k W}^{-1}(K) \times \bar D_{1/2}^N) \subset M$.
		Choosing an $X_j$ as above, we can now proceed exactly as in the precious argument, obtaining and taking the union of the resulting $A_{(w,z)} \subset W \times D_{1/2}^N \times \R^n$ over all $(w,z) \in K \times \bar B^n_0$ and intersecting with $K \times D^N \times \bar B^n_0$ gives an open neighborhood $G_j$ of $K \times 0 \times \bar B^n_0$ in $K \times D^N \times \bar B^n_0$, such that
		$\Upsilon_0(-,s) \times \id_{\R^n}$ is transverse to $ei_{\bd^j X}$ for all $(w,z) \in K \times \bar B^n_0$.
		Again we can use the Tube Lemma to find a neighborhood $U_j$ of $0$ in $D^N$ so that $K \times U_j \times \bar B^n_0 \subset G_j$, and then for every $s \in U_j$, we know that $\Upsilon_0(-,s) \times \id_{\R^n}$ is transverse to $ei_{\bd^j X}$ at every point of $K \times \R^n$, as points of $K \times \R^n$ outside of $K \times \bar B^n_0$ do not intersect $W_j$.
		Now, as above, by ranging over all of the finite options for $j$ and then similarly considering the $\bd^k W$ with subsets $i_{\bd^k W}^{-1}(K)$, maps $\Upsilon_k$, and balls $\bar B^n_k$, we can then find a neighborhood $U$ of $0$ in $D^N_{1/2}$ such that for all $s \in U$ we have $\Upsilon_k(-,s) \times \id_{\R^n}$ naively transverse to all $ei_{\bd^j X}$ at all points of $\bd^k W \times \R^n$, $k\geq 0$, that map to $K \times \R^n$ via the boundary immersions times $\id_{R^n}$.
	\end{comment}

	Let $\{\mc U_\ell\}$ be a locally finite covering of $M$ such that each $\bar{\mc U_\ell}$ is compact.
	As $r_W \colon W \to M$ is proper, each $r^{-1}_W(\bar {\mc U_\ell})$ is compact in $W$.
	Proceeding as just above with $r_W^{-1}(\bar U_\ell)$ in place of $K$, we can find for each $\ell$ an $\varepsilon_{\ell,k} \leq 1/2$ so that for every $s \in D^N_{\varepsilon_{\ell,k}}$ we have $\Upsilon_k(-,s) \times \id_{\R^n}$ naively transverse to all $ei_{\bd^j X}$ at every $(w,z) \in i_{\bd^k W}^{-1} r^{-1}_W(\bar {\mc U_\ell}) \times \R^n$.
	Let $\varepsilon_\ell = \min\{\varepsilon_{\ell,k} \mid k\geq 0\}$.
	These minima exist as $W$ has finite depth.

	Now, using \cref{L: minimizer}, we choose a smooth function $\phi \colon M \to \R$ such that for all $y \in M$ we have $0<\phi(y)<\epsilon_\ell$ if $y \in \bar{\mc U_\ell}$.
	Let $M \times_\phi D^N = \{(y,s) \in M \times D^N \mid |s|<\phi(y)\}$.
	By our construction, for all $k\geq 0$ we have $\Upsilon_k(-,s) \times \id_{\R^n} \colon \bd^k W \times \R^n \to M \times \R^n$ transverse to $ei_{\bd^j X}$ for at each $(w,s,z) \in \bd^k W \times D^N \times \R^n$ such that $(w,s) \in (r_Wi_{\bd^{k}W} \times \id_{D^N})^{-1}(M\times_\phi D^N) = \{(w,s) \in \bd^k W \times D^N \mid |s|<\phi(r_Wi_{\bd^k W}(w))\}$.

	We now modify our above constructions as follows.
	Let $\hat f: M \times D^N \to M_\epsilon \subset \R^N$ be given by $\hat f(y, s) = y +\phi(y) \eta(y) s$; as $\phi(y)\eta(y)>0$, this is again a submersion onto its image at all points.
	Let $\hat F \colon M \times D^N \to M$ be the composition $M \times D^N \xr{\hat f} M_\epsilon \xr{\pi_\epsilon} M$, and let $\hat \Upsilon_k$ be the composition $\bd^k W \times D^N \xr{i_{\bd^k W} \times \id} W \times D^N \xr{r_W \times \id} M \times D^N \xr{\hat F} M$ for $k\geq 0$.
	Once again employing the Transversality Theorem of \cite[Section 2.3]{GuPo74} as above, for almost all $s \in D^N$ we have $V \times \R^n \xr{r_V \times \id_{\R^n}} M \times \R^n \xr{\hat F(-,s) \times \id_{\R^n}} M \times \R^n$ transverse to $e \colon X \to M \times \R^n$.
	Letting $s_0$ be any such point we define $\hat H \colon M \times I \to M$ to be $\hat H(y,t) = \hat F(y,ts_0)$, and we claim that this $\hat H$ satisfies the conditions required by the proposition.

	The map $\hat H$ is proper again by \cref{L: nearby proper homotopy} because $\phi(y) \eta(y) \leq \eta(y)$.
	The first two conditions of the proposition follow immediately from the construction and preceding observations.
	Let $$\hat h_k = \hat H \circ (r_W i_{\bd^k W} \times \id_I): \bd^k W \times I \to M.$$
	By \cref{L: all transversality is wrt embeddings}, it remains to verify that each $$\hat h_k \times \id_{\R^n} \colon \bd^k W \times I \times \R^n \to M \times \R^n$$ is naively transverse to each $ei_{\bd^j X}$.
	\begin{comment}
		DO WE NEED ANY OF THIS??
		As we already know from the second condition of the proposition that $\hat h(-,1) \times \id_{\R^n}$ is transverse to $e$ and from the hypotheses that $\hat h(-,0)$ is transverse to $e$, it suffices to demonstrated transversality to $e$ of the restriction of $\hat h \times \id_{\R^n}$ to $X \times (0,1) \times \R^n$.
	\end{comment}
	From here, the argument is essentially the same as the end of the proof of \cref{P: ball stability} with the $\R^n$ factor just along for the ride.
	Again we focus primarily on $k=0$ to simplify notation slightly.

	In detail, for $(w,t) \in W \times I$ we can write $\hat h_0 \colon W \times I \to M$ explicitly as
	$$\hat h_0(w,t) = \pi_\epsilon(r_W(w)+\phi(r_W(w))\eta(r_W(w))ts_0).$$
	So, alternatively, we can observe that $\hat h_0 (w,t)$ is the composition
	\begin{equation}\label{E: perturb transverse to map}
		W \times I \xr{\Phi} W \times I \xhookrightarrow{\Psi} W \times D^N \xr{r_W \times \id} M \times D^N \xr{F} M,
	\end{equation}
	with $\Phi(w,t) = (w,\phi(r_W(w))t)$, $\Psi(w,t) = (w,ts_0)$, and noting that on the right we do mean our original $F$ and not $\hat F$.

	As $0 < \phi(r_W(w)) < 1$ for all $w \in W$, the first map $\Phi$ is a diffeomorphism onto its image, which is a neighborhood of $W \times 0$ in $W \times I$, and the map $\Psi$ embeds this into $W \times D^N$ by a product map that is constant in the $W$ direction and nontrivial linear in the second factor.
	The composition of the last two maps is just our earlier map $\Upsilon_0$.
	By construction, the map $r_W \times \id_{D^N}$ now takes the image of $\Psi\Phi$ into $M\times_\phi D^N$ (as $|s_0|<1$), and so at each point $(w,s,z)$ in the image of $\Psi\Phi \times \id_{\R^n}$ if we fix $s$ and consider $\Upsilon_0(-,s) \times \id_{\R^n}$ we get by construction a map on $W \times \R^n$ that is naively transverse at $(w,s,z)$ to each $ei_{\bd^j X}$.
	Let $(w,t) \in W \times I$, let $\Psi \Phi(w,t) = (\xi,s)$, and let $\R s_0$ denote the line in $\R^N = T_sD^N$ spanned by the position vector of $s_0$.
	As $\Phi$ is a diffeomorphism onto its image and $\Psi$ is an embedding that is the identity with respect to $W$ and nontrivial linear on $I$ for each fixed $w$, we see
	that the derivative of $\Psi\Phi$ maps the tangent space $T_{(w,t)}(W \times I)$ onto $T_\xi W \times \R s_0 \subset T_{(\xi,s)}(W \times D^N)$.
	In particular, this image contains $T_\xi W \times 0$, and we have established that $D(\Upsilon_0 \times \id_{\R^n})$ takes $T_w W \times 0 \times T_z\R^n$ to a tangent subspace in $M \times \R^n$ at $(\hat h_0(w,t),z)$ that is transverse to the tangent space there of each $ei_{\bd^j X}$.

	The same argument holds for each $k>0$ replacing $W$ with $\bd^{k}W$ in \eqref{E: perturb transverse to map} and $r_W$ with $r_{\bd^k W}$.
	So we see that $\hat H$ satisfies all the requirements of the proposition.
	\qedhere
	\begin{comment}
		But as $\Phi$ is a diffeomorphism onto its image and $\Psi$ is an embedding that is the identity with respect to $X$, we see $\Psi\Phi$ takes a neighborhood of any $(x,t) \in X \times (0,1)$ to a neighborhood of its image in $X \times \R s_0$, where $\R s_0$ is the line in $\R^N$ spanned by $s_0$.
		In particular, the derivative of $\Psi\Phi$ maps the tangent space to $X \times (0,1)$ at $(x,t)$ onto $ T_xX \times \R s_0 \subset T_{\Psi\Phi(x,t)}(X \times D^N)$.
		So for any $(x,t,z) \in X \times (-1,1) \times \R^n$, the image of $T_{(x,t,z)}(X \times (-1,1) \times \R^n)$ under
		$D(\Psi\Phi \times \id_{\R^n})$ contains $T_xX \times 0 \times \R^n$.
		By our construction, $DH_1$ takes this tangent space to a tangent subspace in $M \times \R^n$ at $\hat h(x,t)$ that is transverse to the images of all $D(ei_{\bd^jW})$.
		The same holds for $k>1$ replacing $X$ with $\bd^{k}V$ and $r_X$ with $r_{\bd^{k}V}$.
		So we see that $\hat h$ satisfies all the requirements of the proposition.
	\end{comment}
\end{proof}

Finally, we prove \cref{T: transverse reps}.

\begin{proof}[Proof of \cref{T: transverse reps}]
	Given \cref{L: compact preimage}, the proof is very analogous to that of \cref{T: transverse complex}.

	Let $r_W \colon W \to M$ be a proper map from a manifold with corners to a manifold without boundary, and let $\uV \in C_*^\Gamma(M)$ (or $C^*_\Gamma(M)$) be a cycle (or cocycle).
	Let $r_V \colon V \to M$ be any representative for $\uV$.
	By \cref{P: perturb transverse to map} there is a proper homotopy $H \colon M \to I$ such that $H(-,1)r_V$ is transverse to $r_W$.
	By \cref{D: universal homotopy,P: universal homotopy}, $H(-,1)r_V$ represents the same homology or cohomology class as $V$.

	Next, suppose $V$ is transverse to $W$ and there is a pre(co)chain $Z$ such that $\bd Z \sqcup -V \in Q(M)$.
	We consider first the case where $V$ is a precochain.
	By \cref{P: perturb transverse to map}, there is a proper homotopy $H$ such that $H(-,1)r_Z$ is transverse to $r_W$ and $V \times I \xr{r_V \times \id_I} M \times I \xr{H} M$ is transverse to $r_W$.
	Let $Z'$ denote the precochain $Z \xr{H(-,1)r_Z} M$, let $V'$ be the precochain $V \xr{H(-,1)r_V} M$, and let $Y$ be the precochain $V \times I \xr{H\circ (r_V \times \id_I)} M$.
	Let $A$ be the precochain $Z' \sqcup -Y$, which is transverse to $r_W$.
	We note that $\bd Z' \sqcup -V'$ is the image of $\bd Z \sqcup -V$ after composing with $H(-,1)$, and so it is in $Q(M)$ by \cref{L: Q preservation}, using that we can co-orient $H$ and $H(-,1)$ as in \cref{S: co-oriented homotopy}.
	We also have $\bd Y = V' \sqcup -V \sqcup B$, where $B$ is the precochain $\bd V \times I \xr{H\circ (r_{\bd V} \times \id_I)} M$ (cf. the proof of \cref{C: homotopy}).
	We now compute
	\begin{align*}
		\bd A \sqcup -V &= \bd Z' \sqcup -\bd Y \sqcup -V\\
		&= \bd Z' \sqcup -(V' \sqcup -V \sqcup B) \sqcup -V\\
		&= \bd Z' \sqcup -V' \sqcup V \sqcup -V \sqcup -B.
	\end{align*}
	We have already noted $\bd Z' \sqcup -V' \in Q(M)$ and $V \sqcup -V$ is trivial.
	Since $V$ represents a cycle, $\bd V \in Q^*(M)$ and hence $B \in Q^*(M)$ by \cref{L: dessicated homotopy}.
	So $\bd A \sqcup -V \in Q^*(M)$ and $A$ is transverse to $r_W$.
	So $A$ is our desired replacement for $Z$.

	In the oriented case, we instead have $\bd Y = (-1)^{v} V' \sqcup (-1)^{v+1} V \sqcup B$, identifying $V \times I$ with $V \times_{pt} I$ and applying \cref{P: oriented fiber boundary} (again cf.\ the proof of \cref{C: homotopy}).
	In this case, we let $A = Z' \sqcup (-1)^{v+1} Y$.
	Then
	\begin{align*}
		\bd A \sqcup -V &= \bd Z' \sqcup (-1)^{v+1}\bd Y \sqcup -V\\
		&= \bd Z' \sqcup (-1)^{v+1}((-1)^{v} V' \sqcup (-1)^{v+1} V \sqcup B) \sqcup -V\\
		&= \bd Z' \sqcup -V' \sqcup V \sqcup -V \sqcup (-1)^{v+1}B,
	\end{align*}
	and this is in $Q(M)$ for the same reasons as above.
\end{proof}


\begin{remark}\label{R: countable trans2}
As in \cref{R: countable trans}, and for the same reasons, the proof of \cref{T: transverse reps} allows use to extend the first part of theorem, concerning finding a representative $V$ of $\uV$ that is transverse to $W$, to finding a $V$ that is transverse to each of a countable collection $W_i$.
\end{remark}




\subsection{The Kronecker pairing and the Universal Coefficient Theorem for geometric cohomology}

When applying the cap product to cohomology and homology classes of the same degree, we can compose with the augmentation map $\aug \colon H_0^\Gamma(M) \to \Z$ of \cref{D: aug} to obtain a bilinear Kronecker pairing
$$H^i_\Gamma(M) \otimes H_i^\Gamma(M) \xr{\nplus} H_0^\Gamma(M) \to \Z.$$
Taking the adjunct then provides a map
$$\alpha: H^i_\Gamma(M) \to \Hom(H_i^\Gamma(M),\Z).$$
Tracing through the definitions, this maps acts by counting the intersection number between a geometric chain and a geometric cochain in the sense of \cref{D: intersection number}.

When $H^i_\Gamma(M)$ is finitely generated, this map fits into a short exact sequence, just as for singular cohomology.

\begin{theorem}\label{T: UCT}
	If $H^i_\Gamma(M)$ is finitely generated, there is a short exact sequence
	\[
	0 \to \Ext\left(H_{i-1}^\Gamma(M),\Z\right) \to H^i_\Gamma(M) \xr{\alpha} \Hom\left(H_i^\Gamma(M),\Z\right) \to 0.
	\]
\end{theorem}

\begin{remark}
	The existence of a Universal Coefficient exact sequence holds even if $H^i_\Gamma(M)$ is not finitely generated, as we know by \cref{T: geometric is singular} that $H^i_\Gamma(M) \cong H^i(M)$, and then we have the usual singular cohomology Universal Coefficient Theorem.
	We can further identify $\Hom(H_i(M),\Z)$ and $\Ext(H_{i-1}(M),\Z)$ with $\Hom(H^\Gamma_i(M),\Z)$ and $\Ext(H^\Gamma_{i-1}(M),\Z)$, also using \cref{T: geometric is singular}.
	The reason we need to invoke finite generation in \cref{T: UCT} is that we use \cref{T: intersection qi}, which has that condition, in the proof. What we lose from \cref{T: UCT} by following instead the approach outlined in this remark is the identification of the map $H^i_\Gamma(M) \to \Hom(H_i^\Gamma(M),\Z)$ with the map $\alpha$ given by counting intersection numbers.
\end{remark}

\begin{proof}[Proof of \cref{T: UCT}]
	Let $M$ have a cubulation $X$, let $\mc I \colon C_{\Gamma \pf X}^*(M) \to K^*(X)$ be the intersection map of \cref{D: intersection homomorphism},
	and let $\mc J \colon K_*(X) \to C^\Gamma_*(M)$ be the map inducing the homology isomorphism of \cref{T: cubical homology iso}.
	We consider the diagram
	\[
	\begin{tikzcd}
		H^i_\Gamma(M) \arrow[r, "\alpha"] & \Hom(H_i^\Gamma(M),\Z) \arrow[dd, "\cong", "\mc J^*"'] \\
		H^i(C^*_{\Gamma \pf X}(M)) \arrow[u, "\cong"'] \arrow[d, "\cong", "\mc I"'] & \\
		H^i(K^*(X)) \arrow[r] & \Hom(H_i(K_*(X)), \Z).
	\end{tikzcd}
	\]
	The vertical maps on the left are isomorphisms by \cref{T: transverse complex,T: intersection qi}, while the right hand vertical map is an isomorphism by \cref{T: cubical homology iso}.
	We claim the diagram commutes.
	In fact, let $V \in PC_\Gamma^i(M)$ represent an element of $H^i(C^*_{\Gamma \pf X}(M))$.
	Then $V$ is transverse to the cubulation, and by definition the path clockwise around the diagram takes $\uV$ to a map that acts on an element $\xi$ of $H_*(K_*(X))$ represented by a $\Z$-linear combination of cubical faces $\sum_j c_j E_j$ by treating each $E_j$ as a geometric chain and forming
	$$\aug\left( V \times_M \sum_jc_jE_j\right) = \sum_jc_j\aug(V \times_M E_j).$$

	On the other hand, by \cref{D: intersection homomorphism}, the composition counterclockwise takes $\uV$ to a map that acts on $\xi$ by $\sum_j c_j I_M(V,E_j)$.
	But $I_M(V,E_j)$ is precisely $\aug(V \times_M E_j)$ by \cref{D: intersection number}.
	So the diagram commutes.

	We know that $K_i(X)$ is a free abelian group, so the bottom map of the diagram is a surjection by the algebraic Universal Coefficient Theorem, with kernel $\Ext(H_{i-1}(K_*(X)),\Z)$.
	The commutativity of the diagram thus implies that the top map of the diagram is a surjection with isomorphic kernel.
	To complete the proof, we again invoke \cref{T: cubical homology iso} to observe $\Ext(H_{i-1}(X),\Z) \cong \Ext(H^\Gamma_{i-1}(X),\Z)$.
\end{proof}

\begin{remark}
	Note that while we obtain the expected Universal Coeficient Theorem relating geometric cohomology and homology, we do not claim to have either an isomorphism or a quasi-isomorphism between $C^i_\Gamma(M)$ and $\Hom(C_i^\Gamma(M),\Z)$.
	In fact, as we do not know $C_*^\Gamma(M)$ to be a complex of free abelian groups (which we leave as an open question), it is not clear $\Hom(C_i^\Gamma(M),\Z)$ fits into a short exact Universal Coefficient-type sequence at all.
\end{remark}

\subsection{The geometric cup product is the usual cup product}\label{S: usual cup}

In this section we show that the geometric cup product agrees with the singular cup product in the sense that there is a natural ring isomorphism between singular cohomology $H^*(-)$ with the usual cup product and geometric cohomology $H^*_\Gamma(-)$ with the cup product $\uplus$.

\begin{theorem}\label{T: intersection is cup product}
	On the category of smooth manifolds without boundary and continuous maps, there are natural isomorphisms of functors $\Phi_p \colon H^p(-) \to H^p_\Gamma(-)$, $p \geq 0$, from singular cohomology to geometric cohomology that are also compatible with cup products.
	In other words, for each manifold without boundary $M$ there is a commutative diagram
	\[
	\begin{tikzcd}
		H^p(M) \otimes H^q(M) \arrow{r}{\smile} \arrow{d}{\Phi_p \otimes \Phi_q} &
		H^{p+q}(M) \arrow{d}{\Phi_{p+q}} \\
		H^p_\Gamma(M) \otimes H^q_\Gamma(M) \arrow{r}{\uplus} & H^{p+q}_\Gamma(M).
	\end{tikzcd}
	\]
\end{theorem}

Our proof is based on an axiomatic characterization of the cup product on manifolds due to Kreck and Singhof \cite[Proposition 12]{Krec10b}.
As the proof of this proposition is only sketched in \cite{Krec10b}, we first fill in the details, restricting ourselves to $\Z$ coefficients and changing Kreck and Singhof's notation a bit to avoid conflicts with our earlier notation.
Before stating the result, we establish some further notation and conventions for this section.

In this section we assume the spheres $S^p$, $p>0$, to each have a fixed orientation.
We also want these orientations to be compatible in the sense that the composition $\nu \colon S^p \times S^q \to S^p \wedge S^q \cong S^{p+q}$ is orientation preserving away from the subspace that is collapsed to form the wedge product.
In particular, if $[S^p]$ and $[S^q]$ are the corresponding fundamental classes, the quotient should take $[S^p] \times [S^q]$ to $[S^{p+q}]$.
This can be arranged, for example, by modeling our spheres as the standardly-oriented cubes with their boundaries collapsed.
For each $p>0$, we let $s_p \in H^p(S^p) \cong \Hom(H_p(S^p), \Z)$ be the cohomology class that evaluates to $1$ on $[S^p]$.
Let $\pi_1 \colon S^p \times S^q \to S^p$ and $\pi_2 \colon S^p \times S^q \to S^q$ be the projections.

We let $K_p = K(\Z,p)$, $p>0$, be the Eilenberg-MacLane spaces, which we can assume have been constructed as CW complexes such that the $p+1$-skeleton of $K_p$ is $S^p$.
Let $\iota_p \in H^p(K_p)$ denote the fundamental class such that if $\phi_p \colon S^p \to K_p$ is the inclusion, then $\phi_p^*(\iota_p) = s_p$.
As the $p+1$ skeleton of $K_p$ is the image of $S^p$ under $\phi_p$, it is standard that $\phi_p^*$ is an isomorphism.
We also let $\mu \colon K_p \times K_q \to K_{p+q}$ be the unique-up-to-homotopy map that extends the collapse map $\nu$.

For $M$ connected, we always assume $H^0(M) \cong \Z$ generated by the class of the cochain $1 \in C^0(M)$,  i.e.\ the cochain that evaluates to $1$ on each singular $0$-simplex.
Then for $\lambda \in \Z$, we write $\lambda$ also for the class $\lambda 1$,.

\begin{proposition}[Kreck and Singhof, Proposition 12 of \cite{Krec10b}]\label{P: Kreck-Singhof pairing}
	Consider singular cohomology $H^*(-)$ as a cohomology theory on smooth manifolds\footnote{As defined in \cite{Krec10b}; see the proof of \cref{T: geometric is singular} above.}.
	Suppose $\star$ is a natural multiplication on $H^*(-)$ such that if $M$ is connected and $\lambda \in H^0(M)$ then $\lambda\star \alpha = \alpha\star \lambda = \lambda\alpha$ for all $\alpha \in H^*(M)$ (and with the obvious extension when $M$ is not connected).
	Then if\footnote{Rather than $s_p \times s_q$, Kreck and Singhof require $\pi_1^*(s_p) \star \pi_2^*(s_q)$ to be the element of $H^{p+q}(S^p \times S^q)$ that evaluates to $1$ on the fundamental class of $S^p \times S^q$, but with our conventions that tensor products of cochains act by $(\alpha \otimes \beta)(x \otimes y) = \alpha(x)\beta(y)$, these are the same cohomology class (c.f.\ \cite[page 245]{Span81} and \cite[Section 3B]{Hatc02}.
	} $\pi_1^*(s_p) \star \pi_2^*(s_q) = s_p \times s_q \in H^{p+q}(S^p \times S^q)$ for all $p,q\geq 1$, the product $\star$ is the cup product.
\end{proposition}

\begin{proof}
	For a smooth manifold $M$, let $\alpha \in H^p(M)$ and $\beta \in H^q(M)$.
	The condition that $\lambda\star \alpha = \alpha\star \lambda = \lambda\alpha$ whem $M$ is connected already guarantees that $\star$ is the cup product when $p$ or $q$ is $0$, so we can suppose $p,q>0$.
	As $H^*$ is ordinary singular cohomology, we know that $\alpha$ and $\beta$ can be represented by maps $\bar \alpha \colon M \to K_p$ and $\bar\beta \colon M \to K_q$ with $\alpha = \bar \alpha^*(\iota_p)$ and $\beta = \bar\beta^*(\iota_q)$.
	Furthermore, $\alpha\smile \beta$ is the pullback of $\iota_{p+q}$ by the composition
	\begin{equation}\label{E: EM cross}
		M \xr{\diag} M \times M \xr{\bar\alpha \times \bar \beta} K_p \times K_q \xr{\mu} K_{p+q},
	\end{equation}
	while similarly $s_p \times s_q$ is the pullback of $\iota_{p+q}$ by
	\begin{equation}\label{E: sphere cross}
		S^p \times S^q \xr{\phi_p \times \phi_q} K_p \times K_q \xr{\mu} K_{p+q};
	\end{equation}
	see \cite[Section 4.3]{Hatc02}.

	As we will want to apply the naturality of $\star$ in the category of smooth manifolds, we will choose manifold replacements for $K_p$, $K_q$, and $K_{p+q}$.
	In particular, suppose we realize $K_p$ as a CW complex by the standard constructions and let $K_p^N$ be the $N$-skeleton of $K_p$ with $N$ much larger than $\dim(M)$.
	Then $K_p^N$ is homotopy equivalent to a finite simplicial complex \cite[Theorem 2C.5]{Hatc02}, and we can embed it simplicially into some Euclidean space and take an open regular neighborhood to get a smooth manifold $\mc K_p$ homotopy equivalent to the $N$-skeleton of $K_p$.
	We define $\mc K_q$ and $\mc K_{p+q}$ analogously, using a large enough skeleton $K^{N'}_{p+q}$ of $K_{p+q}$ for the restriction of $\mu$ to $K_p^N \times K_q^N \to K_{p+q}^{N'}$ to be defined.
	Abusing notation, we continue to write $\bar \alpha$, $\bar \beta$, $\phi_p$, $\mu$, etc.
	for the maps involving these manifold replacements of the $K_*$.
	These replacements will be sufficient for all cohomology and homotopy computations required in what follows.

	Next we make two more preliminary observations.
	The first is that it follows from $\star$ being natural with respect to pullbacks that, when $f^*$ is an isomorphism, the product $\star$ is also natural with respect to $(f^*)^{-1}$, as we see by applying the isomorphism $f^*$ to the claimed identity $(f^*)^{-1}(x)\star (f^*)^{-1}(y) = (f^*)^{-1}(x\star y)$.
	The second is that there is an evident commutative diagram
	\begin{equation}\label{D: projections}
		\begin{tikzcd}
			H^p(\mc K_p) \arrow{r}{\pi_1^*} \arrow[d, "\phi_p^*"] &
			H^p(\mc K_p \times \mc K_q) \arrow[d, "(\phi_p \times \phi_q)^*"] \\
			H^p(S^p) \arrow{r}{\pi_1^*} & H^p(S^p \times S^q)
		\end{tikzcd}
	\end{equation}
	and similarly for $\pi_2$, abusing notation to write $\pi_1$ and $\pi_2$ for the projections to the first and second factors for both pairs of spaces.
	Now we compute:
	\begin{comment}
		Now, we consider the diagram
		\begin{diagram}
			H^{p+q}(M)&\lTo^{\diag^*(\bar \alpha \times \bar \beta)^*}& H^{p+q}(\mc K_p \times \mc K_q)&\lTo^{\mu^*}&H^{p+q}(\mc K_{p+q})\\
			&&\dTo^{(\phi_p \times \phi_q)^*}&&\dTo^{\phi_{p+q}}\\
			&&H^{p+q}(S^p \times S^q)&\lTo^{\nu^*}&H^{p+q}(S^{p+q}).
		\end{diagram}
		As the $p+1$ skeleton of $K_p$ can be taken to be the image of $S^p$ under $\phi_p$, it is standard that the vertical maps are isomorphisms.

		Again by \cite[Section 4.3]{Hatc02}, $\mu^*(\iota_{p+q}) = \iota_p \times \iota_q$, so $\alpha\smile \beta = \diag^*(\bar \alpha \times \bar \beta)^*(\iota_p \times \iota_q)$.
		So we compute as follows.
	\end{comment}
	\begin{align*}
		\alpha\smile \beta& = \diag^*(\bar \alpha \times \bar \beta)^*\mu^*(\iota_{p+q})&\text{see \eqref{E: EM cross}}\\
		& = \diag^*(\bar \alpha \times \bar \beta)^*((\phi_p \times \phi_q)^*)^{-1}(\phi_p \times \phi_q)^*\mu^*(\iota_{p+q})\\
		& = \diag^*(\bar \alpha \times \bar \beta)^*((\phi_p \times \phi_q)^*)^{-1}(s^p \times s^q)&\text{see \eqref{E: sphere cross}}\\
		& = \diag^*(\bar \alpha \times \bar \beta)^*((\phi_p \times \phi_q)^*)^{-1}(\pi_1^*(s_p) \star \pi_2^*(s_q))&\text{by assumption}\\
		& = \diag^*(\bar \alpha \times \bar \beta)^*(((\phi_p \times \phi_q)^*)^{-1}\pi_1^*(s_p) \star ((\phi_p \times \phi)^*)^{-1}\pi_2^*(s_q))&\text{by naturality}\\
		& = \diag^*(\bar \alpha \times \bar \beta)^*(\pi_1^*(\phi_p^*)^{-1}(s_p) \star \pi_2^*(\phi_q^*)^{-1}(s_q))&\text{by diagram \eqref{D: projections}}\\
		& = \diag^*(\bar \alpha \times \bar \beta)^*(\pi_1^*(\iota_p) \star \pi_2^*(\iota_q))\\
		& = (\diag^*(\bar \alpha \times \bar \beta)^*\pi_1^*(\iota_p)) \star (\diag^*(\bar \alpha \times \bar \beta)^*\pi_2^*(\iota_q))&\text{by naturality}\\
		& = \bar\alpha^*(\iota_p) \star \bar\beta^*(\iota_q)&\text{see below}\\
		& = \alpha\star\beta &\text{by definition}.
	\end{align*}
	For the penultimate equality, we have used that the composition of maps $$M \xr{\diag}M \times M \xr{\bar\alpha \times \bar \beta}K_p \times K_q \xr{\pi_1}K_p$$ is just $\bar \alpha$, and similarly for $\bar \beta$.
\end{proof}


	Now, recall that in the proof of \cref{T: geometric is singular}, which established an isomorphism between geometric and singular cohomology, we applied \cite[Theorem 10]{Krec10b}.
	That theorem of Kreck-Singhof shows that there is a natural isomorphism of these cohomology theories on the category of smooth manifolds and continuous maps.
	In fact, it shows there is such an isomorphism extending any given isomorphism of coefficients $\Phi_0 \colon H^0(pt) \to H^0_\Gamma(pt)$ to natural isomorphisms $\Phi_p \colon H^p(-) \to H^p_\Gamma(-)$ for all $p\geq 0$.
	We will here assume $\Phi_0$ chosen so that it takes $1 \in H^0(pt)$ to the element $\underline{pt} \in H^0_\Gamma(pt)$ represented by the identity $pt \to pt$ with tautological co-orientation (see \cref{E: first examples}).

	For the proof of \cref{T: intersection is cup product}, we need to arrange that for all $p\geq 1$ we have $\Phi_p(s_p) = s_p^\Gamma$, where
	$s_p \in H^p(S^p)$ is our preferred generator described above and $s_p^\Gamma \in H^p_\Gamma(S^p)$ is the generator represented by an embedded point with normal co-orientation agreeing with our chosen orientation of $S^p$.
	This will not necessarily be the case for the $\Phi_p$ output by \cref{T: geometric is singular}.
	However, part of the data for a cohomology theory in the Kreck-Singhof theory consists of the natural connecting maps $\delta$ of the Mayer--Vietoris sequence, and part of the output of the theorem is that the isomorphisms $\Phi_p$ commute with these connecting maps.
	Let us write the connecting map for a cohomology theory $h^*$ more explicitly as $h^p(U \cap V) \xr{\delta_p}h^{p+1}(U \cup V)$; we will generally write $\delta_p$ for the connecting map independent of which cohomology theory we are discussing.
	Of course if we replace a given $\delta_p$ by $-\delta_p$ for all spaces, then we still have a natural connecting map, and we will not have affected the exactness of the Mayer--Vietoris sequence.
	If we make such a change, we technically have a new cohomology theory with the same cohomology groups, but \cite[Theorem 10]{Krec10b} will output different isomorphisms $\Phi_p$.
	As \cref{T: intersection is cup product} does not particularly care about the signs of the connecting maps in the Mayer--Vietoris sequence, we will first tinker with the connecting maps in order to arrange that $\Phi_p(s_p) = s_p^\Gamma$ for all $p\geq 1$.
\begin{comment}
	Then we will be able to show the resulting $\Phi_p$ satisfy \cref{T: intersection is cup product}.
\end{comment}

\begin{lemma}\label{L: connecting signs}
	Possibly by changing the signs of the connecting morphisms in the Mayer-Vietoris sequences for geometric cohomology, we can arrange for $\Phi_p(s_p) = s_p^\Gamma$ for all $p \geq 1$, where $\Phi_p \colon H^p(-) \to H^p_\Gamma(-)$ are the isomorphisms output by \cref{T: geometric is singular} given $\Phi_0(1) = \underline{pt}$, as above.
\end{lemma}
\begin{proof}
	Let
	\begin{align*}
		U_p& = \{(x_1,\ldots,x_{p+1}) \in S^p \mid x_{p+1}>-1/2\}\\
		V_p& = \{(x_1,\ldots,x_{p+1}) \in S^p \mid x_{p+1}<1/2\}.
	\end{align*}
	Then the equatorial inclusion $S^{p-1} \into U_p \cap V_p$ is a homotopy equivalence.
	We will abuse notation and let $s_p$ also denote its image under the isomorphism $H^{p}(S^p) \cong  H^{p}(U_{p+1} \cap V_{p+1})$ induced by the homotopy equivalence.
	For $p\geq 1$, we now choose the sign of $\delta_p \colon H^{p}(U_{p+1} \cap V_{p+1}) \to H^{p+1}(S^{p+1})$ so that $\delta_p(s_p) = s_{p+1}$.
	Similarly, for geometric cohomology we arrange for $\delta_p(s_p^\Gamma) = s_{p+1}^\Gamma$.
	For $p = 0$, to avoid confusion let us write $z_- = -1 \in \R$ and $z_+ = 1 \in \R$.
	We let $s_0$ be the element of $H^0(S^0) \cong \Z^2$ that restricts to $1 \in H^0(z_+)$ and $0 \in H^0(z_-)$.
	Similarly, let $s_0^\Gamma \in H^0_\Gamma(S^0)$ be represented by the identity map of $z_+$ with its tautological co-orientation, and then $s_0^\Gamma$ is the element of $H^0_\Gamma(S^0) \cong \Z^2$ that restricts to $\underline{pt} \in H^0_\Gamma(z_+)$ and $0 \in H^0_\Gamma(z_-)$.
	Then we choose the signs of $\delta_0$ so that $\delta_0(s_0) = s_{1}$ and $\delta_0(s_0^\Gamma) = s_{1}^\Gamma$.

	Taking $H^*(-)$ and $H^*_\Gamma(-)$ with these Mayer--Vietoris connecting maps and this $\Phi_0 \colon H^0(pt) \to H^0_\Gamma(pt)$, \cite[Theorem 10]{Krec10b} gives natural isomorphisms $\Phi_p \colon H^p(-) \to H^p_\Gamma(-)$ extending $\Phi_0$ on a point.
	The naturality implies that $\Phi_0(s_0) = s_0^\Gamma$.
	It now follows by induction, using the following diagram due to the commutativity of $\Phi_*$ with the connecting maps, that $\Phi_p(s_p) = s_p^\Gamma$ for all $p$:
	\[
	\begin{tikzcd}
		H^p(S^p) \cong H^p(U_{p+1} \cap V_{p+1}) \arrow{r}{\delta_p} \arrow[d, "\Phi_p"] &
		H^{p+1}(U_{p+1} \cup V_{p+1}) = H^{p+1}(S^{p+1}) \arrow[d, "\Phi_{p+1}"] \\
		H^p_\Gamma(S^p) \cong H^p_\Gamma(U_{p+1} \cap V_{p+1}) \arrow{r}{\delta_p} &
		H_\Gamma^{p+1}(U_{p+1} \cup V_{p+1}) = H^{p+1}_\Gamma(S^{p+1}).
	\end{tikzcd}
	\]
\end{proof}

\begin{corollary}\label{C: sphere product}
	Given $\Phi_p \colon H^p(-) \to H^p_\Gamma(-)$ as in \cref{L: connecting signs}, then $\Phi_{p+q}(s_p \times s_q) = s_p^\Gamma \times s_q^\Gamma$ for all $p,q \geq 1$.
\end{corollary}


\begin{proof}
	Recall the maps $\nu \colon S^p \times S^q \to S^p \wedge S^q \cong S^{p+q}$ and $\mu \colon K_p \times K_q \to K_{p+q}$, defined above.
	We consider the following diagram, which commutes by the naturality of $\Phi_{p+q}$:
	\[
	\begin{tikzcd}
		H^{p+q}(S^p \times S^q) \arrow{d}{\Phi_{p+q}} &
		H^{p+q}(S^{p+q}) \arrow[l, "\nu^*"'] \arrow{d}{\Phi_{p+q}} \\
		H^{p+q}_\Gamma(S^p \times S^q) &
		H_\Gamma^{p+q}(S^{p+q}) \arrow[l, "\nu^*"'].
	\end{tikzcd}
	\]
	Let $s_{p+q}^\Gamma$ be represented by the embedding of a point at $y \in S^{p+q}$, normally co-oriented consistently with the orientation of $S^{p+q}$.
	By possibly choosing a different $y$ if necessary, we can also choose a smooth map homotopic to $\nu$ that maps a Euclidean neighborhood of some point $x \in S^p \times S^q$ by an orientation-preserving diffeomorphism onto a neighborhood of $y$, taking $x$ to $y$ and the complement of the neighborhood of $x$ to the complement of the neighborhood of $y$.
	Then, from the definitions, the pullback of $s_{p+q}^\Gamma$ is the embedding of $x$ into $S^p \times S^q$ with normal co-orientation corresponding to the orientation of $S^p \times S^q$.
	By \cref{E: sphere product}, this is exactly $s_p^\Gamma \times s_q^\Gamma$, i.e.\ $\nu^*(s_{p+q}^\Gamma) = s_p^\Gamma \times s_q^\Gamma$.
	So, recalling that $\Phi_p(s_{p+q}) = s_{p+q}^\Gamma$, we have $\nu^*\Phi_{p+q}(s_{p+q}) = s_p^\Gamma \times s_q^\Gamma$.
	Thus, from the commutativity of the diagram and $\Phi_{p+q}$ being an isomorphism, it suffices to show that $\nu^*(s_{p+q}) = s_p \times s_q$.

	For this, consider the commutative diagram
	\[
	\begin{tikzcd}
		H^{p+q}(K_p \times K_q) \arrow[d, "(\phi_p \times \phi_q)^*"] &
		\arrow[l, "\mu^*"'] H^{p+q}(K_{p+q}) \arrow[d, "\phi_{p+q}^*"] \\
		H^{p+q}(S^p \times S^q) & \arrow[l, "\nu^*"'] H^{p+q}(S^{p+q}).
	\end{tikzcd}
	\]
	As the $p+q+1$ skeleton of $K_p \times K_q$ is $S^p \times S^q$, the vertical maps are isomorphisms.
	And we know from \eqref{E: sphere cross} and the definitions that
	$$(\phi_p \times \phi_q)^*\mu^*(\phi_{p+q}^*)^{-1}(s_{p+q}) = (\phi_p \times \phi_q)^*\mu^*(\iota_{p+q})\\
	= s_p \times s_q,$$
	so we must have $\nu^*(s_{p+q}) = s_p \times s_q$, as needed.
\end{proof}


We can now apply \cref{L: connecting signs,C: sphere product} to prove \cref{T: intersection is cup product}.

\begin{proof}[Proof of \cref{T: intersection is cup product}]
	Applying \cref{L: connecting signs}, we assume isomorphisms $\Phi_p \colon H^p(-) \to H^p_\Gamma(-)$, $p \geq 0$, such that $\Phi_p(s_p) = s_p^\Gamma$ for all $p \geq 0$.

	For each $M$, we can now define pairings $\star$ on $H^p(-) \otimes H^q(-) \to H^{p+q}(-)$ by the composition
	\[
	H^p(M) \otimes H^q(M) \xr{\Phi_p \otimes \Phi_q}
	H^p_\Gamma(M) \otimes H^q_\Gamma(M) \xr{\uplus}
	H^{p+q}_\Gamma(M) \xr{\Phi_{p+q}^{-1}}
	H^{p+q}(M).
	\]
	This pairing is natural, as all the maps are natural.
	We will apply \cref{P: Kreck-Singhof pairing} to show that this is really the cup product, which will prove the theorem.

	We first show that if $M$ is connected and $\lambda \in H^0(M) \cong \Z$ then $\lambda\star \alpha = \alpha\star \lambda = \lambda\alpha$ for all $\alpha \in H^*(M)$.
	Let $pt$ be an arbitrary point in $M$.
	By the naturality of $\Phi_0$, the following diagram, in which the horizontal maps are induced by the inclusion $pt \into M$, commutes:
	\[
	\begin{tikzcd}
		H^0(M) \arrow{r} \arrow[d, "\Phi_0"] & H^0(pt) \arrow[d, "\Phi_0"] \\
		H_\Gamma^0(M) \arrow{r} & H_\Gamma^0(pt)
	\end{tikzcd}
	\]
	The vertical maps are isomorphisms by our application of \cite[Theorem 10]{Krec10b}, and it is standard that the top map is an isomorphism.
	In fact, we can consider $H^0(M)$ as generated by the cochain $1_M$, and this pulls back to the generator $1_{pt} \in H^0(pt)$ (each represented by the map that takes a positively oriented point considered as a singular $0$-chain to $1$).
	It follows that the bottom map is an isomorphism.
	Consider the generator $\uM \in H_\Gamma^0(M)$ given by the identity map of $M$ with its tautological co-orientation $(\beta_M,\beta_M)$; see \cref{E: first examples}.
	This has normal orientation given by the positively-oriented $0$-dimensional normal bundle, so by the pullback construction of \cref{D: pullback coorient}, the pullback to $H^0_\Gamma(pt)$ is similarly represented by $\underline{pt}$, the identity map of $pt$ with its canonical co-orientation.
	As $\Phi_0(1_{pt}) = \underline{pt}$ by assumption, it follows from the commutativity that $\Phi_0(1_M) = \uM$.

	So for $\alpha \in H^p(M)$, we have
	\begin{align*}
		\lambda \star \alpha
		& = \Phi_p^{-1}(\lambda\uM \uplus \Phi_p(\alpha)) \\
		& = \Phi_p^{-1}(\lambda\Phi_p(\alpha)) \\
		& = \lambda\alpha,
	\end{align*}
	using the unital property of $\uplus$ --- see \cref{S: (co)chain properties}.
	The same argument holds for $\alpha\star \lambda$.
	If $M$ has multiple components, then these properties clearly hold component-wise, as needed.

	To apply \cref{P: Kreck-Singhof pairing}, it remains to show that $\pi_1^*(s_p) \star \pi_2^*(s_q) = s_p \times s_q$ for all $p,q\geq 1$.
	For this, we have
	\begin{align*}
		\pi_1^*(s_p) \star \pi_2^*(s_q)& = \Phi_{p+q}^{-1}(\Phi_p(\pi_1^*(s_p))\uplus\Phi_q(\pi_2^*(s_q)))&\text{by definition of $\star$}\\
		& = \Phi_{p+q}^{-1}(\pi_1^*\Phi_p(s_p)\uplus\pi_2^*\Phi_q(s_q))&\text{by naturality of the $\Phi$}\\
		& = \Phi_{p+q}^{-1}(\pi_1^*(s_p^\Gamma)\uplus\pi_2^*(s_q^\Gamma))&\text{by \cref{L: connecting signs}}\\
		& = \Phi_{p+q}^{-1}((s_p^\Gamma \times \underline{S^q})\uplus(\underline{S^p} \times s_q^\Gamma))&\text{by Prop.
			\ref{P: projection pullbacks}}\\
		& = \Phi_{p+q}^{-1}((s_p^\Gamma\uplus \underline{S^p})\times( \underline{S^q}\uplus s_q^\Gamma))&\text{by Cor.
			\ref{C: criss cross} }\\
		& = \Phi_{p+q}^{-1}(s_p^\Gamma \times s_q^\Gamma)&\text{by Cor.
			\ref{C: cup with identity}}\\
		& = s_p \times s_q & \text{by \cref{C: sphere product}}.
	\end{align*}
\end{proof}

\subsection{K\"unneth theorems}

Now that we know that the geometric cup product is naturally isomorphic to the singular chain cup product, we can use this to compare cohomology cross products and obtain the geometric cohomology K\"unneth Theorem.
We begin with the homology K\"unneth Theorem, which is simpler, and then address the cohomology one.

Recall from \cref{T: hom iso map,P: singular smooth cubes} that we have isomorphisms $$H_*(NK_*(M)) \xleftarrow{\cong} H_*(NK^{sm}_*(M)) \xr{\cong} H_*^\Gamma(M),$$ where $NK_*(M)$ is the complex of normalized singular cubical chains and $NK^{sm}_*(M) \subset NK_*(M)$ is the subcomplex generated by smooth singular cubes.
As elements of $NK^{sm}_*(M)$ are represented by linear combinations of smooth maps from cubes and the cross product is represented by taking geometric products, we have the following immediate compatibility of chain cross products. Note that the upper left vertical map is an inclusion as the $NK^{sm}_i(M)$ and $NK_i(M)$ are all free groups generated by the nondegenerate singular cubes.

\begin{lemma}\label{L: chain cross compare}
	Let $M$ and $N$ be manifolds without boundary.
	Then the following diagram commutes:
	\[
	\begin{tikzcd}
		NK_*(M) \otimes NK_*(N) \arrow{r}{\times}  & NK_*(M \times N)  \\
		NK^{sm}_*(M) \otimes NK^{sm}_*(N) \arrow{r}{\times} \arrow[d] \arrow[hookrightarrow]{u}& NK^{sm}_*(M \times N) \arrow[d]\arrow[hookrightarrow]{u} \\
		C_*^\Gamma(M) \otimes C_*^\Gamma(N) \arrow{r}{\times} & C_*^\Gamma(M \times N).
	\end{tikzcd}
	\]
	This induces the commutative diagram
	\[
	\begin{tikzcd}
		H_*(NK_*(M)) \otimes H_*(NK_*(N)) \arrow{r}{\times}  &
		H_*(NK_*(M \times N)) \\
		H_*(NK^{sm}_*(M)) \otimes H_*(NK^{sm}_*(N)) \arrow{r}{\times} \arrow[d, "\cong"] \arrow[u, "\cong"']&
		H_*(NK^{sm}_*(M \times N)) \arrow[d, "\cong"] \arrow[u, "\cong"'] \\
		H_*^\Gamma(M) \otimes H_*^\Gamma(N) \arrow{r}{\times} & H_*^\Gamma(M \times N).
	\end{tikzcd}
	\]
\end{lemma}

\begin{theorem}[K\"unneth Theorem]\label{T: homology kunneth}
	Let $M$ and $N$ be manifolds without boundary.
	There are natural short exact sequences
	\[
	0 \to \bigoplus_{p+q = a} H_p^\Gamma(M) \otimes H_q^\Gamma(N) \xr{\times} H_{p+q}^\Gamma(M \times N) \to \bigoplus_{p+q = a-1} H_p^\Gamma(M)* H_q^\Gamma(N) \to 0
	\]
	that split (non-naturally).
\end{theorem}

\begin{proof}
	As the $C_i^\Gamma(M)$ and $C_i^\Gamma(N)$ are flat by \cref{L: flat}, there is such a split short exact sequence with middle term $H_*(C_*^\Gamma(M) \otimes C_*^\Gamma(N))$ by the algebraic K\"unneth Theorem \cite[Theorem V.2.1]{HS}.
	We must show that $C_*^\Gamma(M) \otimes C_*^\Gamma(N) \xr{\times} C_*^\Gamma(M \times N)$ is a quasi-isomorphism.
	But now in the first diagram of \cref{L: chain cross compare}, the top map is a quasi-isomorphism by
	 \cite[Theorem XI.3.1]{Mas91}.
	Furthermore, the vertical maps on the right are quasi-isomorphisms by \cref{T: hom iso map,P: singular smooth cubes}, and so, as all modules are flat, the vertical maps on the left are also quasi-isomorphisms (e.g.\ apply the K\"unneth Theorem and then the Five Lemma to each vertical map).
	So the bottom horizontal map is a quasi-isomorphism.
\end{proof}

We now turn to cohomology.
We will provide two proofs of the cohomology K\"unneth Theorem, one using our maps $\Phi$ from the Kreck-Singhof argument and one using our intersection maps $\mc I$ in the cubical setting.

For the first, recall that for $\uV \in H^*_\Gamma(M)$ and $\uW \in H^*_\Gamma(N)$ we have the relation
$\uV \times \uW = \pi_M^*(\uV)\uplus\pi_N^*(\uW)$, which follows from \cref{C: cross is cup}, while the same relation is well known to hold in singular cohomology \cite[Corollary 5.6.14]{Span81}.
So the following is immediate from the naturality of our comparison maps $\Phi$ and \cref{T: intersection is cup product}

\begin{proposition}\label{P: cross product is cross product}
	On the category of smooth manifolds without boundary and continuous maps, the isomorphisms $\Phi_p$ from singular cohomology to geometric cohomology are compatible with cross products.
	In other words, for manifolds without boundary $M$ and $N$ there are commutative diagrams
	\[
	\begin{tikzcd}
		H^p(M) \otimes H^q(N) \arrow{r}{\times} \arrow[d,"\Phi_p \otimes \Phi_q"', "\cong"] &
		H^{p+q}(M \times N) \arrow[d, "\Phi_{p+q}"', "\cong"] \\
		H^p_\Gamma(M) \otimes H^q_\Gamma(N) \arrow{r}{\times} &
		H^{p+q}_\Gamma(M \times N).
	\end{tikzcd}
	\]
\end{proposition}

\begin{theorem}[K\"unneth Theorem]\label{T: cohomology kunneth}
	If either $H^i_\Gamma(M)$ is finitely generated for all $i$ or $H^i_\Gamma(N)$ is finitely generated for all $i$, then there are natural short exact sequences
	\[
	0 \to \bigoplus_{p+q = a}H^p_\Gamma(M) \otimes H^q_\Gamma(N) \xr{\times} H^{p+q}_\Gamma(M \times N) \to \bigoplus_{p+q = a+1}H^p_\Gamma(M)* H^q_\Gamma(N) \to 0
	\]
	that split (non-naturally).
\end{theorem}

\begin{proof}
	Using \cref{P: cross product is cross product}, we can form the left part of the diagram
	\[
	\begin{tikzcd}
	\displaystyle\bigoplus_{p+q = a}H^p_\Gamma(M) \otimes H^q_\Gamma(N) \arrow[r,"\times",hook]\arrow[d,"\bigoplus \Phi_p \otimes \Phi_q"',"\cong"]& H^{p+q}_\Gamma(M \times N) \arrow[r,twoheadrightarrow]\arrow[d,"\Phi_{p+q}"',"\cong"]&\displaystyle\bigoplus_{p+q = a+1}H^p_\Gamma(M)* H^q_\Gamma(N) \arrow[d,dashed,"\bigoplus \Phi_p * \Phi_q"',"\cong"]\\
	\displaystyle\bigoplus_{p+q = a}H^p_\Gamma(M) \otimes H^q_\Gamma(N) \arrow[r,dashed,hook,"\times"]& H^{p+q}_\Gamma(M \times N) \arrow[r,dashed,twoheadrightarrow]& \displaystyle\bigoplus_{p+q = a+1}H^p_\Gamma(M)* H^q_\Gamma(N).
	\end{tikzcd}
	\]
	Since we know there is a cohomology K\"unneth Theorem of this form for singular cohomology \cite[Theorem 60.5]{Mun84}, and the vertical maps on the left and middle are isomorphisms, the bottom left horizontal map is also injective.
	It follows that there is an isomorphism between the quotient terms of the two sequences, and the quotient term on the top is $\displaystyle\bigoplus_{p+q = a+1}H^p_\Gamma(M)* H^q_\Gamma(N)$, which is then isomorphic to $\displaystyle\bigoplus_{p+q = a+1}H^p_\Gamma(M)* H^q_\Gamma(N)$ via the maps $\Phi_p*\Phi_q$.

	As the exact sequences are isomorphic and the top one splits, they both split.

	This construction is natural since all of the morphisms involved (except the splitting morphisms) are natural.
\end{proof}

In the preceding argument we had to work directly with cohomology because our maps $\Phi_p$ are only defined on cohomology, not on cochains.
Alternatively, we can approach the cohomology K\"unneth theorem more explicitly and in closer analogy with the proof of \cref{T: homology kunneth} by again utilizing cubulations and intersection maps.

One benefit of cubulations over triangulations is that the product of two cubical complexes is again a cubical complex: the product of two cubes $E$ and $F$ is simply the cube $E \times F$.
If $X$ and $Y$ are cubical complexes, this provides a cubical cross product $K_*(X) \otimes K_*(Y) \to K_*(X \times Y)$.

\begin{lemma}
	Let $M$ and $N$ be manifolds without boundary cubulated by cubical complexes $X$ and $Y$.
	The cubical cross product $K_*(X) \otimes K_*(Y) \to K_*(X \times Y)$ is a quasi-isomorphism.\footnote{Certainly this holds more generally for any cubical complexes $X$ and $Y$, but we will not need the greater generally and this statement allows us to be expedient in citing our previous results.}
\end{lemma}
\begin{proof}
	Consider the diagram
	\[
	\begin{tikzcd}
	K_*(X) \otimes K_*(Y) \arrow[r,"\times"]\arrow[d,"\eta \otimes \eta"] & K_*(X \times Y)\arrow[d,"\eta"]\\
	NK^{sm}_*(X) \otimes NK^{sm}_*(Y) \arrow[r,"\times"]\arrow[d,"\psi \otimes \psi"] & NK^{sm}_*(X \times Y)\arrow[d,"\psi"] \\
	NK_*(X) \otimes NK_*(Y) \arrow[r,"\times"] & NK_*(X \times Y),
	\end{tikzcd}
	\]
	in which the maps $\eta$ and $\psi$ are as in \cref{S: cubical and geometric homology}.

	The vertical maps are quasi-isomorphisms by \cref{T: cubical homology iso,P: singular smooth cubes} and because these are all free modules.
	The bottom maps is a quasi-isomorphism by \cite[Theorem XI.3.1]{Mas91}.
	So the top map is also a quasi-isomorphism.
\end{proof}

Another benefit of the cubical category is that the inverse to the cros

\begin{proposition}\label{P: cross product comparison}
	Let $M$ and $N$ be cubulated manifolds without boundary, and let $M \times N$ have the product cubulation.
	Let $K^*$ denote the complex of cubical cochains for the appropriate cubulation.
	The following diagram commutes
	\[
	\begin{tikzcd}
		C^*_{\Gamma}(M) \otimes C^*_{\Gamma}(N) \arrow{r}{\times} & C^*_{\Gamma}(M \times N) \\
		C^*_{\Gamma\pf}(M) \otimes C^*_{\Gamma\pf}(N) \arrow{r}{\times} \arrow[u, hook] \arrow[d, "\mc I \otimes \mc I"] & C^*_{\Gamma\pf}(M \times N) \arrow[u, hook] \arrow[d, "\mc I"] \\
		K^*(M) \otimes K^*(N) \arrow{r}{\times} & K^*(M \times N).
	\end{tikzcd}
	\]
\end{proposition}

\begin{proof}
	The top square certainly commutes as the vertical maps are inclusions.
	Note that the product of two maps transverse to the cubulation will be transverse to the product cubulation, so the middle horizontal map is well defined.

	Let $V$ and $W$ represent elements of $C^*_{\Gamma\pf}(M)$, and let $E$ and $F$ be cubes of the cubulation.
	We check that the two ways around the bottom square evaluate the same on $E \times F$.
	Applying $\mc I(V \times W)$ to $E \times F$ gives $I_{M \times N}(V \times W, E \times F) = \aug((V \times W)\times_{M \times N}(E \times F))$ by \cref{D: intersection homomorphism,D: intersection number}, while going the other way around the diagram and applying the result to $E \times F$ yields, with our conventions, $I_M(V,E)I_N(W,F) = \aug(V \times_M E)\aug(W \times_N F)$.

	We can now compute
	\begin{align*}
		\aug((V \times W)\times_{M \times N}(E \times F))& = (-1)^{(w+f-n)(m-v)}\aug((V \times_M E) \times (W \times_N F))\\
		& = (-1)^{(w+f-n)(m-v)}\aug(V \times_M E)\aug(W \times_N F).
	\end{align*}
	The first equality is due to \cref{P: cap cross}.
	Note that if either $V \times_M E$ or $W \times_N F$ is not $0$-dimensional then also $(V \times W)\times_{M \times N}(E \times F)$ is not $0$-dimensional, and all three expressions above are $0$.
	Otherwise, the second equality is apparent as the product of a $(-1)^a$-oriented point with a $(-1)^b$-oriented point is a $(-1)^{a+b}$-oriented point.
	In this case we also have $w+f = n$ so that $(-1)^{(w+f-n)(m-v)} = 1$.
\end{proof}

The following corollary is now immediate from \cref{P: cross product comparison,T: transverse complex,T: intersection qi}.

\begin{corollary}
	If $M$ and $N$ are cubulated manifolds without boundary and each $H^i(M)$ and $H^j(N)$ is finitely generated, then the cubical cohomology cross product is isomorphic to the geometric cohomology cross product.
	In particular, we have the following diagram with all vertical maps isomorphisms:
	\[
	\begin{tikzcd}
		H^*_{\Gamma}(M) \otimes H^*_{\Gamma}(N) \arrow{r}{\times} & H^*_{\Gamma}(M \times N) \\
		H^*_{\Gamma\pf}(M) \otimes H^*_{\Gamma\pf}(N) \arrow{r}{\times} \arrow[u, "\cong"] \arrow[d, "\cong", "\mc I \otimes \mc I"'] &
		H^*_{\Gamma\pf}(M \times N) \arrow[u, "\cong"] \arrow[d, "\cong", "\mc I"'] \\
		H^*(K^*(M)) \otimes H^*(K*(N)) \arrow{r}{\times} & H^*(K^*(M \times N)).
	\end{tikzcd}
	\]
\end{corollary}

Furthermore, \cref{P: cross product comparison} now implies a proof of \cref{T: cohomology kunneth} completely analogous to the proof of \cref{T: homology kunneth}.
The finiteness hypothesis is required both for the maps $\mc I \colon C^*_{\Gamma\pf}(-) \to K^*(-)$ to be isomorphisms via \cref{T: intersection qi} and for the cross product $H^*(K^*(M)) \otimes H^*(K*(N)) \xr{\times} H^*(K^*(M \times N))$ to be an isomorphism.



\begin{comment}
	Dev's sketched elementary proof \colon

	Use pullback of cross product after showing that cross product agrees with singular cross product using the cubical structure as intermediary as in flows.
	In particular, given $V$ and $W$ then $V \times W = (V \times M) \cap (M \times V)$ should act on the cube $\sigma \times \tau$ as $W(\sigma)V(\tau)$.

	\begin{definition}
		Suppose $M$ and $N$ are manifolds.
		Then there is a product map $C^*_\Gamma(M) \otimes C^*_\Gamma(N) \to C^*_\Gamma(M \times N)$ that takes geometric cochains $\uW \in C^*_\Gamma(M)$ and $\uV \in C^*_\Gamma(N)$ represented by $r_W \colon W \to M$ and $r_V \colon V \to N$ to the cochain $\uW \times \uV$ represented by $r_W \times r_V \colon W \times V \to M \times N$ with the product co-orientation.
		In other words, if $(\beta_W,\beta_M)$ and $(\beta_V,\beta_N)$ are the corresponding co-orientations of $\uW$ and $\uV$, then the product co-orientation is $(\beta_W \wedge \beta_V,\beta_M \wedge \beta_N)$.
		By linear extension we obtain the \textbf{(external) cross product} map $C^*_\Gamma(M) \otimes C^*_\Gamma(N) \to C^*_\Gamma(M \times N)$.
	\end{definition}

	\begin{lemma}
		The cochain cross product is a well-defined map.
	\end{lemma}

	\begin{proof}
		If $V$ is trivial via $\rho_W \colon W \to W$, then $W \times V$ is trivial via $\rho_W \times \id_V$.
		Similarly, if $W$ is of small rank then so is $W \times V$.
		Since the sum of geometric cochains is represented by disjoint union, we have $(W_1\pm W_2) \times V = (W_1 \times V)\pm (W_2 \times V)$.
		So if $W_1-W_2 \in Q^*(M)$ then $W_1 \times V- W_2 \times V = (W_1-W_2) \times V \in Q^*(M \times N)$.
		So the cross product is independent of representative for $\uW$, and similarly it is independent of the choice of representative for $\uV$.

		To see that we have a chain map, we observe that \red{PUT SOMETHING IN THE CO-ORIENTATION SECTION}.
	\end{proof}

	The following is now standard homological algebra:
	\begin{corollary}
		The external cochain cross product induces an \textbf{(external) cohomology cross product} $H^i_\Gamma(M) \otimes H^j_\Gamma(N) \to H^{i+j}_\Gamma(M \times N)$.
	\end{corollary}

	\begin{theorem}
		Let $M$ and $N$ be manifolds.
		Then the geometric cohomology cross product $H^i_\Gamma(M) \otimes H^j_\Gamma(N) \to H^{i+j}_\Gamma(M \times N)$ is isomorphic to the singular cohomology cross product $H^i(M) \otimes H^j(N) \to H^{i+j}(M \times N)$.
	\end{theorem}

	\begin{proof}
		NEED MORE GENERAL INTERSECTION MAP THEOREM OR TO RESTRICT TO COMPACT MANIFOLDS

	\end{proof}
\end{comment}

\subsection{The geometric cap product is the usual cap product}

Our goal in this section is to use the intersection map $\mc I$ to relate the geometric cap product with the classical singular cap product, using the cubical cap product as an intermediary.
We first discuss formulas for the cubical cup and cap products, relying on known formulas for the singular cubical products.
Then we apply the cubical formulas to the geometric world.

In \cref{T: equivalent cap,C: cap relation} we show that the cubical cap product (and hence the singular cap product) determines the geometric cap product in general, while the geometric cap product determines the cubical cap product if all $H^i(M)$ are finitely generated.
This last condition is needed because \cref{T: intersection qi} only tells us that the intersection map induces cohomology isomorphisms $\mc I \colon H^i_{\Gamma \pf X}(M) \to H^i(K_X^*(M))$, for some cubulation $X$, when $H^i(M)$ is finitely generated.
Unlike the situation with \cref{T: intersection is cup product}, for which we used the Kreck-Singhof theorem for cup products to show that the geometric and singular cup products are always isomorphic, we do not know of an analogue of the Kreck-Singhof theorem for cap products that would allow us to provide compatibility of cap products in full generality.

\subsubsection{Cubical cup and cap products}\label{S: cubical products}

In this section we discuss cup and cap products for cubical and singular cubical homology and cohomology.
This will be needed below for comparing the geometric cap product with the classical cap products.

We first recall from Massey \cite[Chapter XI]{Mas91} some results about the normalized singular cubical chain complexes, which we have been denoting $NK_*(-)$, though we utilize some different notation from Massey.
Just as for the more familiar singular simplicial chains, there is an Eilenberg-Zilber theorem that provides a chain homotopy equivalence between $NK_*(X) \otimes NK_*(Y)$ and $NK_*(X \times Y)$ for any spaces $X$ and $Y$.
Explicit constructions of such homotopy inverse maps are given in \cite[Section XI.5]{Mas91}.
The map $\zeta: NK_*(X) \otimes NK_*(Y) \to NK_*(X \times Y)$ is simply the cross product that takes $S \otimes T$ for representative singular cubes $S: \interval^m \to X$ and $T \colon \interval^n \to Y$ to the product $S \times T \colon \interval^m \times \interval^n = \interval^{m+n} \to X \times Y$.
If $S$ or $T$ is degenerate, so is $S \times T$, so this product is well defined for the normalized complexes.
The homotopy inverse map\footnote{Massey sometimes writes this map as $\eta$.} $\xi: NK_*(X \times Y) \to NK_*(X) \otimes NK_*(Y)$ takes $S \colon \interval^n \to X \times Y$ to
$$\xi(S) = \sum \rho_{H,K}A_H(\pi_1S) \otimes B_K(\pi_2S),$$ where $\pi_i$ is the projection to the $i$th factor.
The precise definitions of $\rho_{H,K}$, $A_H$, and $B_K$ will not need to concern us except to note that $H$ and $K$ are complementary subsets of $\{1,\ldots, n\}$, the sum is over all such partitions, $\rho_{H,K}$ is either $1$ or $-1$ (in fact it is the sign of permutation $HK$), and $A_H$ and $B_K$ are cubical faces of various dimensions of the singular cubes $\pi_1S$ and $\pi_2S$.
Again, this construction is sufficiently compatible with degeneracies to be well defined for the normalized singular cube complexes.
We also observe that if $X$ is a smooth manifold and our input singular cubes are smooth, then all other cubes appearing in the constructions are smooth.

As usual, one then defines cup and cap products (up to one's favorite sign conventions) as follows (using our current sign conventions).
If $\alpha, \beta \in NK^*(X) = \Hom(NK_*(X),\Z)$, then $\alpha\smile \beta \in \Hom(NK_*(X),\Z)$ acts on a normalized singular cube $S$ by
\begin{equation}\label{E: cubical cup}
	(\alpha \smile \beta)(S) = (\alpha \otimes \beta)(\xi(\diag S)),
\end{equation}
with $\diag$ the diagonal map $X \to X \times X$,
while the cap product $\alpha\frown S$ is given by
\begin{equation}\label{E: cubical cap}
	\alpha\frown S = (\id \otimes \alpha) (\xi(\diag S)),
\end{equation}
identifying $NK_*(X) \otimes \Z$ with $NK_*(X)$.

Now suppose that $X$ is a cubical complex, and let $K_*(X)$ and $K^*(X) = \Hom(K_*(X),\Z)$ be the cubical chain and cochain complexes.
If $E, F \in K_*(X)$ are any cubical faces, then $E \times F$ is also a cubical face.
Furthermore, abusing notation by conflating $E$ with its embedding into the cubical complex (thought of as a space), we have $\pi_1\diag(E) = \pi_2\diag(E) = E$, and every face of $E$ is also a cube in the complex.
So now if $M$ is a cubulated manifold without boundary and we abuse notation by using $K_*$ for the cubical chain complex $K_*(M)$ corresponding to the cubulation $X$ and $K_*(M \times M)$ for the cubical complex coming from the product cubulation $X \times X$, we have the following commutative diagrams with the upward arrows being inclusions (note that all complexes are free) and the lower horizontal maps being the restrictions of the top horizontal maps:
\[
\begin{tikzcd}
	NK_*(M) \otimes NK_*(M) \arrow{r}{\zeta} & NK_*(M \times M) \\
	NK^{sm}_*(M) \otimes NK^{sm}_*(M) \arrow{r}{\zeta} \arrow[u, hook] & NK^{sm}_*(M \times M) \arrow[u, hook] \\
	K_*(M) \otimes K_*(M) \arrow{r}{\zeta} \arrow[u, hook] & K_*(M \times M). \arrow[u, hook]
\end{tikzcd}
\]
and
\[
\begin{tikzcd}
	NK_*(M) \otimes NK_*(M) & \arrow[l, "\xi"'] NK_*(M \times M) \\
	NK^{sm}_*(M) \otimes NK^{sm}_*(M) \arrow[u, hook] & \arrow[l, "\xi"'] NK^{sm}_*(M \times M) \arrow[u, hook] \\
	K_*(M) \otimes K_*(M) \arrow[u, hook] & \arrow[l, "\xi"'] K_*(M \times M). \arrow[u, hook]
\end{tikzcd}
\]

The top map in each diagram is a homotopy equivalences, the vertical maps are all quasi-isomorphisms by \cref{P: singular smooth cubes} and the proof of \cref{T: cubical homology iso},
and the complexes are all free, so the horizontal maps are all chain homotopy equivalences \cite[Theorem 46.2]{Mun84}.

Putting this all together, for both the complexes of normalized smooth singular cubical chains and the cubical complexes $K_*$ coming from the smooth cubulations, we may define cup and cap products again by the formulas \eqref{E: cubical cup} and \eqref{E: cubical cap}.
Note that in the case of a geometric cube $E$ coming from a cubulation, $\diag E$ is not a cube in the cubical complex, but that does not matter as in the end formula for $\xi(\diag E)$ we work with $\diag E$ only through its projections $\pi_1(\diag E) = \pi_2(\diag E) = E$.
These products are then compatible with the constructions for normalized singular cubical chains and cochains, i.e.\ the restriction of the cup product is the cup product of the restriction and the appropriate mixed functoriality version of that statement holds for cap products.

\subsubsection{Relating geometric and cubical cap products via the intersection map}

In this section we show that the geometric and cubical cap products are compatible in the sense given below in \cref{T: equivalent cap,C: cap relation}.
Recall that in \cref{R: intersection map extension} we extended the definition of the intersection map $\mc I$ to give us a map $H^*_\Gamma(M) \to H^i(K_X^*(M))$ for the manifold $M$ cubulated by $X$, though we leave $X$ tacit in the following.
We again let $\mc J \colon K_*(X) \cong K^X_*(M) \to C^\Gamma_*(M)$ be the map that takes a cubical face of $X$ to its embedding into $M$; see \cref{T: cubical homology iso}.

\begin{theorem}\label{T: equivalent cap}
	Let $M$ be a smoothly cubulated manifold without boundary.
	Let $\uV \in H^*_\Gamma(M)$ and $W \in H_*(K_*(M))$.
	Then
	$$\uV \nplus \mc J(W) = \mc J(\mc I(\uV)\frown W).$$
\end{theorem}

Here the cap product on the left is our geometric cap product and the cap on the right is the cubical cap product defined in \cref{T: cubical homology iso}.

Before proving the theorem, we note the following corollary.

\begin{corollary}\label{C: cap relation}
	The cubical cap product (and hence the singular cap product) determine the geometric cap product.
	If all $H^i(M)$ are finitely generated, then the geometric cap product determines the cubical cap product.
\end{corollary}

\begin{proof}
	Let $\uV \in H^*_\Gamma(M)$ and $\uW \in H_*^\Gamma(M)$.
	Choose a cubulation of $M$.
	As $\mc J \colon H_*(K_*(M)) \to H_*^\Gamma(M)$ is an isomorphism by \cref{T: cubical homology iso}, we have by \cref{T: equivalent cap}
	$$\uV\nplus \uW = \uV\nplus \mc J(\mc J^{-1}(\uW)) = \mc J(\mc I(\uV)\frown \mc J^{-1}(\uW)).$$
	On the other hand, suppose $V \in H^*(K^*(M))$ and $W \in H_*(K_*(M))$.
	Then $\mc I \colon H^*_\Gamma(M) \to H^*(K^*(M))$ is an isomorphism by \cref{T: intersection qi,R: intersection map extension} when all $H^i(M)$ are finitely generated.
	So then
	$$V\frown M = \mc J^{-1}(\mc J(\mc I\mc I^{-1}(V)\frown W)) = \mc J^{-1}(\mc I^{-1}(V) \nplus \mc J(W)).$$
\end{proof}

We will approach the proof of \cref{T: equivalent cap} through a series of lemmas.
The first two concern transversality.
Then we have a series of lemmas that essentially consist of various reformulations of the cap products, eventually linking together the two terms of \cref{T: equivalent cap}.
Once we have all the lemma established, we explain how to tie them all together to prove the theorem.

\begin{comment}
	\begin{lemma}\red{REMOVE???:}
		Suppose $M$ is a cubulated manifold without boundary and $V \in PC^*_{\Gamma\pf}(M)$.
		Then $\id_M \times r_V \colon M \times V \to M \times M$ is transverse to the product cubulation of $M \times M$.
	\end{lemma}
	\begin{proof}
		Easy = see if I did this earlier somewhere
	\end{proof}

	\red{Note $\mc J$ commutes with $\Delta$ and switch them below.}
\end{comment}

\begin{lemma}\label{L: product transversal}
	Let $M$ be a cubulated manifold without boundary.
	Let $V \in PC^*_{\Gamma}(M)$, and suppose $W \in PC_*(M \times M)$ is represented by a collection of embeddings.
	Then there is a proper universal homotopy $h \colon V \times I \to M$ such that $h(-,1) \colon V \to M$ is transverse to the cubulation and $M \times V \xr{\id_M \times h(-,1)} M \times M$ is transverse $W$ in $M \times M$.
\end{lemma}

\begin{proof}
	As in the proof of \cref{P: ball stability}, we use the transversality techniques of \cite[Section 2.3]{GuPo74}.
	Consider $M$ as embedded in some $\R^N$ with an $\epsilon$-neighborhood $M_\epsilon$ and proper submersion $\pi \colon M_\epsilon \to M$.
	We define $H$ as in proof of \cref{P: ball stability} so that $H \colon V \times D^N \to M$ is the proper universal homotopy given by $H(x,s) = \pi(r_V(x)+\epsilon(x)s)$.
	Then $H$ is a submersion and so transverse to each face of the cubulation, and also $\id_M \times H \colon M \times V \times D^N \to M \times M$ is a submersion and hence transverse to $W$.
	So now by the Transversality Theorem of \cite[Section 2.3]{GuPo74}, for any face $E$ of the cubulation $H(-,s)$ is transverse to $E$ for almost all $s \in D^N$ and, similarly, $\id_M \times H(-,s)$ is transverse $W$ for almost all $s \in D^N$ (and similarly for each stratum of $W$).
	As the cubulation must have a countable number of faces and $W$ has a finite number of strata, there is an $s_0 \in D^N$ such that $H(-,s_0)$ is transverse to the cubulation and $\id_M \times H(-,s_0)$ is transverse $W$.
	Now let $h(x,t) = H(x,ts_0)$.
	This is a proper universal homotopy, and $h(-,1)$ has the required transversality properties.
\end{proof}

\begin{lemma}\label{L: M times transverse diag}
	Suppose $M$ is a manifold without boundary, $V \in PC^*_{\Gamma}(M)$, and $V$ is transverse to $W \in PC_*^\Gamma(M)$.
	Then $\id_M \times r_V \colon M \times V \to M \times M$ is transverse to $\diag r_W \colon W \to M \times M$ in $M \times M$, where $\diag \colon M \to M \times M$ is the diagonal map.
	In particular, $\id_M \times r_V \colon M \times V \to M \times M$ is transverse to $\diag \colon M \to M \times M$.
\end{lemma}

\begin{proof}
	By assumption, if $r_V(x) = r_W(y)$, then $Dr_V(T_xV)+Dr_W(T_yW) = T_{r_V(x)}M$.
	Now suppose that $(z,x) \in M \times V$ maps to $\diag r_W(y)$ with $y \in W$.
	This is equivalent to $r_V(x) = r_W(y) = z$.
	At any $(z,x) \in M \times V$, the image of $D(\id_M \times r_V)$ acting on $T_{(z,x)}(M \times V) = T_zM \oplus T_xV$ is $T_z(M) \oplus Dr_V(T_xV)$.
	While the image of $D(\diag r_W)$ acting on $T_yW$ is $Dr_W(T_yW) \oplus Dr_W(T_yW)$.
	As we know $Dr_V(T_xV)+Dr_W(T_yW) = T_{r_V(x)}M$, when $r_V(x) = r_W(y) = z$ these images together span $T_{(r_W(y),r_W(y))}M \times M$.

	The last statement follows by taking $r_W \colon W \to M$ to be $\id_M \colon M \to M$, which is certainly transverse to any $V$.
\end{proof}

For the next lemmas, we make the following definitions.

\begin{definition}
	For a cubulation of $M$, let $\Delta:K_*(M) \to K_*(M \times M)$ be the chain map given by $\Delta(E) = \zeta\xi(\diag E)$ (see \cref{S: cubical products}).
	This is a chain map because it is the restriction of a chain map of the singular cubical complexes to subcomplexes.
\end{definition}

\begin{definition}
	Let $V \in PC^*_\Gamma(M)$.
	Below we write $M \times V$ for the element of $PC^*_\Gamma(M \times M)$ given by the co-oriented exterior product of $V$ with the identity $\id_M \colon M \times M$ given its tautological co-orientation.
\end{definition}

\begin{lemma}\label{L: image of cubical cap}
	Suppose $M$ is a cubulated manifold without boundary and that $K_*(M)$ is the cubical chain complex with respect to some fixed cubulation.
	Let $V \in PC^*_{\Gamma\pf}(M)$ represent a cocycle, and let $W \in K_*(M)$.
	Suppose $\id_M \times r_V \colon M \times V \to M \times M$ is transverse to $\mc J(\Delta(W))$.
	Then $$\mc J(\mc I(V)\frown W) = \pi_1( (M \times V)\times_{M \times M}\mc J(\Delta(W))) \in C_*^\Gamma(M),$$
	where $\pi_1 \colon M \times M \to M$ is the projection to the first factor.
\end{lemma}

\begin{proof}
	For a cubical face $E$ representing an element of $K_*(M)$, let us writem$\xi(\diag (E)) = \sum_i E_{1i} \otimes E_{2i}$, analogously to Sweedler notation.
	By definition, at the chain/cochain level $\mc I(V)\frown E$ is given by
	$$(1 \otimes \mc I(V))(\xi(\diag (E))) = (1 \otimes \mc I(V))\left(\sum_i E_{1i} \otimes E_{2i}\right) = \sum_i E_{1i} \otimes \mc I(V)(E_{2i}) = \sum_i \mc I(V)(E_{2i})\cdot E_{1i},$$
	where $\mc I(V)(E_{2i})$ is the intersection number of $V$ with $E_{2i}$ by \cref{D: intersection homomorphism}.
	So $\mc J(\mc I(V)\frown W)$ is just the geometric cochain represented by $\sum_i I(V,E_{2i})E_{1i} = \sum_i \aug(V \times_M E_{2i})E_{1i} $, identifying the cubical face $E_{1i}$ with its embedding into $M$.
	Note that we have $I(V,E_{2i}) = 0$ if $V$ and $E_{2i}$ do not have complementary dimension in $M$, so we can take the sum $\sum_i I(V,E_{2i})E_{1i}$ to be over those $i$ such that $E_{2i}$ has complementary dimension to $V$.

	On the other hand, $\mc J(\Delta(E))$ is the geometric chain corresponding to $\sum_i E_{1i} \times E_{2i}$, and, applying our transversality assumption, we have
	\begin{align*}
		(M \times V)&\times_{M \times M}\mc J(\zeta\Delta(E))\\
		& = (M \times V)\times_{M \times M}\left(\sum_i E_{1i} \times E_{2i}\right)\\
		& = \sum_i (-1)^{(v+e_{2i}-m)(m-v)}(M \times_M E_{1i}) \times (V \times_M E_{2i})&\text{by \cref{P: cap cross}}\\
		& = \sum_i (-1)^{(v+e_{2i}-m)(m-v)}E_{1i} \times (V \times_M E_{2i})&\text{by \cref{P: cap with 1}}.
	\end{align*}
	We now consider cases depending on the dimension of $V \times_M E_{2i}$.
	If $\dim(V)+\dim(E_{2i})<\dim (M)$, then $V \times_M E_{2i} = \emptyset$, and the corresponding terms in the above formula vanish.
	Similarly if $\dim(V)+\dim(E_{2i})\geq \dim (M)$ but $V$ and $E_{2i}$ do not intersect.
	For the remaining cases, suppose
	$V \times_M E_{2i}\neq \emptyset$.

	If $V$ and $E_{2i}$ have complementary dimension, then $V \times_M E_{2i}$ is $0$ dimensional, and $\pi_1(
	E_{1i} \times (V \times_M E_{2i}))$ is simply $\aug(V \times_M E_{2i})E_{1i}$.
	Furthermore, $(-1)^{(v+e_{2i}-m)(m-v)} = 1$.

	If $\dim(V \times_M E_{2i})\geq 2$, then when we take the projection,
	$\pi_1(E_{1i} \times (V \times_M E_{2i}))$ has small rank.
	In this case, $\dim(\bd (V \times_M E_{2i}))\geq 1$ (or is empty) and so
	$$\bd(\pi_1(E_{1i} \times (V \times_M E_{2i}))) = \pi_1(\bd E_{1i} \times (V \times_M E_{2i}))\pm \pi_1(E_{1i} \times \bd(V \times_M E_{2i}))$$
	also has small rank, and so these terms are degenerate and vanish in $C^\Gamma_*(M)$.

	Finally, suppose $\dim(V \times_M E_{2i}) = 1$.
	Then again $\pi_1(E_{1i} \times (V \times_M E_{2i}))$ has small rank, as does the boundary term $\pi_1(\bd E_{1i} \times (V \times_M E_{2i}))$.
	The second boundary summand $\pm\pi_1(E_{1i} \times \bd(V \times_M E_{2i}))$ may not have small rank.
	However, since $\dim(V \times_M E_{2i}) = 1$, it must consist of mappings of circles and compact intervals, and, therefore, its boundary consists of (maps to $M$ of) pairs of oppositely oriented points.
	So $E_{1i} \times \bd(V \times_M E_{2i})$ consists of pairs of oppositely oriented copies of $E_{1i}$ mapping to $M \times M$, and once we project via $\pi_1$, these pairs become trivial elements of $C^\Gamma_*(M)$.
	So $\pi_1(E_{1i} \times (V \times_M E_{2i}))$ is also degenerate in this case, and these terms are also $0$ in $C^\Gamma_*(M)$.

	\begin{comment}
		We also have
		$$\bd ( V \times_M E_{2i}) = \pm (\bd V) \times_M E_{2i}\pm V \times_M \bd E_{2i}$$.
		Since $V$ is a cocycle, $\bd V$ is a union of trivial or small rank precochains, and hence so is $(\bd V) \times_M E_{2i}$ by Lemma \ref{L: pullback with Q}.
		If $\dim( V \times_M \bd E_{2i})>0$, then $\pi(V \times_M \bd E_{2i})$ has small rank.
		Finally, suppose $\dim (V \times_M \bd E_{2i}) = 0$.
		Then $\dim(V \times_M E_{2i}) = 1$.
		So $V \times_M E_{2i}$ consists of some number of circles and closed intervals in $E_{2i}$, and $V \times_M \bd E_{2i}$ will be the endpoints of those intervals.
		Since these must occur in pairs of opposite signs, then $E_{1i} \times (V \times_M \bd E_{2i})$ will consist of chains $\pm E_{1i} \times pt$, with the points being the points of $V \times_M \bd E_{2i}$.
		Since the points occur in oppositely signed pairs, under $\pi_1$ these pairs become trivial in $M$.
		Altogether then $\pi_1(\bd ( V \times_M E_{2i}))$ is a disjoint union of trivial and small rank precochains.
		So $V \times_M E_{2i} \in Q_*(M)$.
	\end{comment}

	We conclude that $\pi_1( (M \times V)\times_{M \times M}\mc J(\Delta(E)))$ as an element of $C_*^\Gamma(M)$ can be represented as the sum $\sum_i \aug(V \times_M E_{2i})E_{1i}$ over only those $i$ with $E_{2i}$ of complementary dimension to $V$.
	But this is the same formula we derived for $\mc J(\mc I(V)\frown E)$.
\end{proof}

\begin{lemma}\label{L: diagonal version of intersection}
	Let $M$ be a manifold without boundary.
	Let $V \in PC^*_\Gamma(M)$ and $W \in PC_*^\Gamma(M)$ be transverse.
	Let $\pi_1 \colon M \times M \to M$ be the projection on the first factor.
	Then $$V \times_M W = \pi_1((M \times V)\times_{M \times M} \diag(W)).$$
\end{lemma}

\begin{proof}
	By \cref{L: M times transverse diag}, $\id_M \times r_V \colon M \times V \to M \times M$ is transverse to $\diag r_W \colon W \to M \times M$ in $M \times M$, so both expressions are defined.
	We also have $V = M \times_M V = \diag^*(M \times V)$ by \cref{C: cup with identity} and \cref{P: cross to cup}.
	So we can compute
	\begin{align*}
		V \times_M W& = \pi_1\diag (V \times_M W)&\text{since $\pi_1\diag = \id_M$}\\
		& = \pi_1\diag (\diag^*(M \times V) \times_M W)&\text{by the above}\\
		& = \pi_1((V \times M)\times_{M \times M}\diag(W))&\text{by naturality of cap products.}
	\end{align*}
	For the last equality, see \cref{P: natural cap} and its interpretation in terms of naturality of the cap product in \cref{S: (co)chain properties}.
	\cref{P: natural cap} requires $\id_M \times r_V \colon M \times V \to M \times M$ be transverse to $\diag \colon M \to M \times M$ and $W \to M$ be transverse to the pullback of $M \times V$ by $\diag \colon M \to M \times M$ to
	$(M \times V)\times_{M \times M}M \to M$.
	The first requirement holds by \cref{L: M times transverse diag}.
	For the second transversality requirement, \cref{L: transverse to pullback} says that in the presence of the first transversality condition, this is equivalent to requiring $\id_M \times r_V \colon M \times V \to M \times M$ to be transverse to $\diag r_W \colon W \to M \times M$.
	But this also holds by \cref{L: M times transverse diag} as $V$ and $W$ are transverse.
\end{proof}

\begin{lemma}\label{L: diagonal equivalence}
	Let $M$ be a cubulated manifold without boundary.
	Let $W$ be a cycle in $K_*(M)$, and let $V \in PC^*_\Gamma(M)$ be a cocycle such that $\id_M \times r_V \colon M \times V \to M \times M$ is transverse to $\diag(\mc J(W))$ and $\mc J(\Delta(W))$.
	Then $$\underline{\pi_1((M \times V) \times_{M \times M}\diag(\mc J(W)))} = \underline{\pi_1( (M \times V)\times_{M \times M}\mc J(\Delta(W)))} \in H_*^\Gamma(M).$$
\end{lemma}

\begin{proof}
	If we consider $W$ as an element of $NK_*^{sm}(M)$, then the geometric chain $\diag(\mc J(W))$ is represented by the singular cubical chain $\diag(W)$ and $\mc J(\Delta(M))$ is represented by the singular cubical chain $\zeta\xi\diag(W)$.

	As $\zeta\xi: NK_*^{sm}(M) \to NK_*^{sm}(M)$ is chain homotopic to the identity, $\diag(W)$ and $\zeta\xi\diag(W)$ must be homologous in $NK_*^{sm}(M)$, and so they are also homologous as geometric chains.
	In particular, $\diag(\mc J(W))$ and $\mc J(\Delta(M))$ represent the same element of $H_*^\Gamma(M \times M)$.

	As $V$ is a cocycle, so is $M \times V$.
	It now follows from \cref{T: (co)homology products} that
	$(M \times V)\times_{M \times M} \diag(\mc J(W))$ and $(M \times V)\times_{M \times M} \mc J(\Delta(W))$ represent the same geometric homology class, and so their images under $\pi_1$ represent the same geometric homology class.
\end{proof}

\begin{comment}
	Let $E$ be a face of the cubulation.
	Then we can think of $E$ as corresponding to a smooth singular cubical chain represented by the embedding $S_E \colon \interval^n \to M$.
	Then $\diag(\mc J(W))$ is the geometric chain represented by the smooth singular cubical chain $\diag S_E \colon \interval^n \to M \times M$, while $\mc J(\Delta(M))$ is the geometric chain represented by the smooth singular cubical chain $\zeta\xi(\diag S_E)$.
	So, more generally, $\diag(\mc J(W))$ and $\mc J(\Delta(W)))$ must be homologous in $NK^{sm}(M)$.
	So there is a smooth cubical chain $Z$ with $\bd Z = \diag(\mc J(W))-\mc J(\Delta(W)))$ as normalized smooth cubical chains.
	Now thinking of $H$ as a geometric chain, by Lemma \ref{L: product transversal}, we can find a cocycle $V'$ homologous to $V$ so that $M \times V'$ is transverse to $Z$.
	Now we compute using the boundary formula of Proposition \ref{P: Leibniz cap} and that $M \times V'$ is a cocycle that
	\begin{align*}
		\bd((M \times V')&\times_{M \times M} Z)\\
		& = \pm(\bd(M \times V'))\times_{M \times M} Z +(M \times V')\times_{M \times M} \bd Z \\
		& = (M \times V')\times_{M \times M} (\diag(\mc J(W))-\mc J(\Delta(W)))\\
		& = (M \times V')\times_{M \times M} \diag(\mc J(W)) - (M \times V')\times_{M \times M} \mc J(\Delta(W))
	\end{align*}
	Applying $\pi_1$ and that boundaries commute with maps, we obtained the desired homology.
\end{comment}

\begin{proof}[Proof of \cref{T: equivalent cap}]
	Let us first choose a cubical cycle $W$ representing our given cubical homology class.
	By \cref{L: product transversal}, we can choose a representative $V$ of our geometric cohomology class such that $V$ is transverse to the cubulation (and hence to $W$) and $M \times V$ is transverse to $\diag(\mc J(W)) \sqcup \mc J(\Delta(W))$, which is also represented by a union of embeddings.
	Then by \cref{L: image of cubical cap}, we have
	$$\underline{\mc J(\mc I(V)\frown W)} = \underline{\pi_1( (M \times V)\times_{M \times M}\mc J(\Delta(W)))} \in H_*^\Gamma(M),$$
	and by \cref{L: diagonal equivalence} this equals $\underline{\pi_1((M \times V)\times_{M \times M} \diag(\mc J(W)))}$.
	Then by \cref{L: diagonal version of intersection},
	$\pi_1((M \times V)\times_{M \times M} \diag(\mc J(W))) = V \times_M \mc J(W) \in PC_*^\Gamma(M)$.
	Finally, $V \times_M \mc J(W)$ represents $\uV\nplus \mc J(W)$ by definition.
\end{proof}

\subsubsection{Poincar\'e duality}\label{S: PD}

In \cref{T: PD}, we noticed that geometric homology and cohomology satisfy a very strong form of Poincar\'e duality, as for a closed oriented manifold $M$ we in fact have chain-level identities $C^{m-i}_\Gamma(M) = C_i^\Gamma(M)$ obtained by identifying co-oriented cochains with their corresponding oriented chains, using the orientations induced by the orientation of $M$.
\cref{T: equivalent cap} allows us to observe that this strong version of geometric Poincar\'e is compatible with the classical Poincar\'e duality:

\begin{corollary}[Poincar\'e duality]\label{C: PD}
	Let $M$ be a closed oriented cubulated manifold.
	Let $\underline M \in C_m^\Gamma(M)$ be represented by the orientation-preserving identity map $\id_M \colon M \to M$, and let $[M] \in K_*(M)$ represent the cubical fundamental class.
	Then there is a commutative diagram of isomorphisms
	\[
	\begin{tikzcd}
		H^{n-i}_\Gamma(M) \arrow{r}{\nplus \uM} \arrow[d, "\mc I"] & H_i^\Gamma(M) \\
		H^{n-i}_{cub}(M) \arrow{r}{\frown [M]} & H_i^{cub}(M). \arrow[u, "\mc J"']
	\end{tikzcd}
	\]
\end{corollary}

The proof follows immediately from \cref{T: equivalent cap} and the following lemma.

\begin{lemma}
	Let $M$ be closed, oriented, cubulated, and connected.
	Then $\mc J([M]) = \uM \in H_m^\Gamma(M)$.
\end{lemma}

\begin{proof}
	Let $\uV \in H^m_\Gamma(M)$ be represented by a map $V = pt \into M$ taking the point to the center of an $m$-cube of the cubulation, co-oriented so that its normal co-orientation agrees with the orientation of $M$.
	By \cref{P: cap with identity M}, as $\uM$ and $\mc J([M])$ are both represented by embeddings with the same orientation in a neighborhood of the embedded point $V$, the cap products $\uV\nplus \uM$ and $\uV\nplus \mc J([M])$ in $H_0^\Gamma(M)$ are each represented by the same point with its induced orientation (which by \cref{P: cap of immersions} will be the positive orientation).
	This is a generator of $H_0^\Gamma(M) \cong \Z$, as we can see, for example, via our homology isomorphism $H_*(NK_*(M)) \to H_*^\Gamma(M)$.
	As $H_m^\Gamma(M) \cong H_0^\Gamma(M) \cong \Z$ by the isomorphisms between geometric and singular homology and cohomology, $\uV\nplus \colon H_m^\Gamma(M) \to H_0^\Gamma(M)$ must be injective as we have shown it is not the $0$ map.
	Since we have shown $\uV\nplus\uM = \uV\nplus\mc J([M])$, we have $\uM = \mc J([M])$.
\end{proof}

With $\nplus \uM$ as our Poincar\'e duality map, the relation of \cref{P: compare cup and intersection orientations}, which in \cref{S: mixed formulas} became the chain/cochain formula
$$(\uV\uplus \uW)\nplus \uM = (-1)^{(m-v)(m-w)}(\uV\nplus \uM)\bullet(\uW\nplus \uM) = (\uW\nplus \uM)\bullet(\uV\nplus \uM),$$
demonstrates the usual relationship between intersection products and cup products that is well known for homology classes represented by embedded manifolds, cf.\ \cite[Section VI.11]{Bred97}.
Here we see that this relationship extends not just for intersections of embedded manifolds but to all homology classes.
Of course this is always possible if one takes the above formula as a defining formula for the intersection product, but here we see that the intersection product can always be defined geometrically in terms of fiber products.

\subsubsection{Umkehr maps}

\Cref{C: PD} allows us to make some remarks about umkehr maps, also known as wrong-way or transfer maps, associated to maps of closed oriented manifolds $f \colon N \to M$.
These are maps
\begin{align*}
	f^! \colon H^{n-i}_\Gamma(N) \to H^{m-i}_\Gamma(M)\\
	f_! \colon H_{m-i}^\Gamma(M) \to H_{n-i}^\Gamma(N),
\end{align*}
typically defined by taking a homology or cohomology class, dualizing using Poincar\'e duality, applying $f$ or $f^*$, and then dualizing again; see \cite[Definition VI.11.2]{Bred97}.
We will show that when $M$ and $N$ are closed and oriented, these transfer maps correspond to the pullbacks and pushforwards already encountered in \cref{S: functoriality}, where we only required for cohomology pushforwards that $f$ be proper and co-oriented and for homology pullbacks that $f$ be proper and that $M$ and $N$ be oriented.

\begin{proposition}
	Let $f \colon N \to M$ be a map of closed oriented manifolds.
	We may consider $f$ co-oriented via the orientations of $M$ and $N$.
	Then the following diagrams commute:
	\[
	\begin{tikzcd}
		H^{n-i}_\Gamma(N) \arrow{r}{f} \arrow[d, "\nplus \uN"] & H^{m-i}_\Gamma(M) \arrow[d, "\nplus \uM"] & H^{i}_\Gamma(M) \arrow{r}{f^*} \arrow[d, "\nplus \uM"] & H^{i}_\Gamma(N) \arrow[d, "\nplus \uN"] \\
		H_i^\Gamma(N) \arrow{r}{f} & H_i^\Gamma(M) & H_{m-i}^\Gamma(M) \arrow{r}{f^*} & H_{n-i}^\Gamma(N).
	\end{tikzcd}
	\]
\end{proposition}

\begin{proof}
	We start with the diagram on the left.
	Let $\uV \in H^{n-i}_\Gamma(N)$ represented by a co-oriented map $r_v \colon V \to N$.
	Then $\uV\nplus\uN$ is represented by the same map to $N$ with its induced orientation; see \cref{S: co-orientations}.
	In particular, if $x \in V$ then $V$ is oriented at $x$ by the local orientation $\beta_V$ such that $(\beta_V,\beta_N)$ gives the co-orientation of $r_V$.
	The path down then right is then the composition $fr_V$, considering $V$ with its orientation induced by $r_V$ and the orientation of $N$.
	On the other hand, by the definition in \cref{S: covariant functoriality}, the element $f(\uV)$ in $H^{m-i}_\Gamma(M)$ is represented by $fr_V$ co-oriented by composing the co-orientations of $r_V$ and $f$.
	So if the co-orientation of $r_V$ is again $(\beta_V,\beta_N)$, the co-orientation of $fr_V$ representing $f(\uV)$ is $(\beta_V,\beta_N)*(\beta_N,\beta_M) = (\beta_V,\beta_M)$.
	So $f(\uV)\nplus \uM$ is represented by $fr_V$ with $V$ oriented again by $\beta_V$.
	Thus the diagram commutes.

	For the second diagram, let $r_V \colon V \to M$ represent $\uV \in H^{i}_\Gamma(M)$.
	We can assume up to a homotopy that $f$ is smooth and transverse to $r_V$.
	Then $f^*(\uV)$ is represented by the co-oriented pullback $V \times_M N \to N$, and $f^*(\uV)\nplus\uN$ is represented by this map with the orientation on $V \times_M N$ induced by the pullback co-orientation and the orientation of $N$.
	Meanwhile, $\uV\nplus \uM$ is represented by $r_V$ with the orientation consistent with the given co-orientation and the orientation of $M$.
	Applying $f^*$, \cref{D: cohomology pullback and homology transfer} gives the pullback $f^*(\uV\nplus \uM)$ the same orientation just described for $f^*(\uV)\nplus\uN$.
\end{proof}

\begin{corollary}
	If $f \colon N \to M$ is a map of closed oriented manifolds then
	$f^! = f \colon H^{n-i}(N) \to H^{m-i}(M)$ and
\end{corollary}

\begin{comment}
	We know $H_m^\Gamma(M) \cong \Z$ by the isomorphism between geometric and singular homology.
	We first show that $\uM$ is a generator.

	Let $\uV \in H^m_\Gamma(M)$ be represented by a map $V = pt \into M$ taking the point to the center of an $m$-cube of the cubulation, co-oriented so that its normal co-orientation agrees with the orientation of $M$.
	Then by Proposition \ref{P: cap with identity M}, the cap product $\uV\nplus \uM \in H_0^\Gamma(M)$ is represented by the same embedding of a point with its induced orientation (which by Proposition \ref{P: cap of immersions} will be the positive orientation).
	This is certainly a generator of $H_0^\Gamma(M) \cong \Z$, as we can see, for example, via our homology isomorphism $H_*(NK_*(M)) \to H_*^\Gamma(M)$.
	If there were another class $\uW \in H_m^\Gamma(M)$ with $k\uW = \uM$, $|k|>1$, then $\uV\nplus \uM = \uV\nplus (k\uW) = k(\uV\nplus \uW)$ could not be a generator, providing a contradiction.
	So $\uM$ is a generator of $H_m^\Gamma(M)$.

	Now, as $\mc J$ is a homology isomorphism and $[M]$ is a generator of $H_m^{cub}(M)$ by the classical theory, we must have $\mc J([M]) = \pm\uM$.
	But by Proposition \ref{P: cap with identity M}, as $\uM$ and $\mc J([M])$ are both represented by embeddings with the same orientation in a neighborhood of the embedded point $V$, the cap products $\uV\nplus \uM$ and $\uV\nplus \mc J([M])$ in $H_0^\Gamma(M)$ are again each represented by the same point with its positive orientation.
	So we must have $\mc J([M]) = \uM$ rather than $-\uM$.
\end{comment}

\begin{comment}
	\begin{lemma}
		With the hypotheses of Corollary \ref{C: PD}, $\mc J([M]) = \uM$.
	\end{lemma}
	\begin{proof}
		It suffices to assume $M$ is connected, otherwise we could work component-wise.
		By the preceding lemma and classical theory, we know that $\uM$ and $M$ are respective generators of $H_m^\Gamma(M) \cong H_m(M) \cong \Z$, so, as $\mc J$ induces an isomorphism on homology, we must have $\mc J([M]) = \pm\uM$.

		Let $\uV \in H^m_\Gamma(M)$ be represented by a map $V = pt \into M$ taking the point to the center of an $m$-cube, co-oriented so that its normal co-orientation agrees with the orientation of $M$.
		Then by Proposition \ref{P: cap with identity M}, as $\uM$ and $\mc J([M])$ both have the same orientation in a neighborhood of the embedded point, the cap products $\uV\nplus \uM$ and $\uV\nplus \mc J([M])$ in $H_0^\Gamma(M)$ are each represented by the same embedding of a point with its induced orientation (which by Proposition \ref{P: cap of immersions} will be the positive orientation).
		So we must have $\mc J([M]) = \uM$ rather than $-\uM$.
	\end{proof}
\end{comment} %
	% !TEX root = ../foundations.tex

\section{Questions}

\begin{enumerate}

	\item Is $H^*_\Gamma(X)\cong H^*(X)$ when not finitely generated?

	\item Is the cap product $H^i_\Gamma(M)\otimes H_{i+j}^\Gamma(M)\to H^\Gamma_j(M)$ isomorphic to the singular co(homology) cap product when $H^*(X)$ is not finitely generated?

	\begin{comment}
		\item For a closed oriented $M^m$, is the Poincar\'e duality map $H^i_\Gamma(M)\to H_{m-i}^\Gamma(M)$ that takes a cochain to a chain simply by converting the co-orientation to an orientation (via our standard construction over oriented manifolds) isomorphic to the singular co(homology) cap product with the fundamental class via the isomorphisms of homology and cohomology groups we have developed.
	\end{comment}

	\item What can we say about geometric homology/cohomology with coefficients?

	\item Can we develop theories of relative geometric homology/cohomology?

	\item If we consider only compact objects in $C_\Gamma^*(M)$ is the resulting cohomology theory isomorphic to $H^*_c(M)$?

	\item If we allow objects in $C^*_\Gamma(M)$ to be noncompact but with proper maps to $M$, is the resulting homology theory isomorphic to the standard $H^\infty_*(M)$?

	\item Are the cup and cap products we obtain working with $H^*_c(M)$ and $H^\infty_*(M)$ isomorphic to the standard ones?

	\item Can we show that our convention for co-orientating fiber products is unique for some nice set of properties?

	\item Can we express the map $\Ext(H_{i-1}^\Gamma(M),\Z)\to H^i_\Gamma(M)$ of Theorem \ref{T: UCT} in terms of linking numbers?

\end{enumerate}

	\bibliographystyle{alpha}
	\bibliography{foundations}

	\pagebreak
	% !TEX root = ../foundations.tex

\section*{To-Do}
\begin{enumerate}
	\item Is there a place where we do not use integer coefficients? Should Or(E) just be $\wedge^n(E)$?

	\item Add an argument that our definition of co-orientation agrees with Lipyanskiy's?

	\item Thom isomorphism (what does this mean in this setting - we can't have a singular space, and we do not have relative cohomology)

	\item anibal: diagrams in diagrams.sty are sensitive to having a blank space before and after.
	Make this choice homogeneous all around.

	\item anibal: consider a command \verb|\ie| to make homogeneous the use of i.e.\ since some use a \verb|\ | after.

	\item Greg left the following in in s3 and I commented out.
	The exact position can be found searching for any string in it.

	Dev and Anibal, please check the following arguments carefully as I'm not 100\% confident in it.
	It gives the ``right'' answer but I'm a little uncomfortable divorcing the order of the orientation terms from the order of the manifold terms.
	Of course this happens all the time - even if we think of $\R^2$ as $\R_x \oplus \R_y$ we can still think about the two-form $y \wedge x$, but I'm still a little nervous about maybe having missed a sign somewhere.
	I'm also a little nervous about my trick of taking $a$ and $b$ to be even so that they won't contribute, but the earlier work says that this should be allowable.
	Presumably if I didn't do this there would be a bunch of extra signs that miraculous cancel out, but I'm not so excited about trying that out in detail to see.

	\item Similarly with:

	Add a version for pullbacks (as opposed to fiber products)? Might have some extra signs to figure out.
	Not really needed anywhere I do not think.

\begin{comment}
	\item \sout{Picture for creasing.}
	\item Compactness and orientation assumptions on Theorem 3.13 (transversality constrains preserve q-iso type).

	\item \sout{Treatment of creasing.}
	\item Guillemin-Pollock for mnfds with corner.

	\item Clarify isomorphisms used in orientations and make more explicit how the Lipyanskiy orientations fit.


	\item More on Mayer-Vietoris - check full argument
	\item Poincar\'e Lemma - check new proof
	\item (Anibal) Add a better treatment of ``cst" from \verb|Flows/old/pd_cubical_S2.Feb16.tex| \\
	Greg: Let K be any finite set of cubical faces and let L be a single cubical face. We need $cst(K) \cup cst(L)$ to be $cst(K \cup L)$ (maybe this part is just by definition?) and we need $cst(K) \cap cst(L)$ to be $cst(K ? L)$ where $K ? L$ needs to be some set of faces with cardinality less than or equal to that of K.
\end{comment}

\begin{comment}
\item Reference for pullback of normal bundle is normal bundle of pullback
\item Reference for pullback of tangent spaces is tangent space of pullbacks (argument already given?)
\end{comment}


\section*{To-Do elsewhere}


\end{enumerate}

%%% Local Variables:
%%% TeX-master: "foundations.tex"
%%% End: %
\end{document}