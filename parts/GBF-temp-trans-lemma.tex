\documentclass[12pt]{article}
\paperheight=11in
%\paperwidth=8.5in
\renewcommand{\baselinestretch}{1.04}
\usepackage{amsmath,amsthm,verbatim,amssymb,amsfonts,amscd, graphicx, mathrsfs, hyperref,mathtools,multicol, enumitem,bbm,diagrams}
\usepackage[usenames,dvipsnames]{color}
\usepackage{amstext}
%\usepackage{cite}
%\usepackage[notref]{showkeys}
\usepackage{color}


\newcommand\purple[1]{\marginpar{replacement}\textcolor{Purple}{#1}}     
\newcommand\blue[1]{\marginpar{new}\textcolor{blue}{#1}}                       %
\newcommand\red[1]{\marginpar{??}\textcolor{red}{#1}}                         %
\newcommand\green[1]{\marginpar{delete ok?}\textcolor{green}{#1}}
%\newcommand\ownremark[1]{\marginpar{remark for own use}\textcolor[rgb]{0.5,0.5,0}{#1}}                   %


\topmargin0.0cm
\headheight0.0cm
\headsep0.0cm
\oddsidemargin0.0cm
\textheight23.0cm
\textwidth16.5cm
\footskip1.0cm
\theoremstyle{plain}
\newtheorem{theorem}{Theorem}[section]
\newtheorem{corollary}[theorem]{Corollary}
\newtheorem{lemma}[theorem]{Lemma}
\newtheorem{proposition}[theorem]{Proposition}
\newtheorem*{theorem*}{Theorem}
\newtheorem*{lemma*}{Lemma}
\newtheorem*{proposition*}{Proposition}


\theoremstyle{definition}
\newtheorem{definition}[theorem]{Definition}

\theoremstyle{remark}
\newtheorem{remark}[theorem]{Remark}
\newtheorem{example}[theorem]{Example}
\newtheorem{exercise}[theorem]{Exercise}
\newcommand{\codim}{\text{codim}}
 

\newcommand{\primeset}[1]{#1}
\newcommand{\id}{\textup{id}}
\newcommand{\onto}{\twoheadrightarrow}
\newcommand{\hra}{\hookrightarrow} 
\newcommand{\Td}[1]{\Tilde{#1}}
\newcommand{\td}[1]{\tilde{#1}}
\newcommand{\into}{\hookrightarrow}
\newcommand{\PP}{\mathbb{P}}
\newcommand{\bS}{\mathbb{S}}
\newcommand{\X}{\mathbb{X}}
\newcommand{\Z}{\mathbb{Z}}
\newcommand{\Q}{\mathbb{Q}}
\newcommand{\R}{\mathbb{R}}
\newcommand{\G}{\mathbb{G}}
\newcommand{\N}{\mathbb{N}}
\newcommand{\F}{\mathbb{F}}
\newcommand{\C}{\mathbb{C}}
\newcommand{\D}{\mathbb{D}}

\renewcommand{\L}{\mathbb{L}}
\newcommand{\bd}{\partial}
\newcommand{\pf}{\pitchfork}
\newcommand{\ra}{\rightarrow}
\newcommand{\la}{\leftarrow}
\newcommand{\Ra}{\Rightarrow}
\renewcommand{\H}{\mathbb H}
\newcommand{\rla}{\RightLeftarrow}
\newcommand{\mc}[1]{\mathcal{#1}}
\newcommand{\ms}[1]{\mathscr{#1}}
\newcommand{\bb}[1]{\mathbb{#1}}
\newcommand{\dlim}{\varinjlim}
\newcommand{\vg}{\varGamma}
\newcommand{\blm}[2]{\langle  #1 , #2 \rangle}
\newcommand{\bl}[2]{\left( #1 , #2 \right)}
\newcommand{\vs}{\varSigma}
\newcommand{\holink}{\text{holink}}
\newcommand{\map}{\operatorname{map}}
\newcommand{\hl}{\operatorname{holink}}
\newcommand{\wt}{\widetilde}
\renewcommand{\hom}{\text{Hom}}
\newcommand{\Hom}{\text{Hom}}
\newcommand{\SHom}{\text{\emph{Hom}}}
\newcommand{\Ext}{\text{Ext}}
\newcommand{\mf}{\mathfrak}
\newcommand{\ih}{IH^{\bar p}}
\newcommand{\di}{\text{dim}}
\newcommand{\im}{\text{im}}
\newcommand{\cok}{\text{cok}}
\newcommand{\coim}{\text{coim}}
\newcommand{\bp}{\boxplus}
%\renewcommand{\P}{\mathbb P}
\newcommand{\q}{\mathfrak q}
\newcommand{\supp}{\text{supp}}
\newcommand{\singsupp}{\text{singsupp}}
\newcommand{\Dom}{\text{Dom}}
\newcommand{\LPDO}{\text{LPDO}}
\newcommand{\PsiDO}{\Psi\text{DO}}

\newcommand{\ka}{\kappa}

\newcommand{\fl}{\text{FL}}
\newcommand{\wfl}{\text{WFL}}
\newcommand{\Ker}{\mbox{Kernel }}
%\newcommand{\p}{\mf{p}}
\newcommand{\p}{\mathbbm{p}}
%\newcommand{\p}{\mathpzc{p}}
\newcommand{\Vol}{\text{Vol}}
\newcommand{\uW}{\underline{W}}
\newcommand{\udW}{\underline{\partial W}}
\newcommand{\uV}{\underline{V}}



\newcommand{\sect}[1]{\vskip1cm \noindent\paragraph{#1}}

\newcommand{\ttau}{\text{\texthtt}}

\newcommand{\xr}{\xrightarrow}
\newcommand{\xl}{\xleftarrow}

\DeclareRobustCommand{\zvec}[1]{%
  \mathrlap{\vec{\mkern-2mu\phantom{#1}}}#1%
}

\DeclareMathAlphabet{\mathpzc}{OT1}{pzc}{m}{it}
\newcommand{\cman}{\mathrm{cMan}}

\newcommand{\Or}{{\rm Det}}

\begin{document}

\begin{proposition}\label{P: 3 out of 4 trans}
Let $f:V\to M$, $g:W\to M$, and $h:Z\to M$ be maps from manifolds with corners to a manifold without boundary. Suppose that $W$ is transverse to $Z$ and that $V$ is transverse to $W$ and to $W\times_MZ$. Then $V\times_MW$ is transverse to $Z$. In particular, if $V\times_M(W\times_MZ)$ and $V\times_MW$ are well defined, then so is $(V\times_MW)\times_M Z$.
\end{proposition}
\begin{proof}
We must show that $V\times_MW$ is transverse to $Z$, so we consider points $(v,w)\in V\times_M W$ and $z\in Z$ such that $h(z)$ is equal to $(f\times_Mg)(v,w)$, which by definition is equal to $f(v)=g(w)$. In other words, we consider  $(v,w,z)\in V\times W\times Z$ such that $f(v)=g(w)=h(z)$. 

So suppose $(v,w,z)$ is such a triple, and denote the common image by $m\in M$.  By the transversality assumptions, we know that the images of $D_wg:T_wW\to T_mM$ and $D_zh:T_zZ\to T_mM$ span $T_mM$, i.e.\ that $D_wg$ and $D_zh$ are transverse as linear maps, and similarly that $D_{(w,z)}(g\times_M h):T_{(w,z)}(W\times_M Z)\to T_mM$ is transverse to $D_vf:T_vV\to T_mM$. Furthermore, by  Lemma \ref{L: tangent of pullbacks},  the tangent space of a fiber product is the fiber product of the tangent spaces, so $T_{(w,z)}(W\times_M Z)=T_wW\times_{T_mM}T_zZ$ and $D_{(w,z)}(g\times_M h)=D_wg\times_{T_mM}D_zh$. 

Now by \cite[Propositions~4-9]{RamBas09}, the triple of linear maps $(D_vf,D_wg,D_zh)$ is transverse as a triple of maps, if and only if both $D_wg$ is transverse to $D_zh$ and $D_wg\times_{T_mM}D_zh$ is transverse to $D_vf$. As such statements are independent of how we order the terms, the transversality established in the preceding paragraph also implies that $D_vf$ and $D_wg$ are transverse (which already follows from the hypotheses of the proposition), and $D_vf\times_{T_mM}D_wg$ is transverse to $D_zh$. But this implies, again using Lemma \ref{L: tangent of pullbacks}, that $h$ is transverse to $f\times_Mg$, as desired.  
\end{proof}

\begin{remark}
The end of the preceding proof at first seems to imply that if  $g$ and $h$ are transverse and $g\times_M h$ is transverse to $f$, then $f$ is transverse to $g$ and $f\times_Mg$ is transverse to $h$. Indeed,  \cite[Propositions~4-9]{RamBas09} says this is the case for the linear maps of the tangent spaces. Unfortunately, however, as \cite[Propositions~4-9]{RamBas09} applies only to linear maps, we can apply it only at those points $(v,w,z)\in V\times W\times Z$ where we know that $f(v)=g(w)=h(z)$ so that all three tangent space maps are well defined. So such a result would hold if the only intersections among the maps were such triple intersections. However, as noted in Remark \ref{R: multiproducts}, there could be pairs $(v,w)\in V\times W$ with $f(v)=g(w)$, but with this common image in $M$ not in the image of $h$. At such points, \cite[Propositions~4-9]{RamBas09} cannot tell us anything about the transversality of $V$ and $W$, and so $V\times_MW$ might not be well defined due to failure of transversality, even if $V\times_M (W\times_M Z)$ is. 
\end{remark}


\bibliographystyle{amsplain}
\bibliography{../../bib}




\end{document}
