% !TEX root = ../foundations.tex

\section*{To-Do}
\begin{enumerate}
	\item Is there a place where we don't use integer coefficients? Should Or(E) just be $\wedge^n(E)$?

	\item Add an argument that our definition of co-orientation agrees with Lipyanskiy's?

	\item Thom isomorphism (what does this mean in this setting - we can't have a singular space, and we don't have relative cohomology)




	\item anibal: diagrams in diagrams.sty are sensitive to having a blank space before and after.
	Make this choice homogeneous all around.

	\item anibal: consider a command \verb|\ie| to make homogeneous the use of i.e.\ since some use a \verb|\ | after.


\begin{comment}
	\item \sout{Picture for creasing.}
	\item Compactness and orientation assumptions on Theorem 3.13 (transversality constrains preserve q-iso type).

	\item \sout{Treatment of creasing.}
	\item Guillemin-Pollock for mnfds with corner.

	\item Clarify isomorphisms used in orientations and make more explicit how the Lipyanskiy orientations fit.


	\item More on Mayer-Vietoris - check full argument
	\item Poincar\'e Lemma - check new proof
	\item (Anibal) Add a better treatment of ``cst" from \verb|Flows/old/pd_cubical_S2.Feb16.tex| \\
	Greg: Let K be any finite set of cubical faces and let L be a single cubical face. We need $cst(K) \cup cst(L)$ to be $cst(K \cup L)$ (maybe this part is just by definition?) and we need $cst(K) \cap cst(L)$ to be $cst(K ? L)$ where $K ? L$ needs to be some set of faces with cardinality less than or equal to that of K.
\end{comment}

\begin{comment}
\item Reference for pullback of normal bundle is normal bundle of pullback
\item Reference for pullback of tangent spaces is tangent space of pullbacks (argument already given?)
\end{comment}


\section*{To-Do elsewhere}


\end{enumerate}

%%% Local Variables:
%%% TeX-master: "foundations.tex"
%%% End: