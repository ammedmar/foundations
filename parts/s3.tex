% !TEX root = ../foundations.tex

\section{Orientations and co-orientations}\label{S: orientations and co-orientations}

Manifolds with corners are, in particular, topological manifolds, and so they carry the standard notions of orientability and orientation.
As in singular or simplicial homology, orientations carry sign information in geometric versions of homology theory.
For geometric cohomology, however, it turns out that the natural structures to carry sign information are co-orientations, sometimes called orientations of maps.
Unlike orientations, co-orientation can be ``pulled-back.''

Co-orientations are less familiar than orientations, so it is helpful to keep the following central example in mind: If $W \to M$ is an immersion of manifolds, a co-orientation is equivalent to an orientation of the normal bundle of the image; see \cref{normal co-or} below.
Notice that this condition does not require the orientability of either $W$ or $M$.
In fact, an important case is when neither $W$ nor $M$ is orientable but the monodromies of their orientation bundles around loops in $W$ are either both orientation-preserving or both orientation-reversing; it is in this sense that we have a \textit{co}-orientation.

The special case of (local) immersions is important both for intuition and in practice; for example, the only geometric cochains which evaluate non-trivially on fixed collections of chains through the intersection homomorphisms of \cite{FMS-flows} are local immersions.
However, for complete definitions that cover all cases of interest, it is critical to co-orient more general maps and to do so in a way that provides key properties when forming pullbacks, such as a Leibniz rule when taking boundaries, graded commutativity of cochains, and a simple expression when the maps are immersions.
While co-orientations can be found in many places in the literature, we could not find a careful treatment that guaranteed these key properties.
Therefore, we develop co-orientations in depth in this (long) section.
Some readers may prefer to skip ahead, either considering geometric cohomology only with mod~2 coefficients in a first reading, or perhaps take orientation of the normal bundle as a temporary partial definition, coming back later to understand the general setting.

\subsection{Orientations}\label{S: orientations}

If $W$ is a manifold with corners then it is a topological manifold with boundary and so in the interior of $W$, i.e.\ on $S^0(W)$, we can consider orientability and orientations for $W$ in the usual sense, from either the topological or smooth manifold points of view, which are equivalent \cite[Theorem VI.7.15]{Bred97}.
Following standard conventions, we typically refer to an oriented manifold with corners $W$ with the orientation tacit.
If an orientation on $W$ is understood, then $-W$ refers to $W$ with the opposite orientation.

When $W$ is orientable, so is its (topological) boundary \cite[Lemma 6.9.1]{Bred97}, and since $S^0(W) \cup S^1(W)$ is a smooth manifold with boundary, we can allow an orientation of $W$ to determine an orientation of $S^1(W)$ using standard smooth manifold conventions.
In particular, we choose the following convention, which agrees with that of Joyce \cite[Convention 7.2.a]{Joy12}:

\begin{convention}\label{Con: oriented boundary}
	For a smooth oriented manifold with boundary $N$, we orient $\bd N$ by stipulating that an outward normal vector followed by an oriented basis of $\bd N$ yields an oriented basis for $N$.
\end{convention}

When $W$ is an oriented manifold with corners, we can identify $S^1(W)$ with the interior of $\bd W$, and this convention determines an orientation of $\bd W$.

When we wish to work with orientations symbolically, the following interpretation will be extremely useful.

\begin{definition}\label{D: det bundle}
	Let $E \to M$ be a rank $d$ real vector bundle.
	Define the \textbf{determinant line bundle}
	$\Or(E)$ to be $\bigwedge^d E$.
	If $d = 0$ this is interpreted to be the trivial rank one ``bundle of coefficients.''
	We call the principal $O(1) \cong C_2$ bundle associated to $\Or(E)$ the \textbf{orientation cover}.

	An orientation of $E$ is then a section of the orientation cover associated to $\Or(E)$ or, equivalently, an equivalence class of non-zero sections of $\Or(E)$ such that two sections are equivalent if they differ by multiplication by an everywhere positive scalar function.
\end{definition}

In particular, we thus think of orientations of $M^m$ as $m$-forms in $\Or(TM)$.
We typically use the notation $\beta_M$ to stand for such an $m$-form, and, as we are only ever interested in $m$-forms to represent orientations, we systematically abuse notation by not distinguishing between an $m$-form and its equivalence class.
Thus all expressions such as $\beta_M = \beta_V \wedge \beta_W$ should be interpreted as equality of equivalence classes.

\subsubsection{Orientations of fiber products}

If $V$ and $W$ are oriented manifolds, we orient $V \times W$ in the standard way by concatenating oriented bases of tangent spaces of $V$ with those of $W$.
More generally, if $f \colon V \to M$ and $g \colon W \to M$ are transverse maps with $V$, $W$, and $M$ all oriented and $M$ without boundary, Joyce defines an orientation on the pullback $V \times_M W$ as follows \cite[Convention 7.2b]{Joy12}.
Consider the short exact sequence of vector bundles over $P = V \times_M W$ given by
\begin{equation}\label{E: fiber orientation}
	0 \to TP \xr{D\pi_V \oplus D\pi_W} \pi_V^*(TV) \oplus \pi_W^*(TW) \xr{\pi_V^*(Df)-\pi_W^*(Dg)} (f\pi_V)^*TM \to 0.
\end{equation}
Here $\pi_V$ and $\pi_W$ are the projections of $V \times_M W$ to $V$ and $W$, and $D\pi_V$ is being treated as a map $TP \to \pi_V^*TV$ and similarly for $D\pi_W$.
Analogously, $\pi_V^*(Df)$ is the pullback of the map $Df \colon TV \to TM$ obtained first by treating it as a map $TV \to f^*(TM)$ and then pulling back functorially by $\pi_V^*$, and similarly for $\pi_W^*(Dg)$.
By choosing a splitting, this sequence determines an isomorphism
\begin{equation*}
	TP \oplus (f\pi_V)^*TM\cong\pi_V^*(TV) \oplus \pi_W^*(TW).
\end{equation*}
The choices of orientations determine orientations on all summands in this expression except $TP$.
The orientation on $TP$ is then chosen so that the two direct sums differ in orientation by a factor of $(-1)^{\dim(W)\dim(M)}$.

Much more about the orientation of fiber products can be found in the technical report of Ramshaw and Basch \cite{RamBas09}.
While the focus there is on manifolds without boundary, and sometimes just fiber products of linear maps of vector spaces, the results about orientations extend to manifolds with corners by employing them on the top-dimensional stratum and utilizing their stability property, by which orientation properties of fiber products of linear maps extend to properties of fiber products of transverse manifolds (see \cite[Sections 6.3, 9.1.2, and 9.3]{RamBas09}).
Their orientation of fiber products agrees with Joyce's.
This can be checked directly from the definitions\footnote{Their multiplicative ``fudge factor'' in \cite[Theorem 9.14]{RamBas09} at first appears to be different from Joyce's, but this is only because their conventions utilize what in our notation would be the map $\pi_W^*(Dg)-\pi_V^*(Df)$ rather than $\pi_V^*(Df)-\pi_W^*(Dg)$.} or, as Joyce notes in \cite[Remark 7.6.iii]{Joy12}, axiomatically, as Ramshaw and Basch show that theirs is the unique choice of orientation convention satisfying certain basic expected properties.
It is these properties that determine the sign in the definition.
We state these properties in the following two propositions.

\begin{proposition}\label{P: oriented fiber product basic properties}
	Let $f \colon V \to M$ and $g \colon W \to M$ be transverse maps from oriented manifolds with corners to an oriented manifold without boundary.
	\begin{enumerate}
		\item When $M$ is a point, the oriented fiber product $V \times_M W$ is simply $V \times W$, and in this case the fiber product orientation is consistent with the basic concatenation rule for products.
		\item When one of the maps is the identity $\id_M \colon M \to M$, the projection maps to the other factors are orientation preserving diffeomorphisms
		\begin{equation*}
			M \times_M V = V\quad\text{and}\quad V \times_M M = V
		\end{equation*}
	\end{enumerate}
\end{proposition}

\begin{proposition}\label{P: oriented fiber mixed associativity}
	Let $V$, $W$, and $Z$ be oriented manifolds with corners, and let $M$ and $N$ be oriented manifolds without boundary.
	Then the ``mixed associativity'' formula for oriented fiber products
	\begin{equation}\label{E: mixed associativity fiber orientation}
		(V \times_M W)\times_N Z = V \times_M (W\times_N Z)
	\end{equation}
	holds when given maps
	$$V \xr{f} M\xleftarrow{g} W \xr{h} N\xleftarrow{k} Z$$
	and assuming sufficient transversality for all the fiber products in \eqref{E: mixed associativity fiber orientation} to be well defined (see \cref{R: multiproducts}).
	In this case the map $V \times_M W \to N$ is given by composing the projection from $V \times_M W$ to $W$ with $h$, and similarly for the map $W\times_N Z \to M$.
\end{proposition}

These propositions are evident at the level of spaces.
When taking orientations into account, the first property in \cref{P: oriented fiber product basic properties} is proven in \cite[Sections 9.3.9]{RamBas09} as the ``concatenation axiom,'' and the second is proven in \cite[Sections 9.3.5 and 9.3.6]{RamBas09} as the ``left and right identity axioms.''
The mixed associativity property is proven in \cite[Sections 9.3.7]{RamBas09}.
An important special case of this associativity that we will need below occurs when $M = N$ and $g = h$, so that our initial data is three maps all to $M$.
In this case we have the ordinary associativity
\begin{equation}\label{E: oriented fiber associativity}
	(V \times_M W) \times_M Z = V \times_M (W \times_M Z).
\end{equation}
That these properties determine the orientation rule is the content of \cite[Theorem 9-10]{RamBas09}.
Technically they require for uniqueness two other properties: an Isomorphism Axiom, which says that the construction is consistent across oriented homeomorphisms, and a Stability Axiom, which implies that the orientation can be determined pointwise in a globally consistent manner.
These properties are both implicit in Joyce's global definition of the fiber product orientation.

There is also a commutativity rule proven in \cite[Sections 9.3.8]{RamBas09} that follows from the other properties:

\begin{proposition}\label{P: commute oriented fiber}
	Let $f \colon V \to M$ and $g \colon W \to M$ be transverse maps from oriented manifolds with corners to an oriented manifold without boundary.
	Then, as oriented manifolds,
	\begin{equation*}
		V \times_M W = (-1)^{(m-v)(m-w)}W \times_M V.
	\end{equation*}
\end{proposition}

This means that the canonical diffeomorphism taking $(v,w) \in V \times_M W \subset V \times W$ to $(w,v) \in W \times_M V \subset W \times V$ takes a positively-oriented basis of the tangent space of $V \times_M W$ to a $(-1)^{(\dim M-\dim V)(\dim M-\dim W)}$-oriented basis of the tangent space of $W \times_M V$.
We note that these signs, while note quite in line with the Koszul conventions, agree with those for the intersection product of homology classes in Dold \cite[Section VIII.13]{Dol72}.

Furthermore, with our convention for oriented boundaries, one obtains the following useful identity of oriented manifolds with corners; see \cite[Propositions 7.4 and 7.5]{Joy12}

\begin{proposition}\label{P: oriented fiber boundary}
	Let $f \colon V \to M$ and $g \colon W \to M$ be transverse maps from oriented manifolds with corners to an oriented manifold without boundary.
	Then, as oriented manifolds,
	\begin{equation*}
		\bd (V \times_M W) = (\bd V \times_M W) \sqcup (-1)^{m-v}(V \times_M \bd W).
	\end{equation*}
\end{proposition}

\medskip\noindent\textbf{Fiber products of immersions.}
% anibal: should this be a \subsubsection?

The special case of fiber products with $f \colon V \to M$ and $g \colon W \to M$ embeddings or, a bit more generally, immersions, is always of particular interest, especially for developing intuition.
In the case of embeddings, $V \times_M W$ is simply the intersection of $V$ and $W$ as submanifolds of $M$, and in the immersed case this is true locally, i.e.\ restricting attention to submanifolds of $V$ and $W$ on which $f$ and $g$ are embeddings.
Let us try to understand the orientation of $V \times_M W$ in this setting.

For convenience of notation, let us assume $f$ and $g$ are embeddings and write $P = V \times_M W = V \cap W$.
As orientations are defined via the tangent bundles and as the tangent bundle of a fiber product is the fiber product of the tangent bundles by \cref{L: tangent of pullbacks}, it suffices to consider $f$ and $g$ as linear embeddings of vector spaces.
Then we can identify $V$, $W$, and $P$ as subspaces and write $M = \nu W \oplus P \oplus \nu V$, where $\nu W \subset V$ and $\nu V \subset W$ are complementary subspaces to $P$ within $V$ and $W$ respectively.
In particular, $V = P \oplus \nu W$ and $W = P \oplus \nu V$, and we think of $\nu W$ and $\nu V$ as representing normal bundles to $P$.
With these identifications, the maps $f$ and $g$ are simply the identifications with the appropriate subspaces.

We can now write a splitting $M \to V \oplus W$ of the exact sequence \eqref{E: fiber orientation} in terms of these decomposition as
$$\nu W \oplus P \oplus \nu V \to P \oplus \nu W \oplus P \oplus \nu V,$$
given by $(x,p,y) \mapsto (0, x, -p, -y)$.
The signs are necessary as, in our current linear setting, the surjective map of the short exact sequence is $f-g$.
Our isomorphism $P \oplus M \to V \oplus W$ thus has block form
$$\begin{pmatrix*}[r]
	I&0&0&0\\
	0&I&0&0\\
	I&0&-I&0\\
	0&0&0&-I
\end{pmatrix*},$$
which has determinant $(-1)^{w}$.
By definition, we choose an orientation of $P$ so that this matrix takes the concatenation of the orientation of $P$ with an orientation of $M$ to $(-1)^{wm}$ times the concatenation of the orientations of $V$ and $W$.

\begin{proposition}\label{P: orient intersection}
	Let $\beta_V$, $\beta_W$, and $\beta_M$ be orientations of $V$, $W$, and $M$.
	Then, the fiber product orientation of $P = V \times_M W$ is the unique orientation $\beta_P$ such that\footnote{Note that as $P$ and $\nu W$ are subspaces of $V$, we may consider $\beta_P$ and $\beta_{\nu W}$ as forms over $V$, and similarly for the other expressions that follow.} if we choose orientations $\beta_{\nu W}$ and $\beta_{\nu V}$ for $\nu W$ and $\nu V$ such that $\beta_P \wedge \beta_{\nu W} = \beta_V$ and $\beta_P \wedge \beta_{\nu V} = \beta_W$ then $\beta_{\nu W} \wedge \beta_P \wedge \beta_{\nu V} = (-1)^{w(m+1)}\beta_M$ or, alternatively, $$ \beta_P \wedge \beta_{\nu V} \wedge \beta_{\nu W} = \beta_M.$$
\end{proposition}

\begin{proof}
	Note that if we replace $\beta_P$ with its opposite orientation, then this must also reverse the orientations $\beta_{\nu W}$ and $\beta_{\nu V}$ and hence altogether we get the opposite orientation for $\beta_P \wedge \beta_{\nu V} \wedge \beta_{\nu W}$.
	Thus there is a unique such $\beta_P$ as described.

	It will be more convenient to prove the lemma in the first form, but the second form follows by observing that $$(-1)^{w(m+1)} = (-1)^{wm+w} = (-1)^{wm-w^2} = (-1)^{w(m-w)},$$
	and then
	$$\beta_{\nu W} \wedge \beta_P \wedge \beta_{\nu V} = (-1)^{w(m-w)} \beta_P \wedge \beta_{\nu V} \wedge \beta_{\nu W}$$
	as $P \oplus \nu V = W$ and $\dim(\nu W) = m-w$.

	To prove the first statement, let $(p_1,\cdots,p_a)$ be an ordered basis for $P$ consistent with the orientation described in the lemma; so we can write $\beta_P = p_1 \wedge\cdots\wedge p_a$.
	When we consider each $p_i$ as a vector in $V$, $W$, or $M$, we write $p_i^V$, $p_i^W$, or $p_i^M$.
	We employ a similar convention with the other bases we will consider.
	Let $(x_1,\cdots,x_b)$ and $(y_1,\cdots,y_c)$ be corresponding ordered bases for $\nu W$ and $\nu V$ as described in the lemma, and we can write $\beta_{\nu W}$ and $\beta_{\nu V}$ analogously as for $\beta_P$.
	Then, we have $\beta_M$ represented by $(-1)^{w(m+1)} \beta_{\nu W} \wedge \beta_P \wedge \beta_{\nu V}.$ So our orientation of $P \oplus M$ obtained by concatenation is
	$$(-1)^{w(m+1)} p_1 \wedge\cdots\wedge p_a \wedge x^M_1 \wedge\cdots\wedge x^M_b \wedge p^M_1 \wedge\cdots\wedge p^M_a \wedge y^M_1 \wedge\cdots\wedge y^M_c.$$
	When we apply our matrix above, we obtain the form in $V \oplus W$ given by
	$$(-1)^{w(m+1)} (p^V_1+p^W_1) \wedge\cdots\wedge (p^V_a+p^W_a) \wedge x^V_1 \wedge\cdots\wedge x^V_b \wedge (- p^W_1) \wedge\cdots\wedge (-p^W_a) \wedge (-y^W_1) \wedge\cdots\wedge(- y^W_c).$$
	This expression simplifies to
	$$(-1)^{wm} p^V_1 \wedge\cdots\wedge p^V_a \wedge x^V_1 \wedge\cdots\wedge x^V_b \wedge p^W_1 \wedge\cdots\wedge p^W_a \wedge y^W_1 \wedge\cdots\wedge y^W_c.$$
	But this is now precisely $(-1)^{wm}$ times the concatenation orientation of $V \oplus W$ as desired for definition of the fiber product orientation.
\end{proof}

\begin{example}
	As an example, let $M = \R^3$ oriented by the standard ordered basis $(e_x,e_y,e_z)$.
	Let $V$ be the $z = 0$ plane oriented by the ordered basis vectors $(e_x,e_y)$, and let $W$ be the $x = 0$ plane oriented by the ordered basis vectors $(e_y,e_z)$.
	The intersection $P$ is the $y$ axis.
	We claim that $P$ should be oriented by $-e_y$.
	Indeed, in this case we have $\beta_V = e_x \wedge e_y = -e_y \wedge e_x = (-e_y) \wedge e_x$, so $\beta_{\nu W} = e_x$, and $\beta_W = e_y \wedge e_z = (-e_y) \wedge (-e_z)$, so $\beta_{\nu V} = -e_z$.
	And then $$\beta_P \wedge \beta_{\nu V} \wedge \beta_{\nu W} = -e_y \wedge (-e_z) \wedge e_x = e_x \wedge e_y \wedge e_z = \beta_M,$$
	as required.
\end{example}

\begin{corollary}\label{C: orient complementary intersection}
	Suppose $V$ and $W$ have complementary dimensions so that they intersect in a point.
	Then the point is positively oriented if and only if $\beta_{W} \wedge \beta_{V} = \beta_M$.
\end{corollary}

\begin{proof}
	In this case, $\nu W = V$, $\nu V = W$, and $\beta_P = \pm 1 \in \R$.
	If $\beta_P = 1$, then the formula from \cref{P: orient intersection} becomes exactly the formula of the corollary.
\end{proof}

\begin{example}
	Let $M = \R^2$ with the standard orientation that we can write $e_x \wedge e_y$.
	Let $V$ be the $x$-axis with orientation $e_x$ and $W$ be the $y$-axis oriented by $e_y$.
	Then it is false that $e_x \wedge e_y = \beta_{V} \wedge \beta_{W} = (-1)^{w(m+1)} \beta_M = -e_x \wedge e_y$.
	So the orientation of the intersection point is the negative one.
	This runs against the standard convention for transverse intersections of manifolds of complementary dimension, but we nonetheless favor this overall convention for orienting fiber products due to the properties and uniqueness result of \cite{RamBas09}.
\end{example}

\subsection{Co-orientations}\label{S: co-orientations}

To define co-orientations, we recall our definition of an orientation of a bundle from \cref{D: det bundle} as an equivalence class, up to positive scalar multiplication, of an everywhere non-zero section of the top exterior power of the bundle.
This motivates the following.

\begin{definition}\label{D: co-orientations}
	A \textbf{co-orientation} $\omega_g$ of a \textit{continuous}\footnote{We will most often be interested in the case of $g$ smooth, but continuous co-orientable maps do come up in \cref{S: basic properties} where we consider covariant functoriality of geometric cohomology with respect to continuous maps.} map $g \colon W \to M$ of manifolds with corners is an equivalence class, up to positive scalar multiplication, of a nowhere zero section of the line bundle $\Hom(\Or(TW), \Or(g^*TM)) \cong \Hom(\Or(TW), g^*\Or(TM))$.
	Equivalently, a co-orientation is a choice of isomorphism between the associated orientation cover $\Or(TW)$ and the pullback of the associated orientation cover $\Or(TM)$.
\end{definition}

Thus, if $W$ is connected and $g \colon W \to M$ is co-orientable, there are exactly two co-orientations, which are \textbf{opposite} to one another; we write the opposite of $\omega_g$ as $-\omega_g$.
In particular, for connected $W$ a choice of co-orientation at a single point determines a co-orientation globally when $g$ is co-orientable (analogously to orientations).
Also, just as most manifolds do not possess a preferred orientation, most maps $g \colon W \to M$ do not carry a natural choice of co-orientation.

\begin{definition}\label{D: tautological co-orientation}
	An exception to the lack of a natural choice of co-orientation occurs when $g$ is a diffeomorphism, or more generally a codimension-0 immersion, in which case the top exterior power of $Dg$ provides a \textbf{tautological co-orientation}.
\end{definition}

The local triviality of the determinant line bundle of a manifold means being able to choose a consistent basis vector over sufficiently small neighborhoods.
We call such a choice of basis vector around a point in $W$ a \textbf{local orientation}, and, as for global orientations, often denote a local orientation by $\beta_W$.
Again, abusing notation, we also often allow $\beta_W$ to refer to its equivalence class up to multiplication by a positive scalar.
We identify $\beta_W$ with a local choice of (equivalence class of) non-zero $\dim(W)$-form\footnote{As usual, if $\dim(W) = 0$ we identify $\bigwedge^0 TW$ with the ground field $\R$, and, when forming exterior products, multiplication by a $0$-form is treated as scalar multiplication.} in $\bigwedge^{\dim(W)}TW$ in a neighborhood of a point $x$ in $W$ or, equivalently, a local smoothly varying ordered basis for the fibers of $TW$.
We typically do not specify the point $x$, though when necessary we write $\beta_{W,x}$.
We then use ordered-pair notation for co-orientation homomorphisms, with $\omega_g = (\beta_W, \beta_M)$ being the \textbf{local co-orientation} that sends the local orientation $\beta_W$ at $x \in W$ to a local orientation $\beta_M$ for $g^*(TM)$ up to a positive scalar.
We will often further abuse notation by neglecting the pullback and treat $\beta_M$ as a local orientation at $g(x)$ in $M$.
We also write the opposite co-orientation $(\beta_W,-\beta_M) = (-\beta_W,\beta_M)$ as $-(\beta_W,\beta_M)$.
As a co-orientation at a point completely determines the co-orientation of a co-orientable map for connected $W$, it is useful to cheat further and write $\omega_g = (\beta_W,\beta_M)$ for appropriate $\beta_W$ and $\beta_M$ with the chosen points $x$ implicit.

A manifold is orientable if and only if the orientation cover is trivial.
So if $M$ is orientable, $\Or(g^*(TM))$ is trivial, and a co-orientation of $g \colon W \to M$ implies that $W$ is orientable.
Moreover, an orientation on $M$ along with a co-orientation of $g$ gives rise to an \textbf{induced orientation} of $W$.
Explicitly, if $\beta_M$ denotes the global orientation of $M$, then we orient $W$ at each point by the $\beta_W$ such that $\omega_g = (\beta_W,\beta_M)$.
Conversely, if $M$ and $W$ are both oriented, say by $\beta_M$ and $\beta_W$ respectively, we have the \textbf{induced co-orientation} given by $\omega_g = (\beta_W,\beta_M)$ at each point of $W$.
On the other hand, it is not true that if we have an orientation of $W$ and a co-orientation of $g \colon W \to M$ then we obtain an orientation of $M$.
For example, if $W$ is orientable, any constant map to $M$ is co-orientable, regardless of whether or not $M$ is orientable.

More generally, recall that the fundamental group of a manifold acts on classes of local orientations as the deck transformations of the orientation cover.
A map is co-orientable if a loop in $W$ acts non-trivially on a local orientation of $W$ if and only if
its image in $M$ acts non-trivially.
Explicitly, if $g \colon W \to M$ is co-orientable the fundamental group can change the local orientation pair
$(\beta_W, \beta_M)$ to $(-\beta_W, -\beta_M)$, but these pairs define equivalent co-orientations.
Similarly, to compare local constructions at different points, it is useful to use paths.
Suppose $\gamma \colon I \to W$ is a path with $\gamma(0) = x$ and $\gamma(1) = y$.
We can choose a lift $\td \gamma$ of $\gamma$ to the complement of the $0$-section of $\Or(TW)$ such that $\td \gamma(0)$ is in the equivalence class of $\beta_{W,x}$.
We then define $\gamma_*\beta_{W}$ to be the equivalence class of $\td \gamma(1)$.
Likewise, we define $(g\gamma)_*\beta_M$ by a lift of $g\gamma$ to the complement of the zero section of $\Or(TM)$.
Then $\gamma_*\beta_{W}$ and $(g\gamma)_*\beta_M$ depend on $\gamma$, but if $g \colon W \to M$ is co-orientable the pair $(\gamma_*\beta_{W}, (g\gamma)_*\beta_M)$ is independent of $\gamma$ as this data also determines a non-vanishing lift of $\gamma$ in $\Hom(\Or(TW),g^*\Or(TM))$, which is trivial if $g$ is co-orientable.
In particular, if $g$ is co-oriented and $(\beta_{W}, \beta_M)$ represents the choice of co-orientation locally at $x$, then $\gamma_*(\beta_W,\beta_M) \defeq (\gamma_*\beta_{W}, (g\gamma)_*\beta_M)$ will represent the same co-orientation locally at $y$.

\begin{example}
	Let $g$ be any map $g \colon S^1 \to S^2$.
	As $S^1$ and $S^2$ are orientable, $g$ is co-orientable.
	If we choose a local orientation vector $e_{\theta}$ at any point $x \in S^1$ and latitude/longitude coordinates $\phi,\psi$ at $g(x)$ so that\footnote{Following \cite{BoTu82} we will generally omit the exterior product symbol $\wedge$ when working with explicit coordinate choices.} $e_\phi\, e_\psi$ is a local orientation in a neighborhood of $g(x)$, then the two possible co-orientations for for $g$ can be written $(e_\theta, e_\phi\, e_\psi)$ and $-(e_\theta, e_\phi\, e_\psi) = (-e_\theta, e_\phi \,e_\psi) = (e_\theta,- e_\phi\, e_\psi)$.
	While the notation explicitly references a local orientation at a point, this is sufficient to determine the co-orientation globally.
	In what follows we will often demonstrate properties of co-orientations by showing that they hold locally at an arbitrary point but do not depend on the choice of point.

	As another example, consider the standard embedding $g \colon \R P^2 \into \R P^4$.
	Choosing local orientations $e_1\, e_2$ at some $x \in \R P^2$ and $f_1\,f_2\,f_3\,f_4$ at $g(x)$, the two co-orientations are $(e_1\,e_2, f_1\,f_2\,f_3\,f_4)$ and its opposite.
	If $\gamma$ is a loop that reverses the orientation of $\R P^2$ then it also reverses the orientation of $\R P^4$, so $\gamma_*(e_1\,e_2, f_1\,f_2\,f_3\,f_4) = (-e_1\,e_2,- f_1\,f_2\,f_3\,f_4) = (e_1\,e_2, f_1\,f_2\,f_3\,f_4)$, reflecting that $g$ is co-orientable.

	By contrast, no embedding of the M\"obius strip in $\R^3$ is co-orientable.
\end{example}

\begin{remark}\label{R: cooriented composition}
	Co-oriented maps compose in an immediate way, forming a category.
	Namely, given $V\colon\xr{f} W\colon\xr{g} M$ and co-orientations $\Or(TV) \to \Or(f^*TW)$ and $\Or(TW) \to \Or(g^*TM)$, we simply compose the former with the pullback of the latter via $f^*$, recalling that $f^*(\Or(E)) = \Or(f^*E)$ in a natural way.
	We will refer to this simply as composing co-orientations and write the composition in symbols as $\omega_f*\omega_g$.
	Warning: note that we write the terms in the order $\omega_f*\omega_g$ for the map $g \circ f$.
	This is more convenient when writing out co-orientations using the local orientations as we obtain expressions such as $(\beta_V, \beta_W)*(\beta_W,\beta_M) = (\beta_V,\beta_M)$.
\end{remark}

\begin{notation}\label{N: implicit notation}
	It will be useful in notation to sometimes leave the maps, codomains, and co-orientations all implicit once they have already been established and just write $V$ to represent the co-oriented map $f \colon V \to M$.
	In this case we write $-V$ to refer to the same map with the opposite co-orientation.
\end{notation}

\subsection{Normal co-orientations of immersions and co-orientations of boundaries}\label{S: normal orientatin}

As mentioned, a key example is when $g$ is an immersion, which is co-orientable if and only if its normal bundle is orientable\footnote{Recall that technically all bundles are over $W$, though our convention is to elide that in the notation.
Hence we can consider $W$ to have a normal bundle even if $g$ is merely an immersion and not actually an embedding.
The normal bundle can be identified with $g^*(TM)/TW$ after identifying $TW$ with a sub-bundle of $g^*(TM)$ using the differential.
In any case, locally in the neighborhood of any point of $W$ one has the usual identification of the normal bundle with a tubular neighborhood of the image, which suffices for our purposes here.}.
Specifically, if $g \colon W \to M$ is an immersion, letting $\nu W$ denote the normal bundle, we have $TW \oplus \nu W \cong g^*TM$.
So, taking $w = \dim(W)$ and $m = \dim(M)$, a co-orientation is a nowhere-zero map from $\bigwedge^w TW$ to $\bigwedge^m g^*TM = \bigwedge^m (TW \oplus \nu W) \cong \bigwedge^w TW \otimes \bigwedge^{m-w}\nu W$.
Such a nowhere-zero map exists if and only $\bigwedge^{m-w}\nu W$ is a trivial line bundle.

Although such a co-orientation is a global object over $W$, given an orientation of $\nu W$ we can specify a standard choice of ``normal co-orientation'' through the following local construction:

\begin{definition}\label{normal co-or}
	Let $g \colon W \to M$ be an immersion with normal bundle locally oriented by $\beta_\nu$.
	Define the \textbf{normal co-orientation} associated to $\beta_\nu$ locally as\footnote{Whenever we form wedge products, if one of the terms is an element of $\bigwedge^0 TV$ for some $V$ we treat that term as a $0$-form and interpret $\wedge$ as the fiberwise scalar product.}
	$\omega_{\nu} = (\beta_W, \beta_W \wedge \beta_\nu)$, where $\beta_W$ is any choice of a local orientation of $W$.
\end{definition}

This construction is independent of the choice of $\beta_W$, as reversing the orientation of $\beta_W$ gives
$$(-\beta_W, -\beta_W \wedge \beta_\nu) = (\beta_W, \beta_W \wedge \beta_\nu).$$
If the normal bundle to $W$ is oriented globally on $W$ then the construction is also independent of the point at which it is carried out since if $\gamma$ is a path from $x$ to $y$ with $\beta_W$ and $\beta_\nu$ constructed at $x$ then
\begin{equation*}
	(\gamma_*\beta_{W,x}\, , (g\gamma)_* (\beta_{W,x} \wedge \beta_{\nu,x})) =
	(\gamma_*\beta_{W,x}\, , (\gamma_* \beta_{W,x}) \wedge (g\gamma)_*\beta_{\nu,x}) =
	(\gamma_*\beta_{W,x}\, , (\gamma_* \beta_{W,x}) \wedge \beta_{\nu,y}),
\end{equation*}
using that $\nu$ is assumed oriented and that, via the immersion, we can treat a path in $W$ as a path in $M$.
So if the normal bundle to $W$ is oriented, these local choices determine a global co-orientation of $W \to M$.
If the normal bundle is orientable, one can conversely orient the normal bundle if one is given a co-orientation.
Signs in the theory are highly dependent on choices.
One such choice in this definition is whether to append the local normal orientation before or after the local tangent orientation.

\subsubsection{Quillen co-orientations}\label{S: Quillen}

There is a useful alternative, though equivalent, definition of co-orientations due to Quillen \cite{Quil71} that only involves orientations of normal bundles.\footnote{Quillen's context was slightly different.
He assumed the normal bundles to have complex structures and so called these \textit{complex orientations}.}

\begin{lemma}\label{L: Quillen}
	A map $g \colon W \to M$ is co-orientable if and only if for some $N \in \Z_{\geq 0}$ it factors as the composition $W \into M \times \R^N \to M$ of an embedding and a projection such that the image of $W$ in $M \times \R^N$ has an orientable normal bundle.
\end{lemma}

\begin{proof}
	We first note that such a smooth factorization always exists.
	For example, let $j \colon W \to \R^N$ be an embedding as guaranteed by Whitney's theorem.
	Then define $e \colon W \into M \times \R^N$ by $e(x) = (g(x),j(x))$ and $\pi \colon M \times \R^N \to M$ the projection.
	Clearly $\pi e = g$.

	Next we note that $T(M \times \R^N) \cong \pi^*(TM) \oplus \underline{\R}^N$ where $\underline{\R}^N$ is the trivial $\R^N$ bundle over $M \times \R^N$.
	As $\Or( \underline\R^N)$ is a trivial line bundle,
	\begin{equation*}
		\Or(T(M \times \R^N)) \cong \Or(\pi^*(TM)) \otimes \Or(\underline\R^N) \cong \Or(\pi^*TM).
	\end{equation*}
	Thus $\pi$ is always co-orientable.
	Furthermore, we see that
	\begin{equation*}
		\Or(g^*TM) \cong \Or(e^*\pi^*TM) \cong e^*\Or(\pi^*TM) \cong e^*\Or(T(M \times \R^N)) \cong \Or(e^*T(M \times \R^N)).
	\end{equation*}
	So $e$ is co-orientable if and only if $g$ is co-orientable.
	But $e$ is an immersion, and so it is co-orientable if and only if the normal bundle of the image is trivial by the discussion preceding \cref{normal co-or}.
\end{proof}

\begin{definition}\label{D: Quillen normal or}
	If we orient $\R^N$ by its standard orientation $\beta_E \defeq dx_1 \wedge\cdots\wedge dx_N$, then the projection $\pi \colon M \times \R^N \to M$ has a canonical co-orientation $(\beta_M \wedge \beta_E, \beta_M)$ that is well defined as $\R^N$ is contractible.
	If $g \colon W \to M$ is co-oriented and $e \colon W \into M \times \R^N$ is an immersion with normal bundle $\nu$, we then define the \textbf{compatible normal orientation} or \textbf{Quillen normal orientation} of $\nu$ so that the composition of co-orientations $(\beta_W,\beta_W \wedge \beta_\nu)$ with $(\beta_M \wedge \beta_E,\beta_M)$ is the given co-orientation of $g$.
	In other words, if $(\beta_W,\beta_M)$ is the given co-orientation of $g$, then the compatible normal orientation of $\nu$ is such that $\beta_W \wedge \beta_\nu = \beta_M \wedge \beta_E$ up to positive scalar multiple.

	We sometimes speak of the entire structure $W \into M \times \R^N \to M$ with an orientation of $\nu$ as a \textbf{Quillen co-orientation} for $W \to M$ or as a ``compatible Quillen co-orientation'' if we have already specified a co-orientation for $W \to M$ and we wish to choose the Quillen co-orientation that agrees with it.
\end{definition}

\begin{remark}\label{R: immersion}
	In particular, if $g \colon W \to M$ is an immersion, then, by taking $N = 0$, a co-orientation of $g$ is equivalent to a Quillen orientation of the normal bundle $\nu$ of $W$ in $M$.
	In particular, the co-orientation is given locally by $(\beta_W, \beta_M)$ if and only if $\nu$ is oriented so that $\beta_W \wedge \beta_\nu = \beta_M$ up to positive scalar multiple.
	If $g$ is a codimension-$0$ immersion, then $\nu$ will be $0$-dimensional, and if the co-orientation is the tautological one then $\beta_\nu$ will be the positive orientation at each point.
\end{remark}

\begin{remark}
	Lipyanskiy's definition of co-orientation in \cite{Lipy14} factors a proper map through a map which is surjective onto $TM$, rather than injective from $TW$ as in the Quillen approach.
	An argument similar to the one just given establishes an equivalence between Lipyanskiy's definition and ours in that setting.
%	\red{[GBF: I'd suggest to we give it, or at least sketch it.]}
\end{remark}

\subsubsection{Co-orientations of boundaries}

Given a co-oriented map $g \colon W \to M$ where $W$ is a manifold with corners, we can use the normal co-orientation of $\bd W$ in $W$ together with the composition of co-orientations noted in \cref{R: cooriented composition} to define ``boundary co-orientations'':

\begin{definition}\label{D: boundary co-orientation}
	The \textbf{standard co-orientation of a boundary immersion} $i_{\bd W} \colon \bd W \into W$ is the normal co-orientation associated to the \textit{inward}-pointing\footnote{The outward-pointing normal would also work to provide a co-orientation convention for which the Leibniz formula of \cref{S: co-orient pullbacks} holds.
	However, using an outward normal is not consistent with the intersection map $\mc I$ of \cref{S: intersection map} being a chain map with our other conventions, while using the inward normal does make $\mc I$ a chain map.} orientation of $\nu_{\bd W \subset W}$.

	If $g \colon W \to M$ is co-oriented, the \textbf{induced co-orientation} or \textbf{boundary co-orientation} of the composition $gi_{\bd W}$ is the composition of the standard co-orientation of $i_{\bd W}$ with the given co-orientation of $g \colon W \to M$.
	We write $\bd g \colon \bd W \to M$ to denote $gi_{\bd W}$ with its induced co-orientation.
\end{definition}

We will use induced co-orientations on boundaries to define the differential in the geometric cochain complex.

The following examples will be important in \cref{S: creasing} when we discuss creasing.

\begin{example}\label{E: splitting example 1}
	Suppose $g \colon W \into M$ is the inclusion of a codimension-$0$ submanifold of $M$.
	In this case $TW$ is the pullback of $TM$, and we have the tautological co-orientation of \cref{D: tautological co-orientation} that, slightly abusing notation, we can write as $(\beta_M,\beta_M)$.
	A particularly important pair of examples is given by the inclusions $g^- \colon (-\infty, 0] \into \R$ and $g^+ \colon [0,\infty) \into \R$, each tautologically co-oriented by $(e_1,e_1)$, where $e_1$ is the standard unit vector in $\R$.

	Next consider the submanifold consisting of the point $0 \in \R$.
	It has trivial determinant line bundle, and we can choose the basis element to be the $0$-form $1 \in \bigwedge^0 T0 \cong \R$.
	By \cref{normal co-or, D: boundary co-orientation}, the standard co-orientation of the boundary inclusion $\{0\} \into (-\infty, 0]$ is $(1, 1 \wedge -e_1) = (1, -e_1)$.
	The boundary co-orientation of the inclusion $\{0\} \to \R$ induced by the tautological co-orientation of the inclusion $g^- \colon (-\infty, 0] \to \R$ is then the composition of $(1, -e_1)$ with $(e_1, e_1)$, which is again $(1,-e_1) = -(1,e_1)$ (all bases interpreted in the appropriate spaces).
	As the inward normal to $[0,\infty)$ at $0$ is $e_1$, the inclusion $g^+ \colon [0,\infty) \to \R$ induces the opposite co-orientation $(1, e_1)$ on the inclusion $\{0\} \to \R$.
\end{example}

\begin{example}\label{E: manifold decomposition}
	An important generalization of the preceding example occurs when we have smooth maps $g \colon W \to M$ and $\varphi \colon M \to \R$ with $0$ a regular value of $\varphi g$.
	Consider the spaces $W^0 = (\varphi g)^{-1}(0)$, $W^- = (\varphi g)^{-1}((-\infty,0])$, and $W^+ = (\varphi g)^{-1}([0,\infty))$, which are manifolds with corners.
	In fact, they are the fiber products of the maps $\varphi g \colon W \to \R$ and, respectively, the inclusions of $\{0\}$, $(-\infty, 0]$, and $[0,\infty)$ into $\R$.
	The required transversality is guaranteed by $0$ being a regular value of $\varphi g$.
	In this case, however, rather than consider the maps of these fiber products to $\R$, we consider their maps to $M$ via restrictions of $g$.

	In this setting, the normal bundle of $W^0$ in $W$ has a natural orientation given by pulling back via $\varphi g$ the standard orientation $e_1$ of the normal bundle of $0$ in $\R$, and this determines the \textbf{co-orientation of $W^0$ in $W$ induced by $\varphi$}, which is given by the normal co-orientation $(\beta_{W^0}, \beta_{W^0} \wedge (\varphi g)^*e_1)$.
	We also have the boundary co-orientations of the inclusions of $W^0$ into $W^{\pm}$.
	Analogously to \cref{E: splitting example 1}, the co-orientation on $W^0 \into W$ induced by $\varphi$ is the opposite of co-orientation of $W^0 \into W$ as the (partial) boundary of $W^-$ and agrees with the co-orientation of $W^0 \into W$ as the (partial) boundary of $W^+$.

	Now suppose $g \colon W \to M$ is co-oriented with local representatives $(\beta_W, \beta_M)$.
	By composing this co-orientation with the tautological co-orientations of $W^\pm$ in $W$ as in the preceding example, we can canonically co-orient $g|_{W^\pm}$ by $(\beta_W, \beta_M)$.
	Similarly, by composing the co-orientation of $g$ with the $\phi$-induced co-orientation of $W^0 \into W$ we obtain the \textbf{co-orientation of $g|_{W^0} \colon W^0 \to M$ induced by $\varphi$}.
	As above, the co-orientation of $g|_{W^0} \colon W^0 \to M$ induced by $\varphi$ disagrees with the boundary co-orientation of $g|_{W^0} \colon W^0 \to M$ obtained by treating $W^0$ as part of the boundary of $W^-$, while it agrees with the co-orientation obtained by treating $W^0$ as part of the boundary of $W^+$.

	In the special case of the identity map $\id \colon M \to M$ with the tautological co-orientation, we obtain submanifolds $M^0$ and $M^{\pm}$ of $M$ with co-orientations of their embeddings induced as above.
	It is worth observing that then if we also have $g \colon W \to M$ then $W^0 = M^0 \times_M W$ and $W^{\pm} = M^{\pm} \times_M W$.
	This statement holds purely topologically.
	To ensure manifolds with corners, we observe that $\varphi g$ is transverse to $0$ if and only if $g$ is transverse to $M^0$.
	Our co-orientation computations here for $W^0$ and $W^\pm$ will later be seen to be consistent with a co-orientation we define for fiber products of co-oriented maps.
	In \cref{C: co-orient W0}, we will use that technology to see that $\bd (W^0)$ and $(\bd W)^0$ agree as spaces but their maps to $M$ have opposite co-orientations given our conventions, i.e.\ ``$\bd W^0 = -(\bd W)^0$'' eliding the maps.
\end{example}

\begin{comment}
	\red{TO DO SOMEWHERE: It will be convenient to show that $g \colon W^0 \to M$ with this co-orientation is the pullback of $M^0 \into M$ (co-oriented as defined here) and $g \colon W \to M$ and similarly for $g \colon W^\pm \to M$.
		Also need to rewrite things in other places as $M^0$, $M^\pm$, etc instead of always writing $\varphi^{-1}((\infty,0])$ etc.
		Also need to show that $\bd W^0 = -(\bd W)^0$, which should follow from the Leibniz rule and the first thing the previous sentence.}
\end{comment}

\subsubsection{Co-orientations of boundaries of boundaries}

In order to form a chain complex of geometric cochains, we will need a result about co-orientations of $\bd^2 W$.
Recall from \cref{S: boundaries} that Proposition 2.9 of \cite{Joy12} identifies $\bd^2 W$ with the set of points $(x,\bb_1,\bb_2)$ with $x \in W$ and the $\bb_i$ encoding distinct local boundary components.
The map $i_{\bd^2 W} \colon \bd^2 W \to W$ takes $(x,\bb_1,\bb_2)$ to $x$.
The manifold with corners $\bd^2 W$ is equipped with a canonical diffeomorphism $\rho$ defined by $(x,\bb_1,\bb_2) \to (x,\bb_2,\bb_1)$.

\begin{lemma}\label{L: boundary2}
	Suppose $i_{\bd^2 W} \colon \bd^2 W \to W$ is co-oriented via the composition of boundary co-orientations $\bd^2 W \to \bd W \to W$, and suppose $\rho \colon \bd W \to \bd W$ is given the co-orientation induced by $\bigwedge D\rho$ (see \cref{S: co-orientations}).
	Then $i_{\bd^2 W}$ and $i_{\bd^2 W}\rho$ have opposite co-orientations.
\end{lemma}

\begin{proof}
	It suffices to consider points $(x,\bb_1,\bb_2) \in \bd^2 W$ with $x \in S^2(W)$, as such points fill out the interior of $\bd^2 W$.
	In $W$ such $x$ have neighborhoods of the form $[0,\infty)^2 \times \R^{w-2}$ with $x$ at the origin.
	We identify $[0,\infty)^2$ with the first quadrant of $\R^2$, letting $X$ and $Y$ denote the non-negative $x$ and $y$ axes.
	We let $\bb_X$ and $\bb_Y$ be the corresponding local boundary components.
	Then the preimage in $\bd^2 W$ of a small neighborhood $U$ of $x$ in $S^2(W)$ consists of two copies of $U$ that we can write $(U,\bb_X,\bb_Y)$ and $(U,\bb_Y,\bb_X)$.
	The notation indicates that we think of the first copy of $U$ as embedding into $X \times \R^{w-2} \subset \bd W$ and the second as embedding into $Y \times \R^{w-2} \subset \bd W$.
	The map $i_{\bd W} \colon \bd W \to W$ then identifies the two copies.
	The map $\rho$ simply interchanges them.

	Let $\beta_X$ and $\beta_Y$ denote positively-directed tangent vectors in $X$ and $Y$, and let $\beta_U$ be an arbitrary local orientation of $U$.
	Abusing notation, we also write $\beta_U$ for the corresponding local orientations of $(U,\bb_X,\bb_Y)$ and $(U,\bb_Y,\bb_X)$.
	The induced co-orientation on $\rho$ can be written $(\beta_U,\beta_U)$.
	Up to identifying neighborhoods in $W$ with their local models, the boundary co-orientation of $i_{\bd^2 W}$ on $(U,\bb_X,\bb_Y)$ comes from first mapping it into $X \times \R^{w-2}$ and then into $X \times Y \times \R^{w-2}$.
	So from the definition of boundary co-orientations this co-orientation is $(\beta_U, \beta_U \wedge \beta_X \wedge \beta_Y)$.
	Analogously, the boundary co-orientation of $i_{\bd^2 W}$ on $(U,\bb_Y,\bb_X)$ is $(\beta_U, \beta_U \wedge \beta_Y \wedge \beta_X)$.
	By composition, the co-orientations of $i_{\bd^2 W}\rho$ on $(U,\bb_X,\bb_Y)$ and $(U,\bb_Y,\bb_X)$ are respectively $(\beta_U, \beta_U \wedge \beta_Y \wedge \beta_X)$ and $(\beta_U, \beta_U \wedge \beta_X \wedge \beta_Y)$ as first we interchange then embed.
	But $\beta_U \wedge \beta_X \wedge \beta_Y = -\beta_U \wedge \beta_Y \wedge \beta_X$, which establishes the lemma.
\end{proof}

\begin{remark}\label{R: bd2 oriented}
	A similar argument using \cref{Con: oriented boundary} shows that if $W$ is oriented than $\bd^2 W$ possess an orientation reversing diffeomorphism.
	In this case we observe that of our two copies of $U$, one is oriented by \cref{Con: oriented boundary} so that $\beta_X \wedge \beta_Y \wedge \beta_U$ is the local orientation of $W$ and the other is oriented so that $\beta_Y \wedge \beta_X \wedge \beta_U$ is the orientation of $W$.
	Thus the two copies of $U$ have opposite orientations, and again the diffeomorphism simply interchanges them.
\end{remark}

\subsection{Co-orientation of homotopies}\label{S: co-oriented homotopy}

In this section we develop co-orientations related to homotopies.
As the product of two spaces is the same as their fiber product over a point, we have by equation \eqref{E: product boundary}:
\begin{equation*}
	\bd(W \times I) =
	(\bd W \times I) \sqcup (W \times \bd I) =
	(W \times 1) \sqcup (W \times 0) \sqcup (\bd W \times I).
\end{equation*}
Now recall that in general if we have a map $f \colon V \to M$ then we write $\bd f$ for the composition
$$\bd V \xr{i_{\bd V}} V \xr{f} M.$$
Adopting this convention also for pieces of the boundary we make the following definition:

\begin{definition}\label{D: co-oriented homotopy}
	If $G \colon W \times I \to M$ is a co-oriented map, we say that $G$ is a \textbf{co-oriented homotopy} (or simply a \textbf{homotopy} when working with co-orientations is understood) from $g_0 \colon W \to M$ to $g_1 \colon W \to M$ if $\bd G = g_1 \amalg -g_0\amalg H$, where $g_1$, $-g_0$, and $H$ correspond respectively to the compositions of $G$ with the inclusions into $W \times I$ of $W \times 1$, $W \times 0$, and $\bd W \times I$, taking each with its boundary co-orientation as in \cref{D: boundary co-orientation}.
\end{definition}

Note that, by analogy with homotopies involving oriented manifolds, a homotopy from $g_0$ to $g_1$ involves the oppositely co-oriented $-g_0$ in the boundary formula.
In the oriented case, this arises because if we orient $W$ by, say, $\beta_W$ then to orient $W \times I$ we consider $\beta_W \wedge \beta_I$.
Then at one end of the cylinder $\beta_W$ agrees with the boundary orientation of $\bd (W \times I)$ while at the other end it disagrees.
The situation for co-orientations is analogous.

Although we will not need it, we note that by employing appropriate smoothing near the boundaries in order to accomplish transitivity, co-oriented homotopy can be shown to be an equivalence relation on maps $W \to M$.

In our most common use of homotopies, we begin with a co-oriented map $g \colon W \to M$ and want to construct a homotopic co-oriented map.
For this the following lemma is useful.

\begin{lemma}\label{L: co-orientable homotopies}
	Suppose $g \colon W \to M$ is co-orientable and that $G \colon W \times I \to M$ is a homotopy with $g = G(-,t_0)$ for some $t_0 \in I$.
	Then $G$ is co-orientable.
	Conversely, if $G \colon W \times I \to M$ is co-orientable, then so is $g = G(-,t_0) \colon W \to M$ for any $t_0$.
\end{lemma}

\begin{proof}
	Over $W \times t_0$, we have $\Or(T(W \times I)) \cong \Or(TW \oplus TI) \cong \Or(TW) \otimes \Or (TI) \cong \Or(TW)$, while the restriction of $G^*\Or(TM)$ over $W \times t_0$ is isomorphic to $g^*\Or(TM)$.
	If $G$ is co-orientable then there is a nowhere-vanishing map of line bundles $\Or(T(W \times I)) \to G^*\Or(TM))$, so restricting to $W \times t_0$ and using the above identifications we obtain a nowhere-vanishing map of line bundles $\Or(TW) \to g^*\Or(TM))$ over $W \times t_0$, hence $g$ is co-orientable.
	Conversely,
	if $g$ is co-orientable, there is a nowhere-vanishing map of line bundles $\Or(TW) \to g^*\Or(TM)$ over $W \times t_0$.
	By general bundle theory, any vector bundle $E$ over $W \times I$ is isomorphic to $E_{t_0} \times I$, where $E_{t_0}$ is the restriction of $E$ to $W \times \{t_0\}$.
	So our nowhere-vanishing map of line bundles over $W \times t_0$ extends to a nowhere vanishing map of line bundles $\Or(T(W \times I)) \to G^*\Or(TM))$ over $W \times I$.
	This implies the co-orientability of $G$.
\end{proof}

\begin{definition}\label{D: homotopy co-orientation}
	Suppose $g_0 \colon W \to M$ is co-oriented and $G \colon W \times I \to M$ is a smooth homotopy with $G(-,0) = g_0$.
	Then by the above lemma $G$ is co-orientable and clearly there is exactly one choice of co-orientation for $G$ for which the $W \times 0$ component of $\bd G$ is $-g_0$.
	We call this co-orientation the \textbf{co-orientation on $G$ induced by $g_0$}.
	The map $G$ then determines a co-oriented homotopy from $g_0$ to a co-oriented map $g_1 \colon W \to M$.
	We call this co-orientation on $g_1 = G(-,1)$ the \textbf{induced co-orientation on $g_1$.}
\end{definition}

\begin{remark}\label{R: stationary homotopy}
	In the above scenario, if $g_0$ is co-oriented locally at $x \in W$ by $(\beta_W,\beta_M)$, then at $(x,0) \in W \times I$, the corresponding local co-orientation of $G$ that yields $-g_0$ as a boundary component of $G$ is $(\beta_W \wedge -\beta_I, \beta_M)$, where $\beta_I$ corresponds to the standard orientation of $I$.
	This follows from $(\beta_W,\beta_W \wedge \beta_I)$ being the boundary co-orientation of $W \times 0 \into W \times I$ as $e_1$ is the inward pointing normal at $0 \in I$.
	As we can take $\beta_W \wedge -\beta_I$ to be a consistent orientation along the path given by $\gamma(t) = (x,t)$, we have $\gamma_*(\beta_W \wedge -\beta_I, \beta_M) = (\beta_W \wedge -\beta_I,\gamma_*\beta_M)$, and at this end of the homotopy the induced local co-orientation of $g_1$ at $x$ is $(\beta_W,\gamma_*\beta_M)$.
	If $G$ is stationary along $x \times I$, then the co-orientation for $g_1$ at $x$ is again $(\beta_W,\beta_M)$ so that the co-orientations of $g_0$ and $g_1$ agree at $x$.
	This observation will be useful below in showing that pullback co-orientations are well defined.
\end{remark}

\begin{comment}
	If $\beta_{W,x}$ is a local orientation of $W$ at a point $x \in W$ and $\gamma$ is a path in $W$ with $\gamma(0) = x$, then $\gamma$ determines a local orientation $\gamma_*\beta_{W,x}$ of $W$ at $\gamma(1)$ via any lift of $\gamma$ to the complement of the zero section of $\Or(TW)$.
	Similarly, given a map $g \colon W \to M$ and a local orientation $\beta_{M,g(x)}$ of $M$ at $g(x)$, the path $g\gamma$ determines a local orientation $(g\gamma)_*\beta_{M,g(x)}$ at $g\gamma(1)$.
	Of course $\gamma_*\beta_{W,x}$ and $(g\gamma)_*\beta_{M,g(x)}$ depend on $\gamma$, but the condition that $g$ be co-orientable is precisely the condition that the pair $(\gamma_*\beta_{W,x}, (g\gamma)_*\beta_{M,g(x)})$ be independent of $\gamma$.

	We use an analogous construction when we have a homotopy $G: W \times I \to M$ and a co-orientation of $g_0 = G(-,0) \colon W \to M$ and wish to define a co-orientation of $g_1 = G(-,1) \colon W \to M$.
	Explicitly, if the pair $(\beta_{W,x},\beta_{M,g(x)})$ is a local co-orientation for $g_0$ at $x \in W$, then we define the \emph{induced co-orientation} for $g_1$ at $x$ to be $(\beta_{W,x},G(x,-)_*\beta_M)$.
	Any $g_t = G(-,t)$ can be co-oriented analogously.

	\begin{definition}
		If $g \colon W \to M$ is co-oriented and $G \colon W \times I \to M$ is a homotopy with $G|_{W \times \{t_0\}} = g$, we define the \textbf{homotopy co-orientation on $G \colon W \times I \to M$ induced by $g$} so that if $(\beta_W,\beta_M)$ is a local co-orientation of $g$ at $x \in W$ then $(\beta_W \wedge \beta_{e_1},\beta_M)$ co-orients $G$ at $(x,t_0)$, where $\beta_{e_1}$ is the local orientation of $I$ corresponding to the standard positively-oriented tangent vector.
		As $G$ is co-orientable, this determines a co-orientation for the whole map $G$.
	\end{definition}

	\begin{lemma}
		Let $g_0 \colon W \to M$ be co-oriented, and let $G \colon W \times I \to M$ be a homotopy with $G(-,0) = g_0$.
		Let $G$ and $g_1 = G(-,1)$ have the induced co-orientations.
		Furthermore, let $G_\bd:(\bd W) \times I \to M$ be the composition $(\bd W) \times I \xr{i_{\bd W} \times \id_I} W \times I \xr{G}M$, co-oriented by taking the homotopy orientation induced from $g_0i_{\bd W} \colon \bd W \to M$ with its standard boundary co-orientation obtained from the co-orientation of $g_0$.
		Then $$\bd G = g_1-g_0-G_\bd.$$
	\end{lemma}
\end{comment}

\begin{lemma}\label{L: co-oriented homotopy}
	Suppose $G \colon W \times I \to M$ is a co-oriented homotopy from $g_0$ to $g_1$ so that $\bd G = g_1 \amalg -g_0\amalg H$ as in \cref{D: co-oriented homotopy}.
	Then $H$ is a homotopy from $-\bd g_0$ to $-\bd g_1$.
\end{lemma}

\begin{proof}
	By definition $H$ is co-oriented as a boundary component of the co-oriented map $G$, so it remains to check that the induced co-orientations of the ends of $H$ have the expected signs.
	This could be done directly, but rather we use \cref{L: boundary2}, noting that each copy of $\bd W$ (at the top and bottom of the cylinder) can be considered to be a piece of $\bd^2(W \times I)$.
	In particular, applying this lemma we see the maps $\bd W \to M$ take opposite co-orientations depending on whether we think of them as first mapping $\bd W$ into $W$ and then identifying $W$ as one end of the cylinder versus first including $\bd W$ into $(\bd W) \times I$ and then mapping this to $W \times I$.
	In both cases we follow with the map $G$.
	As we think of $g_1$ as defined on $W \subset W \times I$ and let $\bd g_1$ denote its boundary, we see that the corresponding map from the top of the cylinder $(\bd W) \times I$ must be $-\bd g_1$.
	Similarly, the map at the bottom of the cylinder is $\bd g_0$.
\end{proof}

\begin{comment}
	\begin{proof}
		We will always denote the local co-orientation of $g_0$ by $(\beta_W,\beta_M)$.
		Recall that the standard co-orientation for the boundary inclusion $i_{\bd V} \colon \bd V \to V$ of a manifold with corners $V$ is $(\beta_{\bd V},\beta_{\bd V} \wedge \beta_\nu)$, where $\nu$ is an outward pointing normal vector.
		If $f \colon V \to M$ is co-oriented, then the composite $\bd V \xr{i_{\bd V}}V \xr{f}M$ is co-oriented by composing the co-orientations of the component maps.

		Now consider $W \times 1 \subset W \times I$.
		At $(x,1)$, the induced homotopy orientation is $(\beta_W \wedge \beta_{e_1},\gamma_*\beta_M)$, where $\gamma$ is the path $I \to G(x,-)$.
		Composing with the standard boundary co-orientation $(\beta_W,\beta_W \wedge \beta_\nu)$ gives $(\beta_W,\gamma_*\beta_M)$, as we can take $\beta_\nu = \beta_{e_1}$.
		This is the the induced co-orientation for $g_1$.

		On the other hand, as the positively-oriented tangent vector to $I$ points inward at $0$, the standard boundary co-orientation of $W \times 0 \to W \times I \to M$ is the composition of $(\beta_W, \beta_W \wedge \beta_\nu) = -(\beta_W, \beta_W \wedge \beta_{e_1})$ with $(\beta_W \wedge \beta_{e_1},\beta_M)$.
		Thus as a piece of the boundary, $G \circ i_{W \times 0} = -g_0$.

		Finally, consider a point $(x,0) \in (\bd W) \times I$ and let $\nu_\bd$ be an outward pointing normal to $W$ at $i_{\bd W}(x)$.
		Then the standard co-orientation of $\bd W$ in $W$ is $(\beta_{\bd W},\beta_{\bd W} \wedge \beta_\nu)$ and the standard co-orientation of the composite $\bd g_0 = g_0i_{\bd W} \colon \bd W \to M$ is the composite of $(\beta_{\bd W},\beta_{\bd W} \wedge \beta_\nu)$ with $(\beta_W,\beta_M)$.
		If $\beta_W = \beta_{\bd W} \wedge \beta_{\nu_\bd}$, this composite co-orientation is $(\beta_{\bd W},\beta_M)$, and otherwise it is
		$-(\beta_{\bd W},\beta_M)$.
		The induced homotopy co-orientation of $G_\bd \colon \bd W \times I \xr{i_{\bd W} \times \id}W \times I \to M$ is then $\pm(\beta_{\bd W} \wedge \beta_{e_1},\beta_M)$, as $\beta_W = \beta_{\bd W} \wedge \beta_{\nu_\bd}$ or not.

		On the other hand, consider $(\bd W) \times I$ as part of the boundary of $W \times I$.
		The standard boundary co-orientation for $(\bd W) \times I$ in $W \times I$ is $(\beta_{(\bd W) \times I},\beta_{(\bd W) \times I}\wedge\beta_{\nu_\bd})$.
		This is independent of the choice of $\beta_{(\bd W) \times I}$, so we may take $\beta_{(\bd W) \times I} = \beta_W \wedge \beta_{e_1}$, where $\beta_{e_1}$ is positively-directed in $I$.
		Then
		$$(\beta_{(\bd W) \times I},\beta_{(\bd W) \times I}\wedge\beta_{\nu_\bd}) = (\beta_{\bd W} \wedge \beta_{e_1},\beta_{\bd W} \wedge \beta_{e_1}\wedge\beta_{\nu_\bd}) = -(\beta_{\bd W} \wedge \beta_{e_1},\beta_{\bd W} \wedge \beta_{\nu_\bd}\wedge\beta_{\nu_I}).$$
		Composing with the induced co-orientation $(\beta_W \wedge \beta_{e_1},\beta_M)$ of $G$ gives the composite $(\bd W) \times I \xr{i_{(\bd W) \times I}}W \times I \xr{G} M$ the co-orientation $-(\beta_{\bd W} \wedge \beta_{e_1},\beta_M)$ if $\beta_W = \beta_{\bd W} \wedge \beta_{\nu_\bd}$ and $(\beta_{\bd W} \wedge \beta_{e_1},\beta_M)$ otherwise.
		Thus, altogether, the $(\bd W) \times I$ boundary component of $W \times I$ with its boundary co-orientation is $-G_\bd$.
	\end{proof}
\end{comment}

\subsection{Co-orientations of pullbacks and fiber products}\label{S: co-orient pullbacks}

In this section we define a convention for co-orientations of pullbacks and fiber products.
More specifically, if $f \colon V \to M$ and $g \colon W \to M$ are transverse smooth maps from manifolds with corners to a manifold without boundary and $f$ is co-oriented, we define a co-orientation of the pullback $f^* \colon V \times_M W \to W$.
This does not require $g$ to be co-oriented, but if it is, we can compose to also get a co-orientation of the fiber product $V \times_M W \to M$.
Ultimately this will allow us to define certain products of geometric cochains.

Recall that our canonical realization of the topological pullback $P = V \times_M W$ is defined to be $P = \{(x,y) \in V \times W \mid f(x) = g(y)\}$.
By Joyce \cite{Joy12}, the projections $P \to V$ and $P \to W$ are smooth, and hence so is $f \times_M g \colon P \to M$ given by $(x,y) \to f(x) = g(y)$.
It is not obvious how to define the co-orientations of pullbacks and fiber products, and any such definition will depend on choices of convention.
Our goal in this section is to provide a definition such that co-orientations of fiber products of co-oriented maps possess the following desirable properties:

\begin{enumerate}
	\item If $f$ and $g$ are transverse co-oriented embeddings, then their fiber product is just the (embedding of the) intersection of the images of $V$ and $W$ in $M$.
	If $f$ and $g$ are normally co-oriented (see \cref{normal co-or}), then the intersection should be normally co-oriented with the orientation of the normal bundle of the intersection given by concatenating the orientation for the normal bundle of $V$ followed by the orientation for the normal bundle of $W$.

	\item Graded commutativity: Letting $v = \dim(V)$, $w = \dim W$, and $m = \dim(M)$, we should have $V \times_M W = (-1)^{(m-v)(m-w)}W \times_M V$ as fiber product, using Notation \ref{N: implicit notation}.

	\item Leibniz rule: we should have $\bd (V \times_M W) = (\bd V \times_M W)\amalg (-1)^{m-v}(V \times_M \bd W)$, using Notation \ref{N: implicit notation}.
	This formula will hold for pullbacks as well as fiber products.
\end{enumerate}

Before getting into the specifics of the construction, we need to make the following important observations.

\begin{remark}\label{R: pullback representative}
	While our canonical pullback $P$ has been defined as $V \times_M W = \{(x,y) \in V \times W \mid f(x) = g(y)\}$, categorically the pullback $P$ is technically only well defined up to canonical diffeomorphisms.
	In particular, if $P$ and $P'$ are two specific representatives of the pullback, we have commutative diagrams
	\begin{diagram}[LaTeXeqno]\label{D: comm triangle}
		P &\rTo^\cong& P'\\
		&\rdTo(1,1)\ldTo(1,1)W.
	\end{diagram}
	But, as we have observed in \cref{D: tautological co-orientation}, diffeomorphisms come equipped with natural co-orientations, and so a co-orientation of $P \to M$ determines a unique co-orientation of $P' \to M$ by composition and vice versa.
	Thus when working with co-orientations of pullbacks, we typically think of selecting a fixed representative $P \to W$ to work with for computations, though not necessarily the canonical one.
	This observation shows that we are free to do so, and typically we will do so tacitly.
	This foreshadows the notion of isomorphic representatives of geometric chains and cochains; see Definition \ref{D: equiv, triv, and small}.
\end{remark}

\begin{remark}\label{R: precise commutativity}
	This is also a good place to point out exactly what we mean by writing $V \times_M W = (-1)^{(m-v)(m-w)}W \times_M V$ in our commutativity statement, as $V \times_M W \subset V \times W$ and $W \times_M V \subset W \times V$ are different spaces, though canonically identified via the map $\tau \colon V \times W \to W \times V$ that switches the coordinates.
	This map fits into a commutative diagram of fiber products
	\begin{diagram}[LaTeXeqno]
		V \times_M W &\rTo^\tau& W \times_M V\\
		&\rdTo(1,1)\ldTo(1,1)M.
	\end{diagram}
	Again, $\tau$ has an induced co-orientation from being a diffeomorphism, and so the statement means that co-orientation of the fiber product $V \times_M W \to M$ and the composite co-orientation of $\tau$ and then the co-orientation of $W \times_M W \to M$ should differ by the sign $(-1)^{(m-v)(m-w)}$.

	Putting together this observation with \cref{R: pullback representative}, we will generally be able to identify $V \times_M W$ and $W \times_M W$ via $\tau$ as spaces, and so we typically can leave the map $\tau$ itself tacit and just work with the local orientations on $V \times_M W$ and $M$.
\end{remark}

\begin{comment}
	Achieving all of these properties requires a number of non-obvious choices of conventions, as we shall see.
	We begin by providing some general perspective before proceeding to dive in to the general definition of induced co-orientation of a fiber product.

	At a high level, induced co-orientations of fiber products of transverse co-oriented maps will arise from a common alternative description of $P = V \times_M W$ as the preimage of the diagonal $\Delta M = \{(x,x) \in M \times M\}$ under the product map $f \times g \colon V \times W \to M \times M$.
	In other words, $P = (f \times g)^{-1}(\Delta M)$; the reader can easily verify that this is equivalent to the preceding definition.
	Such an
	identification gives rise to the following exact sequences of bundles over $P$, leaving the pullbacks of the bottom row to $P$ implicit in the notation:

	\begin{equation}\label{pullback exact}
		\begin{tikzcd}
			0 \arrow[r] & TP \arrow[r] \arrow[d] & T (V \times W) \arrow[r] \arrow[d] & \nu_{P \subset V \times W} \arrow[r] \arrow[d,"\cong", "i"'] & 0 \\
			0 \arrow[r] & TM \arrow[r,"D\Delta"] & T (M \times M) \arrow[r] & \nu_{\Delta M \subset M \times M} \arrow[r] & 0.
		\end{tikzcd}
	\end{equation}
	Here the two bundles labeled with $\nu$ are normal bundles, and we use the general fact that, given a smooth map of manifolds $h \colon  A \to B$ with $C$ immersed in $B$ and $h$ transverse to $C$, the normal bundle of $h^{-1}(C)$ in $A$ is the pullback of the normal bundle of $C$ in $B$.
	In the case at hand, $P = (f \times g)^{-1}(\Delta M) \subset V \times W$ and the transversality of $f$ and $g$ implies that $f \times g$ is transverse to $\Delta M$, and so $\nu_{P \subset V \times W}$ is the pullback of $\nu_{\Delta M \subset M \times M}$; we label this isomorphism $i$.

	\red{D: we should have some more extensive differential topology discussion, pulling from Joyce's oeuvre.
		In particular, the general fact above
		should be developed in the manifolds with corners section, having a ``differential topology'' subsection.
	}
	\red{GBF: Should we do this????}

	Next we recall that if the sequence of vector bundles $0 \to K \to A \to C \to 0$ 	is exact, then we have a splitting $A \cong K \oplus C$ so that $\Or(A) \cong \Or(K) \otimes \Or(C)$.
	Such an
	isomorphism is not canonical, though it can be made concrete by, for example, taking the image in $A$ of an oriented basis of $K$ and following it by the
	preimage in $A$ of an oriented basis of $C$ to choose a representative oriented basis of $A$.
	But in our present development we will not fix an isomorphism in such a way, using only that they are isomorphic and
	later defining such isomorphisms implicitly.

	Applying these ideas to the exact sequences of Equation~\eqref{pullback exact},
	we have an isomorphism
	$\Or(TP) \otimes \Or(\nu_{P \subset V \times W}) \cong \Or (T(V \times W)) \cong \Or(TV) \otimes \Or(TW)$ and similarly for the second exact sequence.
	These fit into a not-necessarily-commutative square
	\begin{equation}\label{co-or stuff}
		\begin{tikzcd}
			\Or (TP) \otimes \Or(\nu_{P \subset V \times W}) \arrow[r, "\cong"] \arrow[d, "\gamma \otimes \Or(i)"] & \Or (T V) \otimes \Or (TW) \arrow[d] \\
			\Or (TM) \otimes \Or( \nu_{\Delta M \subset M \times M}) \arrow[r, "\cong"] & \Or (T M) \otimes \Or (TM).
		\end{tikzcd}
	\end{equation}
	%where $\gamma$ is induced by the standard map from the pullback to $M$
	%\red{[GBF: I do not think we can say this - isn't the whole point of this section is to define $\gamma$ - if there were a standard way to induce this we wouldn't need this whole diagram.]}.
	Note that the vertical maps of this diagram are not in general induced by the maps in Diagram \eqref{pullback exact}, just as in general a map $W \to M$ does not determine its co-orientation.
	However, we can choose the horizontal isomorphisms by our choices of splittings of the short exact sequences, we can let the vertical map on the right be the tensor product of the co-orientations of $V$ and $W$, and we can let
	$\Or(i)$ be determined by the canonical identification of $\nu_{P \subset V \times W}$ with the pullback of $\nu_{\Delta M \subset M \times M}$.
	Such choices will then determine a $\gamma$ making the diagram commute, and this will be our co-orientation of $P \to M$.
	Our choices of horizontal isomorphisms are essentially ``sign conventions.''
	We could for example set the top and bottom isomorphisms
	by the sort of ``basis of kernel followed by basis of cokernel'' convention mentioned above, but
	these ``obvious'' choices would not result in our three desired properties.

	BCOMMENT
	\red{Again, isn't the point that we do not know $\gamma$, so how can we fix the diagram to commute and then use that to determine $\gamma$? I think the idea is that we really need to say that the first diagram somehow determines this diagram via some conventions (what are those?).
		Then we know what the maps on the right are because that's just the tensor product of co-orientations of $V$ and $W$.
		On the left we know $i$ since that's canonical somehow (we still need to look up a good reference for that), and then all these other things determine a unique $\gamma$ so that the diagram commutes.
		This $\gamma$ is our co-orientation for $P$.
		So I think this all needs to be clarified.}
	Any such set of choices then yields
	a definition of pullback co-orientation through a diagram chase.
	In concrete terms, fix a local orientation $\beta_M$ of $M$, and then
	use the co-orientations of $f$ and $g$ to identify compatible local orientations $\beta_V$ of $V$ and $\beta_W$ of $W$.
	A fixed identification
	of the normal bundle of $\Delta M$ with the tangent bundle of $M$ then gives a $\beta_{\nu P \subset V \times W}$ which corresponds to $\beta_M$.
	The pullback co-orientation of the map $P \to M$ can then be defined pair $\beta_M$ with a
	local orientation $\beta_P$ of $P$ so that $\beta_P \otimes \beta_{\nu P \subset V \times W}$ maps to $\beta_V \otimes \beta_W$ under
	the top horizontal isomorphism of Equation~\ref{co-or stuff}.
	ECOMMENT

	In order to obtain these properties we will develop additional structure to control the
	isomorphisms in Diagram~\eqref{co-or stuff}.
	We do this first by working at the level of vector spaces and linear maps over a point before expanding to local definitions and then
	back to the global level.
\end{comment}

\subsubsection{Co-orientability of pullbacks and fiber products}

Before defining pullback and fiber product co-orientations, we first want to ensure that pullbacks and fiber products of co-orientable maps are themselves co-orientable.
The following argument about co-orientability will provide a roadmap to defining co-orientations such co-orientations.
We also take the opportunity to observe that pullbacks of proper maps are proper.

\begin{lemma}\label{L: co-orientable pullback}
	Suppose $f \colon V \to M$ and $g \colon W \to M$ are transverse maps of manifolds with corners to a manifold without boundary.
	Then:
	\begin{enumerate}
		\item If $f$ is co-orientable, the pullback $f^* \colon P = V \times_M W \to W$ is co-orientable.
		\item If $f$ is proper, the pullback $f^* \colon P = V \times_M W \to W$ is proper.
	\end{enumerate}
\end{lemma}

Note that $g$ need not be co-orientable or proper for this lemma to apply.

\begin{proof}
	We first show that the pullback is proper.
	Let us label our maps
	\begin{diagram}
		P&\rTo^{\pi_V} & V\\
		\dTo^{f^* = \pi_W}&&\dTo_f\\
		W&\rTo^g&M.
	\end{diagram}

	Suppose $K \subset W$ is compact.
	We have
	\begin{align*}
		\pi_W^{-1}(K)& = \{x \in P \mid \pi_W(x) \in K\}\\
		& \subset \{x \in P \mid g\pi_W(x) \in g(K)\} \\
		& = \{x \in P \mid f\pi_V(x) \in g(K)\} \\
		& = \{x \in P \mid \pi_V(x) \in f^{-1}(g(K))\}.
	\end{align*}
	So $\pi_W^{-1}(K) \subset K \times f^{-1}(g(K)) \subset V \times W$.
	But this is a product of compact sets as $f$ is proper.
	So $\pi_W$ is proper.

	For co-orientations, we use Quillen's definition from \cref{L: Quillen}.
	We factor $f$ as $V \into M \times \R^N \to M$, and then we have the pullback diagram

	\begin{diagram}[LaTeXeqno]\label{D: pullback}
		P&\rTo^{\pi_V} & V\\
		\dTo&&\dTo_e\\
		W \times \R^N&\rTo^{g \times \id}&M \times \R^N\\
		\dTo&&\dTo_{\pi_M}\\
		W&\rTo^g&M.
	\end{diagram}
	The bottom square is evidently a pullback.
	Thus by elementary topology the top square is a pullback diagram if and only if the composite rectangle is a pullback diagram.
	So by letting the top square be a pullback diagram, we obtain the pullback $P$ of $W \xr{g} M\xleftarrow{f} V$.

	Since $f$ is transverse to $g$, we have $g \times \id$ transverse to $e$.
	As $e$ is an embedding, it follows that $P = (g \times \id)^{-1}(e(V))$ is a submanifold of $W \times \R^N$.
	Furthermore, by \cref{L: Quillen}, $e(V)$ has an orientable normal bundle in $M \times \R^N$, and since the pullback of the normal bundle is the normal bundle of the pullback, it follows that the normal bundle of $P$ in $W \times \R^N$ is also orientable.
	Applying \cref{L: Quillen} again, the map $f^* = \pi_W \colon P \to W$ is co-orientable.
\end{proof}

\begin{remark}\label{R: pullback representative 2}
	As foreshadowed in \cref{R: pullback representative}, we here use a different realization of $P = V \times_M W$.
	Thinking of the top square of the diagram as a pullback square, this $P$ is concretely the subset $\{(v,(w,z)) \in V \times (W \times \R^N) \mid e(v) = (g(w),z)\}$.
	If $\pi_1 \colon M \times \R^N \to M$ and $\pi_2 \colon M \times \R^N \to \R^N$ are the projections, we know by definition that $\pi_1(e(v)) = f(v)$, and this is also $g(w)$, so the points in this realization of $P$ also satisfy $f(v) = g(w)$.
	In fact, there is a canonical map between this realization of $P$ and our standard realization $\{(v,w) \in V \times W \mid f(v) = g(w)\}$ given by $(v,(w,z)) \to (v,w)$ with inverse given by $(v,w) \to (v,(w,\pi_2(e(v))))$.

	We also already observed in the above proof that $P$ can be identified with $(g \times \id)^{-1}(e(V)) \subset W \times \R^N$.
	Of course when we think of $P$ as $\{(v,(w,z)) \in V \times (W \times \R^N) \mid e(v) = (g(w),z)\}$, the embedding into $W \times \R^N$ is just the map $(v,(w,z)) \to (w,z)$.
\end{remark}

\begin{corollary}
	If $f \colon V \to M$ and $g \colon W \to M$ are transverse and co-orientable, their fiber product $V \times_M W \to M$ is also co-orientable.
\end{corollary}

\begin{proof}
	By the preceding lemma, the pullback $V \times_M W \to W$ is co-orientable, and the map $g \colon W \to M$ is co-orientable by assumption.
	Now choose co-orientations and compose to get a co-orientation of $P \to M$.
\end{proof}

\subsubsection{Co-orientations of pullbacks and fiber products}\label{S: co-orientation of pullbacks}

The construction in the proof of \cref{L: co-orientable pullback} provides a roadmap to define the co-orientations of pullbacks and fiber products.
For the following definition, refer again to Diagram \eqref{D: pullback}.

\begin{definition}\label{D: pullback coorient}
	Suppose $f \colon V \to M$ and $g \colon W \to M$ are transverse with $f$ co-oriented and the normal bundle $\nu V$ of $e(V) \subset M \times \R^N$ given its Quillen normal orientation as defined in \cref{D: Quillen normal or}.
	Then the pullback $P = V \times_M W = (g \times \id_{\R^N})^{-1}(e(V)) \subset W \times \R^N$ has an oriented normal bundle that is the pullback of $\nu V$, which, by abuse of notation, we also label $\nu V$.
	Let $\beta_P$ and $\beta_W$ be local orientations of $P$ and $W$, and let $\beta_E$ be the standard orientation of $\R^N$.
	Define the \textbf{pullback co-orientation} on $P \to W$ to be the composition of the normal co-orientation $(\beta_P,\beta_P \wedge \nu V)$ with the canonical co-orientation $(\beta_W \wedge \beta_E,\beta_W)$.
	In other words, the pullback co-orientation is $(\beta_P,\beta_W)$ if $\beta_P \wedge \beta_{\nu V} = \beta_W \wedge \beta_E$ up to positive scalar and $-(\beta_P,\beta_W)$ otherwise.

	Following \cref{D: top pullback}, we sometimes write $f^* \colon P \to W$.
	We also sometimes write $P = g^*V$ to emphasize that $P$ is the pullback of $V$ by $g$ to a manifold over $W$.

	If $g$ is co-oriented, define the \textbf{fiber product co-orientation} on $P \to M$ as the composition of the pullback co-orientation with the co-orientation of $g \colon W \to M$.
\end{definition}

In the definition, note that the Quillen orientation of $\nu V$ is determined by the co-orientation of $f$ and the orientation $\beta_E$ is taken to be canonically fixed across all instances.
The other orientations appearing in the definition are $\beta_P$ and $\beta_W$, but the co-orientation of the pullback $f^* \colon P \to W$ does not depend on the particular choices.
For example, suppose we choose $\beta_P$ and $\beta_W$ so that $\beta_P \wedge \beta_{\nu V} = \beta_W \wedge \beta_E$ and hence the pullback co-orientation is $(\beta_P,\beta_W)$.
If we replace $\beta_P$ with $\beta_P' = -\beta_P$, then
$\beta_P' \wedge \beta_{\nu V} = -\beta_P \wedge \beta_{\nu V} = -\beta_W \wedge \beta_E$, so the co-orientation is $-(\beta_P',\beta_W) = -(-\beta_P,\beta_W) = (\beta_P,\beta_W)$.
So the co-orientation is unchanged.
Similarly the definition is independent of our choice of $\beta_W$.
We will show just below that the definition is independent of $N$ and $e$ as well.

\begin{remark}\label{R: co-or restriction or switch}
	It follows from the definition that reversing the co-orientation of $f \colon V \to M$ reverses the co-orientation of $f^* \colon V \times_M W \to W$.
	Furthermore, if $g \colon W \to M$ is co-oriented, then reversing the co-orientation of either $f$ or $g$ reverses the co-orientation of the fiber product $f \times_M g \colon V \times_M W \to M$.

	It is also clear that the definition is consistent under restrictions to open sets.
	In other words if $x \in V$, $y \in W$ with $f(x) = g(y)$, then replacing $V$, $W$, and $M$ with neighborhoods of $x$, $y$, and $f(x) = g(y)$ yields a co-orientation of the restriction of $f^*$ to a neighborhood of $(x,y) \in V \times_M W$ that is consistent with the co-orientation of all of $f^*$, at least so long as we use the same $N$ and a restriction of $e$, though we will now show independence of these choices as well.
\end{remark}

\begin{lemma}\label{L: pullback co well defined}
	The pullback and fiber product co-orientations do not depend on the choices of $N$, $e$, or local orientations of $P$, $V$, $W$, or $M$.
\end{lemma}

\begin{remark}
	The co-orientations \textit{do} depend on the choice of the canonical orientation for $\R^N$, the choice of the standard co-orientation of the projection to be $(\beta_M \wedge \beta_E, \beta_M)$, etc., but these are all universal choices.
	The point is that the pullback and fiber product co-orientations only depend on $f$, $g$, and their co-orientations, after fixing such universal choices that do not depend on $f$ or $g$.
\end{remark}

\begin{proof}[Proof of \cref{L: pullback co well defined}]
	As the local orientations of $P$, $V$, $W$, and $M$ used in the construction all come in pairs (e.g.\ $\beta_P$ in $(\beta_P,\beta_P \wedge \nu_V))$, the construction is independent of those choices.

	Next, suppose we are given an embedding $e \colon V \into M \times \R^N$ and extend it to $e' = (e,0) \colon V \into M \times \R^N \times \R^n$.
	In the construction involving $e$, if we choose $\beta_V$, $\beta_M$ so that $(\beta_V,\beta_M)$ is the co-orientation for $f$, then by the definition of the Quillen orientation, $\nu V$ will be such that $\beta_V \wedge \beta_{\nu V} = \beta_M \wedge \beta_E$ up to positive scalar.
	If we now increase the dimension of the Euclidean factor to $\R^{N+n}$ and write its canonical local orientation as $\beta_{E^N} \wedge \beta_{E^n}$ while extending $e$ to $e'$, we see that $\nu_V$ becomes $\nu_V \oplus \underline{\R}^n$ so that $\beta_{\nu V}$ becomes $\beta_{\nu V} \wedge \beta_{E^n}$.
	Pulling back over $W$ we obtain the pullback co-orientation $(\beta_P,\beta_P \wedge \beta_{\nu V} \wedge \beta_{E^n})*(\beta_W \wedge \beta_{E^N} \wedge \beta_{E^n},\beta_W)$.
	This is $(\beta_P,\beta_W)$ if and only if $\beta_P \wedge \beta_{\nu V} \wedge \beta_{E^n} = \beta_W \wedge \beta_{E^N} \wedge \beta_{E^n}$ up to positive scalar, but this condition is equivalent to having $\beta_P \wedge \beta_{\nu V} = \beta_W \wedge \beta_{E^N}$ up to positive scalar.
	So the pullback co-orientation is unchanged.

	Next suppose that $e_0 \colon V \to M \times R^{N_0}$ and $e_1 \colon V \to M \times R^{N_1}$ are any two embeddings over $f$.
	By the preceding paragraph, by adding Euclidean factors we can assume $N_0 = N_1 = N$ for some sufficiently large $N$ without changing the pullback co-orientations associated to $e_0$ and $e_1$.
	Let $\pi \colon M \times \R^N \to M$ be the projection to $M$.
	As $\pi e_0 = \pi e_1$, the maps $e_0$ and $e_1$ are homotopic over $f$, say by linear homotopies in the Euclidean fibers.
	Let $H \colon V \times I \to M \times \R^N$ be the chosen homotopy.
	Next, by the same argument by which embeddings $e$ exist, there is an embedding $\td H: V \times I \into M \times \R^N \times \R^Q$ for some $Q$ so that if $\td \pi \colon M \times \R^N \times \R^Q \to M \times \R^N$ is the projection then $\td \pi \td H = H$.
	If we let $\td e_0 = \td H(-,0) \colon V \to M \times \R^N \times \R^Q$ then $\td \pi \td e_0 = e_0$.
	If we let $(e_0,0) \colon V \to M \times \R^N \times \R^Q$ denote the map $x \to (e_0(x), 0)$, then there is a homotopy from $\td e_0$ to $(e_0,0)$; in fact as $e_0$ is an embedding and $\td \pi \td e_0 = e_0$, we can let these homotopies be linear in the $\R^Q$ factor and constant in the other factors and this homotopy will be an embedding of $V \times I$.
	We can define $\td e_1$, $(e_1,0)$, and an embedded homotopy between them similarly.

	So we have a sequence of three embedded homotopies, say $F_1$, $F_2$, $F_3$ from $(e_0,0)$ to $\td e_0$, from $\td e_0$ to $\td e_1$, and from $\td e_1$ to $(e_1,0)$, respectively.
	Additionally, $\pi \td \pi:F_j(x,t) = f(x)$ so each homotopy is constant in $I$ when projected to $M$, and in particular each of $(e_0,0)$, $(e_1,0)$, $\td e_0$, and $\td e_1$ is an embedding $V \into M \times \R^{N+Q}$ over $f \colon V \to M$.
	We know from above that the pullback co-orientation obtained from using $(e_0,0)$ and $(e_1,0)$ agree with those obtained from $e_0$ and $e_1$.
	So it suffices to use the homotopies to show successively that $(e_0,0)$, $\td e_0$, $\td e_1$, and $(e_1,1)$ all provide the same pullback co-orientation of $f^*$.

	By \cref{L: co-orientable homotopies,D: homotopy co-orientation}, we can use the embeddings $F_j$ to co-orient each of our constant homotopies.
	In particular, using $F_j$ in place of $e$ in \cref{D: pullback coorient} we obtain three co-orientations of the pullbacks $(V \times I) \times_M W \to W$.
	We will see below in \cref{leibniz}, whose proof is independent of this one, that when accounting for co-orientations, pullback co-orientations satisfy a Leibniz rule and, in particular, two of the signed boundary components of each co-oriented $(V \times I) \times_M W \to W$ will be $(V \times \{0\}) \times_M W \to W$ and $(V \times \{1\}) \times_M W \to W$, occurring with opposite signs.
	In other words, with appropriate choices on the co-orientations of the homotopies, by \cref{D: co-oriented homotopy}, we obtain three sequential co-oriented (constant) homotopies from $f$ to itself.
	It now follows by applying \cref{R: stationary homotopy} sequentially that all four copies of $f$ must have the same co-orientation.
	In particular, this is the case for the co-orientations of $f$ obtained from the embeddings $e_0$ and $e_1$.
\end{proof}

\begin{remark}\label{R: local pullback co-orientations}
	The pullback co-orientation is determined locally in the sense that if $U$ is an open subset of $M$ then the pullback co-orientation of $f^{-1}(U) \times_U g^{-1}(U) \to g^{-1}(U)$ will just be the restriction of the pullback co-orientation of $V \times_M W \to W$.
	This is clear from the construction if we co-orient the local pullback using the Quillen co-orientation of $f^{-1}(U) \to U$ given by $f^{-1}(U) \xhookrightarrow{e|_{f^{-1}(U)}} U \times \R^N \to U$, the restriction of the Quillen co-orientation $V\xhookrightarrow{e}M \times \R^N \times M$ use to co-oriented $f \colon V \to M$.
	But \cref{L: pullback co well defined} says that we are free to do so.
\end{remark}

\begin{remark}\label{R: what products exist}
	We have just shown that, after choosing conventions, the fiber product of two transverse co-oriented maps is co-oriented, and this will eventually lead us to the cup product of geometric cochains.
	Analogously, if $f \colon V \to M$ is co-oriented and $W$ is oriented, then the pullback co-orientation $f^* \colon P = V \times_M W \to W$ provides a way to orient $P$, namely if $\beta_W$ is the given globally-defined orientation of $W$ we can choose $\beta_P$ so that $(\beta_P, \beta_W)$ is the co-orientation of $f^*$.
	This observation will be utilized below in our construction of the cap product.
	However, somewhat surprisingly, the fiber product of two maps with oriented domains cannot necessarily be oriented, and so there is in general no product of geometric chains and hence, in general, no homology product.
	Such oriented fiber products can be formed if the the codomain $M$ is oriented, as in this case there is an equivalence between orientations of domains and co-orientations of maps.
	But this is not always possible when $M$ is not orientable.
	For example, we recall that the intersection of two orientable $\R P^3$s in the non-orientable $\R P^4$ can be a non-orientable $\R P^2$.
\end{remark}

\medskip\noindent\textbf{Functoriality of pullbacks.}
The co-oriented pullback construction is functorial in the following sense.
% Consider a \subsubsection?

\begin{proposition}\label{P: pullback functoriality}
	Suppose $f \colon V \to M$ is co-oriented.
	Then the pullback of $f$ by the identity $\id_M \colon M \to M$ is (diffeomorphic to) $f \colon V \to M$ with the same co-orientation.

	Suppose further that $W$ and $M$ are manifolds without boundary, that $g \colon W \to M$ is transverse to $f$ and that $h \colon X \to W$ is transverse to $V \times_M W \to W$ (or, equivalently by \cref{L: transverse to pullback}, that $gh$ is transverse to $f$).
	Then $(gh)^*V \cong h^*g^*V$ as co-oriented manifolds over $M$.
\end{proposition}

\begin{proof}
	We first note that there is a diffeomorphism between $V$ and $\{(v,x) \in V \times M \mid f(v) = x\}$ given by $v \to (v,f(v))$ and $(v,x) \to v$.
	Then given compatible Quillen co-orientation of $f$, we can form the pullback diagram as
	\begin{diagram}
		V&\rTo^{\id_V} & V\\
		\dTo^e&&\dTo_e\\
		M \times \R^N&\rTo^{\id_{M \times \R^N}}&M \times \R^N\\
		\dTo^{\pi_M}&&\dTo_{\pi_M}\\
		M&\rTo^{\id_M}&M,
	\end{diagram}
	and the conclusion is evident.

	For the second claim, there is a diffeomorphism between $\{(v,x) \in V \times X \mid f(v) = g(h(x))\}$ and $\{((v,w),x) \in (V \times_M W) \times X \mid w = h(x)\}$ given by $(v,x) \to (v,h(x),x)$ and $((v,w),x) \to (v,x)$.
	To see that the last map is well-defined notice that $f(v) = g(h(x))$ as $h(x) = w$, and $f(v) = g(w)$ from the assumption $(v,w) \in V \times_M W$.
	Compatibility of the co-orientations now follows by considering the following diagram and noting that it is equivalent to pull back the normal $\nu V$ to $X \times \R^N$ either in two steps or all at once.
	\begin{diagram}
		(V \times _M W)\times_W X&\rTo^{\pi_{V \times_M W}} &V \times_M W&\rTo^{\pi_V} & V\\
		\dTo&&\dTo&&\dTo_e\\
		X \times \R^N&\rTo^{h \times \id}&W \times \R^N&\rTo^{g \times \id}&M \times \R^N\\
		\dTo&&\dTo&&\dTo_{\pi_M}\\
		X&\rTo^h&W&\rTo^g&M.
	\end{diagram}
\end{proof}

\subsubsection{Fiber products of immersions}\label{S: co-or product immersion}

Pullbacks have particularly nice descriptions when on or both of the maps are embeddings or immersions.
In addition, these special cases are good for building intuition about the more general situation.

\begin{example}\label{E: V embedded}
	When $f \colon V \to M$ is a co-oriented embedding, the pullback $V \times_M W$ is particularly easy to describe.
	In this case, we know from \cref{S: normal orientatin} that the co-orientation is determined by an orientation $\beta_{\nu V}$ of the normal bundle to $V$ in $M$.
	Then, as $f$ is already an embedding, we can take $N = 0$ in \cref{D: pullback coorient}.
	So the pullback $V \times_M W$ is just the submanifold $g^{-1}(V) \subset W$, co-oriented by $(\beta_P,\beta_W)$, where $\beta_P \wedge \beta_{\nu V} = \beta_W$, the $\nu V$ here being the pullback of the normal bundle to $g^{-1}(V)$ in $W$.
	In other words, the co-orientation of the pullback is just the normal co-orientation corresponding to the pulled back orientation of $\nu V$.
	We can say that $V$ and $V \times_M W = g^{-1}(V)$ have compatible normal co-orientations.

	The case where $g$ is embedded instead also has a nice description but requires some more technology.
	We will discuss that case below in \cref{E: W embedded}.
\end{example}

In the key example when both $f$ and $g$ are immersions, the fiber products will locally correspond to intersections of the images in $M$, and in this case our general definition of fiber product co-orientation is compatible with the approach to co-orientations via orientations of normal bundles.

\begin{proposition}\label{P: normal pullback}
	Let $f \colon V \to M$ and $g \colon W \to M$ be transverse co-oriented immersions from manifolds with corners to a manifold without boundary.
	Let $\nu V$ and $\nu W$ denote the respective normal bundles.
	Choose local Quillen orientations $\beta_{\nu V}$ and $\beta_{\nu W}$ so that the normal co-orientations $(\beta_V,\beta_V \wedge \beta_{\nu V})$ and $(\beta_W,\beta_W \wedge \beta_{\nu W})$ agree with the given co-orientations of $f$ and $g$.
	Then, decomposing the normal bundle of the fiber product $P = V \times_M W \to M$ at any point of intersection as $\nu V \oplus \nu W$ and giving it the orientation $\beta_{\nu V} \wedge \beta_{\nu W}$, the fiber product co-orientation agrees with the normal co-orientation, i.e.\
	$$\omega_{f \times_M g} = (\beta_P,\beta_P \wedge \beta_{\nu V} \wedge \beta_{\nu W}).$$
\end{proposition}

That is, if one orients the normal bundle of the intersection by following an oriented basis of the normal bundle of $V$ by one for $W$,
the associated normal co-orientation is the fiber product co-orientation.

\begin{proof}
	It suffices to demonstrate this property in the neighborhood of any intersection point, so we may assume that $f$ and $g$ are embeddings and consider $x \in V$, $y \in W$ with $f(x) = g(y) = z \in M$.
	Locally, for our Quillen co-orientation of $f$ we can apply the definition of the pullback co-orientation with $N = 0$ and the embedding $e \colon V \into M \times \R^N$ to be simply $f$ itself.
	As $N = 0$, in this case $\nu V$ is itself the oriented normal bundle of $e(V) = f(V)$ in $M \times \R^N = M$.
	Pulling back via $g$ to $W$, we obtain the oriented pullback of $\nu V$ (which we also call $\nu V$) as the normal bundle of $P$ in $W$.
	By definition, the co-orientation of $P \to W$ is then the composition of $(\beta_P,\beta_P \wedge \beta_{\nu V})$ with the standard co-orientation of the projection $W \times \R^N$ to $W$, which in this case is the identity.
	The co-orientation of the fiber product is thus the composition of $(\beta_P,\beta_P \wedge \beta_{\nu V})$ with the co-orientation $(\beta_W,\beta_W \wedge \beta_{\nu W})$ of $g$.
	But this last co-orientation is independent of the choice of $\beta_W$, so we can take $\beta_W = \beta_P \wedge \beta_{\nu V}$.
	Thus we see that the fiber product co-orientation of $P \to M$ is $(\beta_P, \beta_P \wedge \beta_{\nu V} \wedge \beta_{\nu W})$, as desired.
\end{proof}

\subsubsection{The Leibniz rule}

We now verify the Leibniz rule.
Note that in the proof we work with a single arbitrary but fixed $e \colon V \into M \times \R^N$ and its restriction to $\bd V$, and so the following theorem holds for any such choice in the definition of the pullback co-orientation.
Consequently, this result completes the proof of \cref{L: pullback co well defined}.

\begin{proposition}[Leibniz rule]\label{leibniz}
	Let $f \colon V \to M$ and $g \colon W \to M$ be transverse maps from manifolds with corners to a manifold without boundary, and suppose $f$ co-oriented.
	Let $V \times_M W \to W$ be the co-oriented pullback.
	Then
	$$\bd (V \times_M W) = (\bd V) \times_M W \bigsqcup (-1)^{\dim(M)-\dim(V)} V \times_M (\bd W),$$
	interpreting each of these pullback spaces as representing its co-oriented map to $W$; see Notation \ref{N: implicit notation}.\footnote{Here we interpret $V\times\bd W \to W$ as the composition of the co-oriented pullback $V\times\bd W \to \bd W$ with the co-oriented boundary immersion $\bd W \to W$.} If $g$ is also co-oriented then this formula also holds as fiber products mapping to $M$.
\end{proposition}

Establishing this directly for immersions, for which we can use the normal co-orientations, is a quick exercise.
The general case requires more care.

\begin{proof}
	The statement at the level of underlying manifolds with corners is \cite[Proposition 6.7]{Joy12}, so we focus on co-orientations.
	The second statement follows from the first by composing each map with the co-oriented map $g \colon W \to M$ and taking the composite co-orientations.
	We will write $\bd P$ when considering the boundary co-orientation of $P = V \times_M W$ and $(\bd V) \times_M W$ or $V \times_M (\bd W)$ when considering these pullback co-orientations.
	In the following arguments, it suffices to consider points in the interiors of $\bd V$ or $\bd W$ as knowing a co-orientation at one point of each component determines it globally; in other words, we can avoid corners.

	By \cref{D: pullback coorient}, the co-orientation $\omega_{f^*}$ of $P \to W$ is $(\beta_P,\beta_W)$ if and only if $\beta_P \wedge \beta_{\nu V} = \beta_W \wedge \beta_E$ up to positive scalar, where $\nu V$ is the pullback to $P$ of the Quillen-oriented normal bundle of $e(V)$ in $M \times \R^N$.
	Recall that by \cref{D: boundary co-orientation}, if $\nu \bd P$ is the inward pointing normal of $\bd P$ in $P$ then $\bd P \to V$ is co-oriented by composing the boundary co-orientation $(\beta_{\bd P},\beta_{\bd P} \wedge \beta_{\nu \bd P})$ with $\omega_{f^*}$.
	Choosing $\beta_P = \beta_{\bd P} \wedge \beta_{\nu \bd P}$ (for an arbitrary choice of $\beta_{\bd P}$), we see $\omega_{\bd P \to W}$ is $(\beta_{\bd P},\beta_W)$ if and only if $\beta_{\bd P} \wedge \beta_{\nu\bd P} \wedge \beta_{\nu V} = \beta_W \wedge \beta_E$.

	Now, consider a point in the interior of $(\bd V) \times_M W \subset V \times_M W = P$.
	By choosing standard choices of coordinate charts in such neighborhoods, we see that we can take $\nu\bd V = \nu \bd P$ to be the same inward pointing normal vector.
	In the construction of the pullback co-orientation for $(\bd V) \times_M W$, we can take $e \colon \bd V \to M \times \R^N$ to be the restriction of the $e \colon V \to M \times \R^N$ used to co-oriented $P$.
	As we have already used $\nu \bd V$ for the normal of $\bd V \subset V$, to avoid confusing the notation, we write $\nu^s\bd V$ ($s$ for stable) for the normal bundle of $e(\bd V) \subset M \times \R^N$.
	Note that $\nu^s\bd V \cong \nu\bd V \oplus \nu V$.
	The Quillen orientation $\beta_{\nu V}$ was chosen so that $\beta_V \wedge \beta_{\nu V} = \beta_{M} \wedge \beta_E$.
	If we choose $\beta_{\bd V}$ so that $\beta_V = \beta_{\bd V} \wedge \beta_{\nu\bd V}$ then the boundary co-orientation of $\bd V \to M$ will be $(\beta_{\bd V},\beta_M)$ so we can then perform the construction of \cref{D: pullback coorient} using $\beta_{\bd V}$.
	Note that as $\beta_{\nu V}$ is chosen so that $\beta_V \wedge \beta_{\nu V} = \beta_M \wedge \beta_E$, we will have also $\beta_M \wedge \beta_E = \beta_{\bd V} \wedge \beta_{\nu\bd V} \wedge \beta_{\nu V}$ and so the Quillen orientation of $\nu^s\bd V$ is $\beta_{\nu^s\bd V} = \beta_{\nu\bd V} \wedge \beta_{\nu V}$.

	So now applying once again \cref{D: pullback coorient} and writing $\beta_{(\bd V) \times_M W} = \beta_{\bd P}$ at our chosen point, the co-orientation is $(\beta_{\bd P},\beta_W)$ if and only if $\beta_{\bd P} \wedge \beta_{\nu^s\bd V} = \beta_W \wedge \beta_E$ up to positive scalar.
	But $\beta_{\bd P} \wedge \beta_{\nu^s\bd V} = \beta_{\bd P} \wedge \beta_{\nu\bd V} \wedge \beta_{\nu V}$.
	So the co-orientation is $(\beta_{\bd P},\beta_W)$ if and only if $\beta_{\bd P} \wedge \beta_{\nu\bd V} \wedge \beta_{\nu V} = \beta_W \wedge \beta_E$ up to positive scalar, which we established above to be precisely the description for this component of $\bd P \to W$.

	Next, consider a point $x$ in $V \times_M \bd W$.
	Again we can take $\nu\bd W = \nu \bd P$ to be the same inward pointing normal vector, and we write $\beta_{V \times_M W}$ and $\beta_{V \times_M \bd W}$ as $\beta_P$ and $\beta_{\bd P}$ near our point.
	Furthermore, we can choose $\beta_{W}$, $\beta_{\bd W}$, and $\beta_M$ so that $(\beta_W,\beta_M)$ is the co-orientation of $g$ and $(\beta_{\bd W},\beta_W)$ co-orients $i_{\bd W}$, which implies $\beta_W = \beta_{\bd W} \wedge \beta_{\nu\bd W}$.
	We also continue to choose $\beta_{P} = \beta_{\bd P} \wedge \beta_{\nu \bd P} = \beta_{\bd P} \wedge \beta_{\nu \bd W}$.
	By \cref{D: pullback coorient}, applied to $\bd g \colon \bd W \to M$, the co-orientation of the pullback $V \times_M \bd W \to \bd W$ is $(\beta_{\bd P},\beta_{\bd W})$ if and only if $\beta_{\bd P} \wedge \beta_{\nu V} = \beta_{\bd W} \wedge \beta_E$ up to positive scalar as local orientations at the image of $x$ in $\bd W \times \R^N$.
	Considering $\bd W \times \R^N \subset W \times \R^N$ locally (recalling that $x$ is chosen in the interior of $\bd W$), $\beta_{\bd P} \wedge \beta_{\nu V} = \beta_{\bd W} \wedge \beta_E$ if and only if $\beta_{\bd P} \wedge \beta_{\nu V} \wedge \beta_{\nu\bd W} = \beta_{\bd W} \wedge \beta_E \wedge \beta_{\nu\bd W}$.
	But as $\dim(\nu\bd W) = 1$ and $\dim(\nu V) = m+N-v$,
	$$\beta_{\bd P} \wedge \beta_{\nu V} \wedge \beta_{\nu\bd W} = (-1)^{m+N-v}\beta_{\bd P} \wedge \beta_{\nu\bd W} \wedge \beta_{\nu V} = (-1)^{m+N-v}\beta_{\bd P} \wedge \beta_{\nu\bd P} \wedge \beta_{\nu V},$$ and
	$$\beta_{\bd W} \wedge \beta_E \wedge \beta_{\nu\bd W} = (-1)^N\beta_{\bd W} \wedge \beta_{\nu\bd W} \wedge \beta_E = (-1)^N\beta_{W} \wedge \beta_E.$$
	So the co-orientation of $V \times_M \bd W \to W$ is $(\beta_{\bd P},\beta_{\bd W})$, and hence by composition the co-orientation of $V \times_M \bd W \to \bd W$ is $(\beta_{\bd P},\beta_{W})$,
	if and only if $\beta_{\bd P} \wedge \beta_{\nu\bd P} \wedge \beta_{\nu V} = (-1)^{m-v}\beta_{W} \wedge \beta_E$ up to positive scalar.
	Comparing with $\omega_{\bd P \to W}$, we thus see that $\omega_{V \times_M \bd W \to W}$ differs from it by a sign of $(-1)^{m-v}$ as desired.
\end{proof}

\subsubsection{Graded commutativity}

We demonstrate here graded commutativity of fiber product co-orientations.
We do so only for proper maps, but this will suffice for the purposes of geometric chains and cochains below.
The reader should recall \cref{R: precise commutativity} for a precise explanation of the statement of the proposition.

\begin{proposition}\label{P: graded comm}
	Suppose $f \colon V \to M$ and $g \colon W \to M$ are transverse co-oriented maps from manifolds with corners to a manifold without boundary.
	Then as co-oriented fiber products over $M$ we have $\omega_{g \times_M f} = (-1)^{(m-v)(m-w)} \omega_{f \times_M g}$, or, using Notation \ref{R: precise commutativity},
	$$V \times_M W = (-1)^{(m-v)(m-w)} W \times_M V.$$
\end{proposition}

Before proving the proposition, we put it to use in the following example.

\begin{example}\label{E: embedded}
	In \cref{E: V embedded}, we considered pullback co-orientations $V \times_M W \to W$ when $V \into M$ was embedded.
	In this example, we discuss the case where $W$ is embedded.

	Let $f \colon V \to M$ and $g \colon W \to M$ be transverse maps from manifolds with corners to a manifold without boundary.
	Suppose $f$ is co-oriented and $g \colon W \to M$ is an embedding.
	Let $(x,y) \in V \times_M W$.
	Even though $g$ might not be co-oriented, let us choose a contractible neighborhood $U$ of $y$ in $W$ and an arbitrary co-orientation $(\beta_U,\beta_M)$ on the restriction of $g$ to $U$.
	For the remainder of the argument, we fix this local orientation $\beta_M$ at $f(x) \in M$, and we choose $\beta_V$ at $x$ so that $(\beta_V,\beta_M)$ is the co-orientation of $V$ at $x$.

	Although we are interested in $V \times_M U$, the definition of the pullback co-orientation makes it easier to work with $U \times_M V$ when $U$ is embedded; see \cref{E: V embedded}.
	As we have chosen a co-orientation for $U \into M$ and $V \to M$ comes with a co-orientation, we can consider
	the fiber product $U \times_M V \to M$.
	If we choose an orientation $\beta_{\nu U}$ of the normal bundle to $U$ at $y$ so that $\beta_U \wedge \beta_{\nu U} = \beta_M$, then this fiber product is $P_U = f^{-1}(U) = f^{-1}(U) \xr{f} M$ co-oriented by $(\beta_P,\beta_M)$, where $\beta_P$ is chosen so that $\beta_P \wedge \beta_{\nu U} = \beta_V$, as usual letting $\nu U$ here also stand for its pullback as a normal bundle of $P_U$ in $V$.
	At this point, if we had chosen the opposite co-orientation for $U \into M$, we would have the opposite fiber product co-orientation.

	Now, we apply \cref{P: graded comm}.
	Technically we need $g|_U$ to be proper, but we can achieve this by assuming that $U$ is the intersection of $W$ with some open neighborhood $\mc U$ of $y$ in $M$.
	Then we can replace the co-oriented embedding $U \into M$ with the proper co-oriented embedding $U \into \mc U$ and $f$ with its restriction to $f^{-1}(\mc U)$.
	As pullback co-orientations are determined locally as noted in \cref{R: local pullback co-orientations}, this will suffice.
	For simplicity, though, we maintain our original notations.

	By \cref{P: graded comm}, we have $U \times_M V = (-1)^{(m-v)(m-w)}V \times_M U$ as co-oriented fiber products over $M$.
	So the fiber product $V \times_M U \to M$ is represented by $f^{-1}(U) \xr{f}M$ with co-orientation at $(x,y)$ given by $(-1)^{(m-v)(m-w)}(\beta_P,\beta_M)$.
	But decomposing the fiber product as the pullback $V \times_M U \to U$ and the inclusion $U \into M$, we can also decompose the co-orientation as $(-1)^{(m-v)(m-w)}(\beta_P,\beta_M) = (-1)^{(m-v)(m-w)}(\beta_P,\beta_U)*(\beta_U,\beta_M)$.
	As $(\beta_U,\beta_M)$ is our co-orientation for $g|_U$, the pullback co-orientation must be $(-1)^{(m-v)(m-w)}(\beta_P,\beta_U)$.
	Furthermore, if we had chosen the opposite co-orientation for $g|_U$, that would reverse the signs of both $(\beta_U,\beta_M)$ and of the fiber product, so in either case the pullback co-orientation is $(-1)^{(m-v)(m-w)}(\beta_P,\beta_U)$.
	In other words, this description of the co-oriented pullback $V \times_M U \to U$ is independent of our choice of co-orientation for $g|_U$, and so it extends globally to $V \times_M W \to W$.
	And explicitly it can be described as follows:
	$V \times_M W \to W$ is just the inclusion $f^{-1}(W) \into W$ co-oriented by $(-1)^{(m-v)(m-w)}(\beta_P,\beta_W)$, where, if we write the co-oriented of $f$ locally by $(\beta_V,\beta_M)$ and choose any local orientation $\beta_{\nu W}$ for the normal bundle of $W$ in $M$, then $\beta_P \wedge \beta_{\nu W} = \beta_V$ and $\beta_W \wedge \beta_{\nu W} = \beta_M$.
	In other words, we can say that $W \subset M$ and $V \times_M W = f^{-1}(W) \subset V$ have compatible local normal co-orientations up to the sign $(-1)^{(m-v)(m-w)}$.

	It is a nice exercise to confirm that this agrees with the computation of \cref{P: normal pullback} when $V$ and $W$ are both embedded.
\end{example}

\cref{P: graded comm} will be proven through a sequence of lemmas.
One common theme is to first observe that the theorem will be true if it is true at any one point in each connected component of $P$.
As fiber product co-orientations are well defined globally, if they agree or disagree at any one point then they must agree or disagree on the entire connected component.
By \cref{R: precise commutativity} we can also generally identify $V \times_M W$ with $W \times_M V$ as spaces in what follows and focus only on how the two constructions specify co-orientations.

\begin{lemma}\label{L: im/im}
	If there exists $x \in V$ and $y \in W$ such that $f(x) = g(y)$ and $f$ and $g$ are respectively immersions at $x$ and $y$, then the theorem holds for the connected component containing $(x,y) \in V \times_M W$.
\end{lemma}

\begin{proof}
	As observed just above, it suffices to prove the desired identity at $(x,y)$.
	By \cref{P: normal pullback}, we have $\omega_{f \times_M g} = (\beta_P,\beta_P \wedge \beta_{\nu V} \wedge \beta_{\nu W}),$ noting in the second term that $\beta_P$ is technically the image of the local orientation $\beta_P$ of $P$ now considered as a submanifold of $M$ via the embedding.
	Similarly, \cref{P: normal pullback} gives
	$\omega_{g \times_M f} = (\beta_P,\beta_P \wedge \beta_{\nu W} \wedge \beta_{\nu V})$.
	So the two fiber products differ by $(-1)^{\dim(\nu W)\dim(\nu V)} = (-1)^{(m-v)(m-w)}$ as required.
\end{proof}

\begin{lemma}\label{L: im/sub}
	If there exists $x$ in the interior of $V$ and $y \in W$ such that $f(x) = g(y)$, $f$ is a submersion at $x$, and $g$ is an immersion at $y$, then the theorem holds for the connected component containing $(x,y) \in V \times_M W$.
\end{lemma}

\begin{proof}
	By \cref{R: co-or restriction or switch} and the above observations, it suffices to consider a neighborhood of $x$ and a chart around $f(x)$ with respect to which $f$ agrees locally with the projection $M \times \R^N \to M$.
	For simplicity of notation, we can take all of $M$ to be this chart.
	We can choose the ordering of the coordinates in $\R^N$ so that $(\beta_M \wedge \beta_E,\beta_M)$ agrees with the co-orientation for $f$.
	In this case we let $e \colon V \into M \times \R^N$ be the diffeomorphism realizing the neighborhood of $x$ as $M \times \R^N$.
	The normal bundle $\nu_V$ in this case is $0$-dimensional and positively oriented.
	The pullback to $P$ is all of $W \times \R^N$, and if we also choose $\beta_W$ at $y$ so that $\omega_g = (\beta_W,\beta_M)$, then by \cref{D: pullback coorient} the fiber product co-orientation for $f \times_M g$ is $(\beta_P,\beta_M)$ if and only if $\beta_P = \beta_W \wedge \beta_E$.

	Now consider instead $W \times_M V$, constructed using the same neighborhoods.
	In this case we take $e = g \colon W \to M \times \R^0$.
	The map $f$ is still our projection $M \times \R^N \to M$, and so the pullback is $f^{-1}(W) = W \times \R^N$.
	The normal bundles are $\nu_W$.
	We continue to assume coordinates so that $f$ is co-oriented by $(\beta_M \wedge \beta_E, \beta_M)$.
	By \cref{D: pullback coorient} the fiber product co-orientation for $g \times_M f$ is $(\beta_P,\beta_M)$ if and only if $\beta_P \wedge \beta_{\nu W} = \beta_V$.
	But if $\beta_P = \beta_W \wedge \beta_E$ consistent with the preceding case and $\beta_V = \beta_M \wedge \beta_E$, then we have
	\begin{multline*}\beta_P \wedge \beta_{\nu W} = \beta_W \wedge \beta_E \wedge \beta_{\nu W} = (-1)^{N\dim(\nu W)}\beta_W \wedge \beta_{\nu W} \wedge \beta_E\\ = (-1)^{N\dim(\nu W)}\beta_M \wedge \beta_E = (-1)^{N\dim(\nu W)}\beta_V.\end{multline*}
	Here we have used that $\beta_M = \beta_W \wedge \beta_{\nu W}$ by the choice of Quillen local orientation for $\nu W$ in \cref{D: pullback coorient}, taking into account $N = 0$.
	As $\dim(\nu W) = m-w$ and $N = v-m$, the lemma is follows.
\end{proof}

\begin{lemma}\label{L: sub/sub}
	If there exist $x$ and $y$ in the respective interiors of $V$ and $W$ such that $f(x) = g(y)$ and $f$ and $g$ are respective submersions at $x$ and $y$, then the theorem holds for the connected component containing $(x,y) \in V \times_M W$.
\end{lemma}

\begin{proof}
	As in the preceding lemma, we can choose local coordinates such that $f$ is the projection $M \times \R^N \to M$ and $g$ is the projection $M \times \R^n \to M$.
	Let $\beta_E$ and $\beta_F$ be local orientations of $\R^N$ and $\R^n$ such that $\omega_f = (\beta_V,\beta_M) = (\beta_M \wedge \beta_E,\beta_M)$ and $\omega_g = (\beta_M \wedge \beta_F,\beta_M)$.
	To compute $\omega_{f \times_M g}$, we identify $V$ and $M \times \R^N$ as in the preceding lemma, so the Quillen normal bundle is trivial and positive.
	The pullback is $W \times \R^N \cong M \times \R^n \times \R^N$.
	And by \cref{D: pullback coorient}, taking $\beta_P = \beta_W \wedge \beta_E$, we have $\omega_{f \times_M g} = (\beta_W \wedge \beta_E,\beta_M) = (\beta_M \wedge \beta_F \wedge \beta_E,\beta_M)$.
	Analogously, $\omega_{g \times_M f} = (\beta_V \wedge \beta_F,\beta_M) = (\beta_M \wedge \beta_E \wedge \beta_F,\beta_M)$.
	The lemma follows as $N = v-m$ and $n = w-m$.
\end{proof}

\begin{corollary}\label{C: if full}
	If there exists $x$ and $y$ in the respective interiors of $V$ and $W$ such that $f(x) = g(y)$ and if $f$ and $g$ are of full rank at $x$ and $y$ then \cref{P: graded comm} holds for the connected component containing $(x,y) \in V \times_M W$.
\end{corollary}

\begin{proof}
	If $f \colon V \to M$ is of maximal rank at $x \in V$, it is an immersion or submersion at that point, and similarly for $g$.
	Thus the corollary follows directly from \cref{L: im/im,L: im/sub,L: sub/sub}.
	Not that while the statement of \cref{L: im/sub} assumes $f$ is the submersion and $g$ the immersion, we obtain the opposite case by reversing the roles of $g$ and $f$ in the statement of \cref{P: graded comm}.
\end{proof}

\begin{comment}
	If they are both immersions, apply Lemma \ref{L: im/im}.
	If $f$ is a submersion, it is also a submersion in a neighborhood of $U$ and so it is a submersion on a neighborhood $U$ of $x$.
	Let $A$ be the intersection of $U$ with the interior of $V$.
	By the transversality of $f$ and $g$, if $g(y) = f(x)$, there must be points in a neighborhood of $B$ of $y$ in $W$ that map to $f(A)$, and as a map is a submersion or immersion on an open set, there is a $y'$ in the interior of $B$ that maps to $f(A)$.
	Taking $x'$ in $f^{-1}(y')$, if $g$ is an immersion we apply Lemma \ref{L: im/sub} to $x', y'$, and if $g$ is a submersion we apply Lemma \ref{L: sub/sub}.
	If $f$ is an immersion and $g$ a submersion, we reverse the roles in the argument.
\end{comment}

We now show that for arbitrary transverse intersecting $f$ and $g$ there are always homotopies that maintain these properties while taking $f$ and $g$ each to maps of full rank at the intersection point.

\begin{lemma}\label{L: make full}
	Let $f \colon V \to M$ and $g \colon W \to M$ be transverse proper maps from manifolds with corners to a manifold without boundary.
	Suppose $x \in V$ and $y \in W$ such that $f(x) = g(y)$.
	Then there is a smooth homotopy $F \colon V \times I \to M$ such that $F(-,0) = f$, $F(x,t) = f(x)$ for all $t \in I$, $F(-,t)$ is transverse to $g$ for all $t \in I$, and $DF_{(x,1)}$ has maximal rank at $x$.
\end{lemma}

\begin{proof}
	If $Df$ already has maximal rank at $x$ we can take $F(-,t) = f(-)$.
	So suppose $Df$ does not have maximal rank at $x$.
	We will construct a homotopy fixed outside a small neighborhood of $x$, so we work in local charts, identifying neighborhoods of $x$ and $f(x)$ with $\R^v$ and $\R^m$ so that $x$ and $f(x)$ are at the respective origins.
	We may also choose the charts so that $Df_x(e_i) = e_i$ for $1\leq i\leq k$ for some $k<\dim(V) = v$ and $Df_x(e_i) = 0$ for $i>k$.
	Let $\eta:\R^v \to \R$ be a smooth function that is $0$ outside a compact neighborhood of the origin and $1$ on a neighborhood of the origin.
	Let $z = (z_1,\ldots, z_v)$ be the coordinates of $\R^v$, and let
	$$H(z_1,\ldots, z_v,t) = f(z)+t\eta(z)\left(\sum_{i = k+1}^{\min(v,m)} z_i\right).$$
	Then $H(z,t) = f(z)$ outside a compact neighborhood of the origin in $\R^v$, and so $H$ extends to a homotopy defined on all of $V \times I$.
	Furthermore, $H(0,t) = f(0)$, i.e.\ $H(x,t) = f(x)$.
	Also,
	$DH_{(0,t)}e_i = e_i$ for $i\leq k$ and $te_i$ for $k<i\leq \min(v,m)$.
	Thus $DH(-,t)$ has full rank at $x$ for all $t>0$.
	Finally, as $f$ and $g$ are proper and transverse, there is an $\epsilon$ so that for $0\leq t\leq \epsilon$, $H(-,t)$ and $g$ will be transverse by stability of transversality\footnote{A proof is given within our proof of \cref{T: transverse complex}, below.}.
	Now define $F(-,t) = H(-,\epsilon t)$.
\end{proof}

\begin{proof}[Proof of \cref{P: graded comm}]
	As noted above, it suffices to verify the claim at one point of each connected component of $P$.
	As $f$ and $g$ are transverse, if a component $P_0$ of $V \times_M W$ is nonempty, by \cref{pullback} we can find $x \in V$, $y \in W$, each point in the interior, so that $(x,y) \in P_0$.
	By \cref{L: make full}, we can first perform a homotopy $F \colon V \times I \to M$ of $f$ to $f' \colon V \to M$ and then a homotopy $G \colon W \times I \to M$ of $g$ to $g' \colon W \to M$ so that $f'$ and $g'$ have full rank at $x$ and $y$ respectively.

	First consider $F$.
	As $f$ is co-oriented, $F$ is co-orientable by \cref{L: co-orientable homotopies} and can be co-oriented by \cref{D: homotopy co-orientation}.
	So $(V \times I) \times_M W$ can be co-oriented, and by \cref{leibniz} two of its boundary components will be $f \times_M g: (V \times 0) \times_M W = V \times_M W \to M$ and
	$f' \times_M g:(V \times 1) \times_M W = V \times_M W \to M$, topologically, with the co-orientations given by composing the boundary co-orientations with the co-orientation of $F \times_M g$.
	Similarly, two of the boundary components of $W \times_M (V \times I)$ will be $g \times_M f: W \times_M (V \times 0) = W \times_M V \to M$ and
	$g \times_M f' \colon W \times_M (V \times 1) = W \times_M V \to M$ topologically, with the co-orientations given by $(-1)^{m-w}$ times the compositions of the boundary co-orientations with the co-orientation of $g \times_M F$.
	But now the point is that $(V \times I) \times_M W$ and $W \times_M (V \times I)$ are diffeomorphic with compatible maps to $M$, and so their co-orientations either agree or disagree, and correspondingly the co-orientations of the corresponding boundary components will agree or disagree.
	In other words,
	identifying these pullback as in Diagram \eqref{D: comm triangle}, we see that regardless of the actual co-orientations of $F \times_M g$ and $g \times_M F$ we have $\omega_{f \times_M g} = (-1)^{(m-v)(m-w)}\omega_{g \times_M f}$ if and only if $\omega_{f' \times_M g} = (-1)^{(m-v)(m-w)}\omega_{g \times_M f'}$.
	Analogously using $G$, we have $\omega_{f' \times_M g} = (-1)^{(m-v)(m-w)}\omega_{g \times_M f'}$ if and only if $\omega_{f' \times_M g'} = (-1)^{(m-v)(m-w)}\omega_{g' \times_M f'}$.
	But this last equality is true by \cref{C: if full}.
	\qedhere
\end{proof}

\begin{comment}
	OLD PROOF
	As noted, it suffices to verify the claim at one point of each connected component of $P$.
	As $f$ and $g$ are transverse, if a component $P_0$ of $V \times_M W$ is nonempty, by Theorem \ref{pullback} we can find $x \in V$, $y \in W$, each point in the interior, so that $(x,y) \in P_0$.
	By Lemma \ref{L: make full}, we can first perform a homotopy $F \times I \colon V \times I \to M$ of $f$ to $f' \colon V \to M$ and then a homotopy of $G \colon W \times I \to M$ of $g$ to $g' \colon W \to M$ so that $f'$ and $g'$ have full rank at $x$ and $y$ respectively.
	By the construction of Lemma \ref{L: make full}, these properties will continue to hold if we take the homotopies constructed there to be of arbitrarily short duration, and we can assume the homotopies are constant outside of compact neighborhoods of $x$ and $y$ in the interiors of $V$ and $W$.
	Considering $F$ first, by doing so, we can assure that, as spaces but ignoring co-orientations, the pullback $(V \times I) \times_M W$ will have the form of a cylinder $P \times I$ with $F \times_M g$ being a homotopy from $f \times_M g$ to $f' \times_M g$.
	As $f \times_M g$ is co-orientable, so is this homotopy, and, now considering co-orientations, there is a co-orientation for this homotopy making it a co-oriented homotopy from $f \times_M g$ to $f' \times_M g$ (recall Definition \ref{D: co-oriented homotopy} and Section\label{S: co-oriented homotopy} in general).
	Similarly, $g \times_M F$ can be co-oriented as a co-oriented homotopy from $g \times_M f$ to $g \times_M f'$.
	Identifying these pullback as in Diagram \eqref{D: comm triangle}, we see that $\omega_{f \times_M g} = (-1)^{(m-v)(m-w)}\omega_{g \times_M f}$ if and only if $\omega_{f' \times_M g} = (-1)^{(m-v)(m-w)}\omega_{g \times_M f'}$.
	Analogously, this holds if and only if $\omega_{f' \times_M g'} = (-1)^{(m-v)(m-w)}\omega_{g' \times_M f'}$, which holds by Corollary \ref{C: if full}.
\end{comment}

\begin{comment}
	\begin{corollary}
		Let $f \colon V \to M$ and $g \colon W \to M$ be transverse proper maps from manifolds with corners to a manifold without boundary.
		Suppose $x \in V$ and $y \in W$ such that $f(x) = g(y)$.
		Then there are smooth homotopies $F \colon V \times I \to M$ and $G \colon W \times I \to M$ such that $F(-,0) = f$, $G(-,0) = g$, $F(x,t) = f(x) = g(y) = G(y,t)$ for all $t \in I$, $DF(-,1)$ and $DG(0,1)$ have maximal rank at $x$ and $y$ respectively, and $F(-,t)$ is transverse to $G(-,t)$ for all $t \in I$.
	\end{corollary}
	\begin{proof}
		We can apply Lemma \ref{L: make full} twice in succession with time rescalings, once with a homotopy of $f$ that holds $g$ fixed for $t\in[0,1/2]$ and then with a homotopy of $g$.
	\end{proof}

\end{comment}

\subsubsection{Codimension $0$ and $1$ pullbacks}\label{S: codim 0 and 1 co-or}
The results in this section should be compared with, and justify, the discussion and choices in \cref{E: splitting example 1,E: manifold decomposition}.
They will be useful when working with the creasing construction for geometric cochains.

\begin{proposition}\label{P: codim 0 pullback}
	Let $V$ be an embedded codimension $0$ submanifold with corners in the manifold without boundary $M$, and let $f \colon V \to M$ be the embedding, co-oriented by $(\beta_V,f_*\beta_V)$ at each point of $V$.
	Let $W$ be a manifold with corners and suppose $g \colon W \to M$ is transverse to $f$.
	Then the co-oriented pullback $f^* \colon V \times_M W \to W$ is the inclusion of the codimension $0$ manifold with corners $g^{-1}(V) \into W$, co-oriented by the normal co-orientation of the immersion, which in this case is $(\beta_P,\beta_P)$, identifying the tangent spaces of $V \times_M W$ and $W$.
	Consequently, if $g$ is co-oriented the co-oriented fiber product $V \times_M W \to M$ is just the restriction of the co-oriented map $g$ to the embedded codimension $0$ submanifold with corners $g^{-1}(V)$.
\end{proposition}
\begin{proof}
	It is clear topologically that the pullback is $g^{-1}(V)$, and it must be a codimension $0$ manifold with corners of $W$ by Joyce \cite{Joy12}.
	So we consider co-orientations.
	As $f$ is an embedding we may choose $N = 0$ and $e = f$ in \cref{D: pullback coorient}.
	Identifying $\beta_V$ with $\beta_M$ via the embedding, the Quillen normal bundle of $V$ in $M$ is the positively oriented $\R^0$-bundle.
	So then the definition says that the pullback co-orientation is $(\beta_P,\beta_W)$ when $\beta_P = \beta_W$ up to positive scalar.
\end{proof}

\begin{corollary}\label{C: cup with identity}
	Let $f \colon V \to M$ be a co-oriented map from a manifold with corners to a manifold without boundary, and let $\id_M \colon M \to M$ be the identity.
	Then both co-oriented fiber products $V \times_M M \to M$ and $M \times_M V \to M$ are again $f \colon V \to M$.
\end{corollary}
\begin{proof}
	The first case follows immediately from the preceding lemma, and the other then follows from the graded commutativity of the fiber product, \cref{P: graded comm}.
\end{proof}

\begin{example}
	Let $\varphi \colon M \to \R$ be a smooth function from a manifold without boundary to $\R$, and let $M^+ = \varphi^{-1}([0,\infty)) = [0,\infty)\times_\R M$, $M^- = \varphi^{-1}((-\infty,0]) = (-\infty,0]\times_\R M$, and $M^0 = \varphi^{-1}(0) = 0\times_\R M$.
	Then $M^\pm$ are embedded codimension $0$ manifolds with corners in $M$ and $\bd M^\pm = M^0$.
	Suppose the embeddings $M^\pm \into M$ are given the tautological co-orientations of \cref{D: tautological co-orientation}, and let $g \colon W \to M$ be transverse to $M^\pm$, which in this case is equivalent to being transverse to $M^0$.
	Then let $W^\pm \defeq g^{-1}(M^\pm) = M^\pm \times_M W$, and the pullback co-orientations of $W^\pm \into W$ given by \cref{P: codim 0 pullback} are again just the tautological co-orientations of the codimension $0$ embeddings.
	The compositions $W^\pm \into W \to M$ are then the fiber products, and the co-orientations of \cref{P: codim 0 pullback} agree with those described in \cref{E: manifold decomposition}.
\end{example}

\begin{proposition}\label{P: codim 1 co-orient}
	Suppose $V \subset M$ is a codimension $1$ submanifold with corners with oriented normal bundle $\nu$ in the manifold without boundary $M$.
	Suppose the embedding $f \colon V \into M$ is co-oriented by $(\beta_V,\beta_V \wedge \beta_\nu)$.
	Let $W$ be a manifold without corners and suppose $g \colon W \to M$ is transverse to $f$ so that $g^{-1}(V) = W^0$ is a codimension $1$ submanifold with corners with oriented pullback normal vector bundle $\nu_0$.
	Then the co-oriented pullback $f^* \colon P = V \times_M W \to W$ is the embedding $W^0 \into W$, co-oriented by $(\beta_P,\beta_P \wedge \beta_{\nu_0})$.
\end{proposition}

\begin{proof}
	It is again standard that the pullback is $g^{-1}(V) = W^0$ embedded in $W$ with codimension $1$.
	For the co-orientation, if we take $N = 0$ and $e = f$ in \cref{D: pullback coorient}, then $\nu$ is the same as in that definition and $\nu_0$ is simply the pullback that we just again called $\nu$ there.
	Then by definition the pullback co-orientation is $(\beta_P,\beta_W)$ if and only if $\beta_P \wedge \beta_{\nu_0} = \beta_W$ up to positive scalar, as claimed.
\end{proof}

\begin{example}\label{E: codim 1 pullbacks}
	Continuing to assume $\varphi \colon M \to \R$, let $V = M^0 = \varphi^{-1}(0)$ with normal bundle oriented by the pullback of the standard (positive-direction) orientation of the normal bundle of $0 \in \R$, which determines a co-orientation of the embedding $M^0 \to M$.
	Then by \cref{P: codim 1 co-orient} the pullback co-orientation of $W^0 = M^0 \times_M W \into W$ agrees with the $\varphi$-induced co-orientation of $W^0$ defined in \cref{E: manifold decomposition}.
	We can also confirm now using the Leibniz rule that $$\bd(W^-) = \bd(M^- \times_M W) = (\bd M^-) \times_M W \bigsqcup M^- \times \bd W = M^0 \times_M W \bigsqcup M^- \times_M \bd W = (-W^0) \bigsqcup (\bd W)^-,$$
	using that the orientation of the normal bundle to $M^0$ is outward pointing from $M^-$ and so disagrees with the inward-pointing normal used to co-orient the boundary inclusion.
	Analogously, $$\bd (W^+) = W^0 \bigsqcup (\bd W)^+,$$
	using that the orientation of the normal bundle to $M^0$ is inward pointing for $M^+$.
\end{example}

\begin{comment}
	\begin{proof}
		Again it is clear that $f \times_M g = g|_{W^0}$ as maps.
		Suppose given structural co-orientations.
		As $f$ is an embedding, we have $K_f = 0$, while $V^\perp$ is spanned by $\nu$; note that as $\nu_0$ maps to $\nu$, we write simply $\nu$ in the local decomposition of $TW$.
		The structural co-orientation of $f$ is $(\beta_{W^\perp} \wedge \beta_I, \beta_{W^\perp} \wedge \beta_I \wedge \beta_\nu)$, which agrees with the assumed co-orientation for $f$.
		So the co-orientation of the pullback will be the structural orientation or not according to whether the structural co-orientation of $g$ agrees with the given co-orientation of $g$ or not.
		The structural co-orientation of $g$ is $(\beta_{K_g} \wedge \beta_I \wedge \beta_\nu, \beta_{W^\perp} \wedge \beta_I \wedge \beta_\nu)$, while the structural co-orientation of the pullback is $(\beta_{K_g} \wedge \beta_I, \beta_{W^\perp} \wedge \beta_I \wedge \beta_\nu)$.
		If the given co-orientation for $g$ agrees with the structural orientation, then the claimed co-orientation for $g|_{W^0}$ is the composition of the structural co-orientation for $g$ with $(\beta_{W^0},\beta_{W^0} \wedge \beta_{\nu})$.
		In this last expression we are free to choose any $\beta_{W^0}$ we like, so we can let $\beta_{W^0} = \beta_{K_g} \wedge \beta_I$.
		Then the claimed composite co-orientation is $(\beta_{K_g} \wedge \beta_I, \beta_{W^\perp} \wedge \beta_I \wedge \beta_\nu)$, which agrees with the pullback co-orientation as claimed.
		If the given co-orientation of $g$ disagrees with the structural co-orientation, this changes the sign of both the pullback co-orientation and of the representative of the co-orientation of $g$ used in our composite but not the sign of $(\beta_{W^0},\beta_{W^0} \wedge \beta_{\nu})$.
		So again the pullback co-orientation agrees with the claimed composite.
	\end{proof}
\end{comment}

\begin{remark}
	The formulas in this example are mnemonically convenient, with either all terms involving the symbol $-$ or all terms involving the symbol $+$, though implicitly in the case of the $W^0$ in the formula for $\bd (W^+)$.
\end{remark}

We can now prove the claim from the end of \cref{E: manifold decomposition}.
We express the following corollary using Notation \ref{N: implicit notation}.

\begin{corollary}\label{C: co-orient W0}
	Suppose the hypotheses and notation of \cref{P: codim 1 co-orient} and suppose $V$ is without boundary.
	Then $\bd (W^0) = -(\bd W)^0$ as co-oriented maps to $M$, with $W^0$ and $(\bd W)^0 = (gi_{\bd W})^{-1}(V)$ co-oriented as in \cref{P: codim 1 co-orient} as the pullbacks $V \times_M W \to W$ and $V \times_M \bd W \to W$.
\end{corollary}
\begin{proof}
	By \cref{leibniz}, we have that $$\bd (W^0) = \bd (V \times_M W) = (-1)^{\dim(M)-\dim(V)}V \times_M \bd W = -V \times_M \bd W = -(\bd W)^0$$
	as spaces mapping $M$.
\end{proof}

\subsection{Exterior products}\label{S: exterior products}
In this section we consider products of maps that will eventually become the exterior products in geometric homology and cohomology.
While fiber products, which will eventually be used to define cup and intersection products, require special transversality conditions in order to be defined, these exterior products are always fully defined.
We consider first products of oriented manifolds and then products of co-oriented maps of manifolds.

\subsubsection{Exterior products of maps of oriented manifolds}
If we begin with maps $f \colon V \to M$ and $g \colon W \to N$, then we have the usual product map $f \times g \colon V \times W \to M \times N$ that takes $(v,w) \in V \times W$ to $(f(v),g(w)) \in M \times N$.
If $V$ and $W$ are oriented, say by $\beta_V$ and $\beta_W$, then $V \times W$ possesses the standard product orientation via concatenation of oriented bases that can be described at each point by $\beta_V \wedge \beta_W$.
More technically, $T(M \times N) = \pi_M^*(TM) \oplus \pi_N^*(TN)$ with $\pi_M,\pi_N$ being the projections, but we elide the pullbacks in our notation for local orientations.
By \cref{P: oriented fiber product basic properties}, this convention is consistent with the observation that the exterior product is isomorphic to the fiber product of the unique maps $V,W \to pt$.

We observe the following interplay between fibered and exterior products of maps of oriented manifolds more generally.

\begin{proposition}\label{P: oriented interchange}
	Suppose $f \colon V \to M$ and $g \colon W \to M$ are transverse maps of oriented manifolds with corners to an oriented manifold without boundary and similarly for $h \colon X \to N$ and $k:Y \to N$.
	Then

	$$(V \times X)\times_{M \times N} (W \times Y) = (-1)^{(m-w)(n-y)}(V \times_M W) \times (X\times_NY)$$
	as oriented manifolds.
\end{proposition}
\begin{proof}
	We first note that the transversality assumptions ensure also that $f \times h$ will be transverse to $g \times k$.
	It is straightforward to verify that these are diffeomorphic spaces, so we focus on the orientations.
	For simplicity, let us write
	$P = (V \times X)\times_{M \times N} (W \times Y)$ and $P' = (V \times_M W) \times (X\times_NY)$.
	We then write orientation symbolically as $\beta_P$, etc.
	By definition, and omitting the pullbacks from the notation, $P$ is oriented so that
	$$\beta_P \wedge \beta_{M \times N} = (-1)^{(w+y)(m+n)}\beta_{V \times X} \wedge \beta_{W \times Y},$$
	or, as (non-fiber) products are oriented by concatenation, $$\beta_P \wedge \beta_M\wedge\beta_N = (-1)^{(w+y)(m+n)}\beta_V \wedge \beta_X \wedge \beta_W \wedge \beta_Y.$$
	Here we identify $T(M \times N)$ as a summand by splitting $D(f \times h)-D(g \times k)$.
	Similarly, we have
	\begin{align*}
		\beta_{V \times_M W} \wedge \beta_M & = (-1)^{wm}\beta_V \wedge \beta_W\\
		\beta_{X\times_NY} \wedge \beta_N & = (-1)^{yn}\beta_X \wedge \beta_Y,
	\end{align*}
	using the splittings of $Df-Dg$ and $Dh-Dk$.
	In particular, the signs of $Df$, $Dg$, $Dh$, and $Dk$ in the splitting formulas are consistent in computing the orientations for $P$ and $P'$.
	Wedging the two sides, we have
	$$\beta_{V \times_M W} \wedge \beta_M\wedge\beta_{X\times_NY} \wedge \beta_N = (-1)^{wm+ny}\beta_V \wedge \beta_W\beta_X \wedge \beta_Y.$$
	As $\beta_{P'} = \beta_{V \times_M W}\wedge\beta_{X\times_NY},$ we have
	\begin{align*}
		\beta_{P'} \wedge \beta_M \wedge \beta_N& = (-1)^{m(x+y-n)}\beta_{V \times_M W} \wedge \beta_M\wedge\beta_{X\times_NY} \wedge \beta_N\\
		& = (-1)^{m(x+y-n)+wm+ny}\beta_V \wedge \beta_W \wedge \beta_X \wedge \beta_Y\\
		& = (-1)^{m(x+y-n)+wm+ny+xw}\beta_V \wedge \beta_X \wedge \beta_W \wedge \beta_Y
	\end{align*}
	So $\beta_P$ differs from $\beta_{P'}$ by $-1$ to the power
	$$m(x+y-n)+wm+ny+xw-((w+y)(m+n)).$$
	An elementary computation now shows that this is $(m-w)(n-x)$ as desired.
\end{proof}

\subsubsection{Exterior products of co-oriented maps and applications to fiber products}

Next we define and study an exterior product for co-orientable maps.
In the next subsection, we will see that such products are intimately related to fiber products, just as cup products and exterior products are related in cohomology.
In fact, this relationship will allow us to easily proof some properties about fiber products that we have delayed.

Before getting to co-orientations, we first show that a product of proper maps is proper.

\begin{lemma}\label{L: proper product}
	If $f \colon V \to M$ and $g \colon W \to N$ are proper maps of spaces then the product map $f \times g \colon V \times W \to M \times N$ is proper.
\end{lemma}
\begin{proof}
	Let $\pi_M,\pi_N$ be the projections of $M \times N$ to $M$ and $N$, and similarly for $\pi_V, \pi_W$.
	Let $K$ be a compact subspace of $M \times N$.
	Then $\pi_M(K)$ and $\pi_N(K)$ are compact, and hence so is $\pi_M(K) \times \pi_N(K) \subset M \times N$, and this set contains $K$.
	So $$(f \times g)^{-1}(K) \subset (f \times g)^{-1}(\pi_M(K) \times \pi_N(K)) = f^{-1}(\pi_M(K)) \times g^{-1}(\pi_N(K)).$$ But now $f^{-1}(\pi_M(K))$ and $g^{-1}(\pi_N(K))$ are compact as $f$ and $g$ are proper and so $(f \times g)^{-1}(K)$ is a closed subset of a compact set, hence compact.
\end{proof}

\begin{lemma}
	If $f \colon V \to M$ and $g \colon W \to N$ are co-orientable maps of manifolds with corners then the product map $f \times g \colon V \times W \to M \times N$ is co-orientable.
\end{lemma}
\begin{proof}
	We recall that, by definition, a co-orientation of $f$ is equivalent to a choice of isomorphism between the orientation cover $\Or(TV)$ and the pullback $f^*\Or(TM)$ of the orientation cover $\Or(TM)$, and similarly for $g$.

	If we let $\pi_V, \pi_W$ denote the projections of $V \times W$ to $V$ and $W$, then $T(V \times W) \cong \pi_V^*(TV) \oplus \pi_W^*(TW)$, and so $$\Or(T(V \times W)) \cong \Or(\pi_V^*(TV))\otimes\Or(\pi_W^*(TW)) \cong \pi_V^*\Or(TV)\otimes\pi_W^*\Or(TW).$$ Similarly
	\begin{multline*}(f \times g)^*T(M \times N) \cong (f \times g)^*(\pi_M^*(TM) \oplus \pi_N^*(TN))\\
		 \cong (f \times g)^*\pi_M^*(TM) \oplus (f \times g)^*\pi_N^*(TN)) \cong \pi_V^*f^*(TM) \oplus \pi_W^*g^*(TN),
	\end{multline*}
	using that $\pi_M(f \times g) = f\pi_V \colon V \times W \to M$ and $\pi_N(f \times g) = g\pi_W \colon V \times W \to N$.
	So
	\begin{multline*}
		(f \times g)^*\Or(T(M \times N)) \cong \Or((f \times g)^*T(M \times N))\\ \cong \Or(\pi_V^*f^*TM) \otimes \Or(\pi_W^*g^*TN) \cong \pi_V^*f^*\Or(TM) \otimes \pi_W^*g^*\Or(TN).
	\end{multline*}
	Thus if $\Or(TV) \cong f^*\Or(TM)$ and $\Or(TW) \cong g^*\Or(TN)$, we can construct isomorphisms $\Or(T(V \times W)) \cong (f \times g)^*\Or(T(M \times N))$.
\end{proof}

\begin{definition}\label{D: co-oriented exterior}
	If $f \colon V \to M$ and $g \colon W \to N$ are co-oriented maps of manifolds with corners with co-orientations given by isomorphisms $\phi:\Or(TV) \to f^*\Or(TM)$ and $\psi:\Or(TW) \to g^*\Or(TN)$, we define the \textbf{product co-orientation} of $f \times g \colon V \times W \to M \times N$ by the isomorphism $(-1)^{(m-v)w}\pi_V^*\phi \otimes \pi_W^*\psi$.
	In particular, if at $x \in V$ the co-orientation of $f$ is given locally by $(\beta_V,\beta_M)$ and at $y \in W$ the co-orientation of $g$ is given locally by $(\beta_W,\beta_N)$, then up to standard abuses of notation the product co-orientation is locally represented at $(x,y)$ by $$(-1)^{(m-v)w}(\beta_V \wedge \beta_W,\beta_M \wedge \beta_N).$$

	Following our standard conventions, we often denote this product manifold over $M \times N$ by simply $V \times W$.
\end{definition}

\begin{remark}
	The sign in the definition is not at first obvious, though it will be justified in the following lemmas.
	One way to think of it is as follows: If we we take $V$ and $W$ as immersed submanifolds co-oriented by orienting their normal bundles as in \cref{normal co-or}, then $V \times W$ is also immersed, and at an image point we have $T(M \times N) \cong TV \oplus \nu V \oplus TW \oplus \nu W$.
	The sign $(-1)^{(m-v)w}$ is the sign needed in the orientation to permute this to $TV \oplus TW \oplus \nu V \oplus \nu W$ so that we can properly utilize the normal co-orientation for $\nu(V \times W) \cong \nu V \oplus \nu W$.
	While this argument is essentially heuristic, it is borne out in the computations below.
\end{remark}

\begin{example}\label{E: sphere product}
	Let $S^p$ and $S^q$ be oriented spheres with $p,q>0$.
	Let $V = W = pt$, and let $f \colon V \to S^p$ and $g \colon W \to S^q$ be embeddings to points $x \in S^p$, $y \in S^q$.
	Let $f$ be co-oriented by $(1,\beta_{S^p})$; in other words $V$ is normally co-oriented by orientation of its normal bundle that agrees with the orientation of $S^p$.
	Let $g$ be co-oriented similarly.
	Then $V \times W$ is represented by the embedding of the point to $(x,y) \in S^p \times S^q$ with normal bundle oriented consistently with the product orientation of $S^p \times S^q$.
	There is no extra sign in this case as $\dim(W) = 0$.
\end{example}

\begin{proposition}\label{P: co-oriented exterior unit}
	Let $f \colon V \to M$ be a co-oriented map of manifolds with corners, and let $g:pt \to pt$ be the unique map with the canonical co-orientation.
	Then $f \times g \colon V \times pt \to M \times pt$ and $g \times f:pt \times V \to pt \times M$ are each isomorphic as co-oriented maps of manifolds with corners to $f \colon V \to M$.
\end{proposition}
\begin{proof}
	This is obvious ignoring co-orientations.
	Considering co-orientations, if $f$ is co-oriented at a point by $(\beta_V,\beta_M)$, then the co-orientation of $f \times g$ is simply $(\beta_V \wedge \R,\beta_M \wedge \R) = (\beta_V,\beta_M)$, noting that
	the sign $(-1)^{(m-v)\cdot 0} = 1$ in this case.
	The case $g \times f$ is similar, though due to the transposition the sign is now $(-1)^{(0-0)v}$, which is again $1$.
\end{proof}

\begin{proposition}\label{P: boundary of exterior product}
	Let $f \colon V \to M$ and $g \colon W \to N$ be co-oriented maps of manifolds with corners and suppose $f \times g \colon V \times W \to M \times N$ is given the product co-orientation.
	Then the boundary co-orientation of $V \times W$ as co-oriented maps to $M \times N$ is $$\bd(V \times W) = (\bd V) \times W\bigsqcup (-1)^{m-v}V \times \bd W.$$
\end{proposition}
\begin{proof}
	We know that this expression is an identity ignoring co-orientations, so we must establish the agreement of the co-orientations for each component.
	As usual, it suffices to consider points in the top dimensional strata of $\bd(V \times W)$.
	In what follows, let $(\beta_V,\beta_M)$, $(\beta_W,\beta_N)$, and $(-1)^{(m-v)w}(\beta_V \wedge \beta_W,\beta_M \wedge \beta_N)$ denote the co-orientations of $V$, $W$, and $V \times W$ at the point under consideration.

	Let $\nu$ denote an inward pointing normal to $V \times W$ at such a point.
	Then the inclusion $\bd(V \times W) \to V \times W$ is co-oriented at that point by $(\beta_{\bd(V \times W)},\beta_{\bd(V \times W)} \wedge \beta_\nu)$ for any $\beta_{\bd(V \times W)}$.
	If we choose $\beta_{\bd(V \times W)}$ so that $(\beta_{\bd(V \times W)}\wedge\beta_\nu,\beta_M \wedge \beta_N)$ represents the co-orientation of $V \times W \to M \times N$, then from the definition of the boundary co-orientation, the boundary $\bd(V \times W) \to M \times N$ is co-oriented by $(\beta_{\bd(V \times W)},\beta_M \wedge \beta_N)$.
	We fix such a choice in what follows.

	Now suppose our point is more specifically in the top-dimensional stratum of $(\bd V) \times W$.
	If we choose $\beta_{\bd V}$ so that $\beta_{\bd V} \wedge \beta_\nu = \beta_V$, then $\bd V \to M$ is co-oriented by $(\beta_{\bd V},\beta_M)$ and so $(\bd V) \times W$ is co-oriented by $(-1)^{(m-v+1)w}(\beta_{\bd V} \wedge \beta_W,\beta_M \wedge \beta_N)$.
	On the other hand, the co-orientation of $V \times W$ can then be written $(-1)^{(m-v)w}(\beta_{\bd V} \wedge \beta_\nu \wedge \beta_W,\beta_M \wedge \beta_N) = (-1)^{(m-v)w+w}(\beta_{\bd V} \wedge \beta_W \wedge \beta_\nu,\beta_M \wedge \beta_N)$, so the boundary co-orientation of $\bd(V \times W)$ is $(-1)^{(m-v)w+w}(\beta_{\bd V} \wedge \beta_W,\beta_M \wedge \beta_N)$, which agrees with our co-orientation for $(\bd V) \times W$.

	Next consider a point in the top-dimensional stratum of $V \times \bd W$.
	If we choose $\beta_{\bd W}$ so that $\beta_{\bd W} \wedge \beta_\nu = \beta_W$ then we have $\bd W$ co-oriented by $(\beta_{\bd W},\beta_N)$ and so $V \times \bd W$ is co-oriented by $(-1)^{(m-v)(w-1)}(\beta_{V} \wedge \beta_{\bd W},\beta_M \wedge \beta_N)$.
	On the other hand, the co-orientation of $V \times W$ can now be written $(-1)^{(m-v)w}(\beta_{V} \wedge \beta_{\bd W} \wedge \beta_\nu,\beta_M \wedge \beta_N)$, so the boundary co-orientation of $\bd(V \times W)$ is $(-1)^{(m-v)w}(\beta_{V} \wedge \beta_{\bd W},\beta_{M} \wedge \beta_N)$, which differs from that of $V \times \bd W$ by a factor of $(-1)^{m-v}$.
\end{proof}

\begin{proposition}\label{P: exterior associativity}
	Let $f \colon V \to M$, $g \colon W \to N$, and $h \colon X \to Q$ be co-oriented maps of manifolds with corners.
	Then the co-orientations of $(V \times W) \times X \to M \times N \times Q$ and $V \times (W \times X) \to M \times N \times Q$ agree.
	In other words, forming co-oriented products is associative.
\end{proposition}
\begin{proof}
	Let $v = \dim(V)$, etc.
	If $f,g,h$ are co-oriented by $(\beta_V,\beta_M)$, etc., then both products are co-oriented up to sign by $(\beta_V\wedge\beta_W \wedge \beta_X,\beta_M\wedge\beta_N \wedge \beta_P)$.
	In forming $(V \times W) \times X$ we first have the sign $(-1)^{(m-v)w}$ from $V \times W$, then taking the product with $X$ on the right multiplies by $(-1)^{(m+n-v-w)x}$.
	So the total sign is $(-1)^{(m-v)w+(m+n-v-w)x}$.
	Alternatively, forming $W \times X$ has the sign $(-1)^{(n-w)x}$ and then taking the product with $V$ on the left contributes $(-1)^{(m-v)(w+x)}$.
	So the total sign is $(-1)^{(n-w)x+(m-v)(w+x)}$.
	One readily verifies that these signs agree.
\end{proof}

\red{Dev and Anibal, please check the following arguments carefully as I'm not 100\% confident in it.
	It gives the ``right'' answer but I'm a little uncomfortable divorcing the order of the orientation terms from the order of the manifold terms.
	Of course this happens all the time - even if we think of $\R^2$ as $\R_x \oplus \R_y$ we can still think about the two-form $y \wedge x$, but I'm still a little nervous about maybe having missed a sign somewhere.
	I'm also a little nervous about my trick of taking $a$ and $b$ to be even so that they won't contribute, but the earlier work says that this should be allowable.
	Presumably if I didn't do this there would be a bunch of extra signs that miraculous cancel out, but I'm not so excited about trying that out in detail to see.}

The following lemma provides a nice description of the Quillen co-orientation of product of co-oriented maps.
Among other things, it will help us to next demonstrate a commutativity property for exterior products of co-oriented maps.

\begin{lemma}\label{L: Quillen product co-orientation}
	Let $f \colon V \to M$ and $g \colon W \to N$ be co-oriented maps from manifolds with corners to manifolds without boundary.
	Consider Quillen co-orientations representing $f$ and $g$ via embeddings $e_V \colon V \into M \times \R^a$ and $e_W \colon W \into N \times \R^b \to N$ with $a$ and $b$ even.
	Denote the normal bundles of $V$ and $W$ in $M \times \R^a$ and $N \times \R^b$ by $\nu V$ and $\nu W$.
	Let $T \colon M \times \R^a \times N \times \R^b \to M \times N \times \R^{a+b}$ be the diffeomorphism that interchanges the middle two factors.
	Then
	$T(e_V \times e_W)$ gives an embedding $V \times W \to M \times N \times \R^{a+b}$ with normal bundle isomorphic to the sum of the pullbacks of $\nu V$ and $\nu W$ by the projections of $V \times W \to M \times N \times \R^a \times \R^b$ to either the first and third factor or the second and fourth factors.
	For simplicity, we simply write $\nu V \oplus \nu W$.

	Then the Quillen co-orientation of $f \times g \colon V \times W \to M \times N$ has Quillen co-orientation given by $$\beta_{\nu V \oplus \nu W} = \beta_{\nu V} \wedge \beta_{\nu W},$$
	suitably interpreting the relevant coordinates in $M \times N \times \R^a \times \R^b$.
\end{lemma}

\begin{proof}
	Let $\beta_a$ and $\beta_b$ denote the standard orientations for $\R^a$ and $\R^b$.
	By definition, $\nu V$ and $\nu W$ are oriented so that $\beta_V \wedge \beta_{\nu V} = \beta_M \wedge \beta_a$ and $\beta_W \wedge \beta_{\nu W} = \beta_N \wedge \beta_b$.

	By definition, the Quillen orientation of $\nu V \oplus \nu W$ corresponding to the product co-orientation of $V \times W$ is the orientation $\beta_{\nu V \oplus \nu W}$ such that
	$$(\beta_{V \times W}, \beta_{V \times W} \wedge \beta_{\nu V \oplus \nu W})*(\beta_{M \times N} \wedge \beta_{a+b},\beta_{M \times N}) = (-1)^{(m-v)w}(\beta_V \wedge \beta_W,\beta_M \wedge \beta_N).$$
	Taking $\beta_{M \times N} = \beta_M \wedge \beta_N$ and $\beta_{V \times W} = \beta_V \wedge \beta_W$ and noting $\beta_{a+b} = \beta_a \wedge \beta_b$, this formula becomes
	$$(\beta_V \wedge \beta_W, \beta_V \wedge \beta_W \wedge \beta_{\nu V \oplus \nu W})*(\beta_M \wedge \beta_N \wedge \beta_a \wedge \beta_b,\beta_M \wedge \beta_N) = (-1)^{(m-v)w}(\beta_V \wedge \beta_W,\beta_M \wedge \beta_N).$$
	We also have $$\beta_M \wedge \beta_N \wedge \beta_a \wedge \beta_b = \beta_M \wedge \beta_a \wedge \beta_N \wedge \beta_b,$$ as $a$ is even, so using $\beta_V \wedge \beta_{\nu V} = \beta_M \wedge \beta_a$ and $\beta_W \wedge \beta_{\nu W} = \beta_N \wedge \beta_b$, we require
	$$(\beta_V \wedge \beta_W, \beta_V \wedge \beta_W \wedge \beta_{\nu V \oplus \nu W})*(\beta_V \wedge \beta_{\nu V}\wedge\beta_W \wedge \beta_{\nu W} ,\beta_M \wedge \beta_N) = (-1)^{(m-v)w}(\beta_V \wedge \beta_W,\beta_M \wedge \beta_N).$$
	But now $$\beta_V \wedge \beta_{\nu V}\wedge\beta_W \wedge \beta_{\nu W} = (-1)^{(m-v)w} \beta_V \wedge \beta_{W}\wedge\beta_{\nu V} \wedge \beta_{\nu W},$$
	so, after all that, we see that the Quillen orientation of the normal bundle to $V \times W$ is simply $$\beta_{\nu V \oplus \nu W} = \beta_{\nu V} \wedge \beta_{\nu W},$$
	suitably interpreting the relevant coordinates in $M \times N \times \R^a \times \R^b$.
\end{proof}

\begin{proposition}\label{P: exterior commutativity}
	Let $f \colon V \to M$ and $g \colon W \to N$ be co-oriented maps from manifolds with corners to manifolds without boundary and suppose $f \times g \colon V \times W \to M \times N$ is given the product co-orientation.
	Let $\tau \colon N \times M \to M \times N$ be the diffeomorphism that interchanges coordinates.
	Then the pullback of $V \times W$ by $\tau$ is $$\tau^*(V \times W) = (-1)^{(m-v)(n-w)}W \times V,$$ where $m = \dim M$, etc.
\end{proposition}
We assume $M$ and $N$ to be without corners so that we can properly use the pullback construction, which requires transversality, in the proposition and its proof.
However, the pullback is by a diffeomorphism, so this result should extend without problem to more general settings.
\begin{proof}
	Let $(\beta_V,\beta_M)$ and $(\beta_W,\beta_N)$ be local representations of the co-orientations at some points.
	The product co-orientation of $V \times W \to M \times N$ is $(-1)^{(m-v)w}(\beta_V \wedge \beta_W,\beta_M \wedge \beta_N)$.

	As in \cref{L: Quillen product co-orientation},we consider Quillen co-orientations representing $f$ and $g$ via embeddings $e_V \colon V \into M \times \R^a$ and $e_W \colon W \into N \times \R^b \to N$ with $a$ and $b$ even.
	This is sufficient as we know that the pullback construction is independent of $a$ and $b$ for sufficiently large dimensions by \cref{L: pullback co well defined}.
	Assuming the other notation from \cref{L: Quillen product co-orientation}, the lemma tells us that $f \times g \colon V \times W \to M \times N$ has Quillen co-orientation given by $$\beta_{\nu V \oplus \nu W} = \beta_{\nu V} \wedge \beta_{\nu W},$$
	suitably interpreting the relevant coordinates in $M \times N \times \R^a \times \R^b$.

	Now using this Quillen co-orientation for $f \times g$, we pull back by the diffeomorphism $\tau \colon N \times M \to M \times N$, obtaining the composition we can write $W \times V \into N \times M \times \R^a \times \R^b \to N \times M$.
	The pulled back normal bundle is still oriented in each fiber as $\beta_{\nu V} \wedge \beta_{\nu W}$ (though of course the order of actual local coordinates have now been jumbled around).
	By definition, the pullback co-orientation is $(\beta_W \wedge \beta_V,\beta_N \wedge \beta_M)$ if and only if $$\beta_W \wedge \beta_V \wedge \beta_{\nu V} \wedge \beta_{\nu W} = \beta_N \wedge \beta M \wedge \beta_{a+b},$$
	and as $a$ and $b$ are even this last expression is equal to
	$\beta_N \wedge \beta_b \wedge \beta_M \wedge \beta_{a}.$ But by the previous choices, $\beta_V \wedge \beta_{\nu V} = \beta_M \wedge \beta_a$ and $\beta_W \wedge \beta_{\nu W} = \beta_N \wedge \beta_b$.
	So
	\begin{align*}
		\beta_N \wedge \beta_b \wedge \beta_M \wedge \beta_{a}& = \beta_W \wedge \beta_{\nu W} \wedge \beta_V \wedge \beta_{\nu V}\\
		& = (-1)^{v(n+b-w)}\beta_W \wedge \beta_V \wedge \beta_{\nu W} \wedge \beta_{\nu V}\\
		& = (-1)^{v(n+b-w)+(m+a-v)(n+b-w)}\beta_W \wedge \beta_V \wedge \beta_{\nu V} \wedge \beta_{\nu W}\\
		& = (-1)^{v(n-w)+(m-v)(n-w)}\beta_W \wedge \beta_V \wedge \beta_{\nu V} \wedge \beta_{\nu W},\\
	\end{align*}
	where again we use that $a$ and $b$ are even.

	Therefore, the pullback co-orientation is $(-1)^{v(n-w)+(m-v)(n-w)}(\beta_W \wedge \beta_V,\beta_N \wedge \beta_M)$, which is $(-1)^{(m-v)(n-w)}$ times the product co-orientation of $W \times V$, as claimed.
\end{proof}

The next result relates co-oriented products in which one map is the identity with pullbacks by projections.

\begin{proposition}\label{P: projection pullbacks}
	Let $f \colon V \to M$ be a co-oriented map from a manifolds with corners to a manifold without boundary, and let $\id_N \colon N \to N$ be the identity map of a manifold with corners with the canonical co-orientation.

	\begin{enumerate}
		\item The co-oriented pullback of $V$ by the projection $\pi_1 \colon M \times N \to M$ is $f \times \id_N \colon V \times N \to M \times N$ with its product co-orientation, i.e.\ $\pi_1^*V = V \times N$.

		\item The co-oriented pullback of $V$ by the projection $\pi_2 \colon N \times M \to M$ is $\id_N \times f \colon N \times V \to N \times M$ with its product co-orientation, i.e.\ $\pi_2^*V = N \times V$.
	\end{enumerate}
\end{proposition}
\begin{proof}
	As the projections are submersions, the required transversality conditions to ensure the existence of the pullbacks are met.
	These claims are then clear concerning maps of topological spaces, so we need only verify the co-orientations.

	As in the preceding argument, we start again with an embedding $e \colon V \into M \times \R^a$ to establish a Quillen co-orientation for $f$.
	We again may assume $a$ be even for simplicity.
	We write the co-orientation of $f$ as $(\beta_V,\beta_M)$, and we let $\nu V$ denote the normal to $e(V)$ and orient $\nu V$ so that $\beta_V \wedge \beta_{\nu V} = \beta_M \wedge \beta_a$, writing $\beta_a$ for the standard orientation of $\R^a$.

	For the second statement, the product co-orientation of $\id_N \times f \colon N \times V \to N \times M$ is $(\beta_N \wedge \beta_V,\beta_N \wedge \beta_M)$, as the domain and codomain of $\id_N$ have the same dimension.
	The pullback by the projection $N \times M \to M$ gives us the embedding/projection sequence $N \times V\xhookrightarrow{\id_N \times e} N \times M \times \R^a \to N \times M$, and the oreintations of the pullback of the normal bundle $\nu V$ by $\pi_2 \times \id_{\R^a}$ is again $\beta_{\nu V}$ at each point of $N \times e(V)$.
	So now from the definition, the pullback has the product co-orientation if and only if $\beta_N \wedge \beta_V \wedge \beta_{\nu V} = \beta_N \wedge \beta_M \wedge \beta_a$.
	But $ \beta_V \wedge \beta_{\nu V} = \beta_M \wedge \beta_a$ by assumption, so this holds.

	For the first statement, the product co-orientation of $f \times \id_N \colon V \times N \to M \times N$ is $(-1)^{(m-v)n}(\beta_V \wedge \beta_N,\beta_M \wedge \beta_N)$.
	The pullback by the projection $M \times N \to M$ gives us an embedding/projection sequence $V \times N \into M \times N \times \R^a \to M \times N$ (where the first arrow is the composition of $e \times \id_N$ with a permutation of coordinates), and the orientation of the pullback of the normal bundle $\nu V$ by $\pi_1 \times \id_{\R^a}$ is again $\beta_{\nu V}$.
	So now from the definition, the pullback has the product co-orientation if and only if $\beta_V \wedge \beta_N \wedge \beta_{\nu V} = (-1)^{(m-v)n}\beta_M \wedge \beta_N \wedge \beta_a$.
	But $ \beta_V \wedge \beta_{\nu V} = \beta_M \wedge \beta_a$ by assumption, so
	\begin{align*}
		\beta_V \wedge \beta_N \wedge \beta_{\nu V}& = (-1)^{(m+a-v)n}\beta_V \wedge \beta_{\nu V} \wedge \beta_N\\
		& = (-1)^{(m+a-v)n}\beta_M \wedge \beta_a \wedge \beta_N\\
		& = (-1)^{(m-v)n}\beta_M \wedge \beta_N \wedge \beta_a.\qedhere
	\end{align*}
\end{proof}

The next proposition shows that the exterior product construction is natural.

\begin{proposition}\label{P: natural exterior}
	Let $f \colon V \to M$ and $g \colon W \to N$ be co-oriented maps of manifolds with corners with $M$ and $N$ having no boundaries.
	Let $h \colon X \to M$ and $k:Y \to N$ be maps of manifolds with corners that are transverse to $f$ and $g$ respectively.
	Then $(h \times k)^*(V \times W) = h^*V \times k^*W$ spaces with co-oriented maps to $X \times Y$.
\end{proposition}
\begin{proof}
	It is easy to show that $h \times k$ is transverse to $f \times g$ so we focus on co-orientation.
	As in the preceding proofs, we write the Quillen orientation of the normal to the image of $V \times W \into M \times N \times \R^a \times \R^b$ as $\beta_{\nu V \oplus \nu W} = \beta_{\nu V} \wedge \beta_{\nu W}$.
	Then the pullback $P = (h \times k)^*(V \times W)$ is co-oriented by $(\beta_P,\beta_{X \times Y})$ if and only if we choose $\beta_P$ and $\beta_{X \times Y}$ so that
	$$\beta_{P} \wedge \beta_{\nu V} \wedge \beta_{\nu W} = \beta_{X \times Y} \wedge \beta_{a+b}.$$

	On the other hand, we know $h^*V$ is co-oriented by $(\beta_{h^*V},\beta_X)$ if an only if $\beta_{h^*V} \wedge \beta_{\nu V} = \beta_X \wedge \beta_a$ and $k^*W$ is co-oriented by $(\beta_{k^*W},\beta_Y)$ if an only if $\beta_{k^*W} \wedge \beta_{\nu W} = \beta_Y \wedge \beta_b$.
	Assuming these hold, then $h^*V \times k^*W$ is co-oriented by $$(-1)^{(x-(v+x-m))(w+y-n)}(\beta_{h^*V} \wedge \beta_{k^*W},\beta_X \wedge \beta_Y) = (-1)^{(m-v)(w+y-n)}(\beta_{h^*V} \wedge \beta_{k^*W},\beta_X \wedge \beta_Y).$$

	Now, continuing to assume the equalities of the last paragraph and taking $a$ and $b$ even as usual, we have

	\begin{align*}
		\beta_{h^*V} \wedge \beta_{k^*W} \wedge \beta_{\nu V} \wedge \beta_{\nu W}
		& = (-1)^{(w+y-n)(m-v)}\beta_{h^*V} \wedge \beta_{\nu V} \wedge \beta_{k^*W} \wedge \beta_{\nu W}\\
		& = (-1)^{(w+y-n)(m-v)}\beta_X \wedge \beta_a \wedge \beta_Y \wedge \beta_b \\
		& = (-1)^{(w+y-n)(m-v)}\beta_X \wedge \beta_Y \wedge \beta_a \wedge \beta_b.
	\end{align*}
	So if we take $\beta_P = (-1)^{(w+y-n)(m-v)}\beta_{h^*V} \wedge \beta_{k^*W}$ and $\beta_{X \times Y} = \beta_X \times \beta_Y$, then this also gives us $\beta_{P} \wedge \beta_{\nu V} \wedge \beta_{\nu W} = \beta_{X \times Y} \wedge \beta_{a+b}$.
	Therefore,
	$(h \times k)^*(V \times W)$ is also co-oriented by
	$$(\beta_P,\beta_X \wedge \beta_Y) = ((-1)^{(w+y-n)(m-v)}\beta_{h^*V} \wedge \beta_{k^*W},\beta_X \wedge \beta_Y).$$ We conclude $(h \times k)^*(V \times W) = h^*V \times k^*W$.
\end{proof}

\subsubsection{The relation between co-oriented exterior products and co-oriented fiber products}
Having established these elementary properties for our exterior product, we can now relate co-oriented exterior products to co-oriented fiber products.
These relationships correspond to those in singular cohomology between the exterior and cup products of cochains, though one very nice feature is that in our context these relationships all hold ``on the nose'' at the cochain level and can be proven without any need for Alexander-Whitney maps or any other approximations to the diagonal.
This will also be useful for proving some properties of the co-oriented fiber product that we have deferred so far.

\begin{proposition}\label{P: cross to cup}
	Suppose $f \colon V \to M$ and $g \colon W \to M$ are transverse co-oriented maps from manifolds with corners to a manifold without boundary.
	Let $\diag \colon M \to M \times M$ be the diagonal map $\diag(x) = (x,x)$.
	Then $f \times g$ is transverse to $\diag$, and the pullback of $V \times W \to M \times M$ by $\diag$ is the co-oriented fiber product $V \times_M W \to M$, i.e.\ $$V \times_M W = \diag^*(V \times W).$$
\end{proposition}
\begin{proof}
	It is standard that if $f$ and $g$ are transverse then $f \times g$ is transverse to the diagonal map.
	We briefly recall the argument.
	Suppose $f(x) = g(y) = z \in M$.
	Then $Df(T_xV)+Dg(T_yW) = T_zM$.
	Now suppose $(a,b) \in T_{(z,z)}(M \times M) \cong T_zM \oplus T_zM$.
	Write $a = v_a+w_a$ with $v_a \in Df(T_xV)$ and $w_a \in Dg(T_yW)$.
	Similarly, write $v = v_b+w_b$.
	Then
	\begin{align*}
		(a,b)& = (v_a+w_a,v_b+w_b)\\
		& = (v_a-v_b+v_b+w_a, v_b+w_a-w_a+w_b)\\
		& = (v_a-v_b,-w_a+w_b)+(v_b+w_a, v_b+w_a),
	\end{align*}
	which is in $D(f \times g)(T_xV \times T_yW)+D\diag(T_zM)$.
	Hence the pullback by the diagonal is a manifold with corners.

	This pullback consists precisely of those $(v,w,z) \in V \times W \times M$ such that $(f(v),g(w)) = (z,z)$, which is diffeomorphic to $V \times_M W = \{(v,w) \in V \times W \mid f(v) = g(w)\}$ via the projection $(v,w,z) \to (v,w)$ with inverse $(v,w) \to (v,w,f(v))$.
	So we consider the co-orientations.

	Proceeding as in the proof of \cref{P: exterior commutativity} (with now $M = N$), if $f$ and $g$ are co-oriented (at appropriate points) by $(\beta_V,\beta_M)$ and $(\beta_W,\beta_M)$ and if we we take Quillen co-orientations coming from $V\xhookrightarrow{e_v} M \times \R^a \to M$ and $W\xrightarrow{e_W} M \times \R^b \to M$ (with $a$ and $b$ assumed even) by orienting $\nu_V$ and $\nu_W$ in $M \times \R^a$ and $M \times \R^b$, then the co-orientation of the product $f \times g$ has Quillen co-orientation with the normal bundle to $V \times W$ in $M \times M \times \R^{a+b}$ oriented $\beta_{\nu V} \wedge \beta_{\nu W}$.

	Pulling back by the diagonal, we thus obtain from the definition that the pullback $P$ is co-oriented by $(\beta_P,\beta_M)$ if and only if $\beta_P \wedge \beta_{\nu V} \wedge \beta_{\nu W} = \beta_M \wedge \beta_{a+b}$.

	On the other hand, using the Quillen co-orientation for $V$, the co-orientation of the fiber product $P = V \times_M W \to W \to M$ is $(\beta_P,\beta_M)$ if and only if $\beta_P \wedge \beta_{\nu V} = \beta_W \wedge \beta_a$.

	Note that even though these local orientation forms do not all live on the same spaces, they are essentially just statements about the relative orders of coordinates on the spaces where those coordinates occur; this point of view allows us to elide the pullback maps that would be necessary needed (and messy) to formally pull all our comparisons onto the same space.
	With these thoughts in mind, we can conclude as follows.

	If we assume that we do have $\beta_P \wedge \beta_{\nu V} = \beta_W \wedge \beta_a$ then
	\begin{align*}
		\beta_P \wedge \beta_{\nu V} \wedge \beta_{\nu W}& = \beta_W \wedge \beta_a \wedge \beta_{\nu W}\\
		& = \beta_W \wedge \beta_{\nu W} \wedge \beta_a\\
		& = \beta_M \wedge \beta_b \wedge \beta_a\\
		& = \beta_M \wedge \beta_{a+b}.
	\end{align*}
	So the two co-orientations for $P \to M$ are either both $(\beta_P,\beta_M)$ or both the opposite co-orientation.
	In other words, they agree.
\end{proof}

\begin{corollary}\label{C: fiber natural pullback}
	Suppose $f \colon V \to M$ and $g \colon W \to M$ are transverse co-oriented maps of manifolds with corners to a manifold without boundary, that $N$ is a manifold with corners, and that $h \colon N \to M$ is transverse to $f$, $g$, and $f \times_M g$.
	Then
	$$h^*(V \times_M W) = h^*V \times_N h^*W$$
	as manifolds with co-oriented maps to $N$.
\end{corollary}
\begin{proof}
	Using \cref{P: cross to cup,P: pullback functoriality,P: natural exterior}, and that $\diag_M h = (h \times h) \circ \diag_N$,
	we compute
	\begin{align*}
		h^*(V \times_M W)& = h^*\diag_M^*(V \times W)\\
		& = (\diag_M h)^*(V \times W)\\
		& = ((h \times h) \circ \diag_N)^*(V \times W)\\
		& = \diag_N^*(h \times h)^*(V \times W)\\
		& = \diag_N^*(h^*V \times h^*W)\\
		& = h^*V\times_N h^*W.
	\end{align*}
	To apply \cref{P: pullback functoriality} in the second line, we note that $h$ is transverse to $\diag_M^*(V \times W) = V \times_M W$ by assumption.
	For the fourth line we observe that $h \times h$ is transverse to $f \times g$ because $h$ is transverse to $f$ and $g$, and the composite $(h \times h) \circ \diag_N = \diag_M h$ is transverse to $f \times g$ by our assumptions and \cref{L: transverse to pullback}.
\end{proof}

\begin{corollary}[Associativity of co-oriented fiber products]\label{C: fiber assoc}
	Suppose $f \colon V \to M$, $g \colon W \to M$, and $h \colon X \to M$ are co-oriented maps from manifolds with corners to a manifold without boundary such that the following pairs are transverse (see \cref{R: multiproducts}): $(V,W)$, $(W,X)$, $(V \times_M W,X)$, and $(V,W \times_M X)$.
	Then $$(V \times_M W) \times_M X = V \times_M (W \times_M X)$$ as co-oriented fiber products mapping to $M$.
\end{corollary}
\begin{proof}
	We compute using \cref{P: cross to cup,P: pullback functoriality,P: exterior associativity,P: natural exterior} and that $(\id_M \times \diag)\diag = (\diag\times\id_M)\diag$:

	\begin{align*}
		(V \times_M W) \times_M X& = \diag^*(\diag^*(V \times W) \times X)\\
		& = \diag^*(\diag \times \id_M)^*((V \times W) \times X)\\
		& = ((\diag \times \id_M)\diag)^*((V \times W) \times X)\\
		& = ((\id_M \times \diag)\diag)^*(V \times (W \times X))\\
		& = \diag^*(\id_M \times \diag)^*(V \times (W \times X))\\
		& = \diag^*(V \times \diag^*(W \times X))\\
		& = V \times_M (W \times_M X).\qedhere
	\end{align*}
\end{proof}

\begin{corollary}\label{C: criss cross}
	Suppose $f \colon V \to M$ and $g \colon W \to M$ are transverse co-oriented maps from manifolds with corners to a manifold without boundary and similarly for $h \colon X \to N$ and $k:Y \to N$.
	Then $$(V \times X)\times_{M \times N} (W \times Y) = (-1)^{(m-w)(n-x)} (V \times_M W) \times (X\times_N Y) $$
	as co-oriented fiber products over $M \times N$.
	\red{Add a version for pullbacks (as opposed to fiber products)? Might have some extra signs to figure out.
		Not really needed anywhere I do not think.}
\end{corollary}
\begin{proof}
	With our given transversality assumptions, $f \times h$ is transverse to $g \times k$, so the expression on the left is well defined.
	We then compute using \cref{P: cross to cup,P: pullback functoriality,P: exterior associativity,P: exterior commutativity,P: natural exterior} and letting $\tau$ here be the interchange of the interior $N$ and $M$ in the quadruple product:

	\begin{align*}
		(V \times X)\times_{M \times N} (W \times Y)& = \diag_{M \times N}^*(V \times X \times W \times Y)\\
		& = \diag_{M \times N}^*(\id \times \tau^* \times \id)^*(-1)^{(n-x)(m-w)}(V \times W \times X \times Y)\\
		& = (\diag_M \times \diag_N)^*(-1)^{(n-x)(m-w)}(V \times W \times X \times Y)\\
		& = (-1)^{(n-x)(m-w)}\diag_M^*(V \times W)\times\diag_N^*( X \times Y)\\
		& = (-1)^{(m-w)(n-x)} (V \times_M W) \times (X\times_N Y).
	\end{align*}
	For the third equality, we use that $\diag_M \times \diag_N = (\id_M \times \tau \times \id_N)\diag_{M \times N}$.
\end{proof}

\begin{corollary}\label{C: cross is cup}
	Let $V \to M$ and $W \to N$ be maps from manifolds with corners to manifolds without boundary.
	Let $\pi_M \colon M \times N \to M$ and $\pi_N \colon M \times N \to N$ be the projections.
	Then $$V \times W = \pi_M^*(V)\times_{M \times N}\pi_N^*(W)$$ as co-oriented manifolds mapping to $M \times N$.
\end{corollary}

\begin{proof}
	By \cref{P: projection pullbacks}, $\pi_M^*(V) = V \times N$ and $\pi_N^*(W) = M \times W$, so these are transverse in $M \times N$.
	Then by \cref{C: criss cross,C: cup with identity}, we have

	\begin{align*}
		\pi_M^*(V)\times_{M \times N}\pi_N^*(W)& = (V \times N)\times_{M \times N} (M \times W)\\
		& = (V \times_M M) \times (N\times_N W)\\
		& = V \times W.\qedhere
	\end{align*}
\end{proof}

\subsection{Mixed properties}

In this section we study properties that involve both orientations and co-orientations.
In particular, we are mostly interested in the pullback of a co-oriented map $V \to M$ by a map $W \to M$ with $W$ oriented, in which case the co-orientation of the pullback $V \times_M W \to W$ together with the orientation of $W$ produces an induced orientation on $V \times_M W$ as described in \cref{S: co-orientations}.
As $V \times_M W \to M$ with this orientation will eventually correspond to the cap product when we get to geometric homology and cohomology, we will here refer to this orientation as the \textbf{cap orientation}.

The following results all concern cap orientations on $V \times_M W$.
We note that, by construction, this oriented manifold comes equipped with a map to $W$ and, by composing with the given map $W \to M$, a map to $M$.

We start with the next theorem, which is yet another Leibniz formula.
It will be used in \cref{S: intersection map} to demonstrate that our intersection map $\mc I$, relating geometric cochains to cubical cochains of a cubulation, is a chain map.
This map is a critical component in relating geometric cohomology to other cohomology theories and is also central to the main result about cup products in \cite{FMS-flows}.

\begin{proposition}\label{P: Leibniz cap}
	Let $f \colon V \to M$ and $g \colon W \to M$ be transverse maps of manifolds with corners to a manifold without boundary.
	Suppose $f$ is co-oriented and $W$ is oriented.
	Then\footnote{Note that the signs here agree with those for the boundary of a chain-level cap product in Spanier \cite[Section 5.6.15]{Span81} and Munkres \cite[Section 66]{Mun84}.} $$\bd(V \times_M W) = \left[(-1)^{\dim(V \times_M W)} (\bd V) \times_M W\right] \bigsqcup V \times_M \bd W$$
	as oriented manifolds, giving $V \times_M W$, $(\bd V) \times_M W$, and $V \times_M \bd W$ each their cap orientations.
\end{proposition}
\begin{proof}
	We compute and compare these orientations by first considering the pullback co-orientations as defined in \cref{D: pullback coorient}.
	We proceed by analogy to the proof of the Leibniz rule for the pullback of co-oriented maps in \cref{leibniz}, utilizing the computations already performed there.

	Recall, in brief, from \cref{D: pullback coorient} that to co-oriented the pullback $P = V \times_M W \to W$ we first construct a composition $V\xhookrightarrow{e} M \times \R^N \to M$ and find a Quillen orientation for the normal bundle $\nu V$ of $e(V) \subset M \times \R^N$ as determined by the co-orientation of $V \to M$.
	Then we pull back via $W \times \R^N \to M \times \R^N$ to obtain a normal bundle, also labeled $\nu V$, of $P \subset W \times \R^N$.
	Then we co-orient $P \to W$ locally by $(\beta_P,\beta_W)$ so that $\beta_P \wedge \beta_{\nu V} = \beta_W \wedge \beta_E$, where $\beta_E$ represents the standard orientation of $\R^N$.
	In the case at hand, we can assume $\beta_W$ to represent the global orientation of $W$, and then $\beta_P$ becomes a global orientation for $P$.
	This is the orientation given to $V \times_M W$ in \cref{D: PC products}.

	Let $\nu\bd P$ denote an outward pointing normal vector in the tangent bundle to $P$ at a boundary point of $P$, and let $\beta_{\nu\bd P}$ denote the corresponding orientation.
	Then, by definition, $\bd P$ is oriented at that point by $\beta_{\bd P}$ so that $\beta_{\nu\bd P} \wedge \beta_{\bd P} = \beta_P$.
	In other words, with $\beta_P$, $\beta_W$, $\beta_{\nu V}$, and $\beta_E$ given, $\bd P$ is oriented by $\beta_{\bd P}$ such that $\beta_{\nu\bd P} \wedge \beta_{\bd P} \wedge \beta_{\nu V} = \beta_W \wedge \beta_E$.

	Now, recall that $\bd(V \times_M W) = (\bd V) \times_M W \bigsqcup V \times_M \bd W$ as spaces and consider a point in $(\bd V) \times_M W$.
	By \cref{leibniz}, at such a point the pullback co-orientation of $(\bd V) \times_M W \to W$ agrees with boundary co-orientation of the pullback $P = V \times_M W \to W$, as described again in the preceding paragraph.
	So continuing to let $(\beta_P,\beta_W)$ denote the pullback co-orientation of $P \to W$ and recalling that the boundary co-orientation utilizes the \textit{inward} normal, the boundary co-orientation of $(\bd V) \times_M W \to W$ is the composite $(\beta_{\bd P}, \beta_{\bd P} \wedge -\beta_{\nu\bd P})*(\beta_P,\beta_W)$ for any $\beta_{\bd P}$.
	But if we choose $\beta_{\bd P}$ to represent the orientation of $\bd P$ found above by orienting $P$ and then taking its boundary orientation, we have $\beta_P = \beta_{\nu\bd P} \wedge \beta_{\bd P} = (-1)^{\dim(\bd P)}\beta_{\bd P} \wedge \beta_{\nu\bd P}$.
	So the boundary co-orientation of $(\bd V) \times_M W \to W$ is the composite
	$$(\beta_{\bd P}, \beta_{\bd P} \wedge -\beta_{\nu\bd P})*((-1)^{\dim(\bd P)}\beta_{\bd P} \wedge \beta_{\nu\bd P},\beta_W) = (-1)^{\dim(\bd P)+1}(\beta_{\bd P},\beta_W).$$
	Thus the resulting orientation of $(\bd V) \times_M W$ is $(-1)^{\dim(P)}$ times the orientation of $\bd P$ obtained by taking the oriented boundary of $V \times_M W$.

	Next we consider a point in $V \times_M \bd W$.
	From the Leibniz rule computation for co-orientations, the co-orientation of the pullback $V \times_M \bd W \to \bd W$ is $(\beta_{\bd P},\beta_{\bd W})$ if and only if $\beta_{\bd P} \wedge \beta_{\nu V} = \beta_{\bd W} \wedge \beta_E$.
	If $\nu\bd P$ is an outward normal then this is equivalent to $\beta_{\nu\bd P} \wedge \beta_{\bd P} \wedge \beta_{\nu V} = \beta_{\nu\bd P} \wedge \beta_{\bd W} \wedge \beta_E = \beta_W \wedge \beta_E$.
	But taking $\beta_P$ and $\beta_{\bd P}$ as found above using our given orientation of $\beta_W$, we have $\beta_{\nu\bd P} \wedge \beta_{\bd P} = \beta_P$, and so the condition is equivalent to $\beta_{P} \wedge \beta_{\nu V} = \beta_W \wedge \beta_E$, which holds by our choice above of $\beta_P$.
	So the orientation of $V \times_M \bd W$ agrees with the orientation of $\bd P$ and we obtain overall $$\bd(V \times_M W) = \left[(-1)^{\dim(P)} (\bd V) \times_M W\right] \bigsqcup V \times_M \bd W.$$

	Finally, we use that $\dim(P) = \dim(V)+\dim(W)-\dim(M)$.
\end{proof}

Next we describe the cap orientations when $V \to M$ and $W \to M$ are embeddings.
As we've observed in the cases where either both maps are oriented or both maps are co-oriented, this is often an instructive and important example.

\begin{proposition}\label{P: cap of immersions}
	Let $f \colon V \to M$ and $g \colon W \to M$ be transverse embeddings from manifolds with corners to a manifold without boundary.
	Suppose $f$ is co-oriented and $W$ is oriented.
	Then $P = V \times_M W$ is just the intersection of $V$ and $W$ in $M$.
	If $\beta_W$ is the orientation of $W$ and $\beta_{\nu V}$ is the Quillen orientation of the normal bundle to $V$ in $M$, which at points of $P$ we can identify\footnote{See \cref{L: normal pullback}.} with the normal bundle to $P$ in $W$, then the cap orientation $\beta_P$ of $P$ satisfies $\beta_P \wedge \beta_{\nu V} = \beta_W$.
	If $f$ and $g$ are immersions, then this description holds locally.
\end{proposition}
\begin{proof}
	As $f$ is an embedding, we can take $N = 0$ in the definition of the pullback co-orientation, \cref{D: pullback coorient}.
	Then the pullback is just the inclusion of $P = g^{-1}(V) = V \cap W$ into $W$, and by definition the pullback co-orientation has the form $(\beta_P,\beta_W)$, where $\beta_P \wedge \beta_{\nu V} = \beta_W$ and $\nu V$ here is the pullback of the normal bundle of $V$ in $M$ to be the normal bundle of $V \cap W$ in $W$.
	Furthermore, if we take $\beta_W$ to be the given orientation of $W$, then $\beta_P$ is the induced orientation on the intersection by definition.
	The last statement about immersions follows as we can compute the co-orientations locally.
\end{proof}

The following corollary is particularly important and follows immediately from \cref{P: cap of immersions}.

\begin{corollary}\label{C: complementary cap}
	Let $f \colon V \to M$ and $g \colon W \to M$ be transverse embeddings from manifolds with corners to a manifold without boundary.
	Suppose $f$ is co-oriented, $W$ is oriented, and $\dim(V)+\dim(W) = \dim(M)$.
	Then $V \times_M W$ is the union of intersection points of $V$ and $W$.
	Such a point $x \in V \cap W$ is positively oriented if and only if the Quillen orientation of the normal bundle $\nu V$ of $V$ at $x$ agrees with the orientation of $W$ at $x$, identifying the fiber of $\nu V$ at $x$ with $T_xW$.
	If $f$ and $g$ are immersions, then this description holds locally.
\end{corollary}

The next two propositions will eventually correspond to the unital identities for the cap product for geometric chains and cochains.
The analogues for singular chains and cochains are the cap product with the cochain $1$ and the cap product with a chain representing the fundamental class, though in this case our underlying spaces do not need to be compact.

\begin{proposition}\label{P: cap with 1}
	Let $g \colon W \to M$ be a map from an oriented manifold with corners to a manifold without boundary, and consider $M \to M$ as the identity with the tautological co-orientation.
	Then $M \times_M W = W$ as oriented manifolds.
\end{proposition}
\begin{proof}
	By definition, there is a Quillen co-orientation for $M$ consisting of the sequence of identity maps $M \into M \to M$ with the normal bundle to $M$ in itself being the $0$-dimensional vector bundle, which we consider to have positive orientation at each point.
	It follows from the definition of the pullback that the corresponding Quillen co-orientation for $M \times_M W$ comes from the sequence $g^{-1}(M) = W \into W \to W$ with $W$ also having a $0$-dimensional positively-oriented normal bundle in itself.
	Consequently, the pullback co-orientation for $W \to W$ is the tautological one, and so the induced orientation on $W$ is the given one.
\end{proof}

\begin{proposition}\label{P: cap with identity M}
	Let $f \colon V \to M$ be a co-oriented map from a manifold with corners to an oriented manifold without boundary, and consider $M$ equipped with its identity map $M \to M$.
	Then $V \times_M M$ is $V$ with its induced orientation.
\end{proposition}
\begin{proof}
	This follows directly from \cref{C: cup with identity} and the definitions.
\end{proof}

The next property relates products of pullbacks with pullbacks of products.

\begin{proposition}\label{P: cap cross}
	Let $f \colon V \to M$ and $g:X \to N$ be co-oriented maps from manifolds with corners to manifolds without boundary, and let $h \colon W \to M$ and $k:Y \to N$ be maps with $W$ and $Y$ compacted oriented manifolds with corners.
	Suppose that $V$ is transverse to $W$ and that $X$ is transverse to $Y$.
	Then\footnote{Again our signs agree with the cap product formulas in Spanier, in this case \cite[Section 5.6.21]{Span81}.},

	$$(V \times X)\times_{M \times N} (W \times Y) = (-1)^{(x+y-n)(m-v)} (V \times_M W) \times (X\times_NY),$$
	as oriented manifolds mapping to $M \times N$.
\end{proposition}
\begin{proof}
	Let $\beta_W$ and $\beta_Y$ denote the orientations of $W$ and $Y$.
	Then $W \times Y$ is oriented by $\beta_W \wedge \beta_Y$.

	Now let $P = V \times_M W$ and $P' = X\times_N Y$.
	By definition, $P$ and $P'$ are oriented by the orientations $\beta_P$ and $\beta_{P'}$ such that $(\beta_P,\beta_W)$ and $(\beta_{P'},\beta_Y)$ are the pullback co-orientations for $P \to W$ and $P' \to Y$.
	Then $P \times P'$ is oriented by $\beta_P \wedge \beta_{P'}$.

	Furthermore, using our construction of pullback co-orientations, $\beta_P$ and $\beta_{P'}$ are such that $\beta_P \wedge \beta_{\nu V} = \beta_W \wedge \beta_a$ and $\beta_{P'} \wedge \beta_{\nu X} = \beta_Y \wedge \beta_b$, where $\beta_a$ and $\beta_b$ are the standard orientations Euclidean spaces $\R^a$ and $\R^b$ and we are free to take $a$ and $b$ to be even integers.

	By \cref{L: Quillen product co-orientation}, we have that the Quillen co-orientation of $V \times X \to M \times N$ is represented by an embedding $V \times X \into M \times N \times \R^a \times \R^b$ with normal bundle $\nu V \oplus \nu X$.
	So letting $Q = (V \times X)\times_{M \times N} (W \times Y)$, the orientation $\beta_Q$ is the one such that $(\beta_Q,\beta_W \wedge \beta_Y)$ is the pullback co-orientation, i.e.\ the one such that $\beta_Q \wedge \beta_{\nu V} \wedge \beta_{\nu X} = \beta_W \wedge \beta_Y \wedge \beta_a \wedge \beta_b$.
	But then we compute, using $a$ and $b$ even,

	\begin{align*}
		\beta_W \wedge \beta_Y \wedge \beta_a \wedge \beta_b& = \beta_W \wedge \beta_a \wedge \beta_Y \wedge \beta_b\\
		& = \beta_P \wedge \beta_{\nu V} \wedge \beta_{P'} \wedge \beta_{\nu X}\\
		& = (-1)^{|P'||\nu V|}\beta_P \wedge \beta_{P'} \wedge \beta_{\nu V} \wedge \beta_{\nu X}\\
		& = (-1)^{(x+y-n)(m-v)}\beta_P \wedge \beta_{P'} \wedge \beta_{\nu V \oplus \nu X}.
	\end{align*}
	So $\beta_Q = (-1)^{(x+y-n)(m-v)}\beta_P \wedge \beta_{P'} = (-1)^{(x+y-n)(m-v)}\beta_{P \times P'}$.
\end{proof}

The following technical lemma will be used to prove \cref{P: OC mixed associativity}, which will eventually become the associativity relation among cup and cap products, i.e.\ $(a \cup b) \cap x = a \cap (b \cap x)$.

\begin{lemma}\label{L: same induced}
	Let $f \colon V \to M$ and $g \colon W \to M$ be transverse maps from manifolds with corners to a manifold without boundary.
	Suppose that $f$ is co-oriented and that $W$ and $M$ are oriented, with respective (global) orientations $\beta_W$ and $\beta_M$.
	Suppose we co-orient $g$ by $(\beta_W,\beta_M)$.
	Then the cap orientation of $V \times_M W$ (i.e.\ that induced from the the pullback co-orientation of $V \times_M W \to W$ and the orientation of $W$) is the same as the orientation induced on $V \times_M W$ by the fiber product co-orientation of $V \times_M W \to M$ and the orientation of $M$ .
\end{lemma}
\begin{proof}
	By definition, the orientation of $V \times_M W$ induced from the orientation of $W$ and the pullback co-orientation of $V \times_M W \to W$ is the orientation $\beta_P$ such that $(\beta_P,\beta_W)$ is the pullback co-orientation.
	But then the fiber product co-orientation is the composite $(\beta_P,\beta_W)*(\beta_W,\beta_M) = (\beta_P,\beta_M)$.
	So the orientation induced by the orientation of $M$ and the composite co-orientation is again $\beta_P$.
\end{proof}

\begin{proposition}[Mixed associativity]\label{P: OC mixed associativity}
	Let $f \colon V \to M$ and $g \colon W \to M$ be co-oriented maps from manifolds with corners to a manifold without boundary.
	Let $h \colon Z \to M$ be a map with $Z$ oriented.
	Then, assuming sufficient transversality for all terms to be defined (see \cref{R: multiproducts}),

	$$(V \times_M W) \times_M Z = V \times_M (W \times_M Z),$$
	as oriented manifolds mapping to $M$.
\end{proposition}

Note that both expressions are well defined: On the left, $V \times_M W$ has a fiber product co-orientation, so we can form the cap orientation of the co-oriented $V \times_M W \to M$ over the oriented $Z$.
On the right we first give $W \times_M Z$ its cap orientation and then use that to form the cap orientation of $V \times_M (W \times_M Z)$.

\begin{proof}
	First suppose $M$ is orientable and that we have given it an arbitrary, but fixed, orientation.
	Then applying \cref{L: same induced}, the cap orientation of $(V \times_M W) \times_M Z$ is the same as the orientation induced by the orientation of $M$ and the fiber product co-orientation $(V \times_M W) \times_M Z \to M$, after co-orienting $Z \to M$ with the co-orientation induced by the orientations of $Z$ and $M$.
	Similarly, applying \cref{L: same induced} twice, the cap orientation of $V \times_M (W \times_M Z)$ is the same as that induced by the orientation of $M$ and the iterated fiber product co-orientation $V \times_M (W \times_M Z) \to M$ again coming from the canonical co-orientation of $Z \to M$.
	But now these co-oriented fiber products are the same by \cref{C: fiber assoc}.

	Next, suppose $M$ is not necessarily orientable.
	We know from their constructions that $(V \times_M W) \times_M Z$ and $V \times_M (W \times_M Z)$ are oriented manifolds, and it is not difficult to see that they are diffeomorphic, both being diffeomorphic to $\{(v,w,z) \in V \times W \times Z \mid f(v) = g(w) = h(z)\}$.
	So it suffices to consider these as identical spaces and to show that their induced orientations agree at any arbitrary point.
	If $(v,w,z)$ is such a point, consider its image $a = f(v) = g(w) = h(z) \in M$.
	Let $U$ be a Euclidean neighborhood of $a$, and consider the restrictions of $f$, $g$, and $h$ to $f^{-1}(U)$, $g^{-1}(U)$, and $h^{-1}(U)$.
	The resulting products over $U$ give us the pieces of $(V \times_M W) \times_M Z$ and $V \times_M (W \times_M Z)$ over $U$, and the resulting orientations will be compatible with those of the full manifolds $(V \times_M W) \times_M Z$ and $V \times_M (W \times_M Z)$, as orientations and co-orientations of fiber products are determined locally (see \cref{R: local pullback co-orientations} and the construction of fiber product orientations).
	But as $U$ is orientable, the preceding argument shows that these orientations must agree with each other.
\end{proof}

The following property will eventually manifest itself in geometric (co)homology as the familiar naturality formula for cap products $f_*(f^*(\alpha)\frown x)) = \alpha\frown f_*(x)$.

\begin{proposition}\label{P: natural cap}
	Let $f \colon V \to M$ and $h \colon N \to M$ be transverse maps with $h$ co-oriented, $V$ a manifold with corners and $M$ and $N$ manifolds without boundary.
	Furthermore, let $g \colon W \to N$ be a map from an oriented manifold with corners that is transverse to the co-oriented pullback $V \times_M N \to N$.
	Then the cap orientation induced on $(V \times_M N)\times_N W$ by pulling back the co-oriented map $V \times_M N \to N$ over $W \to N$
	is the same as the cap orientation obtained by pulling back $V \to M$ by the composite $hg \colon W \to M$.
	In other words, $(h^*V)\times_N W = V \times_M W$ as oriented manifolds.
\end{proposition}
\begin{proof}
	Note that $V$ is transverse to $hg \colon W \to M$ by \cref{L: transverse to pullback}, so both expressions are defined.
	It follows directly from \cref{P: pullback functoriality} that the two pullback co-orientations we have described for $(V \times_M N)\times_N W \to W$ agree.
	Therefore, the induced orientations on $(V \times_M N)\times_N W$ agree.
\end{proof}

\subsubsection{Comparing the oriented and co-oriented fiber products}

Suppose $f \colon V \to M$ and $g \colon W \to M$ are two transverse co-oriented maps from manifolds with corners to a manifold without boundary.
Further, suppose $M$ oriented.
Then we know there is a bijection between co-orientations of $f$ and orientations of $V$: an orientation of $V$ induces a co-orientation of $f$ and vice versa.
Of course the same is true of $W$ and $g$.
In this scenario, we have two different ways to orient $V \times_M W$, depending on whether we start by thinking of $V$ and $W$ as oriented or by thinking of $f$ and $g$ as co-oriented.
If we think of $V$ and $W $as oriented, we have the fiber product orientation of $V \times_M W$
discussed in \cref{S: orientations}.
Alternatively, if we think of $f$ and $g$ as co-oriented, we can form the fiber product co-orientation of $V \times_M W \to M$ as in \cref{S: co-orientation of pullbacks} and then consider the induced orientation given the orientation of $M$.

Our goal in this section is to compare these two orientations.
To attempt to avoid confusion, we will write $V \times_M ^oW$ for the fiber product orientation of \cref{S: orientations} and $V \times_M ^cW$ for the co-oriented fiber product or, equivalently, the resulting induced orientation.

The reader might have noticed that a third way to orient $V \times_M W$ is to consider $f$ to be co-oriented and $W$ to be oriented and then form the cap orientation that we studied in detail in the preceding section.
However, we already know this to be identical to $V \times_M ^cW$ by \cref{L: same induced}.
By contrast, these are not always the same as $V \times_M ^oW$.

When we move on to geometric homology, $V \times_M ^oW$ will correspond to the classical intersection product of homology classes, as described for example in \cite[Section VI.11]{Bred97}, while $V \times_M ^cW$ will correspond to the cup product of cohomology classes.
When $M$ is closed and oriented, switching between thinking of $V$ and $W$ as oriented vs.
co-oriented will be the Poincar\'e duality isomorphism.
So this proposition will ultimately demonstrate that the intersection product is Poincar\'e dual to the cup product, up to a sign; see \cref{S: PD}.

\begin{proposition}\label{P: compare cup and intersection orientations}
	Let $f \colon V \to M$ and $g \colon W \to M$ be transverse co-oriented maps from manifolds with corners to an oriented manifold without boundary or, equivalently, suppose $V$, $W$, and $M$ all oriented.
	Then $$V \times_M ^oW = (-1)^{(m-v)(m-w)} V \times_M ^cW$$ as oriented manifolds with corners.
\end{proposition}

The proof will take a bit of work.
Our strategy will be as follows: First, we will prove in \cref{L: compare cup and intersection for immersions} that the result holds when $f$ and $g$ are immersions.
Then we will show, first for co-orientations and then for orientations, that we can replace the fiber product of
$$V \xr{f} M \xleftarrow{g} W$$ with the fiber product
$$V \times \R^{2b}\xhookrightarrow{e \times \id_{\R^{2b}}} M \times \R^{2a} \times \R^{2b} \xhookleftarrow W \times \R^{2a},$$
where $e$ is the embedding $V \into M \times \R^{2a}$ of a Quillen co-orientation of $f$ and the leftward arrow is the identity between the $\R^{2a}$ factors and takes the $W$ factor into $M \times \R^{2b}$ by the embedding map of a Quillen co-orientation of $g$.
By ``replace,'' we mean in the orientated case that the two fiber products are canonically oriented diffeomorphic.
In the co-oriented case we mean that we have a canonical oriented diffeomorphism
between the domains of the two co-oriented fiber products, oriented with their induced orientations coming respectively from the orientation of $M$ and from the concatenation orientation of $M \times \R^{2a} \times \R^{2b}$ using the standard orientations of the Euclidean terms.
The maps of this second fiber product are embeddings for which the proposition holds by \cref{L: compare cup and intersection for immersions}, and so the general case will follow.
The even dimensions of the Euclidean factors are chosen to avoid some extraneous signs in the arguments below.

Before proceeding, let us explain in more detail what we mean by ``canonical'' here and below.
Recall that, as a space, $P = V \times_M W$ can be identified with $\{(v,w) \in V \times W \mid f(v) = g(w)\}$.
Below we will see various fancier embeddings of $P$ in spaces of the form $V \times X \times W \times Y$, with $X$ and $Y$ Euclidean or $I$.
For each such embedding, the projection to $V \times W$ will take the embedding of $P$ back to $P$.
In this way, all versions of $P$ can be canonically identified, and it is these identifications that will yield our orientation preserving diffeomorphisms.
Such identifications have already been discussed in \cref{R: pullback representative,R: pullback representative 2}.
In the latter remark, we provide exactly such a canonical identification between our standard realization of $V \times_M W$ as a subset of $V \times W$ and the version used for co-orienting pullbacks and fiber products.

\begin{comment}

	is already contained in the construction of the co-oriented fiber product, where $V\times^c_MW$ can be considered to be contained in $V \times W \times \R^N$.
	Explicitly we constructed $P = (g \times \id_{\R^N})^{-1}(e(V))$, but as $e$ is an embedding, the $V$ factor of points in $P$ is implicitly determined by $(w,z) \in P \subset W \times \R^N$ as $e^{-1}((g(w),z))$.
	So $P = (g \times \id_{\R^N})^{-1}(e(V))$ is canonically identified with $\{(v,w,z) \in V \times W \times \R^N \mid e(v) = (g(w),z)\}$.
	And as $e$ is part of a Quillen co-orientation, by projecting $M \times \R^N$ to $M$, we have of course $g(w) = f(v)$, so there is a canonical map $\{(v,w,z) \in V \times W \times \R^N \mid e(v) \to \{(v,w) \in V \times W \mid f(v) = g(w)\}$ given by projection to the first two factors.
	This is a diffeomorphism, whose inverse takes $(v,w)$ to $(v,w,z)$ such that $e(v) = (g(w),z)$.

	\red{Put some of this as a remark in the section where pullback of co-orientation is defined.}

	\medskip
\end{comment}

\medskip

We begin with the case where $V$ and $W$ are immersions and then work toward the general case.

\begin{lemma}\label{L: compare cup and intersection for immersions}
	If $f$ and $g$ are transverse immersions of oriented manifolds with corners into an oriented manifold without boundary, then $V \times_M ^oW = (-1)^{(m-v)(m-w)} V \times_M ^cW$.
\end{lemma}
\begin{proof}
	It suffices to consider small neighborhoods on which $f$ and $g$ are embeddings.
	Let $\beta_V$, $\beta_W$, and $\beta_M$ denote the orientations of $V$, $W$, and $M$, respectively.
	Assuming $f$ and $g$ to be co-oriented with the compatible co-orientations, we have the resulting Quillen orientations $\beta_{\nu V}$ and $\beta_{\nu W}$ of the normal bundles of $V$ and $W$.
	Recall that these are defined so that $(\beta_V,\beta_V \wedge \beta_{\nu V})$ and $(\beta_W,\beta_W \wedge \beta_{\nu W})$ are the co-orientations of $f$ and $g$, respectively.
	In this scenario, with $\beta_V$ and $\beta_W$ fixed as the orientations of $V$ and $W$, this is equivalent to requiring $\beta_V \wedge \beta_{\nu V} = \beta_M$ and $\beta_W \wedge \beta_{\nu W} = \beta_M$.
	Again to keep the contexts clear, for the remainder of the argument we will write $\beta^c_{\nu V}$ and $\beta^c_{\nu W}$ for the orientations of $\nu V$ and $\nu W$.

	By \cref{P: normal pullback}, the co-orientation of the fiber product $V\times^cW$ is $(\beta_P,\beta_P \wedge \beta^c_{\nu V} \wedge \beta^c_{\nu W})$.
	In particular, the induced orientation of $V\times^cW$ is the orientation $\beta_P^c$ such that $\beta_P^c \wedge \beta^c_{\nu V} \wedge \beta^c_{\nu W} = \beta_M$.

	On the other hand, in \cref{P: orient intersection}, $\beta^o_P$ is such that if $\beta^o_P \wedge \beta^o_{\nu W} = \beta_V$ and $\beta^o_P \wedge \beta^o_{\nu V} = \beta_W$ then $\beta^o_P \wedge \beta^o_{\nu V} \wedge \beta^o_{\nu W} = \beta_M.$ A priori these may be different orientations of $\nu V$ and $\nu W$ than those of the preceding paragraph, so we use these alternate labels.
	In fact, let us suppose $\beta^o_P$, $\beta^o_{\nu V}$, and $\beta^o_{\nu W}$ chosen so that these expressions all hold.
	Then we have
	$$\beta_M = \beta^o_P \wedge \beta^o_{\nu V} \wedge \beta^o_{\nu W} = \beta_{W} \wedge \beta^o_{\nu W},$$
	so $\beta_W^o = \beta_W^c.$ Similarly,
	$$\beta_M = \beta^o_P \wedge \beta^o_{\nu V} \wedge \beta^o_{\nu W} = (-1)^{(m-v)(m-w)}\beta^o_P \wedge \beta^o_{\nu W} \wedge \beta^o_{\nu V} = (-1)^{(m-v)(m-w)}\beta^o_V \wedge \beta^o_{\nu V},$$
	so $\beta^0_{\nu V} = (-1)^{(m-v)(m-w)}\beta^c_{\nu V}$.

	Thus $$\beta_M = \beta^o_P \wedge \beta^o_{\nu V} \wedge \beta^o_{\nu W} = (-1)^{(m-v)(m-w)}\beta^o_P \wedge \beta^c_{\nu V} \wedge \beta^c_{\nu W}.$$

	We conclude that $\beta^c_P = (-1)^{(m-v)(m-w)}\beta^o_P$.
\end{proof}.

Now we show how to replace a general co-oriented fiber product with a co-oriented fiber product whose maps are embeddings.
In our remaining constructions in this section we take all introduced Euclidean spaces to be even-dimensional to simplify the signs.

\begin{lemma}
	Let $f \colon V \to M$ and $g \colon W \to M$ be transverse co-oriented maps from manifolds with corners to an oriented manifold without boundary.
	Let $V\xhookrightarrow{e}M \times \R^{2a} \to M$ be a Quillen co-orientation for $f$.

	Then $V \times_M W$ (with its orientation induced by the fiber product co-orientation and the orientation of $M$) is canonically orientated diffeomorphic to $V\times_{M \times \R^{2a}}(W \times \R^{2a})$ (with its orientation induced by the fiber product co-orientation and the orientation of $M \times \R^{2a}$).
	Here the maps for the second fiber product are $e \colon V \to M \times \R^{2a}$ and $g\times\id_{\R^{2a}} \colon W \times \R^{2a} \to M \times \R^{2a}$.
	As in the construction of the Quillen co-orientation (\cref{D: Quillen normal or}), we assume $e \colon V \into M \times \R^{2a}$ to be co-oriented so that its composition with the canonical co-orientation $(\beta_{M}\wedge\beta_a,\beta_M)$ is the co-orientation of $f$.
	We also take $g \times \id_{\R^{2a}}$ to be co-oriented by the product co-orientation $(\beta_W \wedge \beta_a,\beta_M \wedge \beta_a)$ if $(\beta_W,\beta_M)$ is the co-orientation of $g$.
	Finally, $M \times \R^{2a}$ is given the product orientation with $\R^{2a}$ having the standard orientation.
\end{lemma}
\begin{proof}
	By the definitions of the induced orientation and the fiber product co-orientation, the orientation of the fiber product $V \times_M W \to M$ is $\beta_P$ where $\beta_P \wedge \beta_{\nu V} = \beta_M\wedge\beta_a$.
	We recall that the $\nu V$ in this formula is actually the pullback of the normal bundle of $e(V) \subset M \times \R^{2a}$ via the map $g\times\id_{\R^{2a}}: W \times \R^{2a} \to M \times \R^{2a}$.
	But this is also precisely the description of the induced fiber product orientation from the co-oriented fiber product $V\times_{M \times \R^{2a}}(W \times \R^{2a})$, treating $e \colon V \into M \times \R^{2a}$ as its own Quillen co-orientation with $N = 0$ in \cref{D: pullback coorient}.
\end{proof}

\begin{corollary}\label{C: co-oriented full transition to embedded}
	Let $f \colon V \to M$ and $g \colon W \to M$ be transverse co-oriented maps from manifolds with corners to an oriented manifold without boundary.
	Let $V\xhookrightarrow{r} M \times \R^{2a} \to M$ and $W\xhookrightarrow{s} M \times \R^{2b} \to M$ be Quillen co-orientations compatible with $f$ and $g$.
	Then the fiber product $V \times_M W$ (with its orientation induced from the co-oriented fiber product) is canonically oriented diffeomorphic to the fiber product $(V \times \R^{2b})\times_{M \times \R^{2a} \times \R^{2b}}(W \times \R^{2a})$ (with its orientation induced from the co-oriented fiber product), in which the first map is $r \times \id_{\R^{2b}}$ and the second map takes $(w,z) \in W \times \R^{2a}$ to $s(w)$ in the first and third coordinates and $z$ in the second coordinate.
\end{corollary}

\begin{proof}
	By the preceding lemma we have $V \times_M W$ canonically oriented diffeomorphic to $V\times_{M \times \R^{2a}}(W \times \R^{2a})$ with $V \to M \times \R^{2a}$ an embedding.
	By \cref{P: graded comm}, the transposition map from this space to $(W \times \R^{2a})\times_{M \times \R^{2a}}V$ is $(-1)^{(m-v)(m-w)}$-orientation preserving, using that all our Euclidean spaces are taken even-dimensional.
	We next observe the map described for $W \times \R^{2a} \to M \times \R^{2a} \times \R^{2b}$ is an embedding whose composition with the projection to $M \times \R^{2a}$ gives a Quillen co-orientation for $g \times \id_{\R^{2a}}$.
	Now we can apply the lemma again and then transpose again.
\end{proof}

Next we show how to replace a general oriented fiber product with an oriented fiber product whose maps are embeddings.
Again, we take all introduced Euclidean spaces to be even-dimensional to simplify the signs.

\begin{lemma}
	Let $f \colon V \to M$ and $g \colon W \to M$ be transverse maps from oriented manifolds with corners to an oriented manifold without boundary.
	Let $V\xhookrightarrow{e}M \times \R^{2a} \to M$ be a factorization of $f$ with $e$ an embedding.
	Then $V \times_M W$ is canonically oriented diffeomorphic to the oriented fiber product $V\times_{M \times \R^{2a}} (W \times \R^{2a})$ of $e \colon V \to M \times \R^{2a}$ and $g \times \id_{\R^{2a}} \colon W \times \R^{2a} \to M \times \R^{2a}$.
\end{lemma}
\begin{proof}
	We will first show that $V \times_M W$ is canonically oriented diffeomorphic to the fiber product $V\times_{M \times \R^{2a}} (W \times \R^{2a})$ with the map $V \to M \times \R^{2a}$ being the composition of $f$ with the inclusion $M = M \times \{0\} \into M \times \R^{2a}$.
	We will write this composite as $f_0$.
	It is clear that $f_0$ and $g \times \id_{\R^{2a}}$ are transverse as $f$ and $g$ are transverse in $M$ and the $\id_{\R^{2a}}$ factor takes care of the $\R^{2a}$ factor of the tangent spaces.
	We also observe that the two fiber products are canonically the same as spaces, as $V \times_M W = \{(v,w) \in V \times W \mid f(v) = g(w)\}$, while the other fiber product is $\{(v,w,0) \in V \times W \times \R^a \mid f(v) = g(w)\}$.

	Now we consider the orientations.
	As the tangent space of the pullback is the pullback of the tangent space, it suffices to assume that all spaces and maps are linear; in the general case we replace all the following spaces with their bundles and all maps with their derivatives.
	For $P = V \times_M W$, we look at the map
	\begin{equation}\label{E: fiber orient}
		\Phi \colon P \oplus M \to V \oplus W
	\end{equation}
	given by $((v,w),x) \to (v,w)+s(x)$, where $s$ is a splitting of the map $V \oplus W \to M$ that takes $(v,w)$ to $f(v)-g(w)$.
	We recall that the orientation of $P$ is chosen so that $\Phi$ is an orientation preserving isomorphism up to a sign of $(-1)^{\dim(W)\dim(M)}$.

	In the case of $V\times_{M \times \R^{2a}} (W \times \R^{2a})$, if we abuse notation and again write $P$ for the fiber product, then we have instead a map
	\begin{equation}\label{E: fiber plus euclidean}
		\Psi: P \oplus M \oplus \R^{2a} \to V \oplus W \oplus \R^{2a},
	\end{equation}
	with $\Psi$ again the restriction to $P$ of the projections to $V$ and $W \oplus \R^{2a}$, while the restriction to $M$ must be a splitting of the map $\Upsilon: V \oplus W \oplus \R^{2a} \to M \oplus \R^{2a}$ that is $f_0$ on the first factor and $-(g \oplus \id_{\R^{2a}})$ on the last two factors.
	We claim that we can take $\Psi((v,w,0),x,z) = (v,w,0)+(s(x),0)-(0,0,z)$.
	This is certainly correct on the $P$ factor.
	For the $M \oplus \R^{2a}$ factor, we must show $\Upsilon \Psi(0,x,z) = (x,z)$.
	We have $\Psi(0,x,z) = (s(x),0)-(0,0,z)$, noting that $s(x) \in V \oplus W$.
	If we write $s(x) = (s_V(x),s_W(x))$, then by definition $f(s_V(x))-g(s_W(x)) = x$.
	So we have \begin{align*}
		\Upsilon((s(x),0)-(0,0,z))& = \Upsilon(s_V(x),s_W(x),z)\\& = f_0(s_V(x))-(g(s_W(x)),z)\\& = (f(s_V(x)),0)-(g(s_W(x)),z)\\& = (f(s_V(x))-g(s_W(x)),z)\\& = (x,z).\end{align*}
	So our definition of $\Psi$ is correct.

	As $\Phi$ and $\Psi$ agree in the first two factors and the dimension of $\R^{2a}$ is even, we see that $\Psi$ is orientation-preserving if and only if $\Phi$ is.
	Furthermore, we have $(-1)^{\dim(W)\dim(M \times \R^{2a})} = (-1)^{\dim(W)\dim(M)}$, so the two fiber product orientations agree in this case.

	Next we must generalize from $f_0 \colon V \to M \times \R^{2a}$ to the general case of $e \colon V \to M \times \R^{2a}$.
	By assumption, $f \colon V \to M$ is the composition of $e$ with the projection $M \times \R^{2a} \to M$, so we may write $e(v) = (f(v), e_{\R}(v))$, and there is a fiberwise homotopy $H \colon V \times I \to M \times \R^{2a}$ from $e$ to $f_0$ given by $H(v,t) = (f(v), te_{\R}(v))$.
	We note that $H$ and $e$ are each transverse to $g \times \id_{\R^{2a}}$.
	Indeed, if $e(v) = (g\times\id_{\R^{2a}})(w,z)$, then $(f(v),e_{\R}(v)) = (g(w),z)$.
	The image of the derivative of $g\times\id_{\R^{2a}}$ at such a point will span $Dg(T_wW) \oplus T(\R^{2a}) = Dg(T_wW) \oplus \R^{2a}$, while the derivative of $e$ will have the form $(Df,De_{\R})$.
	But the image of $D(g\times\id_{\R^{2a}})$ already spans $0\oplus \R^{2a}$, so $De+D(g\times\id_{\R^{2a}}) = (Df,0)+(Dg,0)+(0,\id_{\R^{2a}}) = T(M \times \R^{2a})$.
	The same argument holds for $H(-,t)$ for any fixed $t$, replacing $De_{\R}$ with $tDe_{\R}$.
	But if each $H(-,t)$ is transverse to $g\times\id_{\R^{2a}}$ then so is $H$.

	It follows that we can form the oriented fiber product of $H$ and $g\times\id_{\R^{2a}}$ over $M \times \R^{2a}$.
	In fact, this fiber product is diffeomorphic to $P \times I$: Noting that we have $H(v,t) = (f(v),te_{\R}(v)) = (g(w),z)$ if and only if $f(v) = g(w)$ and $te_{\R}(v) = z$, we obtain a diffeomorphism $P \times I \to (V \times I)\times_{M \times \R^{2a}}(W \times \R^{2a})$ given by $((v,w),t) \to ((v,t),(w, te_{\R}(v))$ with inverse $((v,t),(w,z)) \to ((v,w),t)$.
	The two ends of the cylinder correspond to our two versions of $V\times_{M \times \R^{2a}} (W \times \R^{2a})$, one mapping $V$ by $f_0$ and the other by $e$.
	Hence these spaces are oriented diffeomorphic, and canonically so by our construction.

	Putting this canonical diffeomorphism together with the one constructed above gives the desired oriented canonical diffeomorphism with the original $V \times_M W$.
\end{proof}

\begin{corollary}\label{C: oriented full transition to embedded}
	Let $f \colon V \to M$ and $g \colon W \to M$ be transverse maps from oriented manifolds with corners to an oriented manifold without boundary.
	Let $V\xhookrightarrow{r} M \times \R^{2a} \to M$ and $W\xhookrightarrow{s} M \times \R^{2b} \to M$ be factorizations of $f$ and $g$ with $r$ and $s$ embeddings.
	Then the oriented fiber product $V \times_M W$ is canonically oriented diffeomorphic to the fiber product $(V \times \R^{2b})\times_{M \times \R^{2a} \times \R^{2b}}(W \times \R^{2a})$ in which the first map is $r \times \id_{\R^{2b}}$ and the second map takes $(w,z) \in W \times \R^{2a}$ to $s(w)$ in the first and third coordinates and $z$ in the second coordinate.
\end{corollary}
\begin{proof}
	As in the proof of \cref{C: co-oriented full transition to embedded},
	we apply the preceding lemma to get $V \times_M W$ canonically oriented diffeomorphic to $V\times_{M \times \R^{2a}}(W \times \R^{2a})$.
	Then we use the graded commutativity rule for oriented fiber product as given by \cref{P: commute oriented fiber},
	by which
	the transposition map to $(W \times \R^{2a})\times_{M \times \R^{2a}}V$ is $(-1)^{(m-v)(m-w)}$-orientation preserving, using that all our Euclidean spaces are taken even-dimensional.
	Then we observe the map described for $W \times \R^{2a} \to M \times \R^{2a} \times \R^{2b}$ is an embedding whose composition with the projection to $M \times \R^{2a}$ is $g \times \id_{\R^{2b}}$.
	Now we apply the lemma again and then transpose again.
\end{proof}

\begin{proof}[Proof of \cref{P: compare cup and intersection orientations}]
	By \cref{C: co-oriented full transition to embedded,C: oriented full transition to embedded}, the proposition reduces to \cref{L: compare cup and intersection for immersions}.
\end{proof}

