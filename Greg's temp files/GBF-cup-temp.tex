\documentclass{amsart}
\usepackage{amssymb, amsmath, amsfonts, verbatim, mathtools, stmaryrd, tabularx}
\usepackage{bm}
\usepackage{mathbbol}
\usepackage{ulem} % Used in the To-Do list
\usepackage[all]{xy}
\usepackage{tikz-cd}
\usepackage{caption}
\usepackage{subcaption}
\usepackage{mathrsfs}
\usepackage{babel}
\usepackage{standalone}
\usepackage{tikz}
\usetikzlibrary{decorations.pathmorphing}
\tikzset{snake it/.style={decorate, decoration=snake}}

\usepackage[bookmarks=true, linktocpage=true,
	bookmarksnumbered=true, breaklinks=true,
	pdfstartview=FitH, hyperfigures=false,
	plainpages=false, naturalnames=true,
	colorlinks=true, pagebackref=true,
	pdfpagelabels]{hyperref}
\hypersetup{
	linkcolor=blue, citecolor=blue,
	urlcolor=blue, colorlinks=true,
}
\usepackage[capitalise,noabbrev,nosort]{cleveref}
\newcommand{\creflastconjunction}{, and\nobreakspace} %makes cleveref use serial commas


%\topmargin0.0cm
%\headheight0.0cm
%\headsep0.0cm
%\oddsidemargin0.0cm
%\textheight23.0cm
%\textwidth16.5cm
%\footskip1.0cm

\newtheorem{theorem}{Theorem}[section]
\newtheorem{lemma}[theorem]{Lemma}
\newtheorem{corollary}[theorem]{Corollary}
\newtheorem{proposition}[theorem]{Proposition}

\theoremstyle{definition}
\newtheorem{definition}[theorem]{Definition}
\newtheorem{remark}[theorem]{Remark}
\newtheorem{example}[theorem]{Example}
\newtheorem{convention}[theorem]{Convention}
\newtheorem{notation}[theorem]{Notation}

\newcommand{\defeq}{\stackrel{\mathrm{def}}{=}}
\newcommand{\codim}{\text{codim}}
\newcommand{\mc}[1]{\mathcal{#1}}
\newcommand{\ms}[1]{\mathscr{#1}}
\newcommand{\R}{\mathbb{R}}
\newcommand{\C}{\mathbb{C}}
\newcommand{\Q}{\mathbb{Q}}
\newcommand{\Z}{\mathbb{Z}}
\newcommand{\N}{\mathbb{N}}
\newcommand{\id}{\mathrm{id}}
\newcommand{\e}{\mathbf{e}}
\newcommand{\f}{\mathbf{f}}
\newcommand{\bb}{\mathbf{b}}
\newcommand{\interval}{\mathbb{I}}
\newcommand{\simplex}{\mathbb{\Delta}}
\newcommand{\I}{\square}
\newcommand{\chains}{C_*}
\newcommand{\chainsn}{C_n}
\newcommand{\cochains}{C^*}
\newcommand{\chain}[1]{C_{#1}}
\newcommand{\cochain}[1]{C^{#1}}
\newcommand{\cd}{{\rm cd}}
\DeclareMathOperator*{\colim}{colim}
\newcommand{\ori}{\mathfrak{or}}
\newcommand{\cor}{\mathfrak{cor}}
\newcommand{\sh}{\mathfrak{sh}}
\newcommand{\cs}{{\rm cst}}
\newcommand{\os}{{\rm gst}}
\newcommand{\Or}{{\rm Det}}
\renewcommand{\Im}{{\rm Im} \;}
\newcommand{\im}{\text{im}}
\newcommand{\into}{\hookrightarrow}
\newcommand{\init}{{\rm Init}}
\newcommand{\term}{{\rm Term}}
\newcommand{\sms}{\smallsmile}
\newcommand{\pf}{\pitchfork}
\newcommand{\bd}{\partial}
\newcommand{\td}{\tilde}
\newcommand{\Hom}{\textup{Hom}}
\newcommand\red[1]{\textcolor{red}{#1}}
\newcommand\blue[1]{\textcolor{blue}{#1}}
\newcommand\purple[1]{\textcolor{purple}{#1}}
\newcommand{\uW}{\underline{W}}
\newcommand{\udW}{\underline{\partial W}}
\newcommand{\uV}{\underline{V}}
\newcommand{\uM}{\underline{M}}
\newcommand{\uN}{\underline{N}}
\newcommand{\xr}{\xrightarrow}
\newcommand{\xl}{\xleftarrow}
\newcommand{\uX}{\underline{X}}
%\newcommand{\ker}{\rm ker}
\newcommand{\cok}{\rm cok}
\hyphenation{to-pol-o-gy}
\newcommand{\cman}{\mathrm{cMan}}
\newcommand{\Cre}{\mathrm{Cre}}
\newcommand{\aug}{\mathbf{a}}
\newcommand{\vertices}{{\rm Vert}} % vertices
\newcommand{\jinterval}{\mathbb{J}}
\newcommand{\onto}{\twoheadrightarrow}
\newcommand{\diag}{\mathbf{d}}
\newcommand{\Ext}{\text{Ext}}
\newcommand{\mf}{\mathfrak}

\begin{document}

 NOTE: CLARIFY $K_*(X)$ vs $K_*(M)$ IN PREVIOUS STUFF

ADD CONVENTION ABOUT TENSORS OF COCHAINS

SHOW UNIVERSAL HOMOTOPIES YIELD HOMOLOGY AND COHOMOLOGIES

SIMPLIFY BELOW USING That as chains all that degenerate crud vanishes and then just think of A as a geometric chain so can ignore all the crud.

NEED TO CHOOSE ORIENTATION OF TRACES SO THAT BOUNDARY WORKS OUT


\begin{lemma}\label{L: trans diag cube}
Let $M$ be a  without boundary and let $V$, $W$, and $Z$ be  manifolds with corners. Suppose $f:V\to M$ and $g:W\to M$ are transverse and that $f\times_Mg:V\times_MW\to M$ is transverse to $h:Z\to M$. Then $f\times g:V\times W\to M\times M$ is transverse to the composition $Z\xr{h}M\xr{\diag}M\times M$, where $\diag:M\to M\times M$ is the diagonal map.
\end{lemma}
\begin{proof}
We will apply \cref{L: transverse to pullback}. Consider the pullback diagram
	\[
	\begin{tikzcd}
		& (V \times W)\times_{M\times M} M \arrow[r, "\pi_{V\times W}"] \arrow[d, "\pi_M"] & V\times W \arrow[d, "f\times g"] \\
		Z \arrow[r, "h"] & M \arrow[r, "\diag"]&  M\times M.
	\end{tikzcd}
	\]
As $V$ and $W$ are transverse, we know that $f\times g:V\times W\to M\times M$ and $\diag:M\to M\times M$ are transverse by \cref{P: cross to cup}, and so by \cref{L: transverse to pullback}, the conclusion of the lemma holds if and only if $h$ is transverse to the projection $\diag^*(V\times W)=(V \times W)\times_{M\times M} M \to M$. But now by \cref{P: cross to cup} again, this pullback projection is precisely $f\times_Mg:V\times_MW\to M$. So the needed transversality holds by hypothesis.
\end{proof}



\begin{proposition}
Let $M$ be a smoothly cubulated manifold without boundary with cubulation $X$. Let $\uV,\uW\in H^*_\Gamma(M)$. Then the cohomology classes $\mc I(\uV\uplus\uW)\in H^*(K_X^*(M))$ and $\mc I(\uV) \smile \mc I(\uW)\in H^*(K_X^*(M))$ have the same image under the evaluation map $$H^*(K_X^*(M))\to \Hom(H_*(K_*^X(M)),\Z).$$
\end{proposition}

\begin{proof}
Let $Z$ be a cubical cycle representing an element of $H_*(K_*^X(M))$. As in \cref{S: cubical products}, we can consider $Z$ to also represent a smooth normalized singular cubical chain in $NK_*(M)$. We also have from that section a smooth chain homotopy of normalized singular cubes  in $M\times M$ between $\diag(Z)$ and $\zeta\xi\diag(Z)$. Let $Y$ be a smooth singular cubical chain representing this homotopy. Since $Z$ is a cycle, we have (up to choice of orientation) $\bd Y=\diag(Z)-\zeta\xi\diag(Z)+Q$, where $Q$ is a degenerate singular cubical chain. Recall that we can also think of $\diag(Z)$, $\zeta\xi\diag(Z)$, and $Q$ as representing elements of $PC^\Gamma_*(M\times M)$. For simplicity of notation, we write $\diag(Z)$, $\zeta\xi\diag(Z)$ to represent these manifolds with corners, each of which is a union of signed cubes mapping to $M\times M$.

In order to proceed from here, we need to establish chain and cochain representatives with sufficient transversality to begin computing intersections.

First, choose $f:V\to M$ and $g:W\to M$ representing $\uV$ and $\uW$. Let $h$ be the map representing the cubulation, though of as a disjoint union of top-dimensional cubes mapping to $M$. We can let $X$ denote this disjoint union of cubes mapping to $M$ and note that this contains the same information as the cubulation. By \cref{T: transverse complex}, we can assume that $g:W\to M$ is transverse to the cubulation. So $W\times_MX$ is well defined as a manifold with corners mapping to $M$ (ignoring (co)-orientation considerations for the moment).
Next, let us consider the union $W\sqcup X\sqcup (W\times_M X)$ as a single manifold with corners over $M$. By \cref{P: perturb transverse to map} there is a proper universal homotopy from $f$ to a map $f':V\to M$ that is transverse to this union; $f'$ represents the same cohomology class as $f$ by and application of \cref{L: dessicated homotopy} and the definition of universal homotopy. Abusing notation, let us rechristen this new transverse map $f:V\to M$.  By \cref{P: 3 out of 4 trans}, since $W$ is transverse to $X$ and $V$ is transverse to both $W$ and $W\times_MX$, we also have that $X$ is transverse to $V\times_M W$.

Next, consider $V\times W$ in $M\times M$. As $V$ and $W$ are each transverse to the cubulation $X$, we have $V\times W$ transverse to the product cubulation $X\times X$ by the proof of \cref{P: oriented interchange}.
Also, as $V$ and $W$ are transverse and as $V\times_MW$ is transverse to the cubulation $X$, \cref{L: trans diag cube} shows that $V\times W$ is transverse to the cubulation of the embedded diagonal  $\diag(X)$. So, as $Z$ is a cubical chain with respect to the cubulation $X$ and $\zeta\xi\diag(Z)$ is composed of cubes of the product cubulation $X\times X$, we see that the singular cubical chains $\diag(Z)$ and  $\zeta\xi\diag(Z)$ are transverse to $V\times W$. So now by \cref{P: perturb transverse to map}, there is a universal homotopy $H$ that takes $Y$ (now considered as an element of $PC^\Gamma_*(M\times M)$) to a $Y'$ that is transverse to $V\times W$ and such that the traces of the restrictions of the homotopy to $\diag(Z)$ and  $\zeta\xi\diag(Z)$
are also transverse to  $V\times W$.

We construct an element $A\in PC^\Gamma_*(M\times M)$ from the end of the homotopy and the traces of $\diag(Z)$ and $\zeta\xi\diag(Z)$. To simplify notation, recall from \cref{S: covariant functoriality} that if we have $r_U:U\to M$ representing an element that we call $U\in PC^\Gamma_*(M)$ and if we have a map $F:M\to N$, then we write $F(U)$ for the corresponding element of $PC^\Gamma_*(M)$, represented by the composition $U\xr{r_U}M\xr{F}M$.
Utilizing this notation, also letting $H_1=H(-,1):M\times M\to M\times M$, we let $A\in PC^\Gamma_*(M\times M)$ be the union of the following:
\begin{enumerate}
\item $Y'=H_1(Y)$, i.e.\ if $r_Y:Y\to M\times M$ represents $Y$ as an element of $PC^\Gamma_*(M)$, then $Y'$ is represented by  the composition $Y\xr{r_Y} M\times M\xr{H_1}M\times M$,

\item $-H(\diag(Z)\times I)$, i.e.\ the homotopy $\diag(Z)\times I\xr{r_{\diag(Z)}\times \id_I} M\times M\times I\xr{H}M\times M$ with its opposite orientation,

\item $H(\zeta\xi\diag(Z)\times I)$, i.e.\ the homotopy $\zeta\xi\diag(Z)\times I\xr{r_{\zeta\xi\diag(Z)}\times \id_I} M\times M\times I\xr{H}M\times M$.
\end{enumerate}
Then as observed above, $A$ is transverse to $V\times W$, and we next compute $\bd A$. We have
\begin{enumerate}
\item $\bd Y'=H_1(\bd Y)=H_1(\diag(Z)-\zeta\xi\diag(Z)+Q)=H_1(\diag(Z))-H_1(\zeta\xi\diag(Z))+H_1(Q)$,

\item $-\bd H(\diag(Z)\times I)=-H(\bd (\diag(Z)\times I))=-H_1(\diag(Z))+\diag(Z)-H((\bd\diag(Z))\times I)$,

\item $\bd H(\zeta\xi\diag(Z)\times I)=H(\bd (\zeta\xi\diag(Z)\times I))=H_1(\zeta\xi\diag(Z))-\zeta\xi\diag(Z)+H((\bd\zeta\xi\diag(Z))\times I)$.
\end{enumerate}
As $\bd Z=0$ as a cubical chain, it is trivial as a singular cubical chain or element of $PC_*^\Gamma(M)$, and so the terms involving $\bd\diag(Z)$ or $\bd\zeta\xi\diag(Z)$ are also trivial by \cref{L: dessicated homotopy}. Similarly, $H_1(Q)$ is trivial. So, in $C_*^\Gamma(X)$, we have $\bd Y=\bd A$.




Now, at last we can compute, treating $A$ as a manifold with corners. To fix our indices, we assume as usual that $\dim(M)=m$, $\dim(V)=v$, and $\dim(W)=w$, so $\dim(V\times_MW)= v+w-m$, which is the dimension of an $m-(v+w-m)=2m-v-w$ cochain. So we take $\dim(Z)=2m-v-w$, which implies $\dim(A)=\dim(Y)=2m-v-w+1$. So $\dim((V\times W)\times_{M\times M}A)=(v+w)+(2m-v-w+1)-2m=1$, and its boundary is $0$-dimensional and augments to $0$ by \cref{D: aug}. But, as $V$ and $W$ represent geometric cocycles,
this implies  $$\aug((V\times W)\times_{M\times M}\diag(Z))=\aug((V\times W)\times_{M\times M}\zeta\xi\diag(Z)).$$
We claim the former represents $(\mc I(\uV\uplus\uW))(Z)$ and the latter represents $(\mc I(\uV) \smile \mc I(\uW))(Z)$.

We first consider  $(V\times W)\times_{M\times M}\diag(Z)$, which is a cap product. By \cref{P: natural cap}, this is the same oriented manifold as $\diag^*(V\times W)\times_M Z$, which is $(V\times_MW)\times_M Z$ by \cref{P: cross to cup}.
So it follows from \cref{D: intersection homomorphism} and from the fact that $V\times_MW$ represents $\uV\uplus \uW$, that $$\aug((V\times W)\times_{M\times M}\diag(Z))=\aug((V\times_MW)\times_M Z)=\mc I(\uV\uplus \uW)(Z).$$

Next, consider $(\mc I(\uV) \smile \mc I(\uW))(Z)$. If $Z=\sum_i a_iE_i$, then by the discussion in \cref{S: cubical products}, we have
\begin{align*}
(\mc I(\uV) \smile \mc I(\uW))(Z)
&=\sum a_i (\mc I(\uV) \smile \mc I(\uW))(E_i)\\
&=\sum a_i\sum (\mc I(\uV) \otimes \mc I(\uW))  \rho_{H,K}A_H(E_i) \otimes B_K(E_i)\\
&=\sum a_i\sum \rho_{H,K}\mc I(\uV)(A_H(E_i))\cdot \mc I(\uW)(B_K(E))\\
&=\sum a_i\sum \rho_{H,K}\aug(V\times_M A_H(E_i))\aug(W\times_M B_K(E))\\
&=\sum a_i\sum \rho_{H,K}\aug((V\times W)\times_{M\times M}( A_H(E_i)\times B_K(E)))\\
&=\aug((V\times W)\times_{M\times M}\sum a_i\sum \rho_{H,K}( A_H(E_i)\times B_K(E)))\\
&=\aug((V\times W)\times_{M\times M}\zeta\xi\diag(Z)).\\
\end{align*}
Here the third equality is from the definition of the map $\mc I$ and the fourth utilizes \cref{P: cross product comparison}.
\end{proof}

CHECK CONVENTION ESTABLISHED FOR TENSOR PRODUCT OF COCHAINS ACTING ON CHAINS



We next recall that when $H^i(X)$ is finitely generated, there is an isomorphism $\lambda:T(H^i(X))\to \Hom(T(H_{i-1}(X));\Q/\Z)$, where $T(G)$ is the torsion subgroup of the abelian group $G$. A construction and proof can be found in \cite[Section 8.4.3]{Frie20}. The discussion there is in the context of intersection homology of singular spaces, but the relevant homological algebra applies in general for any cohomology groups coming from a cochain complex $C^*=\Hom(C_*,R)$, where $R$ is a Dedekind domain, $C_*$ is projective in each degree, and the homology and cohomology groups are all finitely generated. In this case, the map $\lambda$ can be described as follows. Let $\alpha$ be a cocycle representing an element of $T(H^i(X))$, and let $w$ be a cycle representing an element of $T(H_{i-1}(X))$. Since $w$ represents a torsion element of $H_{i-1}(X)$, there is a chain $z$ such that $\bd z=rw$ for some integer $r\neq 0$. Then $\lambda(\alpha)(w)=\frac{\alpha(z)}{r}$. It is shown in \cite{Frie20} that this construction is well defined, independent of the choices involved.

In the current setting we will apply this to cubical cohomology with integer coefficients, in which case the homological conditions are satisfied so long as each cohomology group --- or equivalently, each homology group --- is finitely generated.


\begin{proposition}
Let $M$ be a smoothly cubulated manifold without boundary with cubulation $X$, and suppose $H^i(X)$ is finitely generated for each $i$. Let $\uV,\uW\in H^*_\Gamma(M)$. Suppose the cohomology classes $\mc I(\uV\uplus\uW)$ and $\mc I(\uV) \smile \mc I(\uW)$ lie in $T(H^*(K_X^*(M)))$. Then they agree after composition with the map $$\lambda:T(H^*(K_X^*(M)))\to \Hom(H_{*-1}(K_*^X(M)),\Q/\Z).$$
\end{proposition}

UGH, WHAT IF ONE IS TORSION AND THE OTHER ISN'T?

--------------------------
\pagebreak

Restart


We now consider the relationship between the geometric cohomology cup product and the classical cup product, as mediated by our intersection map given a cubical structure on the manifold.


DIVIDE HERE


Our goal now is to prove the following theorem relating cup products in geometric cohomology to cup products in cubical cohomology via the intersection maps of \cref{D: intersection homomorphism}.
Recall our convention from \cref{R: intersection map extension} by which we abuse notation and also write $\mc I$ for the composition $H^*_\Gamma(M) \xleftarrow{\cong} H^*_{\Gamma \pf X}(M) \xr{\mc I} H^*(X)$.

\begin{theorem}\label{T: cup compatibility}
Let $M$ be a manifold without boundary cubulated by the cubical complex $X$.
The following diagram, in which the horizontal compositions are the geometric cohomology cup product and the cubical cohomology cup product, commutes:
	\[
	\begin{tikzcd}
	H^*_\Gamma(M) \otimes H^*_\Gamma(M) \arrow[r,"\times"]\arrow[d,"\mc I \otimes \mc I"]& H^*_\Gamma(M \times M) \arrow[d,"\mc I"]\arrow[r,"\diag^*"] & H^*_\Gamma(M) \arrow[d,"\mc I"]\\
	H^*(X) \otimes H^*(X) \arrow[r,"\times"]& H^*(X \times X) \arrow[r,"\Delta^*"] & H^*(X).
	\end{tikzcd}
	\]
\end{theorem}

The proof of \cref{T: cup compatibility} will rely on some further work.

For the next proposition, we need to further limit the geometric cochains we allow.
So let $M$ be a manifold without boundary cubulated by $X$, and recall that there is a natural chain homotopy between $\zeta \xi \colon NK^{sm}_*(M \times M) \to  NK^{sm}_*(M \times M)$ and the identity  \cite[Section XI.5]{Mas91}; see also \cref{R: smooth Massey}.
The naturality here means that these chain homotopies can be chosen compatibly across our category of manifolds with corners, and we will fix these as the homotopies in what follows.
So $\zeta\xi \diag$, which we denoted $\Delta$ in \cref{D: Delta}, is chain homotopic to $\diag \colon K_*(X) \to NK^{sm}_*(M \times M)$, thinking of $K_*(X)$ as a subcomplex of $NK_*^{sm}(M)$.
Let $P \colon K_i(X) \to NK^{sm}_{i+1}(M \times M)$ be the chain homotopy such that $\diag - \Delta = \bd P + P \bd$, coming from our fixed chain homotopy between $\zeta \xi$ and the identity.

We now define the subcomplex $C^*_{\Gamma \pf P}(M \times M) \subset C^*_{\Gamma \pf (X \times X)}(M \times M) \subset C^*_\Gamma(M \times M)$.
The definition is analogous to our definition of $C^*_{\Gamma \pf X}(M)$ in \cref{D: trans cube} except instead of just being transverse to the cubulation, we require
$PC^*_{\Gamma \pf P} (M \times M)$ to consist of those elements of $V \in PC^*_{\Gamma}(M \times M)$ such that
\begin{enumerate}
\item $V$ is transverse to each characteristic map of the cubulation $X \times X$ of $M \times M$,
\item $V$ is transverse to $\diag(E)$ for every $E$ in $X$, which also implies that $V$ is transverse to $\diag \colon M \to M \times M$,
\item $V$ is transverse to $P(E)$ for every $E$ in $X$, thinking of the cubical chain $P(E)$ as an element of $PC^\Gamma_*(M \times M)$ in the usual way.
\end{enumerate}

Due to the last condition, the maps that each $V$ must be transverse to are not all embeddings.
Nevertheless, a directly analogous construction to the discussion in \cref{D: trans cube} allows us to define $C^*_{\Gamma \pf P} (M \times M)$.

\begin{lemma}\label{L: P-transverse iso}
	The inclusions $C^*_{\Gamma \pf P} (M \times M) \to C^*_{\Gamma \pf X} (M \times M) \to C^*_\Gamma (M \times M)$ are quasi-isomorphisms.
\end{lemma}
\begin{proof}
	The proof that $C^*_{\Gamma \pf P} (M \times M)  \to C^*_\Gamma (M \times M)$ is a quasi-isomorphism is identical to that of \cref{T: transverse complex}, though since we do not need to be transverse only to faces of the cubulation, we use \cref{P: perturb transverse to map} rather than \cref{P: ball stability} to provide the needed transversality arguments.
	For this, we just need to create an appropriate $r_{\mf X}  \colon \mf X \to M$ so that precochains transverse to $\mf X$ will be in $PC^*_{\Gamma \pf P} (M \times M)$.
	So we let $\mf X$ be the disjoint union of
	\begin{enumerate}
		\item the cubes in $X \times X$,
		\item the cubes of $X$,
		\item for each cube $E$ in $X$, the domain of $P(E)$, which will be a collection of cubes, thinking of the cubical chain $P(E)$ as an element of $PC^\Gamma_*(M)$ in the usual way.
	\end{enumerate}
	The map $r_{\mf X}$ then consists of
	\begin{enumerate}
		\item the embeddings of the cubes of $X \times X$ into $M \times M$ via the cubulation,
		\item the embeddings of the cubes of $X$ into $M \times M$ via the cubulation of $M$ and the diagonal map $\diag$,
		\item for each cube $E$ in $X$, the map realizing $P(E)$ as a cubical chain.
	\end{enumerate}
	We note that $\mf X$ is a manifold with corners as it is a countable disjoint union of cubes.
	To see that $r_{\mf X}$ is proper, we observe that each $P(E)$ is a finite cubical chain and, furthermore, by the naturality of the chain homotopies in the proof of the Eilenberg-Zilber theorem through acyclic models (see \cite[Section XI.5]{Mas91}), $P(E)$ will be supported in $E \times E$.
	Therefore, each compact set of $M \times M$ will intersect the image of at most a finite number of the cubes of $\mf X$.

	So, by construction, transversality to $\mf X$ implies membership in $PC^*_{\Gamma \pf P} (M \times M)$, and we conclude inclusion $C^*_{\Gamma \pf P} (M \times M) \to C^*_\Gamma (M \times M)$ is a quasi-isomorphism by the proof of \cref{T: transverse complex} using \cref{P: perturb transverse to map} rather than \cref{P: ball stability}.
	As we already know $C^*_{\Gamma \pf X} (M \times M) \to C^*_\Gamma (M \times M)$ is a quasi-isomorphism by \cref{T: transverse complex}, the lemma follows.
\end{proof}

Next, we notice that the chain map $\mc I \colon C^*_{\Gamma \pf (X\times X)}(M\times M) \to K^*(X \times X)$ of \cref{D: intersection homomorphism} restricts to a chain map on $C^*_{\Gamma \pf P}(M\times M)$.
Furthermore, if we have any map $f \colon N \to M \times M$ that is transverse to every element of $C^*_{\Gamma \pf P}(M\times M)$, then we have a pullback chain map $f^* \colon C^*_{\Gamma \pf P}(M\times M) \to C^*_\Gamma(N)$ given by $f^*(\uV) = \underline{V \times_{M \times M} N}$ as a pullback mapping to $N$, where $V$ is any representative of $\uV$; cf.\ \cref{S: cohomology pullback} and the statement and proof of \cref{P: trans to f}.
In particular, this is the case for $\diag \colon M \to M \times M$, in which case we have a well defined chain map $\diag^* \colon C^*_{\Gamma \pf X\times P}(M\times M) \to C^*_\Gamma(M)$.
In fact, we can say a bit more: If $V \in  C^*_{\Gamma \pf P}(M\times M)$, then we know that $V$ is transverse to the composition $E \into M \xr{\diag} M \times M$ for any face $E$ of $X$.
So by \cref{L: transverse to pullback}, the pullback $V \times_{M \times M} M \to M$ is transverse to the inclusion $E \into M$.
So the image of $\diag^*$ is in the subcomplex $C^*_{\Gamma \pf X}(M)$.
Furthermore, $\diag^* \colon C^*_{\Gamma \pf P}(M \times M) \to C^*_{\Gamma \pf X}(M)$ is a chain map by \cref{leibniz}, as $M$ has no boundary.

This allows us to state our key proposition.

\begin{proposition}\label{P: diag/intersect}
	Let $M$ be a manifold without boundary, cubulated by the cubical complex $X$, and let $C^*_{\Gamma \pf P}(M \times M)$ be as defined above.
	The following diagram commutes up to chain homotopy:
	\[
	\begin{tikzcd}
	C^*_{\Gamma \pf P}(M \times M) \arrow[r,"\diag^*"] \arrow[d,"\mc I"]& C^*_{\Gamma \pf X}(M) \arrow[d,"\mc I"]  \\
	 \arrow[r,"\Delta^*"] K^*(X\times X)&K^*(X).
	\end{tikzcd}
	\]
\end{proposition}
\begin{proof}
	Let $V \in PC^*_{\Gamma \pf P}(M \times M)$ be a representative of $\uV \in C^*_{\Gamma \pf P}(M \times M)$, and let $E$ be a cubical face of $X$.
	We first compute as follows:
	\begin{align*}
		(\mc I \diag^*(V))(E) & = \aug(\diag^*(V) \times_M  E) &\text{\cref{D: intersection homomorphism,D: intersection number}}\\
		&=\aug(V \times_{M \times M} \diag(E)) &\text{\cref{P: natural cap}},
	\end{align*}
	where $\diag(E)$ denotes the map $E \into M \xr{\diag} M \times M$ in the notation of \cref{S: covariant functoriality}.
	On the other hand,
		\begin{align*}
		(\Delta^*\mc I(V))(E) & = \mc I(V)(\Delta (E))\\
		&=\aug(V \times_{M \times M} \Delta (E)).
	\end{align*}
	Note that since $\Delta (E) = \diag(E) - \bd P(E) + P \bd(E)$ and $V \in PC^*_{\Gamma \pf P}(M \times M)$, we have $\Delta(E)$ transverse to $V$, and all other needed transversality for the preceding computations follows from the definitions and discussion above.

	Next, we define $L \colon C^j_{\Gamma \pf P}(M \times M) \to K^{j-1}(X)$ by $$L(\uV)(E) = \aug(V \times_{M \times M} P(E))$$
	for any representative $V$ of $\uV$.
	This is well defined by arguments analogous to those for \cref{P: I is well defined}.
	Then we have
	\begin{align*}
		((dL + L\bd)(V))(E) &=(dL(V))(E) + ((L\bd)(V))(E) \\
			& = L(V)(\bd E) +L(\bd V)(E)\\
			& = \aug(V \times_{M \times M} P(\bd E)) + \aug((\bd V) \times_{M \times M} P(E))\\
			& = \aug(V \times_{M \times M} P(\bd E)) + (-1)^{v+e+1}\aug(\bd(V \times_{M \times M} P(E)))+(-1)^{v+e}\aug(V \times_{M \times M} \bd P(E))\\
			&=\aug(V \times_{M \times M} P(\bd E)) +\aug(V \times_{M \times M} \bd P(E))\\
			&=\aug(V \times_{M \times M} (P\bd + \bd P)(E))\\
			&=\aug(V \times_{M \times M} (\diag - \Delta)(E))\\
			&=\aug(V \times_{M \times M} (\diag(E))) - \aug(V \times_{M \times M} \Delta(E))\\
			&=(\mc I \diag^*(V))(E)- 	(\Delta^*\mc I(V))(E)\\
			&=((\mc I \diag^*- \Delta^*\mc I)(V))(E).
	\end{align*}
	Here in the fourth line we use \cref{P: Leibniz cap}, and in the fifth line we use that $\aug \circ \bd =0$ and that $\aug$ will be trivial unless $\dim(V)+ \dim(E) = \dim(M \times M) \equiv 0 \mod 2$.
	As $E$ and $V$ were arbitrary, we see that we have a chain homotopy between $\mc I \diag^*$ and $\Delta^*\mc I$.
\end{proof}

We obtain the following corollary.

\begin{corollary}\label{C: diag commute}
	The following diagram commutes:
	\[
	\begin{tikzcd}
		&H^*_{\Gamma}(M \times M) \arrow[r,"\diag^*"]& H^*_{\Gamma}(M) \\
	H^*_{\Gamma \pf (X \times X)}(M \times M)  \arrow[rd,"\mc I"']\arrow[ru,"\cong"] & \arrow[l,"\cong"']H^*_{\Gamma \pf P}(M \times M) \arrow[r,"\diag^*"] \arrow[d,"\mc I"] \arrow[u,"\cong"']& H^*_{\Gamma \pf X}(M) \arrow[d,"\mc I"] \arrow[u,"\cong"'] \\
		&H^*(K^*(X\times X))	\arrow[r,"\Delta^*"] & H^*(K^*(X)) .
	\end{tikzcd}
	\]
\end{corollary}
\begin{proof}
	The upper left triangle commutes because the maps are all induced by subcomplex inclusions, and the maps are all isomorphisms by \cref{L: P-transverse iso}.
	The bottom left triangle similarly commutes because $C^*_{\Gamma \pf P}(M \times M) \subset C^*_{\Gamma \pf (X \times X)}(M \times M)$ and, at the chain level, the map $\mc I$ on the right is a restriction of the $\mc I$ map on the left.
	The bottom right square commutes by \cref{P: diag/intersect}.
	Finally, the upper right square commutes at the chain level thanks to our observation above that the chain map $\diag^*$ takes the subcomplex $C^*_{\Gamma \pf P}(M \times M) \subset C^*_\Gamma(M \times M)$ to the subcomplex $C^*_{\Gamma \pf X}(M) \subset C^*_\Gamma(M)$.
\end{proof}

We can now prove \cref{T: cup compatibility}.

\begin{proof}[Proof of \cref{T: cup compatibility}]
We obtain the proof by concatenating the diagrams of \cref{P: cross product comparison} (after taking cohomology) and \cref{C: diag commute} and considering the rectangle formed on the outside.
\end{proof}



\begin{comment}
We also have the following.


\begin{lemma}
	The following diagram commutes:
	\[
	\begin{tikzcd}
		H^*_\Gamma(M) \otimes H^*_\Gamma(M) \arrow[r,"\times"] &H^*_{\Gamma}(M \times M) \\
		H^*_{\Gamma \pf X}(M) \otimes H^*_{\Gamma \pf X}(M) \arrow[r,"\times"]\arrow[d,"\mc I \otimes \mc I"]\arrow[u,"\cong"']  & H^*_{\Gamma \pf (X \times X)}(M \times M)  \arrow[d,"\mc I"]\arrow[u,"\cong"']  \\
		H^*(X) \otimes H^*(X) \arrow[r,"\times"]	&H^*(X\times X) .
	\end{tikzcd}
	\]
\end{lemma}
\begin{proof}
	As the product of two cochains transverse to the cubulation is transverse to the product cubulation PROVE/REF!!!, the diagram at the top commutes at the cochain level.
	For the bottom, we compute

\end{proof}
\end{comment}

\begin{comment}
\begin{proof}
	We consider the expanded diagram
	\[
	\begin{tikzcd}
		H^*_{\Gamma \pf (X \times X)}(M \times M) \arrow[rr,"\diag^*"] \arrow[dd,"\mc I"]&&H^*_{\Gamma \pf X}(M) \arrow[dd,"\mc I"]\\
		&H^*(C^*_{\Gamma \pf P}(M \times M))\arrow[ru,"\diag^*"]\arrow[ld,"\mc I"]\arrow[lu]\\
		H^*(K^*(X\times X)) \arrow[rr,"\Delta^*"] &&H^*(K^*(X)).
	\end{tikzcd}
	\]
	The bottom right quadrilateral commutes by \cref{P: diag/intersect}, while the two triangles commute because $C^*_{\Gamma \pf P}(M \times M)$ is a subcomplex of $C^*_\Gamma(M \times M)$ and the diagonal maps labeled $\diag^*$ and $\mc I$ come from restricting the maps on $C^*_\Gamma(M \times M)$.
	Finally, the upper left diagonal map is an isomorphism by \cref{L: P-transverse iso}, so the corollary follows by a diagram chase.
\end{proof}
\end{comment}

\begin{comment}
\begin{theorem}
	Let $M$ be a manifold without boundary with cubulation $X$. The following diagram commutes, where the cup product on the bottom is that of \cref{S: cubical products}
	\[
	\begin{tikzcd}
	H^*_\Gamma(M)\otimes H_\Gamma^*(M)\arrow[r,"\uplus"]& H^*_{\Gamma \pf X}(M) \\
	H^*_{\Gamma \pf X}(M)\otimes H_{\Gamma \pf X}^*(M) \arrow[u,"\cong"']  \arrow[d,"\mc I \otimes \mc I"] \arrow[r,"\uplus"]& H^*_{\Gamma \pf X}(M) \arrow[d,"\mc I"] \arrow[u,"\cong"'] \\
	H^*(K^*(X)) \otimes H^*(K^*(X))\arrow[r,"\smile"]& H^*(K^*(X)),
	\end{tikzcd}
	\]
\end{theorem}
\begin{proof}
Consider the diagram

\begin{tikzcd}
		H^*_\Gamma(M) \arrow[dd,"\mc I"] & &H^*_{\Gamma}(M \times M) \arrow[ll,"\diag^*"'] \arrow[dd,"\mc I"] \\
		&H^*(C^*_{\Gamma \pf P}(M \times M))\arrow[lu,"d^*"]\arrow[rd,"\mc I"]\arrow[ru]\\
		H^*(K^*(X)) && \arrow[ll,"\Delta^*"'] H^*(K^*(X\times X)).
	\end{tikzcd}

\end{proof}
\end{comment}

\end{document}
