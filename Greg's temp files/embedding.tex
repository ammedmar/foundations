\documentclass[12pt]{article}
\paperheight=11in
%\paperwidth=8.5in
\renewcommand{\baselinestretch}{1.04}
\usepackage{amsmath,amsthm,verbatim,amssymb,amsfonts,amscd, graphicx, mathrsfs, hyperref,mathtools,multicol, enumitem,bbm,cleveref}
\usepackage[usenames,dvipsnames]{color}
\usepackage{amstext}
%\usepackage{cite}
%\usepackage[notref]{showkeys}
\usepackage{color}



\newcommand\purple[1]{\marginpar{replacement}\textcolor{Purple}{#1}}     
\newcommand\blue[1]{\marginpar{new}\textcolor{blue}{#1}}                       %
\newcommand\red[1]{\marginpar{??}\textcolor{red}{#1}}                         %
\newcommand\green[1]{\marginpar{delete ok?}\textcolor{green}{#1}}
%\newcommand\ownremark[1]{\marginpar{remark for own use}\textcolor[rgb]{0.5,0.5,0}{#1}}                   %


\topmargin0.0cm
\headheight0.0cm
\headsep0.0cm
\oddsidemargin0.0cm
\textheight23.0cm
\textwidth16.5cm
\footskip1.0cm
\theoremstyle{plain}
\newtheorem{theorem}{Theorem}[section]
\newtheorem{corollary}[theorem]{Corollary}
\newtheorem{lemma}[theorem]{Lemma}
\newtheorem{proposition}[theorem]{Proposition}
\newtheorem*{theorem*}{Theorem}
\newtheorem*{lemma*}{Lemma}
\newtheorem*{proposition*}{Proposition}


\theoremstyle{definition}
\newtheorem{definition}[theorem]{Definition}

\theoremstyle{remark}
\newtheorem{remark}[theorem]{Remark}
\newtheorem{example}[theorem]{Example}
\newtheorem{exercise}[theorem]{Exercise}
\newcommand{\codim}{\text{codim}}
 

\newcommand{\primeset}[1]{#1}
\newcommand{\id}{\textup{id}}
\newcommand{\onto}{\twoheadrightarrow}
\newcommand{\hra}{\hookrightarrow} 
\newcommand{\Td}[1]{\Tilde{#1}}
\newcommand{\td}[1]{\tilde{#1}}
\newcommand{\into}{\hookrightarrow}
\newcommand{\PP}{\mathbb{P}}
\newcommand{\bS}{\mathbb{S}}
\newcommand{\X}{\mathbb{X}}
\newcommand{\Z}{\mathbb{Z}}
\newcommand{\Q}{\mathbb{Q}}
\newcommand{\R}{\mathbb{R}}
\newcommand{\G}{\mathbb{G}}
\newcommand{\N}{\mathbb{N}}
\newcommand{\F}{\mathbb{F}}
\newcommand{\C}{\mathbb{C}}
\newcommand{\D}{\mathbb{D}}

\renewcommand{\L}{\mathbb{L}}
\newcommand{\bd}{\partial}
\newcommand{\pf}{\pitchfork}
\newcommand{\ra}{\rightarrow}
\newcommand{\la}{\leftarrow}
\newcommand{\Ra}{\Rightarrow}
\renewcommand{\H}{\mathbb H}
\newcommand{\rla}{\RightLeftarrow}
\newcommand{\mc}[1]{\mathcal{#1}}
\newcommand{\ms}[1]{\mathscr{#1}}
\newcommand{\bb}[1]{\mathbb{#1}}
\newcommand{\dlim}{\varinjlim}
\newcommand{\vg}{\varGamma}
\newcommand{\blm}[2]{\langle  #1 , #2 \rangle}
\newcommand{\bl}[2]{\left( #1 , #2 \right)}
\newcommand{\vs}{\varSigma}
\newcommand{\holink}{\text{holink}}
\newcommand{\map}{\operatorname{map}}
\newcommand{\hl}{\operatorname{holink}}
\newcommand{\wt}{\widetilde}
\renewcommand{\hom}{\text{Hom}}
\newcommand{\Hom}{\text{Hom}}
\newcommand{\SHom}{\text{\emph{Hom}}}
\newcommand{\Ext}{\text{Ext}}
\newcommand{\mf}{\mathfrak}
\newcommand{\ih}{IH^{\bar p}}
\newcommand{\di}{\text{dim}}
\newcommand{\im}{\text{im}}
\newcommand{\cok}{\text{cok}}
\newcommand{\coim}{\text{coim}}
\newcommand{\bp}{\boxplus}
%\renewcommand{\P}{\mathbb P}
\newcommand{\q}{\mathfrak q}
\newcommand{\supp}{\text{supp}}
\newcommand{\singsupp}{\text{singsupp}}
\newcommand{\Dom}{\text{Dom}}
\newcommand{\LPDO}{\text{LPDO}}
\newcommand{\PsiDO}{\Psi\text{DO}}

\newcommand{\ka}{\kappa}

\newcommand{\fl}{\text{FL}}
\newcommand{\wfl}{\text{WFL}}
\newcommand{\Ker}{\mbox{Kernel }}
%\newcommand{\p}{\mf{p}}
\newcommand{\p}{\mathbbm{p}}
%\newcommand{\p}{\mathpzc{p}}
\newcommand{\Vol}{\text{Vol}}
\newcommand{\uW}{\underline{W}}
\newcommand{\udW}{\underline{\partial W}}
\newcommand{\uV}{\underline{V}}



\newcommand{\sect}[1]{\vskip1cm \noindent\paragraph{#1}}

\newcommand{\ttau}{\text{\texthtt}}

\newcommand{\xr}{\xrightarrow}
\newcommand{\xl}{\xleftarrow}

\DeclareRobustCommand{\zvec}[1]{%
  \mathrlap{\vec{\mkern-2mu\phantom{#1}}}#1%
}

\DeclareMathAlphabet{\mathpzc}{OT1}{pzc}{m}{it}
\newcommand{\cman}{\mathrm{cMan}}

\newcommand{\Or}{{\rm Det}}

\begin{document}


\begin{lemma}\label{L: pullback immersion}
	Suppose $f \colon V \to M$ and $g \colon W \to M$ are transverse maps from manifolds with corners to a manifold without boundary and that $f$ is an embedding. 
	Then the pullback map $f^* \colon V \times_M W \to W$ is a smooth immersion that takes $V \times_M W$ bijectively onto $g^{-1}(f(V))$.
\end{lemma}
\begin{proof}
	Let $P = V \times_M W$, which by definition is $\{ (v,w) \in V \times W \mid f(v) = g(w) \}$, and $f^*$ is just the projection to $W$. 
	As $f$ is injective, for any pair $(v,w) \in P$, the element $w$ uniquely determines $v$ as $v = f^{-1}(g(w))$, so the maps $(v, w) \mapsto w$ and $w \mapsto (f^{-1}(g(w)), w)$ establish a bijection between $P$ and $g^{-1}(f(V))$. 
	
	The map $f^*$ is smooth by \cite[Section 6]{Joy12}.
	
	Now, by \cref{L: tangent of pullbacks}, the tangent space of the pullback is the pullback of the tangents space, so if $(v,w) \in P$, then $T_{(v,w)}P = T_vV \times_{T_{f(v)}M} T_wW$. 
	Suppose $(a,b) \in T_{(v,w)}P$ with $a \in T_vV$ and $b \in T_wW$. 
	Let $\pi_V \colon P \to V$ and $\pi_W \colon P \to W$ be the restrictions of the projections to $P$, in which case we know that $\pi_W = f^*$, the pullback. 
	By assumption, $Dg_v(b) = Df_w(a)$. 
	So if $D\pi_W(a,b) = b = 0$, then we must have $Df_w(a)=0$, but this means $a = 0$ as $f$ is an embedding.
	So if $D\pi_W(a,b) = 0$, then $(a,b) = 0$, so $\pi_W = f^*$ is an immersion.
\end{proof}

\begin{corollary}
	Suppose $f \colon V \to M$ and $g \colon W \to M$ are transverse maps from manifolds with corners to a manifold without boundary and that $f$ is an embedding. 
	If the pullback map $f^* \colon V \times_M W \to W$ is proper, then it is an embedding onto a closed submanifold with corners of $W$.
\end{corollary}
\begin{proof}
By \cref{L: pullback immersion}, the map $f^*$ is injective, and every proper injective map is a closed map and a homeomorphism onto its image \cite[Proposition 10.1.2]{Bou98}.
\end{proof}


\begin{lemma}
With the assumptions of the preceding lemma, $f^*$ is a local embedding, i.e.\ for any $p \in V \times_M W$ there is a neighborhood $U$ such that $f^*|_U$ is an embedding.
\end{lemma}
\begin{proof}
We identify $V$ with its embedded image in $M$.
We also write $\pi$ rather than $f^*$ for simplicity of notation in what follows.
 
Let $p = (v,w) \in V \times_M W$, so $\pi(p)=w$. 
Let $(A, \alpha)$ and $(B, \beta)$ be charts whose images containing $p$ and $w$, respectively; by choosing a smaller $A$ we may assume $\pi(\alpha(A)) \subset \beta(B)$. 
By definition of smoothness \cite[Definition 3.1]{Joy12}, the map $\beta^{-1}\pi\alpha \colon A \to B$ is smooth, meaning that there are neighborhoods $A'$ and $B'$ of $A$ and $B$ in their respective Euclidean spaces such that $\beta^{-1}\pi\alpha$ extends to a smooth map $\pi': A' \to B'$. 
As $\pi$ is an immersion by 


Need to prove this week - use version for manifolds and local chart extension crap:
1. There are charts $A$ and $B$ so that $A \to V \times_M W \to W \to B$ is smooth. 
2. So there are ngbds $A'$ and $B'$ of $A$ and $B$ and $A' \to B'$ that is smooth in usual sense.
3. By assumption, derivative should be injective at $p$, so $A'$ has a nice embeddings in $B'$, possibly by making things smaller by usual manifold theory.  
So $A'$ is a homeo onto its image in B'.
Let $P'$, $W'$ be images of charts. 
Then we have 
P' -> W'
|     |
A  -> B
|     |
A' -> B'

bottom is homeo onto image, so restricts to A homeo onto image (should be able to cite this)
and top verticles are homeos, so top map is homeo onto image.
\end{proof}

\begin{corollary}
	Suppose $f \colon V \to M$ and $g \colon W \to M$ are transverse embeddings from manifolds with corners to a manifold without boundary. Then the fiber product $V \times_M W \to M$ is locally an embedding. 
\end{corollary}
\begin{proof}
By the preceeding lemma, the pullback $V \times_M W \to W$ is locally an embeddings, and the fiber product is the composition of the pullback with the embedding $g$.
\end{proof}


\begin{corollary}
	Suppose $f \colon V \to M$ and $g \colon W \to M$ are transverse maps from manifolds with corners to a manifold without boundary and that $f$ is an embedding. 
	If the pullback map $f^* \colon V \times_M W \to W$ is proper, then it is an embedding.
\end{corollary}
\begin{proof}
Mirror MO argument
\end{proof}


Idea for gxid pullback being good:

Lemma: V properly embeds in Euclidean space (generalize standard thing)

Lemma: if V maps to M x R and one component is proper, the whole map is proper.


Overall alternative idea: see if it's possible to emulate manifold proof

Other alternative idea: Embedd W in manifold with boundary

First idea: look in Margalef-Roig





----------

\begin{proposition}
Let $H \colon W \times I \to M$ and $f \colon V \to M$ be maps from manifolds with corners to a manifold without boundary. Suppose that $f$ is proper and that $g = H(-,0)$ are transverse. Suppose further that $H$ is a constant homotopy outside a compact set $K \subset W$. Then there is an $\epsilon>0$ such that $H(-,t)$ is transverse to $f$ for every $0\leq t <\epsilon$. 
\end{proposition}

\begin{proof}
We will show that for every point $x \in K$ there is an open neighborhood $U_x$ of $x$ in $W \times I$ such that $H(-,t)$ is transverse to $f$ at $x$ for all $(x,t) \in U_x$. 
Then $\cup_x U_x$ covers a neighborhood of $K$ in $W \times I$, and by the Tube Lemma (REF) there is therefore an $\epsilon$ such that $K \times [0,\epsilon) \subset \cup_x U_x$. 
As $H$ is constant in $t$ off of $K$ and $g = H(-,0)$ is transverse to $K$, this $\epsilon$ will suffice. 

So suppose $x \in K$. 
First, suppose $g(x) \cap f(V) = \emptyset$. 
By assumption, $V$ is a closed subset of $M$; thus we can take $U_x=H^{-1}(M-V)$.



Next suppose that $g(x) = f(y)$ for some $y \in V$. 
Let $z\in M$ be this common value. Using a chart to establish local coordinates for $M$ about $z$, we can identify $Df_x(T_xV)$ with a $v$-dimensional subspace of  
\end{proof}




\begin{lemma}\label{L: proper product}
	Let $f \colon X \to Y$ be an arbitrary map and $g \colon X \to Z$ a proper map. 
	Then the map $(f,g) \colon X \to Y \times Z$ given by $(f,g)(x)=(f(x),g(x))$ is proper.
\end{lemma}
\begin{proof}
	Let $K \subset Y \times Z$ be compact. 
	Let $\pi_Y,\pi_Z$ be the projectiond of $Y \times Z$ to $Y$ and $Z$. Then 
	\begin{align*}
		(f,g)^{-1}(K) &= \{x \in X \mid (f(x),g(x)) \in K \}\\
		&\subset \{ x \in X \mid f(x) \in \pi_X(K), g(x) \in \pi_Y(K)\}\\
		&\subset \{ x \in X \mid g(x) \in \pi_Y(K)\}\\
		&= g^{-1}(\pi_Y(K)).
	\end{align*} 
	As $K$ is compact and $g$ is proper, $g^{-1}(\pi_Y(K))$ is compact. 
	So $(f,g)^{-1}(K)$ is a closed subset of a compact set, and so compact.
\end{proof}

\begin{corollary}\label{C: embed V}
Let $g \colon W \to M$ be a map from a manifold with corners to a manifold without boundary. 
Then for some $N$ there exists a closed embedding $e \colon W \to M \times \R^N$ such that, if $\pi \colon M \times \R^N \to M$ is the projection, then $\pi e = g$. 
\end{corollary}
\begin{proof}
By \cite[Corollary 11.3.10]{MaDo92}, there is a closed embedding $\alpha \colon W \to \R^N$ for some $N$. 
Then by \cref{L: proper product} the map $(g,\alpha)$ is proper, and as $\alpha$ is an immersion, so is $(g,\alpha)$.
In fact, by \cite[Theorem 3.2.6]{MaDo92} and the fact that $M$ and $\R^N$ do not have boundary, a map is an immersion in this context if and only if it is injective on tangent spaces, and the map $\pi_{\R^N}(g,\alpha)$ equals the embedding $g$, so $D(g,\alpha)$ must be injective at each point.
Similarly, the map $(g,\alpha)$ is itself  injective.  
But just as for ordinary manifolds, an injective proper immersion is a closed embedding \cite[Proposition 3.3.4]{MaDo92}, and it is clear from the construction that $\pi (g,\alpha) = g$.
\end{proof}


\begin{lemma}
	The fiber product of two transverse immersions $f \colon V \to M$ and $g \colon W \to M$ from manifolds with corners to a manifold without boundary is an immersion and so, locally in a neighborhood of each point, an embedding.
\end{lemma}
\begin{proof}
	By \cref{L: tangent of pullbacks}, the tangent space of the pullback is the pullback of the tangents space, so if $(v,w) \in P$, then $T_{(v,w)}P = T_vV \times_{T_{f(v)}M} T_wW$. 
	Suppose $(a,b) \in T_{(v,w)}P$ with $a \in T_vV$ and $b \in T_wW$. 
	Let $\pi_V \colon P \to V$ and $\pi_W \colon P \to W$ be the restrictions to $P$ of the projections from $V \times W$.
	Then, suppressing the points for the remainder of the argument, $D(f \times_M g) = D(g\pi_W) = D(f \pi_V)$.
	But $D(g \pi_W)(a,b)=Dg\circ D\pi_W (a,b)=Dg(b)$, and $Dg$ is injective as $g$ is an immersion. 
	So if $D(g \pi_W)(a,b)=0$, then $b=0$, and also $a=0$ by the same computation with $D(f \pi_V)$.
	So $D(f \times_M g)$ is injective, and $f \times_M g$ is an immersion by \cite[Proposition 3.2.6]{MaDo92}.
	It is therefore a local embedding by \cite[Proposition 3.2.13]{MaDo92}.  
\end{proof}

\begin{corollary}
Suppose $f \colon V \to M$ and $g \colon W \to M$ are transverse immersions of manifolds with corners to a manifold without boundary. Suppose $v \in V$ and $w \in W$ with $f(v) = g(w)$. 
Then there are open neighborhoods $U_v$ of $v$ in $V$ and $U_w$ of $w$ in $W$ such that the fiber product $f|_{U_v} \times_M g|_{U_w}$ is a local embedding onto $f(U_v) \cap g(U_w)$. 
\end{corollary}

\begin{proof}
In general, if $a: X \to Z$ and $b: Y \to Z$ are injective maps, then $X \times_Z Y = \{(x,y) \mid f(x) = g(y)\} = \{ z \in a(x) \cap b(Y) \}$. 
By \cite[Proposition 3.2.13]{MaDo92}, $f$ and $g$ are local embeddings, so we can choose neighborhoods $U_v$ and $U_w$ on which $f$ and $g$ are embeddings. 
Now applying the preceding lemma to $f|_{U_v}$ and $g|_{U_w}$, the map $f|_{U_v} \times_M g|_{U_w}$ from $U_v \times_M U_w$ to $M$ has image $f(U_v) \cap (U_w)$ and restricts to an embedding in a neighborhood of each point. 
\end{proof}


\bibliographystyle{amsplain}
\bibliography{../../bib}




\end{document}
