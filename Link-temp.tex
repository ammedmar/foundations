\documentclass[12pt]{article}
\paperheight=11in
\paperwidth=8.5in
\renewcommand{\baselinestretch}{1.05}
\usepackage{amsmath,amsthm,verbatim,amssymb,amsfonts,amscd,diagrams, graphicx, mathrsfs, mathtools}
\usepackage[usenames,dvipsnames]{color}
\usepackage{amstext}
%\usepackage{cite}
%\usepackage[notref]{showkeys}
\usepackage{color}
\usepackage{hyperref}
\usepackage{cleveref}
\usepackage{tikz-cd}

\newcommand\purple[1]{\marginpar{replacement}\textcolor{Purple}{#1}}
\newcommand\blue[1]{\marginpar{new}\textcolor{blue}{#1}}                       %
\newcommand\red[1]{\marginpar{??}\textcolor{red}{#1}}                         %
\newcommand\green[1]{\marginpar{delete ok?}\textcolor{green}{#1}}
%\newcommand\ownremark[1]{\marginpar{remark for own use}\textcolor[rgb]{0.5,0.5,0}{#1}}                   %


\topmargin0.0cm
\headheight0.0cm
\headsep0.0cm
\oddsidemargin0.0cm
\textheight23.0cm
\textwidth16.5cm
\footskip1.0cm


\setcounter{tocdepth}{3}

\theoremstyle{plain}
\newtheorem{theorem}{Theorem}[section]
\newtheorem{corollary}[theorem]{Corollary}
\newtheorem{lemma}[theorem]{Lemma}
\newtheorem{sublemma}[theorem]{Sublemma}
\newtheorem{proposition}[theorem]{Proposition}

\theoremstyle{definition}
\newtheorem{definition}[theorem]{Definition}

\theoremstyle{remark}
\newtheorem{remark}[theorem]{Remark}
\newtheorem{example}[theorem]{Example}
\newcommand{\codim}{\textup{codim}}
\newarrow{ul}---->
\newarrow{Backwards}<----

\newcommand{\uW}{\underline{W}}
\newcommand{\udW}{\underline{\partial W}}
\newcommand{\uV}{\underline{V}}
\newcommand{\uM}{\underline{M}}
\newcommand{\uN}{\underline{N}}

\newcommand{\ilim}{\varprojlim}
\newcommand{\Lk}{\textup{Lk}}
\newcommand{\xr}{\xrightarrow}
\newcommand{\xl}{\xleftarrow}
\newcommand{\onto}{\twoheadrightarrow}
\newcommand{\hra}{\hookrightarrow}
\newcommand{\Td}[1]{\Tilde{#1}}
\newcommand{\td}[1]{\tilde{#1}}
\newcommand{\mr}[1]{\mathring{#1}}
\newcommand{\ul}[1]{\underline{#1}}
\newcommand{\into}{\hookrightarrow}
\newcommand{\Z}{\mathbb{Z}}
\newcommand{\X}{\mathbb{X}}
\newcommand{\Q}{\mathbb{Q}}
\newcommand{\R}{\mathbb{R}}
\newcommand{\G}{\mathbb{G}}
\newcommand{\N}{\mathbb{N}}
\newcommand{\F}{\mathbb{F}}
\newcommand{\C}{\mathbb{C}}
\newcommand{\D}{\mathbb{D}}
\renewcommand{\L}{\mathbb{L}}
\newcommand{\bd}{\partial}
\newcommand{\pf}{\pitchfork}
\newcommand{\ra}{\rightarrow}
\newcommand{\la}{\leftarrow}
\newcommand{\Ra}{\Rightarrow}
\renewcommand{\H}{\mathbb H}
\newcommand{\rla}{\RightLeftarrow}
\newcommand{\mc}[1]{\mathcal{#1}}
\newcommand{\ms}[1]{\mathscr{#1}}
\newcommand{\bb}[1]{\mathbb{#1}}
\newcommand{\dlim}{\varinjlim}
\newcommand{\vg}{\varGamma}
\newcommand{\blm}[2]{\langle  #1 , #2 \rangle}
\newcommand{\bl}[2]{\left( #1 , #2 \right)}
\newcommand{\vs}{\varSigma}
\newcommand{\holink}{\textup{holink}}
\newcommand{\map}{\operatorname{map}}
\newcommand{\hl}{\operatorname{holink}}
\newcommand{\wt}{\widetilde}
\renewcommand{\hom}{\textup{Hom}}
\renewcommand{\S}{\mathbb{S}}
\newcommand{\Hom}{\textup{Hom}}
\newcommand{\SHom}{\textup{\emph{Hom}}}
\newcommand{\Ext}{\textup{Ext}}
\newcommand{\mf}{\mathfrak}
\newcommand{\ih}{IH^{\bar p}}
\newcommand{\di}{\textup{dim}}
\newcommand{\im}{\textup{im}}
\newcommand{\cok}{\textup{cok}}
\newcommand{\coim}{\textup{coim}}
\newcommand{\bp}{\boxplus}
\renewcommand{\i}{\mathfrak i}
\renewcommand{\j}{\mathfrak j}
\newcommand{\ka}{\kappa}
\renewcommand{\Cup}{\smile}
\renewcommand{\Cap}{\frown}
\newarrow{Onto}----{>>}
\newarrow{Equals}=====
\newcommand{\fl}{\textup{FL}}
\newcommand{\wfl}{\textup{WFL}}
\newcommand{\Ker}{\mbox{Kernel }}
\newcommand{\sgn}{\textup{sgn}}
\newcommand{\sect}[1]{\vskip1cm \noindent\paragraph{#1}}
\newcommand{\id}{\textup{id}}
\newcommand{\IAW}{\textup{IAW}}
\newcommand{\pt}{\textup{pt}}
\newcommand{\diag}{\mathbf{d}}
\newcommand{\aug}{\mathbf{a}}

\begin{document}

\begin{example}
In this extended example we will use the cohomology Mayer-Vietoris sequence to find geometric generators for the cohomology of a link complement in $S^3$.

Let $L = \amalg_{a \in A} L_a$ be a finite collection of disjoint embedded circles in $S^3$, and let $M = S^3 - L$.
We take each $L_a$ to be oriented as determined by its embedding, giving the domain $S^1$ its standard orientation.
Let $U_a$ be an open tubular neighborhood of $L_a$; we may assume that the $U_a$ are also pairwise disjoint and that each is oriented as a subspace of $S^3$.
Then each $U_a$ is diffeomorphic to $S^1 \times B^2$, with $B^2$ being the open $2$-disk, and we can identify each $U_a - L_a$ with $S^1 \times S^1 \times (0,1)$.
We choose these diffeomorphisms so the $(0,1)$ factor runs outward from the link, $S^1 \times \phi \times t$ is a parallel to $L_a$ with the parallel orientation that does not link $L_a$, and $\theta \times S^1 \times t$ is a meridian of $L_a$, oriented according to the right hand rule: if one curves the fingers of their right hand around the meridian with fingers pointing in the direction of its orientation, then the thumb will point along the direction of $L_a$.
If $\ell$ and $m$ denote an oriented longitude and meridian and $\beta_\ell$ and $\beta_m$ are their orientations, then this means that $\beta_\ell \wedge \beta_m \wedge \beta_{(0,1)}$, with the standard orientation on $(0,1)$, will be the \emph{opposite} of the orientation of $S^3$.

We first need to build up some information about the generators of the cohomology of $U_a$ and $U_a - L$, which starts by considering $S^1$ and $S^1 \times S^1$.
By \cref{E: cohomology of spheres}, we know that $H^0_\Gamma(S^1) \cong \Z$, and we can take the identity map $\id \colon S^1 \to S^1$ with its tautological co-orientation to represent the generator; this is equivalent to the normal co-orientation assigning the positive co-orientation to the $0$-dimensional normal bundle.
We also know $H^1_\Gamma(S^1) \cong \Z$, and we can take as generator the inclusion of a point; let us suppose the co-orientation is the normal co-orientation in which the orientation of the normal bundle coincides with the orientation of $S^1$.
By the K\"unneth Theorem (\cref{T: cohomology kunneth}), and using our standard abuse of allowing the domain to represent the map, it therefore follows that $H^1_\Gamma(S^1 \times S^1) \cong \Z \oplus \Z$ is generated by $S^1 \times pt$ and $pt \times S^1$, while $H^2_\Gamma(S^1 \times S^1) \cong \Z$ is generated by $pt \times pt$.
By \cref{L: Quillen product co-orientation}, both $S^1 \times pt$ and $pt \times S^1$ have normal co-orientations such that the normal bundle is oriented consistently with its corresponding factor; e.g.\ the normal bundle of $S^1 \times pt$ is oriented so that identifying a fiber $\theta \times \R$ in the obvious way as a subset of $\theta \times S^1$, the orientations of $\R$ and $S^1$ agree.
The map $pt \times pt \to S^1 \times S^1$ is normally co-oriented by the orientation of the normal bundle that agrees with the product orientation of $S^1 \times S^1$.

Next, we consider the deformation retractions $U_a \to L_a \cong S^1$ and $U_a - L_a \to L_a \times S^1$.
Pulling back by the first homotopy equivalence, we have $H^1_\Gamma(U_a)$ represented by $pt \times B^2$ with the normal bundle oriented consistently with the orientation of $L_a$, and  $H^2_\Gamma(U_a) = H^3_\Gamma(U_a) = 0$. Similarly, pulling back by the second retraction, we have $H^1_\Gamma(U_a - L_a)$ generated by the classes of $S^1 \times pt \times (0,1)$ and $pt \times S^1 \times (0,1)$, while $H^2_\Gamma(U_a - L_a)$ is generated by the class of $pt \times pt \times (0,1)$.
In all cases, the co-orientation is given by the oriented pullback of the normal bundle on $S^1$ or $L_a \times S^1$.

We can now turn to the cohomology of $M$ itself.
As $M$ is connected but not closed, we already know from \cref{E: first examples} that $H^3_\Gamma(M) = 0$ and $H^0_\Gamma(M) \cong \Z$ with generator represented by the identity map.
So it remains to consider $H^1_\Gamma(M)$ and $H^2_\Gamma(M)$.

Consider the Mayer-Vietoris sequence for the pair $M = S^3 - L$ and $\amalg U_a$, with union $S^3$ and intersection $\amalg (U_a - L_a)$.
Since we know that $H^1_\Gamma(S^3) = H^2_\Gamma(S^3)=0$, we obtain from the Mayer-Vietoris sequence an isomorphism
$$H^1_\Gamma(\amalg U_a) \oplus H^1_\Gamma(S^3 - L) \xr{j} H^1_\Gamma( \amalg(U_a - L_a)).$$
We also know that $H^3_\Gamma(M) = H^3_\Gamma(\amalg U_a) = H^2_\Gamma(\amalg U_a) = 0$, so we have an exact sequence
$$0 \to H^2_\Gamma(S^3 - L) \xr{j} H^2_\Gamma( \amalg(U_a - L_a)) \xr{\delta} H^3_\Gamma(S^3) \to 0.$$

Let us first consider the isomorphism $H^1_\Gamma(\amalg U_a) \oplus H^1_\Gamma(S^3 - L) \xr{j} H^1_\Gamma( \amalg(U_a - L_a)).$
As observed above, each $H^1_\Gamma(\amalg U_a)$ can be taken to be generated by the class of an inclusion
$pt \times B^2 \into S^1 \times B^2$, and these maps restrict to inclusions $pt \times S^1 \times (0,1)  \into S^1 \times S^1 \times (0,1) \cong U_a - L_a$.
This accounts for half the summands of $H^1_\Gamma( \amalg(U_a - L_a))$.
The remainder of $H^1_\Gamma( \amalg(U_a - L_a))$ must therefore come from $H^1_\Gamma(M)$, and if we can find maps representing elements of $H^1_\Gamma(M)$ that restrict to $S^1 \times pt \times (0,1)  \into S^1 \times S^1 \times (0,1) \cong U_a - L_a$, then these must be linearly independent and generate $H^1_\Gamma(M)$.
But now recall that any knot $K$ in $S^3$ possesses a Seifert surface, i.e.\ a compact oriented embedded manifold whose boundary is $K$ \cite{REF FOR SMOOTH}.
Seifert surfaces are usually assumed to be connected, but that will not be necessary for us.
So given a component $L_a$ of our link, let $S_a$ be a Seifert surface for it; we may also assume that $S_a$ is transverse to all of the other components of $L$.
If we let $\hat S_a = S_a - (S_a \cap L)$, then the embedding of $\hat S_a$ into $S^3 - L$, with the co-orientation determined by the orientations of $S_a$ and $S^3$, is a precochain.

Furthermore, if we consider a closed collar of $\bd S_a$, which we can write as $L_a \times [0,1]$ with $L_a \times 0 = L_a$, then $L_a \times 1$ is a parallel copy of $L_a$.
Furthermore, as $S_a$ is assumed to be embedded in $S^3$, it must be the case that $S_a - (L_a \times [0,1))$, whose boundary is $L_a \times 1$, does not intersect $L_a$. So $L_a$ and its translate $L_a \times 1$ do not link each other (see \cite[Section 5.D]{Ro76} for a general discussion of linking numbers).
Consequently, we can identify $U_a$ with $S^1 \times B^2$ in such a way that $S_a \cap (U_a-L_a) \cong S^1 \times pt \times (0,1)$.
So the cochains represented by the $\hat S_a$ must generate $H^1_\Gamma(M)$.

To be a bit more specific for future use, let $\beta_{L_a}$ be the orientation of $L_a$. Then by \cref{Con: oriented boundary}, the orientation $\beta_{S_a}$ of $S_a$ will be such that $\beta_{\nu_a} \wedge \beta_{L_a} = \beta_{S_a}$, where $\nu_a$ is the outward pointing normal to $S_a$ at the boundary.
So at such a point the induced co-orientation of the embedding of $S_a$ is $(\beta_{S_a},\beta_{S^3}) = (\beta_{\nu_a} \wedge \beta_{L_a},\beta_{S^3})$.
Thus if we let $n_a$ be a normal vector to $S_a$, the corresponding normal co-orientation of $S_a$ is such that $\beta_{\nu_a} \wedge \beta_{L_a} \wedge \beta_{n_a} = \beta_{S^3}$.
So we see the normal co-orientation is equivalent to the standard righthand rule for orienting the normal to a surface that cobounds an oriented curve in $\R^3$, and this agrees with our convention for generators of $H^1_\Gamma(U_a-L_a)$.

Next we consider the sequence for $H^2_\Gamma$.
As $H^3_\Gamma(S^3) \cong \Z$ and $H^2_\Gamma(\amalg(U_3-L_3)) \cong \Z^{|A|}$, the short exact sequence above splits, and we must have abstractly that $H^2_\Gamma(M) \cong \Z^{|A|-1}$.
We claim in this case that $H^2_\Gamma(M)$ is generated by the classes of smooth curves that run from one link component to another.
More precisely, suppose that $\gamma_{ab} \colon [0,1] \to S^3$ is a curve with $\gamma_{ab}(0) \in L_a$, $\gamma_{ab}(1) \in L_b$, and $\gamma_{ab}((0,1)) \cap L = \emptyset$.
By restricting the domain to $(0,1)$ and using the co-orientation induced by the standard orientations of $(0,1)$ and $S^3$, we obtain a precochains $\hat \gamma_{ab}$.
Now recall that $H^2_\Gamma(U_a-L_a)$ is generated by the class of $pt \times pt \times (0,1)$.
We can clearly arrange $\hat \gamma_{ab}$ and our identification of $U_a - L_a$ with $S^1 \times S^1 \times (0,1)$ so that the restriction of $\hat \gamma_{ab}$ to $U_a - L_a$ has this form.
The co-orientation of $\gamma_{ab}$ coming from the orientations is $(\beta_{(0,1)},\beta_{S^3})$, but near $L_a$, we have $\beta_{S^3} = \beta_{U_a - L_a} = \beta_{\ell_a} \wedge \beta_{m_a} \wedge \beta_{(0,1)} = \beta_{(0,1)} \wedge \beta_{\ell_a} \wedge \beta_{m_a}$.
So near $L_a$, this is precisely the normal co-orientation of $pt \times pt \times (0,1)$ coming from the pullback to $U_a - L_a$ of the generator $pt \times pt$ of $H^2(S^1 \times S^1)$.



So if we write these generators of $H^2_\Gamma(\amalg(U_a-L_a))$ as $\underline{x_a}$ and choose consistent co-orientations, we see that $j(\underline{\hat \gamma_{ab}}) = \underline{x_a} - \underline{x_b}$.
Now let us order our indexing set $A$ and relabel as $A = \{1, \ldots , n\}$.
We then have that the set $\mc G = \{\underline{\hat \gamma_{i,i+1}}\}_{i=1}^{n-1}$ maps onto the set $\{\underline{x_i}-\underline{x_{i+1}}\}_{i=1}^{n-1}$, which spans a subgroups of $H^2_\Gamma(\amalg(U_a-L_a)) \cong \Z^n$ such that the quotient is $\Z$.
Thus the elements of $\mc G$ generate $H^2_\Gamma(M) \cong \Z^{n-1}$.
We also see from the exact sequence that $\underline{\hat \gamma_{i,i+1}}+\underline{\hat \gamma_{i+1,i+2}} = \underline{\hat \gamma_{i, i+2}}$, as their difference maps to $0$ but $j$ is injective.
More generally, it follows that:
\begin{enumerate}
\item $\underline{\hat \gamma_{ab}} + \underline{\hat \gamma_{bc}} =\underline{\hat \gamma_{a,c}}$, for any $a,b,c \in A$,
\item $\underline{\hat \gamma_{ab}} = -\underline{\hat \gamma_{ba}}$,
\item $\underline{\hat \gamma_{aa}}=0$.
\end{enumerate}


So, in summary, the generators of $H^1_\Gamma(M)$ can be represented by Seifert surfaces, while the elements of $H^2_\Gamma(M)$ can be realized as arcs between neighboring components, subject to the above relations.

Of course the real power of geometric cohomology is that we can also derive consequences for the \emph{ring} $H^*_\Gamma(M)$ from geometry.
For example, if it is possible to choose Seifert surfaces $S_a$ simultaneously for all $a$ that are disjoint or intersect only in closed curves, then all cup product on $H^1_\Gamma(M)$ vanish.
On the other hand, if two Seifert surfaces $S_a$ and $S_b$ intersect in just an arc between $L_a$ and $L_b$, then this corresponds to the cup product of two generators of $H^1_\Gamma(M)$ being a generator of $H^2_\Gamma(M)$.


\end{example}

\begin{comment}
\begin{example}
In this extended example we will use the cohomology Mayer-Vietoris sequence to find geometric generators for the cohomology of a link complement in $S^3$.
So let $L = \amalg_{a \in A} L_a$ be a finite collection of disjoint embedded circles in $S^3$, and let $M = S^3 - L$.
We take each $L_a$ to be oriented as determined by its embedding, giving the domain $S^1$ its standard orientation.
Let $U_a$ be an open tubular neighborhood of $L_a$; we may assume that the $U_a$ are also pairwise disjoint and that each is oriented as a subspace of $S^3$.
Then each $U_a$ is diffeomorphic to $S^1 \times D^2$, with $B^2$ being the open $2$-disk, and we can identify each $U_a - L_a$ with $S^1 \times S^1 \times (0,1)$.
We choose these diffeomorphisms so the $(0,1)$ factor runs outward from the link, $S^1 \times \phi \times t$ is a parallel to $L_a$ with the parallel orientation that does not link $L_a$, and $\theta \times S^1 \times t$ is a meridian of $L_a$, oriented according to the right hand rule: if one curves the fingers of their right hand around the meridian with fingers pointing in the direction of the orientation, then the thumb will point along the direction of the $L_a$.
If $\ell$ and $m$ denote an oriented longitude and meridian and $\beta_\ell$ and $\beta_m$ are their orientations, then this means that $\beta_\ell \wedge \beta_m \wedge \beta_{(0,1)}$, with the standard orientation on $(0,1)$, will be the \emph{opposite} of the orientation of $S^3$.

We first need to build up some information about the generators of the cohomology of $U_a$ and $U_a - L$, which starts by considering $S^1$ and $S^1 \times S^1$.
By \cref{E: cohomology of spheres}, we know that $H^0_\Gamma(S^1) \cong \Z$, and we can take the identity map $\id \colon S^1 \times S^1$ with its tautological co-orientation to represent the generator; this is equivalent to the normal co-orientation assigning the positive co-orientation to the $0$-dimensional normal bundle.
We also know $H^1_\Gamma(S^1) \cong \Z$, and we can take as generator the inclusion of a point; let us suppose the co-orientation is the normal co-orientation induced by the standard orientation of $S^1$.
By the K\"unneth Theorem (\cref{T: cohomology kunneth}) and using our standard abuse of allowing the domain to represent the map, it therefore follows that $H^1_\Gamma(S^1 \times S^1) \cong \Z \oplus \Z$ is generated by $S^1 \times pt$ and $pt \times S^1$, while $H^2_\Gamma(S^1 \times S^1) \cong \Z$ is generated by $pt \times pt$.
By \cref{L: Quillen product co-orientation}, both $S^1 \times pt$ and $pt \times S^1$ have normal co-orientations such that the normal bundle is oriented consistently with its corresponding factor; e.g.\ the normal bundle of $S^1 \times pt$ is oriented so that identifying a fiber $\theta \times \R$ in the obvious way as a subset of $\theta \times S^1$, the orientations of $\R$ and $S^1$ agree.
The map $pt \times pt \to S^1 \times S^1$ is normally co-oriented by the orientation of the normal bundle that agrees with the product orientation of $S^1 \times S^1$.

Next, we consider the deformation retractions $U_a \to L_a \cong S^1$ and $U_a - L_a \to L_a \times S^1$.
Pulling back by the first homotopy equivalence, we have $H^1_\Gamma(U_a)$ represented by $pt \times B^2$ with the normal bundle oriented consistently with the orientation of $L_a$, and  $H^2_\Gamma(U_a) = H^2_\Gamma(U_a) = 0$. Similarly, pulling back by the second retraction, we have $H^1_\Gamma(U_a - L_a)$ generated by the classes of $S^1 \times pt \times (0,1)$ and $pt \times S^1 \times (0,1)$ and $H^2_\Gamma(U_a - L_a)$ generated by the class of $pt \times pt \times (0,1)$.
In all cases, the normal co-orientation of the pullback is simply the pullback of the normal co-orientation.

We can now turn to the cohomology of $M$ itself.
As $M$ is connected but not closed, we already know from \cref{E: first examples} that $H^0_\Gamma(M) \cong \Z$ with generator represented by the identity map and $H^3_\Gamma(M) = 0$ with generator represented by the embedding of any point.
So it remains to consider $H^1_\Gamma(M)$ and $H^2_\Gamma(M)$.

Consider the Mayer-Vietoris sequence for the pair $M = S^3 - L$ and $\amalg U_a$, with union $S^3$ and intersection $\amalg (U_a - L_a)$.
Since we know that $H^1_\Gamma(S^3) = H^2_\Gamma(S^3)=0$, we obtain an isomorphism
$$H^1_\Gamma(\amalg U_a) \oplus H^1_\Gamma(S^3 - L) \xr{j} H^1_\Gamma( \amalg(U_a - L_a)).$$
We also know that $H^3_\Gamma(M) = H^3_\Gamma(\amalg U_a) = H^2_\Gamma(\amalg U_a) = 0$, so we have an exact sequence
$$0 \to H^2_\Gamma(S^3 - L) \xr{j} H^2_\Gamma( \amalg(U_a - L_a)) \xr{\delta} H^3_\Gamma(S^3) \to 0.$$

Let us first consider the isomorphism $H^1_\Gamma(\amalg U_a) \oplus H^1_\Gamma(S^3 - L) \xr{j} H^1_\Gamma( \amalg(U_a - L_a)).$
As observed above, each $H^1_\Gamma(\amalg U_a)$ can be taken to be generated by the class of an inclusion
$pt \times B^2 \into S^1 \times B^2$, and these maps restrict to inclusions $pt \times S^1 \times (0,1)  \into S^1 \times S^1 \times (0,1) \cong U_a - L_a$.
This accounts for half the summands of $H^1_\Gamma( \amalg(U_a - L_a))$.
The remainder of $H^1_\Gamma( \amalg(U_a - L_a))$ must therefore come from $H^1_\Gamma(M)$, and if we can find maps representing elements of $H^1_\Gamma(M)$ that restrict to $S^1 \times pt \times (0,1)  \into S^1 \times S^1 \times (0,1) \cong U_a - L_a$, then these must be linearly independent and generate $H^1_\Gamma(M)$.
But now recall that any knot $K$ in $S^3$ possesses a Seifert surface, i.e.\ a compact oriented embedded manifold whose boundary is $K$ \cite{REF FOR SMOOTH}.
Seifert surfaces are usually assumed to be connected, but that will not be necessary for us.
So given a component $L_a$ of our link, let $S_a$ be a Seifert surface for it; we may also assume that $S_a$ is transverse to all of the other components of $L$.
If we let $\hat S_a = S_a - (S_a \cap L)$, then the embedding of $\hat S_a$ into $S^3 - L$, with the co-orientation determined by the orientations of $S_a$ and $S^3$, is a precochain.

Furthermore, if we consider a closed collar of $\bd S_a$, which we can write as $L_a \times [0,1]$ with $L_a \times 0 = L_a$, then $L_a \times 1$ is a parallel copy of $L_a$.
Furthermore, as $S_a$ is assumed to be embedded in $S^3$, it must be the case that $S_a - (L_a \times [0,1))$, whose boundary is $L_a \times 1$, does not intersect $L_a$. So $L_a$ and its translate $L_a \times 1$ do not link each other (see \cite[Section 5.D]{Ro76} for a general discussion of linking numbers).
Consequently, we can identify $U_a$ with $S^1 \times B^2$ in such a way that $S_a \cap (U_a-L_a) \cong S^1 \times pt \times (0,1)$.
So the cochains represented by the $\hat S_a$ must generate $H^1_\Gamma(M)$.

To be a bit more specific for future use, let $\beta_{L_a}$ be the orientation of $L_a$. Then by \cref{Con: oriented boundary}, the orientation $\beta_{S_a}$ of $S_a$ will be such that $\beta_{\nu_a} \wedge \beta_{L_a} = \beta_{S_a}$, where $\nu_a$ is the outward pointing normal to $S_a$ at the boundary.
So at such a point the induced co-orientation of the embedding of $S_a$ is $(\beta_{S_a},\beta_{S^3}) = (\beta_{\nu_a} \wedge \beta_{L_a},\beta_{S^3})$.
Thus if we let $n_a$ be a normal vector to $S_a$, the corresponding normal co-orientation of $S_a$ is such that $\beta_{\nu_a} \wedge \beta_{L_a} \wedge \beta_{n_a} = \beta_{S^3}$.
So we see the normal co-orientation is just the standard righthand rule for orienting the normal to a surface that cobounds an oriented curve in $\R^3$, and this agrees with our convention for generators of $H^1_\Gamma(U_a-L_a)$.

Next we consider the sequence for $H^2_\Gamma$.
As $H^3_\Gamma(S^3) \cong \Z$ and $H^2_\Gamma(\amalg(U_3-L_3)) \cong \Z^{|A|}$, the short exact sequence above splits, and we must have abstractly that $H^2_\gamma(M) \cong \Z^{|A|-1}$.
We claim in this case that $H^2_\Gamma(M)$ is generated by the classes of smooth curve that run from one link component to another.
Indeed, suppose that $\gamma_{ab} \colon [0,1] \to S^3$ is a curve with $\gamma_{ab}(0) \in L_a$, $\gamma_{ab}(1) \in L_b$, and $\gamma_{ab}((0,1)) \cap L = \emptyset$.
By restricting the domain to $(0,1)$ and using the co-orientation induced by the standard orientations of $(0,1)$ and $S^3$, we obtain a precochains $\hat \gamma_{ab}$.
Now recall that $H^2_\Gamma(U_a-L_a)$ is generated by the class of $pt \times pt \times (0,1)$.
We can clearly arrange $\hat \gamma_{ab}$ and our identification of $U_a - L_a$ with $S^1 \times S^1 \times (0,1)$ so that the restriction of $\hat \gamma_{ab}$ to $U_a - L_a$ has this form.

So if we write the generators of $H^2_\Gamma(\amalg(U_a-L_a))$ as $x_a$ and choose consistent co-orientations, we see that, possibly up to a universal sign, $j(\underline{\hat \gamma_{ab}}) = x_a - x_b$.
Now let us order our indexing set $A$ and relabel $A = \{1, \ldots , n\}$.
We then have that the set $\mc G = \{\underline{\hat \gamma_{i,i+1}}\}_{i=1}^{n-1}$ maps onto the set $\{x_i-x_{i+1}\}_{i=1}^{n-1}$, which spans a subgroups of $H^2_\Gamma(\amalg(U_a-L_a))$ such that the quotient is $\Z$.
Thus the elements of $\mc G$ generate $H^2_\Gamma(M) \cong \Z^{n-1}$.
We also see from the exact sequence that $\underline{\hat \gamma_{i,i+1}}+\underline{\hat \gamma_{i+1,i+2}} = \underline{\hat \gamma_{i, i+2}}$, as their difference maps to $0$, but $j$ is injective.
More generally, $\underline{\hat \gamma_{i,i+1}} + \cdots + \underline{\hat \gamma_{i+k-1, i+k}} =\underline{\hat \gamma_{i,i+k}}$.

So, in summary, the generators of $H^1_\Gamma(M)$ can be represented by Seifert surfaces, while the elements of $H^2_\gamma(M)$ can be realized as arcs between neighboring components, subject to the above relations.

Of course the real power of geometric cohomology is that we can also derive consequences for the \emph{ring} $H^*_\Gamma(M)$ from geometry.
For example, if it is possible to choose Seifert surfaces $S_a$ simultaneously that are disjoint or intersect only in closed curves, then all cup product on $H^1_\Gamma(M)$ vanish.
On the other hand, if two Seifert surfaces $S_a$ and $S_b$ intersect in just an arc between $L_a$ and $L_b$, then this corresponds to the cup product of two generators of $H^1_\Gamma(M)$ being a generator of $H^2_\Gamma(M)$.
\end{comment}


--------------------------------------------
Alex

We saw in \cref{T: PD} that Poincar\'e duality takes a very strong chain-level form in the world of geometric chains and cochains.
It is not clear that we can provide quite the same chain-level version of Alexander duality, but nonetheless the duality in homology and cohomology can be expressed by pleasingly geometric maps, at least when our subspaces of $S^n$ are taken to be embedded submanifolds.
This is the case, for example, when studying links.

So for this section let us assume an underlying sphere $S^n$ and subspace $L$ that is the union of smoothly embedded closed manifolds, not necessarily of the same dimension.
We recall that Alexander duality is the statement that $\td H_i(L) \cong \td H^{n-i-1}(S^n-L)$ for all $i \geq 0$.
The standard proof REF proceeds by using relative homology, Lefschetz duality, excision, and some homotopy equivalences. We will define two maps that will be inverses in most dimensions, though we'll have to take some extra care when $i=0$ or $n-i$.

First, let $\mc K_i = \ker(H_i^\Gamma(L) \to H_i^\Gamma(S^n)$, where the map is induced by the inclusion $L \into S^n$.
In all degrees $i>0$, this will simply be $H_i(\Gamma(L))$.
For $0<i<n$, this follows immediately from the computation of the homology of spheres REF.
In dimension $n$, we observe that $H_n^\Gamma(L) = 0$ by \cref{E: dimension range}, as all the manifold components of $L$ must have dimension $<n$.
In degree $0$, if $L$ has $\ell$ components, we have the obvious surjection $H_0^\Gamma(L) \cong \Z^\ell
\to H_0^\Gamma(S^n) \cong \Z$, so the kernel $\mc K_0$ is isomorphic to $\Z^{\ell-1}$.
Generators of $\mc K_0$ can be represented by prechains consisting of pairs of points in separate components of $L$ with opposite orientations, as if they were in the same component then they would represent $0$ in $H_0^\Gamma(L)$.
If $g_{ij}$ is a homology class represented by a prechain with a positive point in component $L_i$ and a negative point in component $L_j$, then we also have the relations $g_{ij}+g_{jk}=g_{ik}$.
So we can take $\mc K_0$ to be the free abelian group on generators $g_{1i}$ of $i>1$ or on generators $g_{i,i+1}$ for $1\leq i\leq \ell-1$.

Now, let $V$ be a prechain representing an element of $\mc K_i$.
We can think of $V$ as being a prechain in $L$ or in $S^n$, and as it represents the trivial homology class in $S^n$, there is by \cref{R: cycles and boundaries} some $Z \in PC^\Gamma_{i+1}(S^n)$ such that $\bd Z \amalg -V \in Q_*(S^n)$.
Now let $\mathring Z = r_Z^{-1}(S^n-L)$, and let $\mathring r_Z$ be the restriction of $r_Z$ to be a map from $\mathring Z$ to $S^n-L$.
The space $\mathring Z$ is an open subset of $Z$, and so it is a manifold with corners.
As $Z$ and $S^n$ are oriented, we have the induced co-orientation on $r_Z$ and hence on $\mathring r_Z$.  We also note that $\mathring r_Z$ is proper: if $C \subset S^n-L$ is compact, then $r_Z^{-1}(C) = \mathring r_Z^{-1}(C)$ is compact, as all maps with compact domains are proper.
Alternatively, we can observe that $\mr r_Z$ is simply the pullback to $S^n-L$ of $Z$ after we convert it to a precochain by taking the induced co-orientation.
So $\mathring r_Z \colon \mathring Z \to S^N - L$ is a precochian.
In fact, it is a precocyle, as by \cite{leibniz} we know $\bd{\mr Z}$ is the pullback to the boundaryless $S^n-L$ of $\bd Z$, which is in $Q^*(S^n-L)$ because $\bd Z \amalg -V$ pulls back to an element of $Q(S^n-L)$ by \cref{L: pullback map Q}, while the pullback of $V$ is $\emptyset$.
We will show that this construction gives a well defined homomorphism $\mc Z \colon \mc K_i \to H^{n-i-1}_\Gamma(S^n-L)$.

\begin{proposition}
The map $\mc Z \colon \mc K_i \to H^{n-i-1}_\Gamma(S^n-L)$ is a well-defined homomorphism.
In particular, it does not depend on the choices involved in the definition.
\end{proposition}
\begin{proof}
Once we show that $\mc Z$ is independent of the specific choices of $V$ (representing $\uV \in H_i(L)$) and $Z$, it will follow immediately that $\mc Z$ is a homomorphism as given $V,W\in PC_i^\Gamma(L)$ representing homology classes and $Z$ and $Y$ as constructed above from $V$ and $W$ respectively, then we may represent $\mc Z(\uV + \uW)$ by $\mathring Z \amalg \mr Y$, which will represent $\underline{\mr Z}+\underline{\mr Y}\in H^{n-i-1}_\Gamma(S^n-L)$.

So suppose that $V,V' \in PC_i^\Gamma(L)$ represent the same homology class $\uV$ in $\mc K_i$, possibly with $V = V'$.
Let $Z,Z' \in PC_{i+1}(S^n)$ be such that $\bd Z \amalg -V, \bd Z' \amalg -V' \in Q_*(S^n)$.
As $\uV=\underline{V'} \in H_i^\Gamma(L)$, there is an $S \in PC_{i+1}(L)$ such that $\bd S \amalg -V \amalg V' \in Q_*(L)$.
We note that prechains in $Q_*(L)$ also represent elements of $Q_*(S^n)$.
So now consider $S \amalg -Z \amalg Z'$.
By taking boundaries and rearranging, we have $$\bd (S \amalg -Z \amalg Z')\amalg -V \amalg V' \amalg V \amalg -V' = (\bd S \amalg -V \amalg V')\amalg -(\bd Z \amalg -V) \amalg (Z'\amalg -V')\in Q_*(S^n).$$
But  $-V \amalg V' \amalg V \amalg -V'$ is trivial, so by \cref{L: Lipy12} we have $S \amalg -Z \amalg Z'\in Q_*(S^n)$, i.e.\ $S \amalg -Z \amalg Z'$ represents an $i+1$ cycle by \cref{R: cycles and boundaries}.

We first suppose $i+1\neq n$, so that $H_{i+1}^\Gamma(S^n) = 0$ and there exists $Y \in PC_{i+1}(S^n)$ such that $\bd Y \amalg -(S \amalg -Z \amalg Z') \in Q_*(S^n)$.
Let us now restrict to $S^n-L$ by converting orientations of domains to the induced co-orientations of their maps and then performing the pullback to $S^n-L$.
As $S$ maps into $L$, we have $\mr S = \emptyset$.
As pullbacks of elements in $Q$ remain in $Q$ by \cref{L: pullback map Q} and using the Leibniz rule \cred{leibniz}, we obtain that $$\bd \mr Y \amalg \pm (-\mr Z \amalg \mr Z') \in Q_*(S^n-L).$$
The $\pm$ allows us to avoid thinking about any signs that might be introduced by switching from chains to cochains in two different indices, and it is sufficient for $\mr Z$ and $\mr Z'$ to represent the same cohomology class, again by \cref{R: cycles and boundaries}.




\end{proof}

















\bibliographystyle{amsplain}
\bibliography{../bib}



Several diagrams in this paper were typeset using the \TeX\, commutative
diagrams package by Paul Taylor.


\end{document}
