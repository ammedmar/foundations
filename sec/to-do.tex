% !TEX root = ../foundations.tex

\section*{To-Do}

\begin{enumerate}
	\item Is there a place where we do not use integer coefficients? Should Or(E) just be $\wedge^n(E)$?

	\item Add an argument that our definition of co-orientation agrees with Lipyanskiy's?

	\item Thom isomorphism (what does this mean in this setting - we can't have a singular space, and we do not have relative cohomology)

	\item anibal: diagrams in diagrams.sty are sensitive to having a blank space before and after.
	Make this choice homogeneous all around.

	\item anibal: consider a command \verb|\ie| to make homogeneous the use of i.e.\ since some use a \verb|\ | after.

	\item Greg left the following in in s3 and I commented out.
	The exact position can be found searching for any string in it.

	Dev and Anibal, please check the following arguments carefully as I'm not 100\% confident in it.
	It gives the ``right'' answer but I'm a little uncomfortable divorcing the order of the orientation terms from the order of the manifold terms.
	Of course this happens all the time - even if we think of $\R^2$ as $\R_x \oplus \R_y$ we can still think about the two-form $y \wedge x$, but I'm still a little nervous about maybe having missed a sign somewhere.
	I'm also a little nervous about my trick of taking $a$ and $b$ to be even so that they won't contribute, but the earlier work says that this should be allowable.
	Presumably if I didn't do this there would be a bunch of extra signs that miraculous cancel out, but I'm not so excited about trying that out in detail to see.

	\item Similarly with:

	Add a version for pullbacks (as opposed to fiber products)? Might have some extra signs to figure out.
	Not really needed anywhere I do not think.

	\item From s5 I commented out this:

	GBF: Might want to try to combine those into a single lemma somewhere at some point, but it looks like it might be less messy, if a bit redundant, not to.

	\item also:

	Fix tikz arrows?

\begin{comment}
	\item \sout{Picture for creasing.}
	\item Compactness and orientation assumptions on Theorem 3.13 (transversality constrains preserve q-iso type).

	\item \sout{Treatment of creasing.}
	\item Guillemin-Pollock for mnfds with corner.

	\item Clarify isomorphisms used in orientations and make more explicit how the Lipyanskiy orientations fit.


	\item More on Mayer-Vietoris - check full argument
	\item Poincar\'e Lemma - check new proof
	\item (Anibal) Add a better treatment of ``cst" from \verb|Flows/old/pd_cubical_S2.Feb16.tex| \\
	Greg: Let K be any finite set of cubical faces and let L be a single cubical face. We need $cst(K) \cup cst(L)$ to be $cst(K \cup L)$ (maybe this part is just by definition?) and we need $cst(K) \cap cst(L)$ to be $cst(K ? L)$ where $K ? L$ needs to be some set of faces with cardinality less than or equal to that of K.
\end{comment}

	\begin{comment}
		\item \sout{Picture for creasing.}
		\item Compactness and orientation assumptions on Theorem 3.13 (transversality constrains preserve q-iso type).

		\item \sout{Treatment of creasing.}
		\item Guillemin-Pollock for mnfds with corner.

		\item Clarify isomorphisms used in orientations and make more explicit how the Lipyanskiy orientations fit.

		\item More on Mayer-Vietoris - check full argument
		\item Poincar\'e Lemma - check new proof
		\item (Anibal) Add a better treatment of ``cst" from \verb|Flows/old/pd_cubical_S2.Feb16.tex| \\
		Greg: Let K be any finite set of cubical faces and let L be a single cubical face. We need $cst(K)\cup cst(L)$ to be $cst(K\cup L)$ (maybe this part is just by definition?) and we need $cst(K)\cap cst(L)$ to be $cst(K ? L)$ where $K ? L$ needs to be some set of faces with cardinality less than or equal to that of K.
	\end{comment}

	\begin{comment}
		\item Reference for pullback of normal bundle is normal bundle of pullback
		\item Reference for pullback of tangent spaces is tangent space of pullbacks (argument already given?)
	\end{comment}

\end{enumerate}