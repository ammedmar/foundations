\documentclass[12pt]{article}
\paperheight=11in
\paperwidth=8.5in
\renewcommand{\baselinestretch}{1.05}
\usepackage{amsmath,amsthm,verbatim,amssymb,amsfonts,amscd,diagrams, graphicx, mathrsfs, mathtools}
\usepackage[usenames,dvipsnames]{color}
\usepackage{amstext}
%\usepackage{cite}
%\usepackage[notref]{showkeys}
\usepackage{color}
\usepackage{hyperref}
\usepackage{cleveref}

\newcommand\purple[1]{\marginpar{replacement}\textcolor{Purple}{#1}}
\newcommand\blue[1]{\marginpar{new}\textcolor{blue}{#1}}                       %
\newcommand\red[1]{\marginpar{??}\textcolor{red}{#1}}                         %
\newcommand\green[1]{\marginpar{delete ok?}\textcolor{green}{#1}}
%\newcommand\ownremark[1]{\marginpar{remark for own use}\textcolor[rgb]{0.5,0.5,0}{#1}}                   %


\topmargin0.0cm
\headheight0.0cm
\headsep0.0cm
\oddsidemargin0.0cm
\textheight23.0cm
\textwidth16.5cm
\footskip1.0cm


\setcounter{tocdepth}{3}

\theoremstyle{plain}
\newtheorem{theorem}{Theorem}[section]
\newtheorem{corollary}[theorem]{Corollary}
\newtheorem{lemma}[theorem]{Lemma}
\newtheorem{sublemma}[theorem]{Sublemma}
\newtheorem{proposition}[theorem]{Proposition}

\theoremstyle{definition}
\newtheorem{definition}[theorem]{Definition}

\theoremstyle{remark}
\newtheorem{remark}[theorem]{Remark}
\newtheorem{example}[theorem]{Example}
\newcommand{\codim}{\textup{codim}}
\newarrow{ul}---->
\newarrow{Backwards}<----


\newcommand{\ilim}{\varprojlim}
\newcommand{\Lk}{\textup{Lk}}
\newcommand{\xr}{\xrightarrow}
\newcommand{\xl}{\xleftarrow}
\newcommand{\onto}{\twoheadrightarrow}
\newcommand{\hra}{\hookrightarrow}
\newcommand{\Td}[1]{\Tilde{#1}}
\newcommand{\td}[1]{\tilde{#1}}
\newcommand{\into}{\hookrightarrow}
\newcommand{\Z}{\mathbb{Z}}
\newcommand{\X}{\mathbb{X}}
\newcommand{\Q}{\mathbb{Q}}
\newcommand{\R}{\mathbb{R}}
\newcommand{\G}{\mathbb{G}}
\newcommand{\N}{\mathbb{N}}
\newcommand{\F}{\mathbb{F}}
\newcommand{\C}{\mathbb{C}}
\newcommand{\D}{\mathbb{D}}
\renewcommand{\L}{\mathbb{L}}
\newcommand{\bd}{\partial}
\newcommand{\pf}{\pitchfork}
\newcommand{\ra}{\rightarrow}
\newcommand{\la}{\leftarrow}
\newcommand{\Ra}{\Rightarrow}
\renewcommand{\H}{\mathbb H}
\newcommand{\rla}{\RightLeftarrow}
\newcommand{\mc}[1]{\mathcal{#1}}
\newcommand{\ms}[1]{\mathscr{#1}}
\newcommand{\bb}[1]{\mathbb{#1}}
\newcommand{\dlim}{\varinjlim}
\newcommand{\vg}{\varGamma}
\newcommand{\blm}[2]{\langle  #1 , #2 \rangle}
\newcommand{\bl}[2]{\left( #1 , #2 \right)}
\newcommand{\vs}{\varSigma}
\newcommand{\holink}{\textup{holink}}
\newcommand{\map}{\operatorname{map}}
\newcommand{\hl}{\operatorname{holink}}
\newcommand{\wt}{\widetilde}
\renewcommand{\hom}{\textup{Hom}}
\renewcommand{\S}{\mathbb{S}}
\newcommand{\Hom}{\textup{Hom}}
\newcommand{\SHom}{\textup{\emph{Hom}}}
\newcommand{\Ext}{\textup{Ext}}
\newcommand{\mf}{\mathfrak}
\newcommand{\ih}{IH^{\bar p}}
\newcommand{\di}{\textup{dim}}
\newcommand{\im}{\textup{im}}
\newcommand{\cok}{\textup{cok}}
\newcommand{\coim}{\textup{coim}}
\newcommand{\bp}{\boxplus}
\renewcommand{\i}{\mathfrak i}
\renewcommand{\j}{\mathfrak j}
\newcommand{\ka}{\kappa}
\renewcommand{\Cup}{\smile}
\renewcommand{\Cap}{\frown}
\newarrow{Onto}----{>>}
\newarrow{Equals}=====
\newcommand{\fl}{\textup{FL}}
\newcommand{\wfl}{\textup{WFL}}
\newcommand{\Ker}{\mbox{Kernel }}
\newcommand{\sgn}{\textup{sgn}}
\newcommand{\sect}[1]{\vskip1cm \noindent\paragraph{#1}}
\newcommand{\id}{\textup{id}}
\newcommand{\IAW}{\textup{IAW}}
\newcommand{\pt}{\textup{pt}}
\newcommand{\diag}{\mathbf{d}}
\newcommand{\aug}{\mathbf{a}}

\begin{document}


\subsubsection{Splitting}
In this section, we begin by consolidating and extending some results previously encountered in \cref{E: manifold decomposition,S: codim 0 and 1 co-or}, namely the splitting of a manifold with corners over a manifold without boundary $r_W \colon W \to M$ into pieces $W^+$ and $W^-$ with common boundary $W^0$ using a function $\phi \colon M \to \R$.
In this subsection we use our previous discussions to establish the facts we will need in the context of prechains and precochains, and in the next subsection we will utilize these spaces to perform creasing.

Our standard assumptions throughout this will be that $M$ is a manifold without boundary and $\phi \colon M \to \R$ is a smooth map having $0$ as a regular value in the classical sense, i.e.\ for all $x \in \phi^{-1}(0)$ the differential $D_x\phi$ is nonzero.
We also assume an element of $PC_*^\Gamma(M)$ or $PC^*_\Gamma(M)$ represented by $r_W \colon W \to M$ such $0$ is also a regular value for $\phi r_W$, meaning by \cref{D: regular value} that $\phi r_W$ is transverse to $0$, which is equivalent to assuming $0$ is a regular value in the classical sense for the restriction of $\phi r_W$ to each stratum of $W$.
We let $M^0 = \phi^{-1}(0)$, $M^- = \phi^{-1}((\infty,0])$, and $M^+ = \phi^{-1}([0,\infty))$, and analogously for $W^0$, $W^-$, and $W^-$.
We sometimes write $M^\pm$ for statements that could involve either $M^+$ or $M^-$, and similarly for $W$.

\begin{remark}
	For simplicity of notation, we primarily use $0$ as our value in $\R$ at which to perform splittings, though it should be clear that we could split $W$ using any appropriately regular value to obtain analogous results.
\end{remark}

\begin{lemma}\label{L: 0 transverse M0}
	Zero is a regular value for $\phi r_W$ if and only if $r_W$ is transverse to the inclusion of $M^0$ into $M$.
\end{lemma}
\begin{proof}
	As $0$ is a regular value for $\phi$, by classical differential topology $M^0$ is an embedded codimension-one submanifold of $M$, and in a neighborhood of each point of $M^0$, the map $\phi$ behaves up to diffeomorphisms like the standard projection of $\R^m$ to the first coordinate; see Section 1.4 \cite{GuPo74}.
	In particular, at each $z \in M^0$, we have $T_zM^0 = \ker(D_z\phi)$.
	So the linear subspace spanned by a vector $v \in T_zM$ is transverse to $T_zM^0$ if and only if its image under $D_z\phi$ is non-zero.
	It follows that $r_W$ is transverse to $M^0$ if and only if $\phi r_W$ is transverse to $0$.
\end{proof}


\begin{lemma}\label{L: pm0 as fiber products}
	With our standing assumptions, there are diffeomorphisms
	\begin{align*}
		M^0 & \cong 0 \times_{\R} M &W^0 & \cong 0 \times_{\R} W & W^0& \cong M^0 \times_M W \\
		M^- & \cong (-\infty,0] \times_{\R} M &W^- & \cong (-\infty,0] \times_{\R} W& W^-  &\cong M^- \times_M W\\
		M^+ & \cong [0,\infty) \times_{\R} M & W^+ & \cong [0,\infty) \times_{\R} W& W^+ &\cong M^+ \times_M W.\\
	\end{align*}
	In particular, these spaces are all manifolds with corners.
	Note that it is possible for some of these spaces to be empty.
\end{lemma}
\begin{proof}
	These diffeomorphisms are discussed above in \cref{E: manifold decomposition}.
\end{proof}

The rightmost column above demonstrates $W^0$, $W^-$, and $W^+$ as fiber products over $M$.
We can use these descriptions to realize these spaces as prechains or precochains.
Note: depending on $\phi$ and $r_W$, some of these spaces may be empty, in which case appropriate versions of the following statements hold vacuously.

\begin{lemma}
	Suppose $W \in PC^*_\Gamma(M)$, and $0$ is a regular value of $\phi \colon M \to \R$ and $\phi r_W \colon W \to \R$.
	Let the inclusions $M^\pm \into M$ have their tautological co-orientations, and give $M^0$ the co-orientation determined by its normal vector field oriented by pulling back the normal vector field over $0$ in $\R$ co-oriented by the positively-directed normal vector.
	Then $W^0$, $W^-$, and $W^+$, defined as the fiber products of $M^0$, $M^-$, and $M^+$ with $W$, are elements of $PC^*_\Gamma(M)$.
	Furthermore,
	\begin{align*}
		\bd(W^-) &=  -(W^0) \bigsqcup (\bd W)^- \\
		\bd (W^+) &= W^0 \bigsqcup (\bd W)^+\\
		(\bd W)^0 &= -\bd (W^0).
	\end{align*}
\end{lemma}
\begin{proof}
	As $M^\pm$ and $M^0$ are all closed subsets of $M$, their inclusions are all proper maps, so with the co-orientations assigned above, they all represent elements of $PC^*_\Gamma(M)$.
	By \cref{L: 0 transverse M0}, the map $r_W \colon W \to M$ is transverse to $M^0$ and $M^\pm$.
	It follows that $W^0$, $W^-$, and $W^+$, defined as the fiber products of $M^0$, $M^-$, and $M^+$ with $W$, are elements of $PC^*_\Gamma(M)$ by \cref{D: PC products,L: product preserves iso}.
	The co-orientations have been discussed previously in \cref{S: codim 0 and 1 co-or}, and consequently we have the boundary computations from \cref{E: codim 1 pullbacks,C: co-orient W0}.
\end{proof}


\begin{lemma}\label{L: W0 chain}
	Suppose $W \in PC_*^\Gamma(M)$, and $0$ is a regular value of $\phi \colon M \to \R$ and $\phi r_W \colon W \to \R$.
	Let $\R$, $(-\infty, 0]$, and $[0,\infty)$ have their standard orientations, and give $0$ its positive orientation.
	Now orient $W^0$, $W^-$, and $W^+$ by realizing them as oriented fiber products of $0$,  $(-\infty, 0]$, and $[0,\infty)$ with $\phi r_W \colon W \to \R$ over $\R$.
	Then the restrictions of $r_W \colon W \to M$ to $W^0$, $W^-$, and $W^+$ realize elements of $PC_*^\Gamma(M)$.
	When $W^\pm$ are nonempty, their orientations agree with the orientation of $W$.
	Furthermore, as elements of $PC_*^\Gamma$ we have
	\begin{align*}
	\bd(W^-) &=  (W^0) \bigsqcup (\bd W)^- \\
	\bd(W^+) &= -W^0 \bigsqcup (\bd W)^+\\
	(\bd W)^0 &= -\bd (W^0).
	\end{align*}
\end{lemma}

\begin{proof}
	Since $W$ is compact, so will be $W^0$, $W^-$, and $W^+$ as closed subsets of $W$.
	We have orientations of $W^0$, $W^-$, and $W^+$ as defined above, and so restricting $r_W$ we have elements of $PC_*^\Gamma(M)$.

	As the orientations of the fiber products are determined locally, we can consider a point in the interior of $W^+$ with an open neighborhood $N$ also in the interior of $W^+$.
	The orientation of $N$ in $W^+$ will be consistent with its orientation in the restricted fiber product $[0,\infty) \times_\R N$, which will be the same as its orientation in $(0,\infty) \times_{(0,\infty)} N$.
	But this is the same as the initial orientation of $N$ as a subset of $W$ by \cref{P: oriented fiber product basic properties}.
	The same argument holds for $W^-$.

	We then have by \cref{P: oriented fiber boundary}, our conventions for oriented boundaries, and the standard computations for the boundaries of $(-\infty,0]$ and $[0,\infty)$ that
	\begin{equation*}
			\bd W^- = \bd ((-\infty,0] \times_\R W) = (0 \times_\R W)  \sqcup ((-\infty,0] \times_\R \bd 	W) = W^0 \sqcup (\bd W)^-,
	\end{equation*}
	and
	\begin{equation*}
			\bd W^+ = \bd ([0,\infty) \times_\R W) = (-0 \times_\R W)  \sqcup ([0,\infty) \times_\R \bd W) 	= (-W^0) \sqcup (\bd W)^+.
	\end{equation*}
	We also compute
	\begin{equation*}
			\bd W^0 = \bd (0 \times_\R W) = - 0 \times_\R \bd W = -(\bd W)^0.
\end{equation*}
\end{proof}

Lastly, we will occasionally need to consider the preimage of an interval $[p,q] \subset \R$.
If $p < q$ are regular values for $\phi$, then $\phi^{-1}([p,q])$ will be an embedded manifold with boundary in $M$ that we denote $M^{[p,q]}$.
If further $\phi r_W$ is transverse $p$ and $q$ then we can form $W^{[p,q]} = M^{[p,q]} \times_M W$.
We also let $W^p = \phi^{-1}(p) \times_M W$, and similarly for $q$.

\begin{lemma}
	If $p < q$ are regular values for $\phi \colon M \to \R$, if $W \in PC^*_\Gamma(M)$, and if the inclusion $M^{[p,q]} \into M$ is given its tautological co-orientation, then
	$W^{[p,q]} \in PC^*_\Gamma(M)$ and
	$$\bd W^{[p,q]} = W^p \sqcup -W^q \sqcup (\bd W)^{[p,q]}.$$
	Similarly, if $W \in PC_*^\Gamma(M)$, then so is $W^{[p,q]}$, its orientation agrees with that of $W$, and
	$$\bd W^{[p,q]} = -W^p \sqcup W^q \sqcup (\bd W)^{[p,q]}.$$
\end{lemma}
\begin{proof}
	We note that we only need to check transversality at $p$ and $q$, as the inclusion of the interior of $M^{[p,q]}$ to $M$ is transverse to any map.

	In the precochain case, $W^{[p,q]} = M^{[p,q]} \times_M W \in PC^*_\Gamma(M)$ by \cref{D: PC products,L: product preserves iso}, and the boundary formula follows from the Leibniz rule analogously to our computations above for $W^\pm$.

	In the prechain case, we observe that $W^{[p,q]} \cong (\phi r_W)^{-1}([p,q])$ by similar arguments as for $W^\pm$ in \cref{E: manifold decomposition}, so $W^{[p,q]}$ is compact.
	The agreement of orientation and the boundary formula are similar to the argument of \cref{L: W0 chain}.
\end{proof}



\bibliographystyle{amsplain}
\bibliography{../bib}



Several diagrams in this paper were typeset using the \TeX\, commutative
diagrams package by Paul Taylor.


\end{document}
